The following literature review describes challenges in \gls{MSR} multiphysics
modeling, followed by a review of existing multiphysics simulation software
developed for modeling \glspl{MSR}, with a focus on simulating steady-state
operation and transient
accident scenarios which typically occur on the order of a few minutes to
several hours. Long-term dynamics such as fuel burnup and structural corrosion
are outside the scope of this work. Lastly, this chapter reviews the existing
capabilities of Moltres, the MSR simulation software to be developed in this
work.

\section{Challenges in \gls{MSR} Multiphysics Modeling}

While modeling \glspl{MSR} is not necessarily more difficult than modeling
solid-fueled reactors, we must adapt our software tools to accurately model the
unique phenomena found in these circulating-fuel reactors. The differences in
the challenges of simulating \glspl{MSR} compared to solid-fueled reactors stem
mainly from the liquid fuel form of the fuel salt \cite{huff_identifying_2019,
diamond_phenomena_2018}. Firstly, liquids generally exhibit greater thermal
expansion per unit change in temperature than solids. A decrease in density of
the fuel medium increases the likelihood of neutrons escaping the fuel region
and being absorbed by non-fissile material elsewhere in the reactor.
Consequently, combined with the temperature-dependent Doppler broadening of
resonance capture cross sections, \glspl{MSR} possess stronger negative fuel
temperature reactivity feedback than their solid-fueled counterparts. These
phenomena ultimately result in strong interactions between the neutron fluxes
and core temperatures given that neutron fluxes affect core temperatures
through fission heat generation and core temperatures in turn affect neutron
fluxes through the mechanisms as described prior.

Secondly, with the fuel
salt also serving the role of providing cooling in the core, velocity flow
profiles in the fuel salts strongly impact the temperature distribution via
advection-dominated heat transfer. This contrasts with the relatively static
temperature distribution shapes in fuel pins and other forms of solid fuel
matrixes physically separated from the coolant. Therefore, multiphysics
software must employ \textit{tight coupling schemes} to couple the neutronics
and thermal-hydraulics governing equations to accurately capture the strong
multiphysics interactions in transient \gls{MSR} simulations. Tightly coupled
numerical models handle multiphysics interactions by either updating all state
variables simultaneously in one monolithic solve (\textit{full coupling}) or
iterating among all state variables (\textit{fixed point iterations}) until the
solution converges in every timestep \cite{keyes_multiphysics_2013}. Full
coupling tends to be more computationally expensive because it combines all
physics equations into a single large system of equations to be solved
simultaneously, whereas fixed point iterations involve solving smaller systems
separately and iteratively updating the state variables until convergence.
However, full coupling schemes can outperform fixed point iterative methods in
some multiphysics problems as they boast better stability and convergence
rates. In contrast, \textit{Loosely coupled schemes} solve each set of
single-physics equations sequentially using state variable data from the
previous timestep without any iterations within every timestep. Aufiero et al.
\cite{aufiero_development_2014} showed that the loose coupling approach failed
to accurately reproduce an increase in reactor power in an \gls{MSR} in
response to a 150 pcm reactivity insertion.

Lastly, 

\section{Multiphysics MSR Simulation Tools}

In recent years, several simulation tools have been developed for full-core
modeling of fast-spectrum \glspl{MSR}. One approach involves coupling
single-physics neutronics and thermal-hydraulics software via a script which
handles data transfers between the software. For example, researchers at
the Delft University of Technology coupled 3D neutron diffusion software
DALTON \citep{boer_validation_2010} and CFD software HEAT
\citep{de_zwaan_static_2007} to perform a safety analysis of the \gls{MSFR}
\citep{fiorina_modelling_2014}. In a later effort from the same institute,
\cite{tiberga_discontinuous_2019} coupled PHANTOM-$S_N$ and DGFlows
in their participation in the
CNRS Benchmark study \citep{tiberga_results_2020}. Another multiphysics
package was developed at the Paul Scherrer Institute (PSI) coupling
thermal-hydraulics system software TRACE \citep{nrc_trace_2007} with the nodal
neutron diffusion software PARCS \citep{downar_parcs_2010} for the safety
analysis of the \gls{MSFR} \citep{pettersen_coupled_2016}. With modern
advancements in computing hardware and growing access to high-performance
computing, others \citep{laureau_transient_2017,blanco_neutronic_2020} have
developed multiphysics packages coupling CFD software
OpenFOAM with Monte Carlo particle transport software Serpent 2, thus allowing
for high-fidelity neutronics calculations in transient reactor analyses.

Another approach towards creating \gls{MSR} simulation tools involves
developing stand-alone multiphysics software which can perform both
neutronics and thermal-hydraulics calculations. Among earlier efforts,
\cite{nicolino_coupled_2008} and \cite{zhang_development_2009} recognized the
need for more robust multiphysics coupling techniques and higher-fidelity
thermal hydraulics solutions to accurately capture complex flow profiles in
pool-type \glspl{MSR}. They each independently developed unnamed multiphysics
simulation tools and demonstrated their tools with non-moderated \gls{MSR}
designs. Later, \cite{li_transient_2015} demonstrated the steady-state and
transient analysis capabilities of COUPLE, a neutronics and thermal-hydraulics
software developed at the Karlsruhe Institute of Technology.  Others opted to
leverage general multiphysics software platforms such as OpenFOAM
\citep{openfoam_openfoam_2021} and \gls{MOOSE}
\citep{gaston_physics-based_2015}.
Several parallel efforts at various European institutes have developed coupled
neutronics and thermal-hydraulics tools in OpenFOAM. These tools share some
similarities such as implementing the $SP_N$ simplified $P_N$ neutron transport
model and leveraging OpenFOAM's turbulent flow modeling capabilities.
Differences include the aforementioned external coupling capability with
Serpent 2 from the Centre national de la scientifique (CNRS)
\citep{blanco_neutronic_2020}, fuel compressibility modeling and helium bubble
tracking capability from Politecnico di Milano (PoliMi)
\citep{cervi_development_2019}, and fuel performance analysis capability from
PSI \citep{fiorina_creation_2018}. Within the modular MOOSE finite element
framework, \gls{MSR} simulation tools include Moltres
\citep{lindsay_introduction_2018}, the subject of this work and Rattlesnake
\citep{wang_rattlesnake_2021}, both of which rely on MOOSE's built-in
incompressible flow physics module. Rattlesnake is primarily a radiation
transport application but the MOOSE framework facilitates multiphysics coupling
with other MOOSE-based applications such as Pronghorn for thermal-hydraulics
such that all applications share the same data structure, thus eliminating
computational costs from external data transfers and optionally allowing for
all physics to be solved simultaneously instead of sequentially through
fixed-point iterations. Similarly, Moltres benefits from the highly-integrated
cross compatibility within the ``ecosystem'' of MOOSE-based applications.

\section{Moltres}
