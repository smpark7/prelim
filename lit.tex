\glspl{MSR} possess unique characteristics which render existing \gls{LWR}
analysis software inappropriate for \gls{MSR} analysis. Legacy \gls{LWR}
software typically scale poorly on modern high-performance computing
clusters and do not support complex geometries beyond regular \gls{LWR} fuel
assembly lattices. Furthermore, \glspl{MSR} feature strong multiphysics
coupling which force segregated solvers into taking smaller timesteps to
maintain accuracy. This chapter provides a brief history of \glspl{MSR},
followed by a discussion of the challenges in \gls{MSR} multiphysics modeling
for reactor accident analysis. Next, this chapter presents a literature
review of existing multiphysics simulation software developed for \glspl{MSR}
analysis. This work focuses on software for analysing short-term reactor
dynamics which requires the ability to accurately simulate various transient
scenarios such as reactor start-up and coast-down, load-following operations,
steady-state operation, and accident analysis. Long-term dynamics such as fuel
burnup and structural corrosion fall outside the scope of this work. Lastly,
this chapter provides a literature review of turbulence modeling and control
rod modeling, and their relevance in \glspl{MSR} modeling.

\section{\gls{MSR}}

The first \gls{MSR}, named the \gls{ARE}, dates back to the 1940s
as part of the US Aircraft Nuclear Propulsion program
\cite{rosenthal_molten-salt_1970}. Researchers recommended molten fluoride
salts in particular for high uranium solubility, chemical stability, low vapor
pressure even at high temperatures, good heat transfer properties,
resistance against radiation damage, and reduced corrosive effects on some
common structural material \cite{rosenthal_molten-salt_1970}. They
subsequently built the 2.5 MW$_{\text{th}}$ ARE reactor at \gls{ORNL}, where
it achieved criticality on November 1954 and generated 100 MWh over nine days.
The reactor ran on enriched uranium in a molten salt mixture of NaF,
ZrF$_4$, and UF$_4$ with BeO neutron moderators. The aircraft program
ultimately never came to fruition as the development of intercontinental
ballistic missiles effectively eliminated the need for long-range
nuclear-powered bomber aircraft.

However, the successful demonstration of the \gls{ARE} spurred further
research into adapting \glspl{MSR} for civilian power generation
\cite{rosenthal_molten-salt_1970}. One key finding from the
research was that the thorium fuel cycle had a better breeding ratio than the
$^{238}$U-to-$^{239}$Pu fuel cycle in thermal-spectrum reactors.
Ultimately, these efforts culminated in the design, construction, and
successful operation of the \gls{MSRE}, a graphite-moderated thermal
\gls{MSR}. The \gls{MSRE} had a
graphite-moderated design with a LiF-BeF$_2$-ZrF$_4$-UF$_4$ fuel salt mixture,
initially rated at 10 MW$_{\text{th}}$ but later restricted to 8
MW$_{\text{th}}$ due to a miscalculation of heat transfer capabilities
\cite{haubenreich_experience_1970}. In January 1969, the \gls{MSRE} became the
first reactor to run on $^{233}$U fuel.

Building on their experience with the \gls{MSRE}, \gls{ORNL} proposed a
new program for the construction and operation of a demonstration reactor
based on the \gls{MSBR} concept that they had
developed \cite{macpherson_molten_1985}. The \gls{MSBR} is a thermal-spectrum,
single fluid reactor with fertile $^{232}$Th isotopes mixed directly into the
FLiBe molten salt for $^{233}$U breeding \cite{gehin_liquid_2016}. Like the
\gls{MSRE}, the \gls{MSBR} relies on continuous online reprocessing to add
fertile material and remove fission product neutron poisons. Researchers
estimated the doubling time (the minimum amount of time required to produce
enough fissile material to start up another \gls{MSBR}) to be
approximately 22 years. However, \gls{ORNL} failed to secure funding for the
new program in their two attempts in 1972 and 1974. Nevertheless, from a
technical perspective, two independent
technology evaluation and design studies of the \gls{MSR} had reported
favorably on the promise of the system \cite{macpherson_molten_1985}.

In spite of this setback, research into \glspl{MSR} continued through the late
1970s. In 1980, \gls{ORNL} published a report describing a new \gls{MSR}
concept, called the \gls{DMSR} \cite{gehin_liquid_2016} with denatured
$^{235}$U fuel (i.e. low-enriched uranium). The \gls{ORNL} researchers
developed this design in response to the fuel reprocessing restrictions
introduced by President Ford in 1976. The \gls{DMSR} would operate as a
once-through
converter system without fuel reprocessing. While the fuel consists of 19.75
\% high-assay low-enriched uranium, the initial core loading includes thorium
to boost its conversion ratio throughout its lifetime. It has a continuous
online feed consisting of \gls{HALEU} to maintain criticality, and denatured
$^{235}$U to keep uranium enrichment levels below nuclear non-proliferation
policy thresholds. The design also includes a gas sparging system for removing
gaseous fission products, while noble metals plate out onto the walls of
the coolant loop. The older \gls{MSBR} design had a significant drawback; the
extensive neutron damage in the graphite moderator necessitated frequent
replacement (every four years) throughout its operational lifetime. The
\gls{DMSR} avoids this issue by running on a lower power density while
maintaining
the overall power output of 2250 MW$_{\text{th}}$. As a result, researchers
projected that the graphite moderator would last for the entirety of the
\gls{DMSR}'s design lifetime.

A concurrent program at the UK Atomic Energy Authority led the
development of a 2500 MW$_{\text{e}}$ lead-cooled Molten Chloride Fast Reactor
concept \cite{smith_assessment_1974}. It is a dual fluid system,
with separate loops for the fuel salt and the blanket salt. The blanket is a
1 m-wide tank surrounding the core. The absence of moderators and the choice
of chloride over fluoride salt resulted in a relatively hard neutron spectrum
which favors $^{239}$Pu breeding over the thorium cycle. The UK researchers
performed some experiments to study molten salt chemistry but they did not
build reactor prototypes. The UK program eventually shut down just like its
US counterpart partly due to the successful demonstration of the Prototype
Fast Reactor which had achieved criticality in 1974.

Following a lull lasting through the late 20th century, researchers at
\gls{CNRS} began research into \glspl{MSR} in 1997
\cite{heuer_simulation_2010}. Starting from the \gls{MSBR} design, they
performed parametric studies based on safety, breeding, and
other performance metrics \cite{mathieu_thorium_2006}. Their efforts
culminated in the \gls{MSFR} concept, a fast-spectrum breeder \gls{MSR}
designed to run on the thorium fuel cycle \cite{merle_optimized_2007}. In
2008, the Generation IV International Forum highlighted the \gls{MSFR} among
other \gls{MSR} designs for further development \cite{gif_generation_2008}.
The \gls{MSFR} has also benefited from collaborative research through three
European-funded projects, the \gls{EVOL} \cite{euratom_final_2015},
\gls{SAMOFAR} \cite{kloosterman_20_2017}, and \gls{SAMOSAFER}
\cite{cordis_severe_nodate} projects. Under the \gls{EVOL}
project, researchers further optimized the \gls{MSFR} design based on
neutronic and thermal-hydraulic safety analyses. They also diversified
start-up fuel options beyond the original $^{233}$U-based fuel composition to
include $^{235}$U, $^{239}$Pu, and \gls{TRU}, and established fuel
reprocessing schemes for the \gls{MSFR}. The \gls{SAMOFAR} project, which
started approximately two years after the end of the \gls{EVOL} project,
supported more comprehensive safety assessments of the reactor and the
reprocessing plant, and funded a number of experiments for validation of the
\gls{MSFR}'s safety features. The ongoing \gls{SAMOSAFER} project funds further
research activities with the goal of achieving modeling, analysis, and
design improvements on various aspects of \gls{MSR} operation and safety. As
opposed to the channel-type design of the \gls{MSRE} and \gls{MSBR}, the
\gls{MSFR} reactor core consists of a single, large channel through which the
salt flows as shown in Figure \ref{fig:msfr}. The omission of graphite channels
avoids operational and design issues associated with graphite neutron damage
and parasitic absorption and results in
a harder neutron spectrum than the \gls{MSRE} and \gls{MSBR}.

\begin{figure}[htb!]
	\centering
	\includegraphics[width=.7\columnwidth]{msfr}
	\caption{Schematic diagram of the \gls{MSFR}. Retrieved from 
	\cite{allibert_7_2016}.}
	\label{fig:msfr}
\end{figure}

China and India have also started national programs supporting \gls{MSR}
development. China launched their \gls{TMSR} program in 2011 to develop and
construct both solid-fueled and liquid-fueled \gls{TMSR} designs
\cite{zou_research_2019}. Latest updates at the time of writing indicate that a
2-MW liquid-fueled prototype will complete construction by August
2021, with tests expected to start the following month. This milestone places
China ahead of other groups in the context of modern \gls{MSR} \gls{RD}
efforts. India also signaled their interest especially in thorium-based
reactors given their inherent and vast thorium reserves
\cite{jayaram_overview_1987}. They have developed conceptual designs of the
\gls{IMSBR}, expected to run on a fast/epithermal neutron spectrum to avoid
using graphite moderators given their tendency to deform under high neutron
fluence.

In the US, Southern Company received funding from \gls{DOE}'s
\gls{ARDP} \cite{doe_office_2021} to design, construct, and operate the
\gls{MCRE}, a prototype chloride salt-based \gls{MSR} relevant to TerraPower's
\gls{MCFR} \cite{terrapower_mcfr_2020}. This project builds on a prior
five-year cost-sharing project for the development of the \gls{MCFR} involving
TerraPower, Southern Company, Oak Ridge National Laboratory, Idaho National
Laboratory, the Electric Power Research Institute, and Vanderbilt University.
The \gls{MCFR} is a fast-spectrum \gls{MSR} similar to the \gls{MSFR}.
Canada-based Terrestrial Energy is also developing their \gls{IMSR}
\cite{leblanc_18_2017}, a small modular \gls{MSR} based on the \gls{MSRE}. The
replaceable \gls{IMSR} core-unit which holds the reactor core, pumps, heat
exchangers, and control rods, runs for approximately 7 years before it is shut
down and replaced after a cool-down period.

\gls{MSR} modeling software play an important role in supporting
\gls{MSR} development. They accelerate reactor design and optimization by
enabling engineers to iterate through numerous design changes. \gls{MSR}
modeling software are also essential tools in reactor safety analysis and
licensing efforts as engineers must demonstrate and verify their \gls{MSR}
designs remain safe under various accident scenarios. The next section
identifies challenges in \gls{MSR} multiphysics modeling with respect to
reactor accident analysis.

\section{Challenges in MSR Multiphysics Modeling} \label{sec:challenges}

While modeling \glspl{MSR} is not necessarily more difficult than modeling
solid-fueled reactors, we must adapt our software tools to accurately model the
unique phenomena found in these circulating-fuel reactors. The differences in
the challenges of simulating \glspl{MSR} compared to solid-fueled reactors stem
mainly from the liquid fuel form of the fuel salt \cite{huff_identifying_2019,
diamond_phenomena_2018}.

Firstly, liquids generally exhibit greater thermal
expansion per unit change in temperature than solids. A decrease in density of
the fuel medium increases the likelihood of neutrons escaping the fuel region
and being absorbed by non-fissile material elsewhere in the reactor.
Consequently, combined with the temperature-dependent Doppler broadening of
resonance capture cross sections, \glspl{MSR} possess stronger negative fuel
temperature reactivity feedback than their solid-fueled counterparts
\cite{elsheikh_safety_2013}. These
phenomena ultimately result in strong interactions between the neutron fluxes
and core temperatures given that neutron fluxes affect core temperatures
through fission heat generation and core temperatures in turn affect neutron
fluxes through the mechanisms as described prior.

Secondly, with the fuel
salt also serving the role of providing cooling in the core, velocity flow
profiles in the fuel salts strongly impact the temperature distribution via
advection-dominated heat transfer \cite{diamond_phenomena_2018}. This contrasts
with the relatively static temperature distribution shapes in fuel pins and
other forms of solid fuel matrixes physically separated from the coolant; 
changes in coolant and the resultant changes in heat transfer rates in
solid-fueled reactors are often reduced to empirical correlations governing
convective heat transfer between the fuel cladding and the coolant. On the
other hand, the velocity flow profile has a more direct effect on the
temperature distribution and the temperature-dependent neutron cross sections
in \glspl{MSR}.

Lastly, \glspl{DNP} flow freely within the primary coolant loop as opposed to
being held in place as in solid-fueled reactors. Thus, the delayed neutron
source distribution varies significantly depending on the flow profile. In
addition,
the reactor loses some delayed neutrons from out-of-core \gls{DNP} decay. These
delayed neutrons are considered lost as they're emitted in subcritical regions
and are unlikely to contribute to further fission reactions. The reduced
delayed neutron fraction in the core strongly impacts the transient response of
the reactor in various accident scenarios. Thus, accurate \gls{MSR} transient
simulations require accurate modeling of \gls{DNP} drift.

As a result, multiphysics
software must employ \textit{tight coupling schemes} to couple the neutronics
and thermal-hydraulics governing equations to accurately capture the strong
multiphysics interactions in transient \gls{MSR} simulations. Tightly coupled
numerical models handle multiphysics interactions by either updating all state
variables simultaneously in one monolithic solve (\textit{full coupling}) or
iteratively updating all state variables (\textit{fixed point iterations})
until the solution converges in every timestep \cite{keyes_multiphysics_2013}.
Full coupling tends to be more computationally expensive because it combines
all physics equations into a single large system of equations to be solved
simultaneously, whereas fixed point iterations involve operator splitting to
separate the system of equations into smaller systems based on their associated
physics, solving smaller systems separately, and iteratively updating the state
variables until convergence. Fixed point iterative methods are often less
stable, less accurate, and have poorer
convergence rates since these methods make iterative corrections
without any regard to potentially destabilizing modes introduced by the
multiphysics coupling \cite{keyes_multiphysics_2013}. Notably, proven
techniques exist for improving the performance of fixed point iteration
coupling schemes for many relevant computational multiphysics research fields,
including reactor analysis \cite{ragusa_consistent_2009}. While fully coupled
schemes deal with solving a large system of equations, they can outperform
fixed point iterative methods in some multiphysics problems through superior
stability and convergence rates. 

In contrast to tight coupling schemes, \textit{loose coupling schemes}
solve each set of single-physics equations using state variable
data from the previous timestep without iterative corrections within every
timestep. Loosely coupled schemes are inappropriate for modeling \glspl{MSR}
given the strong coupling between the neutronics and thermal-hydraulics.
Aufiero et al. \cite{aufiero_development_2014} demonstrated a loose coupling
approach that failed to reproduce the expected increase in reactor power
in an \gls{MSR} in response to a 150 pcm reactivity insertion.

Consequently, most MSR multiphysics simulation tools employ, at
the very least, tight coupling schemes through fixed point iterations. The
next section explores various numerical coupling approaches in existing
multiphysics \gls{MSR} simulation tools.

\section{MSR Multiphysics Simulation Tools} \label{sec:msr-tools}

In recent years, several simulation tools have been developed for full-core
modeling of fast-spectrum \glspl{MSR}. \textit{Tightly coupled} approaches,
through segregated solvers, involve coupling separate single-physics neutronics
and thermal-hydraulics software. For example, researchers at
the \gls{TUD} coupled the 3D neutron diffusion software DALTON
\cite{boer_validation_2010} and the \gls{CFD} software HEAT
\cite{de_zwaan_static_2007} to perform a safety analysis of the \gls{MSFR}
\cite{fiorina_modelling_2014}. In a later effort from the same institute,
Tiberga et al. \cite{tiberga_discontinuous_2019} coupled PHANTOM-$S_N$ and
DGFlows in their participation in the CNRS benchmark study
\cite{tiberga_results_2020}. The CNRS benchmark, named after the \gls{CNRS}
where it was originally developed, facilitates code-to-code verification of
\gls{MSR} multiphysics software \cite{aufiero_testing_2018}. Another
multiphysics package was developed at the Paul Scherrer Institute (PSI)
coupling the thermal-hydraulics system software \gls{TRACE}
\cite{nrc_trace_2007} with the
nodal neutron diffusion software \gls{PARCS} \cite{downar_parcs_2010} for the
safety analysis of the \gls{MSFR} \cite{pettersen_coupled_2016}. Coupling
single-physics software to form an integrated multiphysics tool allows
researchers to leverage on older, well-validated, single-physics software.
These single-physics software are also highly optimized for solving specific
types of \glspl{PDE} relevant to the investigated system.

With modern advancements in computing hardware and growing access to
high-performance computing systems, others have developed multiphysics tools by
coupling the \gls{CFD} software OpenFOAM
\cite{the_openfoam_foundation_ltd_openfoam_2021} with the Monte Carlo particle
transport software
Serpent \cite{leppanen_serpent_2014}, thus achieving high-fidelity neutronics
calculations in transient reactor analyses. Laureau et al.
\cite{laureau_transient_2017} developed an innovative technique called the
\gls{TFM} method through the introduction of additional time-dependence
operators to conventional fission matrices typically used to accelerate source
convergence in Monte Carlo neutronics calculations. The \gls{TFM} method
pre-calculates three \glspl{TFM} of the reactor system in Serpent and
interpolates the matrix values during the actual transient calculations to
incorporate the effects of temperature-induced density change and Doppler
effect on the neutron cross sections and ultimately the neutron flux. Blanco
\cite{blanco_neutronic_2020} took a more integrated approach by
compiling Serpent as an internal \texttt{C}-based function within OpenFOAM's
\texttt{C++}-based framework. This approach reduced the amount of required data
transfers between Serpent and OpenFOAM as both software have access to shared
memory during runtime. Their integrated tool employs the Quasi-Static
method for transient neutronics calculations and runs Serpent Monte Carlo
calculations several times per timestep until convergence is reached.

Another \gls{MSR} simulation approach involves developing ``all-in-one''
multiphysics software which handle all multiphysics calculations and data
transfer internally. Among earlier efforts, Nicolino et al.
\cite{nicolino_coupled_2008} and Zhang et al. \cite{zhang_development_2009}
recognized the
need for more robust multiphysics coupling techniques and higher-fidelity
thermal hydraulics solutions to accurately capture complex flow profiles in
pool-type \glspl{MSR}. They each independently developed unnamed multiphysics
simulation tools and demonstrated their tools with non-moderated \gls{MSR}
designs. Later, Li et al. \cite{li_transient_2015} demonstrated the
steady-state and transient analysis capabilities of COUPLE, a neutronics and
thermal-hydraulics software developed at the Karlsruhe Institute of Technology.
Others adopted extensible software frameworks for developing numerical solvers
to develop multiphysics reactor analysis software. Examples of these software
frameworks include the commercial COMSOL
Multiphysics\textsuperscript{\textregistered} software
\cite{comsol_ab_comsol_nodate}, the aforementioned open-source CFD toolbox
OpenFOAM, and the open-source finite-element
framework \gls{MOOSE} \cite{gaston_physics-based_2015}. Researchers at
\gls{PoliMi} developed a \gls{MSR} simulation tool in COMSOL and
modeled the \gls{MSBR} as a single axisymmetric fuel channel with a uniform
flow profile \cite{cammi_multi-physics_2011}, followed by the \gls{MSRE} core
also as a single axisymmetric fuel channel with parabola-shaped laminar flow
\cite{cammi_dimensional_2012}. They later expanded on their approach by
modeling the \gls{MSRE} upper plenum, downcomer and lower plenum, primary heat
exchanger, and secondary heat exchanger as 0D systems (lumped-parameter model),
and substituting the 2D fuel channel with a 3D fuel channel which more closely
resembled the actual fuel channels in the \gls{MSRE}
\cite{zanetti_geometric_2015}. Beyond graphite-moderated \glspl{MSR}, they
also modeled the \gls{MSFR} in the same publication which featured \gls{TUD}'s
DALTON + HEAT coupled multiphysics tool.
More recently, several European institutes (\gls{CNRS}, \gls{PoliMi},
and the \gls{PSI}) have also developed coupled neutronics and
thermal-hydraulics tools in OpenFOAM. Their tools share some
similarities such as implementing the $SP_N$ simplified $P_N$ neutron transport
model and leveraging OpenFOAM's turbulent flow modeling capabilities.
Differences include fuel compressibility modeling and helium bubble tracking
capabilities from \gls{PoliMi} \cite{cervi_development_2019}, fuel
performance analysis capability from \gls{PSI} \cite{fiorina_creation_2018},
and the aforementioned external coupling capability with Serpent from
\gls{CNRS} \cite{blanco_neutronic_2020}.

Finally, within the MOOSE framework, simulation tools capable of modeling
\glspl{MSR} include: Rattlesnake \cite{wang_rattlesnake_2021}; and Moltres
\cite{lindsay_moltres_2017}\textemdash the subject of this work.
Rattlesnake primarily tackles radiation transport problems, but the MOOSE
framework facilitates multiphysics coupling
with MOOSE-based applications for other physics
such that all applications share the same data structure. This eliminates
computational costs from external data transfers and optionally allowing for
\textit{fully coupled} solves in which the application solves all physics
simultaneously. Similarly, Moltres benefits from the highly-integrated
cross-compatibility
within the ecosystem of MOOSE-based applications. Abou-Jaoude et al.
\cite{abou-jaoude_coupled_2020} coupled Rattlesnake with Pronghorn, another
MOOSE-based application for advanced reactor thermal-hydraulics modeling, to
demonstrate several steady-state \gls{MSR} simulation capabilities defined in
the CNRS benchmark. Lindsay et al.
\cite{lindsay_introduction_2018} first demonstrated Moltres' \gls{MSR} modeling
capabilities on 2D axisymmetric and 3D Cartesian models of the \gls{MSRE} with
fixed velocity flow on a fully coupled neutronics and thermal-hydraulics solve.
We later demonstrated some of Moltres' more recent developments through
modeling a 2D axisymmetric model of the \gls{MSFR} for steady-state operation
and transient accident analysis \cite{park_advancement_2020}. The latter study
introduced looped \gls{DNP} flow, coupling the \gls{DNP} drift and temperature 
advection-diffusion to incompressible flow, and decay heat modeling
capabilities. The proposed work aims to continue Moltres development for
multiphysics \gls{MSR} analysis. Chapter \ref{chap:moltres} describes Moltres
and its existing capabilities for reactor analysis.

\section{Turbulence Modeling in MSRs}

In fluid dynamics, turbulent flow is characterized by unsteady, irregular, and
chaotic fluid motion as opposed to neat, parallel flow layers in laminar flow
\cite{pope_turbulent_2000}. The transition from laminar to turbulent flow
typically occur at Reynolds numbers between 2000 and 4000, depending on the
setup \cite{pope_turbulent_2000}. Turbulent flows are expected in \glspl{MSR}.
Kedl \cite{kedl_fluid_1970} reports expected Reynolds numbers in the \gls{MSRE}
ranging from 1000 in the regular fuel coolant channels to over 10000 in the
flow distributor volute and core wall cooling annulus regions. For the
\gls{MSFR}, salt flow in the central core region is highly turbulent and
reaches Reynolds numbers on the order of $10^5$.

\subsection{Turbulence Models}

Numerous types of turbulence models exist for various turbulent flow
applications. The most common turbulence models can be classified into the
following categories by order of increasing computational complexity:
%
\begin{itemize}
    \item \gls{RANS}-based models
    \begin{itemize}
        \item Eddy viscosity models
        \begin{itemize}
            \item Algebraic models
            \item One- and two-equation models
        \end{itemize}
        \item \gls{RSM}
    \end{itemize}
    \item \gls{DES}
    \item \gls{LES}
    \item \gls{DNS}
\end{itemize}

\gls{RANS}-based models are based on the \gls{RANS} equations obtained from
applying time-averaging to the equations of fluid flow. The \gls{RANS}
equations separate flow into time-averaged $U$ and fluctuating $u$ components
and can be writtin in Einstein notation and Cartesian coordinates as:
%
\begin{align}
    \frac{\partial U_i}{\partial t} + U_j \frac{\partial u_i}{\partial x_j} =&
    -\frac{1}{\rho} \frac{\partial P}{\partial x_i} + \nu \nabla^2 U_i -
    \frac{\partial \langle u_i u_j \rangle}{x_j}
    \shortintertext{where}
    \langle \cdot \rangle =& \mbox{ time-averaging operator,} \nonumber \\
    \rho =& \mbox{ fluid density,} \nonumber \\
    P =& \mbox{ time-averaged pressure field,} \nonumber \\
    \nu =& \mbox{ kinematic viscosity.} \nonumber
\end{align}

Eddy viscosity models, which comprise of the most widely used turbulence models
in use today \cite{rodi_turbulence_2017}, operate on the eddy viscosity
hypothesis which states that the Reynolds stresses in the \gls{RANS} equations
are given by:
%
\begin{align}
    \langle u_iu_j \rangle =& \frac{2}{3}k \delta_{ij} - \nu_T \left(
    \frac{\partial U_i}{\partial x_j} + \frac{\partial U_j}{\partial x_i}
    \right)
    \shortintertext{where}
    k =& \mbox{ mean turbulent kinetic energy,} \nonumber \\
    \delta_{ij} =& \mbox{ Kronecker delta,} \nonumber \\
    \nu_T =& \mbox{ eddy viscosity.} \nonumber \\
\end{align}

The various eddy viscosity models mainly differ in their approach towards
the closure problem of calculating the eddy viscosity. As the name suggests,
algebraic models rely on algebraic equations to calculate the eddy viscosity
distribution directly from flow variables. As a result, algebraic models are
the least computationally intensive models for turbulence. Algebraic models
can be further categorized into two types: uniform eddy viscosity models
and mixing length models. Uniform eddy viscosity models apply a uniform eddy
viscosity throughout the problem domain. The uniform eddy viscosity is
calculated from flow parameters such as the characteristic velocity, the
characteristic flow width, and empirically determined turbulent Reynolds
number. Given that eddy viscosities usually vary significantly in most types of
flow, uniform eddy viscosity models have a very limited range of applicability
\cite{pope_turbulent_2000}. Mixing length models add a level of complexity by
relating the eddy viscosity to spatially-varying flow parameters such as the
mean velocity gradient (Prandtl \cite{prandtl_7_1925} and Cebeci-Smith
\cite{smith_numerical_1967} models) or the mean rate of strain (Baldwin-Lomax
\cite{baldwin_thin-layer_1978} model) and an empirical mixing length parameter.
Combined with empirical data for the mixing length parameter, these
models provide better approximations of free shear flows, but still
underperform for more complex flows involving flow separation and significant streamline curvature.

One- and two-equation turbulence models introduce differential equations to
describing turbulence quantities such as the turbulence kinetic energy and the
turbulence rate of dissipation to obtain the eddy viscosity distribution. The
most common and best performing one-equation model is the Spalart-Allmaras
model which provides an equation for the eddy viscosity directly with several
closure coefficients and functions \cite{wilcox_turbulence_2006}. The
Spalart-Allmaras model is considered ``complete'' as it does not involve any
adjustable coefficients or functions. Calibrated for free shear flows in
aeronautical applications, the model performs modestly better than algebraic
models in these applications \cite{pope_turbulent_2000}, but it still deviates
significantly from experimental data
for separated flows \cite{wilcox_turbulence_2006}.

Investigations with
one-equation models reveal the need for an extra equation to account for
turbulent length scales separately from turbulent velocity. Thus, two-equation
models became the most widely adopted turbulence model in the late 20th century
\cite{pope_turbulent_2000}. Two-equation models include the $k$-$\epsilon$,
$k$-$\omega$, and $k$-$\tau$ models. The variables $k$, $\epsilon$, $\omega$,
and $\tau$ correspond to turbulent kinetic energy, turbulent dissipation,
specific turbulent dissipation rate, and turbulent time scale, respectively.
While none of these models perform universally well, they are generally more
accurate than the algebraic and one-equation models. Successive contributions
and modifications to the two-equation models through the years have also
improved their performance in predicting various types of turbulent flow. Their
moderate computational expense compared to expensive, high-fidelity turbulence
models favor their adoption in most commercial \gls{CFD} software for
engineering applications \cite{pope_turbulent_2000}.

\glspl{RSM} directly computes the individual components $\langle u_i u_j
\rangle$ of the Reynolds stress tensor instead of approximating it with a
single, isotropic eddy viscosity term. As a consequence, \glspl{RSM} provide
more realistic predictions for flows with significant rotational motion and
sudden changes in the mean strain rate, albeit at greater computational
expense, compared to the one- and two-equation models
\cite{wilcox_turbulence_2006}. Smaller improvements are observed in modeling
free shear flows and backward-facing step flows \cite{wilcox_turbulence_2006}.

Due to the much higher computational cost for \gls{DES}, \gls{LES}, and
\gls{DNS}, these models have limited applicability in routine, high-Reynolds
number engineering problems today. However, given their high accuracy, these
models are useful for flow problems with relatively simple geometries and at
low Reynolds numbers and validating the lower-fidelity turbulence models
\cite{zhiyin_large-eddy_2015}.

\subsection{Turbulence Models in MSR Simulation Tools}

For MSR modeling, the $k$-$\epsilon$ and $k$-$\omega$ turbulence models are the
most commonly used models as shown in published work with COMSOL
\cite{fiorina_modelling_2014}, OpenFOAM \cite{aufiero_development_2014}, and
\gls{TUD}'s in-house codes \cite{fiorina_modelling_2014,tiberga_results_2020}.
Podila et al. \cite{podila_cfd_2019} performed \gls{CFD} simulations of the
\gls{MSRE} core with six different turbulence models, namely a Spalart-Allmaras
model, two variants of the $k$-$\epsilon$ model, a $k$-$\omega$ model, and two
variants of \glspl{RSM}. Their results showed relatively small differences
in graphite and fuel temperatures among different turbulence models. However,
they observed significant differences in the turbulence intensities near the
wall. Given the lack of experimental data for model validation, the authors
could not make a clear assessment of the models' accuracies. Nevertheless, the
close agreement of the fuel temperatures imply that the discrepancies in the
turbulent intensities near the wall have a negligible impact on the overall
distribution of advected quantities in the \gls{MSRE}. Podila et al.
\cite{podila_cfd_2019} opted to use a $k$-$\epsilon$ model for subsequent
calculations in their work given its lowest computational cost and the close
agreement in the temperature distributions.

Amongst other \gls{MSR} simulation tools, the $k$-$\epsilon$ and $k$-$\omega$
turbulence models are the most commonly used models as shown in published work
with COMSOL \cite{fiorina_modelling_2014}, OpenFOAM
\cite{aufiero_development_2014}, and \gls{TUD}'s in-house codes
\cite{fiorina_modelling_2014,tiberga_results_2020}. Fiorina et al.
\cite{fiorina_modelling_2014} compared the flow distribution from both models
in a 2D axisymmetric \gls{MSFR} geometry and observed that the $k$-$\omega$
model produced a wider recirculation zone near the outer wall, an additional
recirculation zone near the top wall, and significantly higher maximum
temperatures within the former recirculation zone. Thus, their work highlights
the difficulties of modeling separated flows and calls for extra attention
towards the choice of turbulence model in \glspl{MSR}.

\section{Movable Control Rod Modeling in MSRs}

Control rods 
