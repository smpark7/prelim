This chapter provides an overview of Moltres as a multiphysics
simulation software for molten salt reactors. 
Section \ref{sec:moltres-features}
describes general software features of Moltres, Section
\ref{sec:moltres-physics} expands on the physics models in Moltres, and Section
\ref{sec:moltres-previous} describes the current capabilities in Moltres.

\section{General Features} \label{sec:moltres-features}

This section discusses the general software features of Moltres. The
discussion specifically focuses on robustness, extensibility, and ease of use
as these characteristics represent the hallmarks of a good multiphysics
software \cite{keyes_multiphysics_2013}. These criteria also apply for
assessing \gls{MSR} simulation tools since the nonlinearity of \gls{MSR}
multiphysics analysis and the complexity of advanced reactor designs
necessitate robust, scalable, and flexible computational tools.

Moltres draws many advantages from being developed on MOOSE
\cite{permann_moose_2020}. MOOSE is an open-source, finite-element,
multiphysics framework developed at \gls{INL}. The framework provides a
user-friendly interface for developing multiphysics software through
\gls{OOP} in \texttt{C++} to modularize various
functions relevant to finite-element, multiphysics solvers. In this approach,
MOOSE and MOOSE-based applications break down \glspl{PDE} into individual terms
and store them as individual \texttt{C++ objects} referred to as
\texttt{Kernels}. These \texttt{Kernels} contain functions for calculating
their weak form residual and Jacobian
contributions and other relevant functions required to solve a given
\gls{PDE}. \gls{OOP} in MOOSE simplifies software development
since developers can write new \texttt{Kernels} as child classes in
\texttt{C++} derived from existing \texttt{Kernels} (base classes) which share
similar physics to inherit common functions.
The same philosophy applies for all other systems in MOOSE such as
the \texttt{BCs}, \texttt{Materials}, and \texttt{Postprocessor}
systems for handling relevant boundary conditions, material properties, and
postprocessing calculations, respectively. Overall, this approach also saves
researchers time and effort as they are unencumbered by the technical details
and complexities involved in programming efficient computational tools for
numerical analysis.

\begin{figure}[htb!]
	\centering
	\includegraphics[width=.7\columnwidth]{moose}
	\caption{Structure of MOOSE and its dependencies.}
	\label{fig:moose}
\end{figure}

MOOSE relies on two other open-source libraries: libMesh
\cite{kirk_libmesh_2006} for its \gls{FEM} capabilities on unstructured mesh,
and PETSc \cite{satish_petsc_2019} for its non-linear solvers and
preconditioning routines. By extension, Moltres gains access to these
sophiscated numerical analysis tools and benefits from their continuous
development. Figure
\ref{fig:moose} shows how MOOSE serves as an interface between physics
applications and libMesh/PETSc. MOOSE supports
modeling on up to three-dimensional (3-D) unstructured meshes for a wide range
of mesh file formats, including the commonly used Exodus II file format. For
2-D meshes, users can opt between Cartesian and polar RZ coordinates. MOOSE
also supports parallel computing through the \gls{MPI} library to leverage
modern high-performance computing for large multiphysics simulations.

Moltres benefits from the highly-integrated cross compatibility within the
ecosystem of MOOSE-based applications. MOOSE facilitates multiphysics coupling
among all MOOSE-based applications by providing a common framework for shared
data access and file input/output, thus eliminating computational costs from
data transfers and allowing for fully coupled solves. For example, Moltres
couples with the \texttt{Navier-Stokes} module \cite{peterson_overview_2018}
from MOOSE for fully coupled reactor simulations modeling neutronics and
thermal-hydraulics with incompressible flow. Section
\ref{sec:challenges} highlighted the advantages of using fully coupled schemes
for modeling strongly coupled systems such as the coupled
neutronics and thermal-hydraulics in \glspl{MSR}. Moltres can also
couple to other MOOSE-based applications in a similar fashion with ease. In
addition, MOOSE
provides the option for either tight or loose coupling through the
\texttt{MultiApp} system \cite{gaston_physics-based_2015}. Tight coupling
schemes can outperform fully coupled schemes in weakly coupled systems in which
the computational expenses of fully coupled schemes outweigh the savings from
running fewer Newton iterations due to the superior convergence rate. Loose
coupling schemes are useful for accelerating time-dependent simulations of
stable systems towards steady state in which only the steady-state
configuration is of interest to the user. This typically occurs in the later
stages of \gls{MSR} simulations when the delayed neutron precursor
concentrations converge slowly due to their relatively large decay half-lives.
Furthermore, segregated solves through the \texttt{MultiApp} system enables
Moltres to introduce delayed neutron precursor drift and non-uniform
temperature distributions into criticality search simulations. Regardless of
which coupling scheme is best, MOOSE-based applications provide the flexibility
to switch among the schemes as users see fit. For time-dependent simulations,
MOOSE provides more than ten different implicit and explicit timestepping
schemes. The default, most common scheme is the first-order backward Euler
method which offers excellent solver stability for stiff \glspl{PDE}.

Lastly, Moltres is an open-source \gls{LGPL} software hosted on
GitHub \cite{github_build_2017}. Open-sourcing software provides ease of access
and expands the userbase. These characteristics promote software quality
through increased feedback on users' needs and transparency for peer review.
Open-source software accelerate research progress by supporting research
collaboration and sharing best software practices. Supporting Moltres'
continued development, Moltres relies on GitHub for online version control with
continuous integrated testing to protect its existing capabilities.

In summary, Moltres provides robust, yet flexible coupling capabilities to
model strongly coupled neutronics and thermal-hydraulics in \glspl{MSR}. As a
MOOSE-based application, Moltres is highly extensible by means of coupling to
other MOOSE-based applications and benefits from MOOSE's user-friendly
interface for software development and general ease of use.

\section{Physics Models} \label{sec:moltres-physics}

This section describes the various physics models available in Moltres to model
coupled neutronics and thermal-hydraulics in \glspl{MSR}. Section \ref{sec:nts}
discusses the neutronics model in Moltres, while Section \ref{sec:th} discusses
the thermal-hydraulics model.

\subsection{Neutronics} \label{sec:nts}

\subsubsection{Multigroup Neutron Diffusion Equations}

Moltres solves the multigroup neutron diffusion equations for the neutron
flux solution within the problem domain. These equations are derived from the
neutron transport equation in the diffusion-dominated limit with Fick's law of
diffusion and further simplified by discretizing the continuous neutron energy
variable into a finite number of energy groups \cite{bell_nuclear_1970,
duderstadt_nuclear_1976}. The time-dependent multigroup neutron
diffusion equations with $G$ energy groups and $I$ delayed neutron precursor
groups are given by:
%
\begin{align}
    \frac{1}{v_g} \frac{\partial \phi_g}{\partial t} =& \nabla \cdot D_g
    \nabla \phi_g - \Sigma^r_g \phi_g +
    \sum^G_{g' \neq g} \Sigma^s_{g' \rightarrow g} \phi_{g'} \nonumber \\
    &+ \chi^p_g \sum^G_{g'=1} \left( 1-\beta \right) \nu \Sigma^f_{g'}
    \phi_{g'} + \chi^d_g \sum^I_i \lambda_i C_i \label{eq:neutron} \\
    %
    \shortintertext{where}
    v_g =& \text{ average speed of neutrons in group $g$,} 
    \nonumber \\
    \phi_g =& \text{ neutron flux in group $g$,}
    \nonumber \\
    t =& \text{ time,} \nonumber \\
    D_g =& \text{ diffusion coefficient of neutrons in} \nonumber \\
    &\text{ group $g$,} \nonumber \\
    \Sigma^r_g =& \text{ macroscopic removal cross section for} \nonumber \\
    &\text{ neutrons from group $g$,} \nonumber \\
    \Sigma^s_{g' \rightarrow g} =& \text{ macroscopic scattering cross section
    for neutrons from} \nonumber \\
    &\text{ groups $g'$ to $g$,} \nonumber \\
    \chi^p_g =& \text{ prompt fission spectrum for neutrons in} \nonumber \\
    &\text{ group $g$,} \nonumber \\
    G =& \text{ total number of discrete neutron groups,} \nonumber \\
    \nu_g =& \text{ average number of neutrons produced per} \nonumber \\
    &\text{ fission,} \nonumber \\
    \Sigma^f_{g} =& \text{ macroscopic fission cross section for neutron}
    \nonumber \\
    &\text{ in group $g$,} \nonumber \\
    \chi^d_g =& \text{ delayed fission spectrum for neutrons in} \nonumber \\
    &\text{ group $g$,} \nonumber \\
    I =& \text{ total number of delayed neutron precursor} \nonumber \\
    &\text{ groups,} \nonumber \\
    \beta =& \text{ total delayed neutron fraction.} \nonumber
\end{align}

In spite of forming only around 0.7\% of all neutrons emitted, delayed neutrons
play outsized roles in reactor kinetics. The relatively long half-lives of
delayed neutron precursors gives us ample time in adequately designed reactors
to control reactor power output and intervene in case of power excursions.
The precursor concentration balance equations for $I$ precursor
groups are given by:
%
\begin{align}
    \frac{\partial C_i}{\partial t} =& \beta_i \sum^G_{g'=1} \nu \Sigma^f_{g'}
    \phi_{g'} - \lambda_i C_i - \vec{u} \cdot \nabla C_i + \nabla \cdot
    D_{\text{P}} \nabla C_i \label{eq:precursor} \\
    %
    \shortintertext{where}
    \beta_i =& \text{ delayed neutron fraction of precursor group $i$,}
    \nonumber \\
    \lambda_i =& \text{ average decay constant of delayed neutron} \nonumber \\
    &\text{ precursors in precursor group $i$,} \nonumber \\
    C_i =& \text{ concentration of delayed neutron precursors in}
    \nonumber \\
    &\text{ precursor group $i$,} \nonumber \\
    \vec{u} =& \text{ molten salt flow velocity vector,}
    \nonumber \\
    D_{\text{P}} =& \text{ effective diffusion coefficient of the delayed}
    \nonumber \\
    &\text{ neutron precursors.} \nonumber
\end{align}

These two equations are largely similar to conventional formulations of the
multigroup neutron diffusion equations with delayed neutrons for most reactor
types. The only differences are in the last two terms in Equation
\ref{eq:precursor}
which represent the advection and diffusion terms, respectively, to model the
movement of \gls{DNP} in liquid-fuel \glspl{MSR}.

As shown in Equations \ref{eq:neutron} and \ref{eq:precursor}, Moltres requires
group constant data from dedicated high-fidelity neutronics software such as
the NEWT module in SCALE \cite{dehart_reactor_2011}, Serpent
\cite{leppanen_serpent_2014}, or OpenMC \cite{romano_openmc:_2015}. These group
constant data are the neutron energy group $g$ values for $v_g$, $D_g$,
$\Sigma^r_g$, $\Sigma^s_{g' \rightarrow g}$, $\chi^p_g$, $\chi^d_g$,
$\Sigma^f_{g}$, and $\nu\Sigma^f_{g}$, and precursor group $i$ values for
$\beta_i$ and $\lambda_i$. Users
can run a Python script in Moltres' Github repository which automatically reads
user-provided SCALE/Serpent/OpenMC output data files and creates
Moltres-compatible JSON format files containing all required group constant
data. Moltres allows for an arbitrary number of neutron energy groups $G$ and
precursor groups $I$ as long as the user provides the necessary group constant
data. In practice, $I$ depends on the nuclear data library used to generate
group constants---the JEFF \cite{plompen_joint_2020} and ENDF
\cite{brown_endfb-viii0_2018} data libraries define eight precursor groups
and six precursor groups, respectively.

In multiphysics reactor simulations, we model the coupling between neutronics
and thermodynamics through temperature-dependent group constants. To sample
group constants at different temperatures in Moltres, users must provide group
constant data measured at more than one temperature (e.g. 800K--1500K at 100K
intervals). Users can then choose from linear spline, cubic spline, or monotone
cubic interpolation methods available in Moltres to interpolate the group
constant data for values falling within the provided temperature range. 

\subsubsection{Neutronics Boundary Conditions}

Moltres provides two types of boundary conditions for neutron fluxes; these are
conventionally known as the vacuum and reflective boundary conditions given,
respectively, as:

\begin{align}
    \frac{\partial \phi}{\partial x_i} + \frac{\phi}{2} =& 0 \label{eq:vacuum}
    \shortintertext{and}
    \frac{\partial \phi}{\partial x_i} =& 0
    \shortintertext{where}
    x_i =& \mbox{ spatial coordinate perpendicular to the boundary.} \nonumber
\end{align}

The vacuum boundary condition typically applies to the external boundaries of
the reactor beyond which lies low-interaction media such as air, while the
reflective boundary condition is useful for exploiting symmetries in the
model geometry such as along the axial boundary in axisymmetric geometries. The
reflective boundary condition is also equivalent to the more generally known
homogeneous Neumann boundary condition. Relevant boundary conditions for
delayed neutron precursors include the homogeneous Neumann boundary condition
along fuel salt-structural interfaces; and outflow/inflow boundary
conditions along the outlet/inlet boundaries through which the precursors
flow as they circulate the fuel salt loop.

\subsection{Thermal-Hydraulics} \label{sec:th}

\subsubsection{Incompressible Flow}

Moltres largely relies on MOOSE's \texttt{Heat} \texttt{Conduction} and
\texttt{Navier-Stokes} physics modules for its thermal-hydraulics modeling
capabilities. While the \texttt{Navier-Stokes} module supports both
compressible and incompressible flow modeling, this work focuses on
multiphysics coupling in \glspl{MSR} with incompressible flow. The
time-dependent incompressible Navier-Stokes equations for velocity $\vec{u}$
with the Boussinesq approximation for buoyancy-driven flow are given as:

\begin{align}
    \text{Momentum equation: } \rho \frac{\partial \vec{u}}{\partial t} =&
    -\rho (\vec{u}
    \cdot \nabla) \vec{u} - \nabla p + \mu \nabla^2 \vec{u}
    + \rho \alpha \vec{g} \left(T - T_{\text{ref}} \right)
    \label{eq:momemtum}
    \shortintertext{and}
    \text{Mass equation: } \nabla \cdot \vec{u} =& 0
    \label{eq:divergence}
    \shortintertext{where}
    \rho =& \text{ fluid density,} \nonumber \\
    p =& \text{ pressure,} \nonumber \\
    \mu =& \text{ dynamic viscosity,} \nonumber \\
    \alpha =& \text{ coefficient of thermal expansion,} \nonumber \\
    \vec{g} =& \text{ gravitational force vector,} \nonumber
    \\
    T =& \text{ fluid temperature,} \nonumber \\
    T_{\text{ref}} =& \text{ reference temperature at which the nominal}
    \nonumber \\
    &\text{ density is provided.} \nonumber
    \nonumber
\end{align}

Velocity variables and advected quantities such as temperature are susceptible
to numerical node-to-node oscillations
commonly observed when resolving advection-dominated flows using continuous
Galerkin methods \cite{kuhlmann_lid-driven_2018}. The \texttt{Navier-Stokes}
module provides the \gls{SUPG} stabilization scheme
\cite{brooks_streamline_1982} for the velocity and temperature variables to
minimize these oscillations. The module also provides the \gls{PSPG}
stabilization scheme \cite{hughes_new_1986} which enables equal-order
discretizations. Peterson et al. \cite{peterson_overview_2018}
provides further detail on the implementation of these stabilization schemes in
the \texttt{Navier-Stokes} module.

\subsubsection{Temperature Advection-Diffusion}

Lastly, Moltres solves for the temperature distribution through the temperature
advection-diffusion equation given by:

\begin{align}
    \rho c_{p} \frac{\partial T}{\partial t} =& - \rho c_p \vec{u}
    \cdot \nabla T + \nabla \cdot \left(k \nabla T \right) + Q_f - Q_s
    \label{eq:temp}
    \shortintertext{and}
    Q_f =& \sum^G_{g=1} \epsilon_g \Sigma_g^f \phi_g
    \shortintertext{where}
    c_p =& \text{ specific heat capacity of molten salt,} \nonumber \\
    k =& \text{ effective thermal conductivity of molten salt,} \nonumber \\
    Q_f =& \text{ fission heat source,} \nonumber \\
    \epsilon_g =& \text{ average fission energy released by neutrons in group
    $g$,} \nonumber \\
    Q_s =& \text{ heat sink/removal.} \nonumber
\end{align}

$Q_f$ represents the fission heat source term and is calculated by taking the
sum of neutron group fluxes multiplied by their respective macroscopic fission
cross sections and the average fission energy released per fission.

\subsubsection{Thermal-Hydraulics Boundary Conditions}

The \texttt{Navier-Stokes} module provides the following types of boundary
conditions for the velocity and temperature variables:

\begin{align}
    \text{Dirichlet: }& & u \ \left(\text{or } T\right) =& c & \\
    \text{Homogeneous Neumann: }& & \frac{\partial u}{\partial x_i} \
    \left(\text{or } \frac{\partial T}{\partial x_i}\right) =& 0 & \\
    \text{``No boundary condition'' outflow: }& &
    \mu \left[ \nabla \vec{u} + \left(\nabla \vec{u} \right)^T \right] \cdot
    \hat{n} =& 0 \ \text{ (velocity)} & \\
    \text{``No boundary condition'' outflow: }& &
    k \nabla T \cdot\hat{n} =& 0 \ \text{ (temperature)} &
    \shortintertext{where}
    & & c =& \text{ user-defined constant value,} & \nonumber \\
    & & \hat{n} =& \text{ unit normal vector to the boundary.} & \nonumber
\end{align}

The Dirichlet boundary condition can be used to set the inlet velocities and
temperatures, and no-slip conditions along solid boundaries, while the
homogeneous Neumann boundary condition is commonly
imposed along the outlet boundary. However, the latter approach tends to
artificially influence upstream behavior, especially in developing flow. The
``no boundary condition'' outflow boundary condition by Griffiths
\cite{griffiths_no_1997} has been shown to reduce such upstream errors.

\subsection{Precursor Loop} \label{sec:moltres-loop}

Moltres also accounts for the decay of
\glspl{DNP} outside the active core region by simulating its flow in a
separate 1-D pipe geometry. This outer loop pipe calculation is implicitly
coupled to the active core simulation through Picard iterations in MOOSE's
MultiApp functionality and inlet/outlet boundary values.
The outer loop region is assumed to be subcritical to minimize neutron
irradiation upon heat exchangers, pumps, and other equipment. Therefore, the
only significant neutronics-related phenomena are the drift and decay of
\glspl{DNP}. The governing equation for the \glspl{DNP} is:
%
\begin{align}
    \frac{\partial C_i}{\partial t} =& - \lambda_i C_i - u
    \frac{\partial C_i}{\partial x}.
    \label{eq:dnploop}
\end{align}
%
Equation \ref{eq:dnploop} is derived from equation \ref{eq:precursor} by
removing the fission \gls{DNP} source and diffusion terms, and the conversion
of the advection into a 1-D form.

The governing equation
for temperature, derived from equation \ref{eq:temp}, is:
%
\begin{align}
    \rho c_{p} \frac{\partial T}{\partial t} =& - \rho c_p u
    \frac{\partial T}{\partial x} - Q_{hx} \label{eq:temploop}
    \intertext{where}
    Q_{hx} =& \text{heat removal rate through the heat exchanger [W].} 
    \nonumber
\end{align}

In the outer loop region, the fission heat source term is replaced with a heat
exchanger sink term $Q_{hx}$.

\subsubsection{Boundary Conditions}

Table \ref{table:loopbc} summarizes the boundary conditions for all variables
on the inlet and outlet of the 1-D outer loop region. The inlet boundary
conditions are all Dirichlet boundary conditions. The inlet boundary values
are set by the outflow from the active core region that this inlet is
connected to in the actual reactor geometry. The outlet boundary conditions
are all outflow boundary conditions as shown in Table \ref{table:loopbc}.

\begin{table}[htbp!]
    \small
	\caption{Boundary conditions in the 1-D outer loop geometry. $u$
	represents the 1-D velocity in this region.}
	\centering
	\begin{tabular}{ l l c}
		\toprule
		Variable & Boundary & Boundary Condition \\
        \midrule
        \multirow{2}{*}{Delayed neutron precursor concentration $C_i$} &
        Inlet (Core) & $C_i = c$ \\
        & Outlet (Core) & $u \cdot C_i = 0$ \\
        \midrule
        \multirow{2}{*}{Temperature $T$} &
        Inlet (Core) & $T = c$ \\
        & Outlet (Core) & $u \cdot T = 0$ \\
		\bottomrule
	\end{tabular}
	\label{table:loopbc}
\end{table}

\subsubsection{Core and Outer Loop Coupling}

This subsection details the delayed neutron precursors, and
temperature coupling between the core and outer loop regions. 

At every timestep, Moltres calculates weighted averages of the
temperature and the precursors at the outlet. These values are weighted by the
outflow velocity values at the outlet according to the following equation:
%
\begin{align}
    \overline{\psi} =& \frac{\int_\mathcal{C} \psi(x_j) u(x_j) dx_j}{
    \int_\mathcal{C} u(x_j) dx_j} \\
    \intertext{where}
    \psi =& \text{ variable to be weighted [ - ]} \nonumber \\
    \mathcal{C} =& \text{ outlet boundary area [ - ]} \nonumber \\
    u =& \text{ outflow velocity perpendicular to the outlet boundary
    [m$\cdot$s$^{-1}$],} \nonumber \\
    x_j =& \text{ spatial coordinate parallel to the outlet boundary.}
    \nonumber
\end{align}

Moltres transfers this outflow value from the core region to the 1-D
outer loop region, to be used as the boundary value for the inhomogeneous
Dirichlet boundary
condition at the inlet. Likewise, the outflow value from the outer
loop region is used for the inflow value in the central core region. No
averaging is required for this step as the outer loop region is a 1-D system.
This approach that the inflow temperature and \gls{DNP} are uniform at the
inlet. The Picard iterations within every timestep ensure that the two systems
are implicitly coupled.

\section{Previous Work} \label{sec:moltres-previous}

This section discusses some previous work with Moltres to illustrate its
various capabilities and coupling approaches for multiphysics \gls{MSR}
modeling and simulation. Section \ref{sec:msre} summarizes work by Lindsay et
al. \cite{lindsay_introduction_2018} in modeling the \gls{MSRE}, and Section
\cite{park_advancement_2020} summarizes my previous work in modeling the
\gls{MSFR}.

\subsection{Introduction to Moltres and Modeling the MSRE} \label{sec:msre}

This section follows work by Lindsay et al. in \textit{Introduction to Moltres:
An Application for Simulation of Molten Salt Reactors}
\cite{lindsay_introduction_2018}.

In 2017, Lindsay et al. first introduced
Moltres to the \gls{MSR} community for multiphysics simulations of \glspl{MSR}.
Their work showcased Moltres' neutron diffusion model, thermal-hydraulics
model, and their coupling in the \gls{MOOSE} framework. The authors
demonstrated these capabilities by running time-dependent simulations of 2D
axisymmetric and 3D \gls{MSRE} models until the flux, precursor, and
temperature distributions reached steady state.

\begin{figure}[htb!]
	\centering
	\includegraphics[width=.45\columnwidth]{msre-geometry}
	\caption{Schematic diagram of the 2D \gls{MSRE} geometry adopted by
	Lindsay et al. \cite{lindsay_introduction_2018}.}
	\label{fig:msre-geometry}
\end{figure}

Figure \ref{fig:msre-geometry} shows the fuel channels and moderator regions of
the 2D \gls{MSRE} geometry that Lindsay et al. adopted for their study.
They ran a two-group neutron diffusion model with six precursor groups and
vacuum boundary conditions on the outer boundaries governed by Equations
\ref{eq:neutron}, \ref{eq:precursor}, and \ref{eq:vacuum} shown in Section
\ref{sec:nts}. They modeled precursor drift due to fuel salt flow by imposing
fixed uniform flow upwards through the fuel channels shown in Figure
\ref{fig:msre-geometry}. For their thermal-hydraulics model, they employed a
governing equation for temperature in the fuel salt equivalent to Equation
\ref{eq:temp} with fixed uniform flows while also imposing a cosine-shaped heat
source term representing heat dissipation from gamma and neutron irradiation in
the graphite moderator region. In addition, all governing equations were fully
coupled and solved simultaneously as a single system of equations with implicit
Euler time-stepping to accurately and efficiently resolve the strong coupling
expected between the neutronics and temperature.

\begin{figure}[htb!]
	\centering
	\includegraphics[width=.45\columnwidth]{2d_gamma_heating_group1}
	\includegraphics[width=.45\columnwidth]{2d_gamma_heating_group2}
	\caption{Neutron group 1 and 2 fluxes in the 2D axisymmetric \gls{MSRE}
	model from Lindsay et al. \cite{lindsay_introduction_2018}.}
	\label{fig:msre-flux}
\end{figure}

\begin{figure}[htb!]
	\centering
	\includegraphics[width=.45\columnwidth]{2d_gamma_heating_pre1_scaled}
	\includegraphics[width=.45\columnwidth]{2d_gamma_heating_pre6_scaled}
	\caption{Longest- and shortest-lived precursor concentrations ($\lambda =
	1.24\times 10^{-2}$s$^{-1}$ and $3.07$s${-1}$, respectively) in the 2D
	axisymmetric \gls{MSRE} model from Lindsay et al.
	\cite{lindsay_introduction_2018}.}
	\label{fig:msre-precursor}
\end{figure}

\begin{figure}[htb!]
	\centering
	\begin{minipage}[b]{0.45\columnwidth}
	    \includegraphics[width=\columnwidth]{2d_gamma_heating_temp}
	    \caption{Temperature distribution in the 2D
	    axisymmetric \gls{MSRE} model from Lindsay et al.
	    \cite{lindsay_introduction_2018}.}
	    \label{fig:msre-temp}
	\end{minipage}
	\hfill
	\begin{minipage}[b]{0.45\columnwidth}
	    \includegraphics[width=\columnwidth]{combined_msre_moltres_axial_temps}
	    \caption{Moltres \cite{lindsay_introduction_2018} and \gls{ORNL}
	    \gls{MSRE} \cite{briggs_molten-salt_1964} axial temperature
	    distributions in the hottest fuel channel and adjacent graphite.}
	    \label{fig:msre-temp-plot}
	\end{minipage}
\end{figure}

Figure \ref{fig:msre-flux} shows the fast and thermal neutron fluxes
corresponding to group 1 and 2 in the 2D \gls{MSRE} model. As expected, the
fluxes exhibit general cosine shapes in the axial and radial directions. We
also observe minor oscillations in the radial direction coinciding with the
regular fuel and moderator lattice. The fuel regions favor the fast flux while
the moderator regions favor the thermal flux.

Figure \ref{fig:msre-precursor}
shows the longest- and shortest-lived precursor concentrations in the fuel
channels. With a long half-life of 55.9 s relative to the 6.91 s it takes for
salt to flow from bottom to top, longest-lived precursor concentration peaks
outside the model domain. By contrast, the shortest-lived precursor
concentration closely follows the cosine shape of the neutron fluxes which
dictate where the precursors are born.

Finally, Figure \ref{fig:msre-temp}
shows the temperature distribution in the 2D \gls{MSRE} model. The temperature
naturally peaks near the outlet due to upward advection. The moderator regions
experience hotter temperatures than the fuel regions due to radiative heating
and the relatively inefficiency of heat conduction in the graphite compared to
advection in the fuel salt.

\begin{figure}[htb!]
	\centering
	\begin{subfigure}[h]{0.45\columnwidth}
	    \includegraphics[width=\columnwidth]{combined_msre_moltres_radial}
	    \caption{Radial fluxes at reactor half-height.}
	    \label{fig:msre-flux-radial}
	\end{subfigure}
	\hfill
	\begin{subfigure}[h]{0.45\columnwidth}
	    \includegraphics[width=\columnwidth]{combined_msre_moltres_axial}
	    \caption{Axial fluxes along the core centerline ($r=0$ cm).}
	    \label{fig:msre-flux-axial}
	\end{subfigure}
	\caption{The fast and thermal fluxes from
	Moltres \cite{lindsay_introduction_2018} and the \gls{ORNL} \gls{MSRE}
	design calculations \cite{briggs_molten-salt_1964}.}
\end{figure}

The corresponding neutronics and thermal-hydraulics results from their 3D model
show good qualitative agreement with the 2D results. Given the lack of
\gls{MSR} experimental data, they compared their 2D model results with
\gls{MSRE} design calculations performed using legacy software in 1963-1964 at
\gls{ORNL}. Their neutron flux and temperature distribution results showed good
qualitative agreement with \gls{ORNL} data. The authors attributed
discrepancies to the following differences in the two modeling approaches: the
absence of axial heat conduction and the use of 32 neutron groups in the
\gls{ORNL} calculations, and the exclusion of control rod thimbles in the
Moltres calculations.

\subsubsection{Critical Assessment} \label{sec:msre-critique}

Ultimately, Lindsay et al. achieved their goal of ``introducing Moltres as a
simulation tool'' \cite{lindsay_introduction_2018}. Their work illustrated the
development of neutronics and thermal-hydraulics models in Moltres and
demonstrated, along with advanced capabilities from the \gls{MOOSE} framework,
fully-coupled simulations of the \gls{MSRE} with implicit time-stepping. The
decent qualitative agreement observed between Moltres and the \gls{ORNL}
\gls{MSRE} calculations proved that Moltres was capable of simulating
\glspl{MSR} with some simplifying assumptions. Furthermore, Moltres was a
relatively newly developed software at the time of submission.

On the other hand, significant improvements could be made to Moltres to better
model multiphysics phenomena in \glspl{MSR}. For instance, replacing the
fixed uniform salt flow with a proper flow profile governed by fluid flow
equations would accurately capture precursor and temperature advection.
Temperature advection has a particularly large impact on the temperature
distribution in the fuel salt since molten salts generally have large Prandtl
numbers, which measures the ratio of convective to conductive heat transfer.
The flow-modeling feature would be of even greater importance when modeling
pool-type \glspl{MSR} which consist of a single large fuel salt region in the
reactor core.

Another essential feature for modeling \glspl{MSR}, which was already under
development at the time of publication, is a precursor loop
system to recirculate precursors back into the core. While some precursors
decay outside the core, others survive long enough to recirculate back into the
core. The loop system would provide a more accurate estimate of the delayed
neutron fraction as opposed to discarding all precursors which flowed out of
the core. In transient simulations involving sudden increases in the neutron
flux, precursors recirculating into the core can induce observable jumps and
dips in the power output due to the associated reactivity insertion from the
delayed neutrons.

Additionally, Moltres would also benefit from a decay heat model which Lindsay
et al. also mention in their work \cite{lindsay_introduction_2018}, especially
for accident transient analyses. While decay heat from fission product decays
represents a small fraction ($\sim5\%$) of total power output, this heat source
can be significant in unprotected loss of flow or loss of secondary cooling
accidents. Therefore, understanding residual heat generation from fission
product decays in \gls{MSR} is essential in preventing further structural
failure in the aftermath of an accident.

Lastly, the authors rightly recognized two other avenues for future work: more
rigorous validation and verification of Moltres' capabilities by comparison
with more detailed experimental data and other modern \gls{MSR} modeling
efforts; and the study of transient simulation cases investigating control rod
ejection, single channel blockage, loss of flow, and loss of secondary cooling.

\subsection{Steady-State and Transient Simulations of the MSFR}
\label{sec:msfr}

This section follows my previous work in \textit{Advancement and Verification
of Moltres for Molten Salt Reactor Safety Analysis}
\cite{park_advancement_2020}, which I shall refer to as ``this work'' in this
section.

This work represents a continuation of the work by Lindsay et al.
\cite{lindsay_introduction_2018} in the development of Moltres as an \gls{MSR}
simulation tool. New features introduced include the coupling of the existing
neutronics and temperature equations to incompressible Navier-Stokes equations
to model flow dynamics, the precursor loop system, and the decay heat model.
I later demonstrated these features through steady-state and transient
simulations of a 2D axisymmetric \gls{MSFR} model.

Moltres depends on the \texttt{Navier-Stokes} module in \gls{MOOSE} for
incompressible flow modeling capabilities. Efforts towards coupling
the neutronics capabilities to the \texttt{Navier-Stokes} module were already
underway during the peer review process of the work described in Section
\ref{sec:msre}. Section \ref{sec:th} describes the governing equations for
incompressible flow and temperature advection-diffusion. At the same time,
the precursor loop system was being developed to model
precursor recirculation into the core. The precursor loop system leverages on
the \texttt{MultiApp} system in \gls{MOOSE} to couple a 1D pipe model to the
core to simulate precursor flow outside the core. The core and the pipe models
are coupled via inlet and outlet boundary conditions, as detailed in Section
\ref{sec:moltres-loop}. The loop system can also accommodate a pointwise heat
exchanger model with heat removal rate as a function of time
and/or auxiliary variables in the Moltres simulation.



Navier-stokes flow
precursor loop
decay heat
2D MSFR
accident transient analysis

\subsubsection{Critical Assessment} \label{msfr-critique}

