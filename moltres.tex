This chapter provides an abridged overview of Moltres as a multiphysics
simulation software for molten salt reactors. 
Section \ref{sec:moltres-features}
describes general software features of Moltres, Section
\ref{sec:moltres-physics} expands on the physics models in Moltres, and Section
\ref{sec:moltres-previous} describes the current capabilities in Moltres.

\section{General Features} \label{sec:moltres-features}

This section discusses the general software features of Moltres. The
discussion specifically focuses on robustness, extensibility, and ease of use
as these characteristics represent the hallmarks of a good multiphysics
software \cite{keyes_multiphysics_2013}. These criteria also apply for
assessing \gls{MSR} simulation tools since the nonlinearity of \gls{MSR}
multiphysics analysis and the complexity of advanced reactor designs
necessitate robust, scalable, and flexible computational tools.

Moltres draws many advantages from being developed on MOOSE
\cite{permann_moose_2020}. MOOSE is an open-source, finite-element,
multiphysics framework developed at \gls{INL}. The framework provides a
user-friendly interface for developing multiphysics software through
\gls{OOP} in \texttt{C++} to modularize various
functions relevant to finite-element, multiphysics solvers. In this approach,
MOOSE and MOOSE-based applications break down \glspl{PDE} into individual terms
and store them as individual \texttt{C++ objects} referred to as
\texttt{Kernels}. These \texttt{Kernels} contain functions for calculating
their weak form residual and Jacobian
contributions and other relevant functions required to solve a given
\gls{PDE}. \gls{OOP} in MOOSE simplifies software development
since developers can write new \texttt{Kernels} as child classes in
\texttt{C++} derived from existing \texttt{Kernels} (base classes) which share
similar physics to inherit common functions.
The same philosophy applies for all other systems in MOOSE such as
the \texttt{BoundaryCondition}, \texttt{Materials}, and \texttt{Postprocessor}
systems for handling relevant boundary conditions, material properties, and
postprocessing calculations, respectively. Overall, this approach also saves
researchers time and effort as they are unencumbered by the technical details
and complexities involved in programming efficient computational tools for
numerical analysis.

\begin{figure}[htb!]
	\centering
	\includegraphics[width=.7\columnwidth]{moose}
	\caption{Structure of MOOSE and its dependencies.}
	\label{fig:moose}
\end{figure}

MOOSE relies on two other open-source libraries: libMesh
\cite{kirk_libmesh_2006} for its \gls{FEM} capabilities on
unstructured mesh, and PETSc \cite{satish_petsc_2019} for its non-linear
solvers and preconditioning routines. By extension, Moltres gains access to
these libraries for their \gls{PDE} solving capabilities. Figure
\ref{fig:moose} shows how MOOSE serves as an interface between physics
applications and libMesh/PETSc. MOOSE supports
modeling on up to three-dimensional (3-D) unstructured meshes for a wide range
of mesh file formats, including the commonly used Exodus II file format. For
2-D meshes, users can opt between Cartesian and polar RZ coordinates. MOOSE
also supports parallel computing through the \gls{MPI} library to leverage
modern high-performance computing for large multiphysics simulations.

Moltres benefits from the highly-integrated cross compatibility within the
ecosystem of MOOSE-based applications. MOOSE facilitates multiphysics coupling
among all MOOSE-based applications by providing a common framework for shared
data access and file input/output, thus eliminating computational costs from
data transfers and allowing for fully coupled solves. For example, Moltres
couples with the \texttt{Navier-Stokes} module \cite{peterson_overview_2018}
from MOOSE for fully coupled reactor simulations modeling neutronics and
thermal-hydraulics with incompressible flow. Section
\ref{sec:challenges} highlighted the advantages of using fully coupled schemes
for modeling strongly coupled systems such as the coupled
neutronics and thermal-hydraulics in \glspl{MSR}. Moltres can also
couple to other MOOSE-based applications in a similar fashion with ease. In
addition, MOOSE
provides the option for either tight or loose coupling through the
\texttt{MultiApp} system \cite{gaston_physics-based_2015}. Tight coupling
schemes can outperform fully coupled schemes in weakly coupled systems in which
the computational expenses of fully coupled schemes outweigh the savings from
running fewer Newton iterations due to the superior convergence rate. Loose
coupling schemes are useful for accelerating time-dependent simulations of
stable systems towards steady state in which only the steady-state
configuration is of interest to the user. This typically occurs in the later
stages \gls{MSR} simulations when the delayed neutron precursor concentrations
converge slowly due to their relatively large decay half-lives. Furthermore,
segregated solves through the \texttt{MultiApp} system enables Moltres to
introduce delayed neutron precursor drift and non-uniform temperature
distributions into criticality search simulations. Regardless of which
coupling scheme is best, MOOSE-based applications provide the flexibility
to switch among the schemes as users see fit. For time-dependent simulations,
MOOSE provides more than ten different implicit and explicit timestepping
schemes. The default, most common scheme is the first-order backward Euler
method which offers excellent solver stability for stiff \glspl{PDE}.

Lastly, Moltres is an open-source \gls{LGPL} software hosted on
GitHub \cite{github_build_2017}. Open-sourcing software provides ease of access
and expands the userbase. These characteristics promote software quality
through increased feedback on users' needs and transparency for peer review.
Open-source software accelerate research progress by supporting research
collaboration and sharing best software practices. Supporting Moltres'
continued development, Moltres relies on GitHub for online version control with
continuous integrated testing to protect its existing capabilities.

In summary, Moltres provides robust, yet flexible coupling capabilities to
model strongly coupled neutronics and thermal-hydraulics in \glspl{MSR}. As a
MOOSE-based application, Moltres is highly extensible by means of coupling to
other MOOSE-based applications and benefits from MOOSE's user-friendly
interface for software development and general ease of use.

\section{Physics Models} \label{sec:moltres-physics}

This section describes the various physics models available in Moltres to model
coupled neutronics and thermal-hydraulics in \glspl{MSR}.

\subsection{Neutronics}

\subsubsection{Multigroup Neutron Diffusion Equations}

Moltres solves the multigroup neutron diffusion equations for the neutron
flux solution within the problem domain. These equations are derived from the
neutron transport
equation in the diffusion-dominated limit with Fick's law of diffusion and
further simplified by discretizing the continuous neutron energy variable into
a finite number of energy groups. The time-dependent multigroup neutron
diffusion equations with $G$ energy groups and $I$ delayed neutron precursor
groups are given by:
%
\begin{align}
    \frac{1}{v_g} \frac{\partial \phi_g}{\partial t} =& \nabla \cdot D_g
    \nabla \phi_g - \Sigma^r_g \phi_g +
    \sum^G_{g' \neq g} \Sigma^s_{g' \rightarrow g} \phi_{g'} \nonumber \\
    &+ \chi^p_g \sum^G_{g'=1} \left( 1-\beta \right) \nu \Sigma^f_{g'}
    \phi_{g'} + \chi^d_g \sum^I_i \lambda_i C_i \label{eq:neutron} \\
    %
    \shortintertext{where}
    v_g =& \text{ average speed of neutrons in group $g$,} 
    \nonumber \\
    \phi_g =& \text{ neutron flux in group $g$,}
    \nonumber \\
    t =& \text{ time,} \nonumber \\
    D_g =& \text{ diffusion coefficient of neutrons in} \nonumber \\
    &\text{ group $g$,} \nonumber \\
    \Sigma^r_g =& \text{ macroscopic cross section for removal of} \nonumber \\
    &\text{ neutrons from group $g$,} \nonumber \\
    \Sigma^s_{g' \rightarrow g} =& \text{ macroscopic cross section of
    scattering from} \nonumber \\
    &\text{ groups $g'$ to $g$,} \nonumber \\
    \chi^p_g =& \text{ prompt fission spectrum for neutrons in} \nonumber \\
    &\text{ group $g$,} \nonumber \\
    G =& \text{ total number of discrete neutron groups,} \nonumber \\
    \nu_g =& \text{ average number of neutrons produced per} \nonumber \\
    &\text{ fission,} \nonumber \\
    \Sigma^f_{g} =& \text{ macroscopic fission cross section for neutron}
    \nonumber \\
    &\text{ in group $g$,} \nonumber \\
    \chi^d_g =& \text{ delayed fission spectrum for neutrons in} \nonumber \\
    &\text{ group $g$,} \nonumber \\
    I =& \text{ total number of delayed neutron precursor} \nonumber \\
    &\text{ groups,} \nonumber \\
    \beta =& \text{ total delayed neutron fraction.} \nonumber
\end{align}

In spite of forming only around 0.7\% of all neutrons emitted, delayed neutrons
play outsized roles in reactor kinetics. The relatively long half-lives of
delayed neutron precursors gives us ample time in adequately designed reactors
to control reactor power output and intervene in case of power excursions.
The precursor concentration balance equations for $I$ precursor
groups are given by:
%
\begin{align}
    \frac{\partial C_i}{\partial t} =& \beta_i \sum^G_{g'=1} \nu \Sigma^f_{g'}
    \phi_{g'} - \lambda_i C_i - \vec{u} \cdot \nabla C_i + \nabla \cdot
    D_{\text{P}} \nabla C_i \label{eq:precursor} \\
    %
    \shortintertext{where}
    \beta_i =& \text{ delayed neutron fraction of precursor group $i$,}
    \nonumber \\
    \lambda_i =& \text{ average decay constant of delayed neutron} \nonumber \\
    &\text{ precursors in precursor group $i$,} \nonumber \\
    C_i =& \text{ concentration of delayed neutron precursors in}
    \nonumber \\
    &\text{ precursor group $i$,} \nonumber \\
    \vec{u} =& \text{ molten salt flow velocity vector,}
    \nonumber \\
    D_{\text{P}} =& \text{ effective diffusion coefficient of the delayed}
    \nonumber \\
    &\text{ neutron precursors.} \nonumber
\end{align}

These two equations are largely similar to conventional formulations of the
multigroup neutron diffusion equations with delayed neutrons for most reactor
types. The only differences are in the last two terms in Equation \ref{eq:dnp}
which represent the advection and diffusion terms, respectively, to model the
movement of \gls{DNP} in liquid-fuel \glspl{MSR}.

As shown in Equations \ref{eq:neutron} and \ref{eq:precursor}, Moltres requires
group constant data from dedicated high-fidelity neutronics software such as
the NEWT module in SCALE \cite{dehart_reactor_2011}, Serpent
\cite{leppanen_serpent_2014}, or OpenMC \cite{romano_openmc:_2015}. These group
constant data are the neutron energy group $g$ values for $v_g$, $D_g$,
$\Sigma^r_g$, $\Sigma^s_{g' \rightarrow g}$, $\chi^p_g$, $\chi^d_g$,
$\Sigma^f_{g}$, and $\nu\Sigma^f_{g}$, and precursor group $i$ values for
$\beta_i$ and $\lambda_i$. Users
can run a Python script in Moltres' Github repository which automatically reads
user-provided SCALE/Serpent/OpenMC output data files and creates
Moltres-compatible JSON format files containing all required group constant
data. Moltres allows for an arbitrary number of neutron energy groups $G$ and
precursor groups $I$ as long as the user provides the necessary group constant
data. In practice, $I$ depends on the nuclear data library used to generate
group constants---the JEFF \cite{plompen_joint_2020} and ENDF
\cite{brown_endfb-viii0_2018} data libraries define eight precursor groups
and six precursor groups, respectively.

In multiphysics reactor simulations, we model the coupling between neutronics
and thermodynamics through temperature-dependent group constants. To sample
group constants at different temperatures in Moltres, users must provide group
constant data measured at more than one temperature (e.g. 800K--1500K at 100K
intervals). Users can then choose from linear spline, cubic spline, or monotone
cubic interpolation methods available in Moltres to interpolate the group
constant data for values falling within the provided temperature range. 

\subsubsection{Boundary Conditions}

Moltres provides two types of boundary conditions for neutron fluxes; these are
conventionally known as the vacuum and reflective boundary conditions given,
respectively, as:

\begin{align}
    \frac{\partial \phi}{\partial x_i} + \frac{\phi}{2} =& 0
    \shortintertext{and}
    \frac{\partial \phi}{\partial x_i} =& 0
    \shortintertext{where}
    x_i =& \mbox{ spatial coordinate perpendicular to the boundary.} \nonumber
\end{align}

The vacuum boundary condition typically applies to the external boundaries of
the reactor beyond which lies low-interaction media such as air, while the
reflective boundary condition is useful for exploiting symmetries in the
model geometry such as the axial boundary in axisymmetric geometries. The
reflective boundary condition is also equivalent to the more generally known
homogeneous Neumann boundary condition. Relevant boundary conditions for
delayed neutron precursors include the homogeneous Neumann boundary condition
along fuel salt-structural interfaces; and outflow/inflow boundary
conditions along the outlet/inlet boundaries through which the precursors
flow as they circulate the fuel salt loop.

\subsection{Thermal-Hydraulics}

\subsubsection{Temperature Advection-Diffusion}

\subsubsection{Incompressible Navier-Stokes Flow}

\subsubsection{Boundary Conditions}

\subsection{Precursor Loop}

\section{Previous Work} \label{sec:moltres-previous}

\subsection{MSRE}

\subsection{MSFR}

\subsection{CNRS Benchmark}

%\section{Motivation}

A liquid-fueled \gls{MSR} is a class of advanced nuclear reactor which has fissile material
dissolved in a molten salt mixture. This fuel salt mixture doubles as the primary reactor coolant
for transferring heat from the reactor core to the primary heat exchangers. Figure \ref{fig:msr}
shows a schematic diagram of a thermal-spectrum, channel-type \gls{MSR}. In channel-type
\glspl{MSR}, the fuel salt flows through vertical channels in the reactor core next to neutron
moderators (e.g., graphite, heavy water, molten sodium hydroxide). 
%
\begin{figure}[htb!]
	\centering
	\includegraphics[width=.7\columnwidth]{msr}
	\caption{Schematic diagram of the \gls{MSR} concept. Retrieved from
	\cite{doe_technology_2002}.}
	\label{fig:msr}
\end{figure}

The \gls{MSR} is one of six advanced reactor designs selected at the \gls{GIF} for improved safety,
sustainability, efficiency, and cost over the current generation of predominantly \glspl{LWR}.
Due to the high thermal expansion coefficient, \glspl{MSR} possess an inherently robust
safety feature in the strong negative fuel temperature coefficient of
reactivity \cite{elsheikh_safety_2013}. This reactivity coefficient limits the
maximum temperature that the reactor core would experience in an accident
scenario such as an unprotected reactivity insertion because the subsequent
rise in core temperatures induces a significant drop in reactivity which
quickly neutralizes the initial reactivity insertion. \glspl{MSR} also
operate at a large thermal margin to boiling and can rely on natural
circulation in the event of a pump failure. As a last resort, many \gls{MSR}
designs incorporate a drain plug consisting of actively-cooled frozen salt, which
melts when the core temperatures exceed safety thresholds. The hot molten salt
in the core would then flow down into a drain tank designed to hold the fuel salt in
a subcritical configuration to disrupt any further chain fission reactions.

Some \glspl{MSR} like the \gls{MSBR} or the \gls{MSFR} can
incorporate the thorium fuel cycle for improved sustainability arising from the
use of abundant natural thorium resources and reduced transuranic waste
\cite{heuer_towards_2014}. The latter consequence also reduces costs
associated with long-term nuclear waste storage. In addition, the ability to
operate near atmospheric pressures eliminates the need for a thick pressure
vessel and drives down construction costs, while online fuel reprocessing
reduces reactor downtime during reactor operation \cite{dolan_1_2017}.
We can make further economic arguments supporting \glspl{MSR} in the context of the
carbon-constrained future envisioned in \gls{IEA}'s \gls{NZE} roadmap
\cite{iea_net_2021}. The roadmap for minimizing carbon emissions requires solar \gls{PV}- and
wind-dominated energy markets, which can bring about highly variable electricity generation. The
resulting volatility in electricity prices
encourages the construction of heat storage and peak power
production plants. At the same time, demand for carbon-neutral
fuel will rise as electrification is economically unfeasible
for some industries such as the aviation and marine sectors, which depend on
energy-dense fuels for propulsion. As described by Forsberg
\cite{forsberg_market_2020}, the most cost-efficient options for the
aforementioned resources (heat storage, peak power
production, and hydrogen fuel) all require high-temperature heat.
This requirement favors \glspl{MSR} which are expected to deliver heat at higher average
temperatures than \glspl{LWR} and other high-temperature advanced reactors.
\glspl{MSR} may also possess an edge over other high-temperature advanced reactors from
synergistic benefits of siting molten salt heat storage facilities near the power generation
facility.

\subsection{Past \& Present \gls{MSR} Research \& Development}

\gls{ORNL} researchers first conceived the \gls{MSR} concept in pursuit of a high-temperature
liquid fuel reactor for the US Aircraft Nuclear Propulsion program in
the 1950s \cite{rosenthal_molten-salt_1970}. They
built the first ever operational \gls{MSR}, the 2.5 MW$_{\text{th}}$
\gls{ARE} reactor at \gls{ORNL}. The successful demonstration of the \gls{ARE} spurred further
research into adapting \glspl{MSR} for civilian power generation
\cite{rosenthal_molten-salt_1970}. Continued \gls{MSR} research efforts culminated in the design,
construction, and successful operation of the 8-MW$_{\text{th}}$, thermal-spectrum \gls{MSRE} with
graphite channels and a LiF-BeF$_2$-ZrF$_4$-UF$_4$ fuel salt mixture
\cite{haubenreich_experience_1970}. In addition to other operational achievements, the
\gls{MSRE} became the first reactor to run on $^{233}$U fuel bred from $^{232}$Th. Building on
their experience with the \gls{MSRE}, \gls{ORNL} proposed a new program for the construction and
opeartion of a demonstration reactor based on the \gls{MSBR} concept that they had developed
\cite{macpherson_molten_1985}. The \gls{MSBR} is a thermal-spectrum \gls{MSR} with fertile
$^{232}$Th isotopes mixed directly into the \gls{FLiBe} molten salt for $^{233}$U breeding. Like
the \gls{MSRE}, the \gls{MSBR} was to rely on continuous online salt reprocessing to add fertile
material and remove fission product neutron poisons.
However, the \gls{MSBR} project was canceled prior to the demonstration stage in
favor of the \gls{LMFBR}, which had benefited from a head start in development and stronger
political backing \cite{macpherson_molten_1985}.

Following a relative lull lasting until the late 1990s, renewed research efforts and the \gls{GIF}
provided new impetus for \gls{MSR} research and development. As of the end of 2022, numerous
\gls{MSR} designs exist at various stages of development. Leading \gls{MSR} designs, in terms of
development, licensing, and/or demonstration, include the \gls{MSFR} \cite{merle_optimized_2007},
\gls{MCFR} \cite{terrapower_terrapower_2021}, TMSR-LF1 \cite{zhang_review_2018}, and \gls{IMSR}
\cite{leblanc_18_2017}. The \gls{MSFR} is a fast-spectrum breeder reactor developed through
collaborative efforts from European institutes with funding support from the
European Union. Figure \ref{fig:msfr} shows a schematic diagram of the \gls{MSFR}. As opposed to
the multi-channel design of the \gls{MSRE} and \gls{MSBR}, the
\gls{MSFR} core consists of a large molten salt pool without graphite moderators to avoid
frequent graphite replacements and positive graphite temperature reactivity feedback. The
\gls{MCFR} is a similar pool-type reactor under active development at TerraPower. TerraPower and
Southern Company embarked on a joint project to design, construct, and operate a prototype
\gls{MCRE} design with funding support from \gls{DOE}'s \gls{ARDP}. \gls{CAS} launched the
\gls{TMSR} program in 2011 to develop and construct both solid-fueled and liquid-fueled \gls{TMSR}
designs \cite{zou_research_2019}. They finished construction of TMSR-LF1, a 2-MW$_{\text{th}}$
liquid-fueled prototype, in August 2021 and received approval for reactor commissioning in August
2022. Lastly, Canada-based Terrestrial Energy is also developing their \gls{IMSR}, a small modular
\gls{MSR} based on the \gls{MSRE}, which passed a joint technical review carried out by the
Canadian and US nuclear regulators in July 2022.
%
\begin{figure}[htb!]
	\centering
	\includegraphics[width=.7\columnwidth]{msfr}
	\caption{Schematic diagram of the \gls{MSFR}. Retrieved from 
	\cite{allibert_7_2016}.}
	\label{fig:msfr}
\end{figure}

\subsection{\gls{MSR} Modeling \& Simulation}

With the renewed global interest in \glspl{MSR}, \gls{MSR} modeling software play an important role
in supporting \gls{MSR} development.
Accurate reactor modeling capabilities are important because they
accelerate reactor design and optimization by enabling quicker iteration through numerous design
changes. \gls{MSR} modeling software are also essential tools in reactor safety analysis and
licensing efforts as reactor developers must demonstrate and verify the \gls{MSR} systems perform
as designed and remain safe under various accident scenarios.

While modeling \glspl{MSR} is not necessarily more difficult than modeling
solid-fueled reactors, we must adapt our software tools to accurately model the
unique phenomena found in these circulating-fuel reactors. The differences in
the challenges of simulating \glspl{MSR} compared to solid-fueled reactors stem
mainly from the liquid fuel form of the fuel salt \cite{diamond_phenomena_2018,
huff_identifying_2019}.

Liquids generally exhibit greater thermal
expansion per unit change in temperature than solids. A decrease in density of
the fuel medium increases the likelihood of neutrons escaping the fuel region
and being absorbed by non-fissile material elsewhere in the reactor.
Consequently, combined with the temperature-dependent Doppler broadening of
resonance capture cross sections, \glspl{MSR} possess stronger negative fuel
temperature reactivity feedback than their solid-fueled counterparts
\cite{elsheikh_safety_2013}. These
phenomena ultimately result in strong interactions between the neutron fluxes
and core temperatures given that neutron fluxes affect core temperatures
through fission heat generation and core temperatures in turn affect neutron
fluxes through the mechanisms as described prior.

With the fuel
salt also serving the role of providing cooling in the core, velocity flow
profiles in the fuel salts strongly impact the temperature distribution via
advection-dominated heat transfer \cite{diamond_phenomena_2018}. This contrasts
with the relatively static temperature profiles in fuel pins and
other forms of solid fuel matrixes, which are physically separate from the coolant.

\glspl{DNP} flow freely within the primary coolant loop as opposed to
being held in place as in solid-fueled reactors. Thus, the delayed neutron
source distribution varies significantly depending on the flow profile and
velocity. In addition, the reactor loses some delayed neutrons from out-of-core
\gls{DNP} decay. These delayed neutrons are considered lost as they're emitted
in subcritical regions and are unlikely to contribute to further fission
reactions in the active core. The reduced delayed neutron fraction in the core
contributes to a greater prompt power spike following a reactivity insertion
event compared to solid-fueled reactors, absent any temperature reactivity
feedback.

Molten-salt flow along various parts of the coolant loop may fall within the turbulent flow
regime, characterized by chaotic eddies, vortices, and other flow instabilities.
Turbulent flow effects further complicate multiphysics interactions of flow with the temperature
and \gls{DNP} distribution. Turbulent flow effects contribute significantly to advection-dominated
heat and particle transfer in molten salt systems, thereby causing enhanced mixing. Therefore,
multiphysics software for \gls{MSR} analysis require adequate flow modeling capabilities with
support for \gls{DNP} drift and some form of turbulence modeling.

\subsection{Moltres for Multiphysics \gls{MSR} Analysis}

Moltres \cite{lindsay_moltres_2017} is an open-source multiphysics reactor simulation software
developed specifically with the considerations for \gls{MSR} characteristics in mind. Moltres is
built on the \gls{MOOSE} \cite{permann_moose_2020} open-source finite-element framework,
which facilitates multiphysics coupling between different
\gls{MOOSE}-based and \gls{MOOSE}-wrapped applications. The \gls{MOOSE} framework also provides
Moltres with advanced meshing and numerical solver capabilities through interfacing with libMesh
\cite{kirk_libmesh_2006} and PETSc \cite{satish_petsc_2019} open-source libraries. Therefore,
Moltres supports up to 3-D unstructured meshes, scales well on high-performance computing systems,
and provides a flexible multiphysics coupling system, which can be tailored for the type of
system being analyzed.

Moltres models coupled neutronics and thermal-hydraulics in reactors. While
generally applicable to most reactor concepts, much of
Moltres' development focuses on meeting the needs of \gls{MSR} multiphysics simulations.
Together with \gls{MOOSE}'s \texttt{Heat}
\texttt{Conduction} and \texttt{Navier-Stokes} \cite{peterson_overview_2018}
modules, Moltres solves the multigroup neutron diffusion
equations, for an arbitrary number of energy and precursor groups, and
thermal-hydraulics equations simultaneously on the same mesh (or separately solved and coupled
through fixed-point iterations if desired).

Lindsay et al. \cite{lindsay_introduction_2018}
demonstrated Moltres' multiphysics \gls{MSR} modeling capabilities with 1D salt
flow in 2D-axisymmetric and 3D models of the \gls{MSRE}. The neutron flux and
temperature distributions agreed qualitatively with legacy
\gls{MSRE} data albeit with some minor quantitative discrepancies due to
simplifications and assumptions in the reactor geometry. I demonstrated Moltres' capabilities for
1) looping of \gls{DNP} drift back into the reactor core, 2) coupling the \gls{DNP}
drift to numerically calculated salt flow profiles within the reactor core,
and 3) a decay heat model to simulate decay heat from fission products, with a 2-D axisymmetric
design of the \gls{MSFR} in my Master's thesis \cite{park_advancement_2020}.

While the Moltres
model of the \gls{MSFR} showed good agreement with other studies in most steady-state and transient
simulation cases, the Moltres model showed significant discrepancies during pump-initiated
transient scenarios in the absence of a proper turbulence model. Instead, the model applied a
uniform eddy viscosity assumption, which proved to be inadequate under non-steady flow. In order to
advance Moltres as a multiphysics simulation software for \gls{MSR} analysis, Moltres requires a
turbulence modeling capability to capture adequate turbulent flow phenomena and its interactions
with other physics present in \glspl{MSR}.

\subsection{\gls{VV} of Multiphysics \gls{MSR} Models and Software}

\gls{VV} of simulation software and models are important steps in simulation software development
\cite{sargent_verification_2010}. Verification is the process of checking whether a software
and its implementation accurately represents the conceptual description and specifications.
Validation is the process of checking whether a model is an accurate representations of the real
world within the range of its intended uses. For reactor software, verification is commonly
performed by comparison with other reactor software designed to run the same type of reactor
simulations. On the other hand, validation is performed by comparing numerical results from
a simulation model to experimental data from the corresponding live test. The validity of a model,
and the results derived from it, depends on the outcome of both verification and validation.

The important multiphysics phenomena in \glspl{MSR} for \gls{VV} are salt flow-induced
\gls{DNP} drift and the strong coupling between neutron flux and temperature advection-diffusion.
Notable \gls{VV} studies on \gls{MSR} modeling include work by Delpech et al.
\cite{delpech_benchmark_2003}, Tiberga et al. \cite{tiberga_results_2020}, Fratoni et al.
\cite{fratoni_molten_2020}, and Brovchenko et al. \cite{brovchenko_neutronic_2019} in relation to
the \gls{MSRE} and the \gls{MSFR} designs.



Delpech et al. \cite{delpech_benchmark_2003}
published one of the first modern \gls{VV} work for \gls{MSR} multiphysics modeling. Collaborators
from six institutions modeled \gls{MSRE} pump start-up, pump coast-down, and natural
circulation transients to assess and validate their models and codes for studying the effects of
salt flow on the reactivity and power. Given the wide range of neutronics methods from
multidimensional Monte Carlo methods to zero-dimensional point reactor kinetics, some deviations
were observed between different codes. Neverthless, all results showed generally good agreement
with \gls{MSRE} experimental data from \gls{ORNL}.

As mentioned in Subsection \ref{sec:msr-tools}, Tiberga et al. \cite{tiberga_results_2020}
published the CNRS benchmark for the verification of \gls{MSR} simulation tools designed for
fast-spectrum \gls{MSR} modeling. In contrast with the multi-channel \gls{MSRE} and its derivative
designs, the CNRS benchmark has a 2 m$\times$2 m problem domain of homogeneous fuel salt mimicking
the large salt pool in fast-spectrum \gls{MSR} designs. The CNRS benchmark consists of three phases
starting with single-physics calculations in Phase 0, followed by problems which gradually
introduce multiphysics coupling in Phase 1, and lastly time-dependent pertubation problems in Phase
2. Thus, the benchmark provides a systematic approach aimed towards helping code developers
identify sources of discrepancies which may otherwise be masked by error cancellation or other
dominant sources of discrepancies. The final steady-state and time-dependent problems involve
studying the effects of natural circulation and lid-driven flow on the reactivity and power output.
Aside from the problem specifications, Tiberga et al. also published the associated group
constant cross section data required by deterministic neutronics solvers to perform neutronics
calculations. Four institutions participated in the benchmarking exercise with neutron diffusion,
$SP_N$, and $S_N$-based solvers for the neutronics calculations. Their measured neutronics and
\gls{TH} parameters showed excellent agreement within up to 2.5\% discrepancy from their combined
average.

Neutronic benchmark studies of the \gls{MSRE} and the \gls{MSFR} by Fratoni et al.
\cite{fratoni_molten_2020} and Brovchenko et al. \cite{brovchenko_neutronic_2019} measured the
delayed neutron losses due to the decay of \glspl{DNP} flowing out of the active core region.
Fratoni et al. sought to establish a standard validation platform for \gls{MSR} neutronics
simulation tools with \gls{MSRE} experimental data for inclusion in the \gls{IRPhEP} handbook.
They characterized and validated a model of the \gls{MSRE} in the Monte Carlo particle transport
code Serpent. On the other hand, the \gls{MSFR} benchmark by Brovchenko et al. featured results
from multiple \gls{MSR} simulation tools by several collaborators. Their assessment found that
the choice of nuclear database for the cross sections and decay data has the most significant
impact on the neutronics results.

While these publications have plugged significant technical gaps, more can be done to develop
open \gls{VV} procedures for \gls{MSR} multiphysics modeling. For instance, the
CNRS benchmark does not assess the loss of delayed neutrons due to the decay of \glspl{DNP} flowing
out of the active core region. This phenomenon is important as the delayed neutron fraction in the
core directly impacts the transient power response in unprotected accident scenarios. Meanwhile,
the neutronic benchmark studies by Fratoni et al. \cite{fratoni_molten_2020} and Brovchenko et al.
\cite{brovchenko_neutronic_2019} did not provide standardized group constant data required by most
deterministic multiphysics \gls{MSR} simulation tools. Therefore, it is difficult to isolate
discrepancies arising from code implementations as opposed to discrepancies from using different
nuclear databases or stochastic uncertainties in Monte Carlo simulations. A good model verification
procedure for delayed neutron loss should ideally provide well-defined problems and the necessary
input data. It is also helpful to perform model verification studies on simpler problems like the
bare homogeneous problem domain in the CNRS benchmark before embarking on more complicated
validation studies which require accurate models of the reference experiments.

\section{Objectives}

This thesis demonstrates latest capabilities of Moltres
\cite{lindsay_introduction_2018}.
In particular, this thesis presents two more recent
developments in Moltres, namely fully integrating \gls{MOOSE}'s incompressible
Navier-Stokes module into Moltres, and introducing a
decay heat model.
The main objective of this thesis is to verify Moltres'
latest capabilities in modeling multiphysics, steady-state, and transient
behavior of fast-spectrum \glspl{MSR} through the study of the \gls{MSFR}
concept. Code-to-code verification is an important exercise in software
development for ensuring that the application produces accurate and reliable
results. This thesis covers the \gls{MSFR} concept mainly because it has been
studied extensively with readily available data in the literature to verify
against. The \gls{MSFR} design also features interesting flow
patterns that greatly affect the steady-state and transient behavior. This
present work will first present a verification of Moltres' \gls{MSFR}
diffusion neutronics against the Monte Carlo neutron transport software
Serpent 2, followed by a verification of
the coupled neutronics/thermal-hydraulics steady-state and accident transient
results against two sets of results published by
Fiorina et al. \cite{fiorina_modelling_2014}. The two sets of results arose
from a collaborative benchmarking exercise by researchers at Politecnico di
Milano and Technical University of Delft with two separate \gls{MSR}
simulation tools. Section \ref{chap:lit} discusses these tools
in greater detail. The
secondary objective is to identify areas of improvement in Moltres for future
development.

\section{Outline}

The outline of this thesis is as follows. Chapter 2 discusses the history and
features of \glspl{MSR}, and a literature review of existing \gls{MSR}
simulation tools. The chapter also covers the \gls{MSFR} concept in greater
detail. Chapter 3 details the software and the general modeling
approach for generating the results in this thesis. Chapter 4 provides a
neutronics assessment by comparing key neutronics parameters from Moltres'
eigenvalue calculations to Serpent's Monte Carlo calculations. Chapter 5
presents steady-state results of coupled neutronics/thermal-hydraulics
\gls{MSFR} simulations in Moltres. Chapter 6 presents accident transient
simulation results for unprotected reactivity insertions, unprotected loss of
heat sink, unprotected loss of flow, and unprotected pump overspeed. Lastly,
Chapter 7 summarizes the key findings in this thesis
and posits some potential avenues for future work.

\section{Description of the CNRS Benchmark} \label{sec:benchmark}

The CNRS Benchmark \cite{tiberga_results_2020} is a numerical
benchmark for multiphysics software dedicated to modeling \glspl{MSR}. It
consists of three phases and eight steps in total. Each
step is a well-defined subproblem for systematically assessing the
capabilities of \gls{MSR} software and pinpointing sources of discrepancies
between software. Phase 0 consists of three single-physics problems in fluid
dynamics, neutronics, and temperature. Phase 1 consists
of four coupled steady-state problems. Lastly, Phase 2 consists of one
coupled, time-dependent problem.

\begin{figure}[htb!]
	\centering
	\includegraphics[width=.6\columnwidth]{cnrs-geometry}
	\caption{2m$\times$2m 2D domain of the CNRS Benchmark. $U_{lid}$
	represents the velocity along the top boundary. For comparison, various quantities are
	measured along the centerlines AA' and BB'. From Tiberga et
	al. \cite{tiberga_results_2020}.}
	\label{fig:cnrs-geometry}
\end{figure}

As shown in Figure \ref{fig:cnrs-geometry}, the domain geometry is a
2m$\times$2m square cavity filled with LiF-BeF$_2$-UF$_4$ molten salt at an
initial temperature of 900K \cite{tiberga_results_2020}.
Standard vacuum boundary conditions apply for neutron flux along all
boundaries whereby outgoing neutrons are considered lost, while homogeneous
boundary conditions apply for delayed neutron precursors. No-slip boundary
conditions apply for velocity variables in the cavity, except along the top
boundary for Steps 0.1, 0.3, 1.1, 1.2, and 1.4, which impose forced flow in the
form of lid-driven
cavity flow. For the temperature variable, all boundaries are insulated, and we
simulate salt cooling with the following volumetric heat sink equation:
%
\begin{align}
    q'''(\vec{r}) &= \gamma \left(900 - T(\vec{r})\right) \label{eq:cnrs-heat}
    \shortintertext{where}
    q''' &= \mbox{volumetric heat sink [W$\cdot$m$^{-3}$],}
    \nonumber \\
    \gamma &= \mbox{heat transfer coefficient [W$\cdot$m$^{-3}\cdot$K$^{-1}$],}
    \nonumber \\
    T(\vec{r}) &= \mbox{temperature at point $\vec{r}$ [K].} \nonumber
\end{align}

Tiberga et al. \cite{tiberga_results_2020} used Serpent 2
\cite{leppanen_serpent_2014} with the JEFF-3.1 library
\cite{koning_jeff-31_2006} to generate multigroup neutronics data for the
LiF-BeF$_2$-UF$_4$ salt in the domain at 900K, which they condensed into six
energy groups and eight precursor groups. We direct readers to their paper for
the group constant data \cite{tiberga_results_2020}. In addition, the
benchmark prescribes the following equations to govern the temperature
dependence in the cross sections and the neutron diffusion coefficients:
%
\begin{align}
    \Sigma_i (T) &= \Sigma_i(T_{ref})
    \frac{\rho_{fuel}(T)}{\rho_{fuel}(T_{ref})}
    \shortintertext{and}
    D (T) &= D(T_{ref})
    \frac{\rho_{fuel}(T_{ref})}{\rho_{fuel}(T)}
    \shortintertext{where}
    \Sigma_i &= \mbox{relevant macroscopic cross section [cm${-1}$],}
    \nonumber \\
    D &= \mbox{neutron diffusion coefficient [cm$^2\cdot$s$^{-1}$],}   
    \nonumber \\
    \rho_{fuel} &= \mbox{density of the fuel salt [kg$\cdot$m$^{-3}$],}
    \nonumber \\
    T_{ref} &= \mbox{reference temperature} = 900\mbox{ K}. \nonumber
\end{align}

The benchmark also prescribes incompressible Navier-Stokes flow with the
Boussinesq approximation for evaluating the salt flow in the
domain but does not restrict the type of neutronics model.
Table \ref{table:benchmark} lists the relevant input parameters and observables.

\begin{table*}[tp!]
	\caption{Input parameters and observables of each benchmark step.}
	\centering
	\footnotesize
    \begin{tabular}{p{.05\textwidth} p{.1\textwidth} p{.3\textwidth} p{.45\textwidth}}
		\toprule
        \textbf{Step} & \textbf{Name} & \textbf{Input parameters} & \textbf{Observables} \\
		\midrule
        0.1 & Velocity field &
		\begin{itemize}[nosep,noitemsep,left=0pt,
		                before={\begin{minipage}[t]{\hsize}},
                        after ={\end{minipage}}]
		    \item $U_{lid} = 0.5$ m$\cdot$s$^{-1}$
		\end{itemize}\vspace*{-\baselineskip}\mbox{} &
		\begin{itemize}[nosep,noitemsep,left=0pt,
		                before={\begin{minipage}[t]{\hsize}},
                        after ={\end{minipage}}]
		    \item Velocity components $(u_x,u_y)$ along AA' and BB'
		\end{itemize}\vspace*{-\baselineskip}\mbox{} \\
        \midrule
        0.2 & Neutronics &
        \begin{itemize}[nosep,noitemsep,left=0pt,
		                before={\begin{minipage}[t]{\hsize}},
                        after ={\end{minipage}}]
		    \item $U_{lid} = 0$ m$\cdot$s$^{-1}$
		    \item $T = 900$ K
		    \item $P = 1$ GW
		\end{itemize} &
		\begin{itemize}[nosep,noitemsep,left=0pt,
		                before={\begin{minipage}[t]{\hsize}},
                        after ={\end{minipage}}]
		    \item Fission rate density $\sum^6_g \Sigma_{f,g} \phi_g(\vec{r})$ along AA'
            \item Reactivity $\rho$
		\end{itemize}\vspace*{-\baselineskip}\mbox{} \\
        \midrule
        0.3 & Temperature &
        \begin{itemize}[nosep,noitemsep,left=0pt,
		                before={\begin{minipage}[t]{\hsize}},
                        after ={\end{minipage}}]
		    \item Fixed flow field from Step 0.1 for
		    $U_{lid} = 0.5$ m$\cdot$s$^{-1}$
		    \item Fixed heat source distribution
		    $\sum^6_{g} \epsilon_g \Sigma_{f,g} \phi_g(\vec{r})$ from Step 0.2
		    \item $\gamma = 10^6$ W$\cdot$m$^{-3}\cdot$K$^{-1}$
		\end{itemize} &
		\begin{itemize}[nosep,noitemsep,left=0pt,
		                before={\begin{minipage}[t]{\hsize}},
                        after ={\end{minipage}}]
		    \item Temperature $T$ along AA' and BB'
		\end{itemize}\vspace*{-\baselineskip}\mbox{} \\
        \midrule
        1.1 & Circulating fuel &
        \begin{itemize}[nosep,noitemsep,left=0pt,
		                before={\begin{minipage}[t]{\hsize}},
                        after ={\end{minipage}}]
		    \item Fixed flow field from Step 0.1 for
		    $U_{lid} = 0.5$ m$\cdot$s$^{-1}$
		    \item $T = 900$ K
		    \item $P = 1$ GW
		\end{itemize} &
		\begin{itemize}[nosep,noitemsep,left=0pt,
		                before={\begin{minipage}[t]{\hsize}},
                        after ={\end{minipage}}]
		    \item Delayed neutron source $\sum^8_i \lambda_i C_i$ along AA' and BB'
		    \item Reactivity change between Step 1.1 and Step 0.2,
		    $\Delta \rho = \rho - \rho_{s_{0.2}}$
		\end{itemize}\vspace*{-\baselineskip}\mbox{} \\
        \midrule
        1.2 & Power coupling &
        \begin{itemize}[nosep,noitemsep,left=0pt,
		                before={\begin{minipage}[t]{\hsize}},
                        after ={\end{minipage}}]
		    \item Fixed flow field from Step 0.1 for
		    $U_{lid} = 0.5$ m$\cdot$s$^{-1}$
		    \item $P = 1$ GW
		    \item $\gamma = 10^6$ W$\cdot$m$^{-3}\cdot$K$^{-1}$
		\end{itemize}\vspace*{-\baselineskip}\mbox{} &
		\begin{itemize}[nosep,noitemsep,left=0pt,
		                before={\begin{minipage}[t]{\hsize}},
                        after ={\end{minipage}}]
		    \item Temperature $T$ along AA' and BB'
            \item Reactivity change between Step 1.2 and Step 1.1,
            $\Delta\rho = \rho - \rho_{s_{1.1}}$
            \item Change in fission rate density
            $\sum^6_g \Sigma_{f,g} \phi_g(\vec{r}) -
            \left[\sum^6_g \Sigma_{f,g} \phi_g(\vec{r})\right]_{s_{0.2}}$
		\end{itemize} \\
        \midrule
        1.3 & Buoyancy &
        \begin{itemize}[nosep,noitemsep,left=0pt,
		                before={\begin{minipage}[t]{\hsize}},
                        after ={\end{minipage}}]
		    \item $P = 1$ GW
		    \item $U_{lid} = 0$ m$\cdot$s$^{-1}$
		    \item $\gamma = 10^6$ W$\cdot$m$^{-3}\cdot$K$^{-1}$
		\end{itemize}\vspace*{-\baselineskip}\mbox{} &
		\begin{itemize}[nosep,noitemsep,left=0pt,
		                before={\begin{minipage}[t]{\hsize}},
                        after ={\end{minipage}}]
		    \item Velocity components $(u_x, u_y)$ along AA' and BB'
            \item Temperature $T$ along AA' and BB'
            \item Delayed neutron source $\sum^8_i \lambda_i C_i$ along AA' and BB'
            \item Reactivity change from Step 0.2
        $\Delta\rho = \rho - \rho_{s_{0.2}}$
		\end{itemize} \\
        \midrule
        1.4 & Full coupling &
        \begin{itemize}[nosep,noitemsep,left=0pt,
		                before={\begin{minipage}[t]{\hsize}},
                        after ={\end{minipage}}]
		    \item $\gamma = 10^6$ W$\cdot$m$^{-3}\cdot$K$^{-1}$
		    \item $P$ variable in the range $[0,1]$ GW with a step of 0.2 GW
		    \item $U_{lid}$ variable in the range $[0,0.5]$ m$\cdot$s$^{-1}$
		    with a step of 0.1 m$\cdot$s$^{-1}$
		\end{itemize} &
		\begin{itemize}[nosep,noitemsep,left=0pt,
		                before={\begin{minipage}[t]{\hsize}},
                        after ={\end{minipage}}]
		    \item Reactivity change between Step 1.4 and Step 0.2,
		    $\Delta\rho = \rho - \rho_{s_{0.2}}$, for all permutations of $P$
		    and $U_{lid}$ values
		\end{itemize}\vspace*{-\baselineskip}\mbox{} \\
        \midrule
        2.1 & Forced convection transient &
        \begin{itemize}[nosep,noitemsep,left=0pt,
		                before={\begin{minipage}[t]{\hsize}},
                        after ={\end{minipage}}]
		    \item $\gamma = 10^6$ W$\cdot$m$^{-3}\cdot$K$^{-1}$
            \item Steady-state solution from Step 1.4 for $U_{lid} = 0.5$
        m$\cdot$s$^{-1}$ and $P = 1.0$ GW
		\end{itemize} &
		\begin{itemize}[nosep,noitemsep,left=0pt,
		                before={\begin{minipage}[t]{\hsize}},
                        after ={\end{minipage}}]
		    \item Power gain and shift as a function of the perturbation frequency
		\end{itemize}\vspace*{-\baselineskip}\mbox{} \\
		\bottomrule
	\end{tabular}
	\label{table:benchmark}
\end{table*}

Step 2.1 studies the transient response of the fully coupled nonlinear system.
Linear perturbation analyses are performed by introducing periodic
perturbations to the heat transfer coefficient $\gamma$ and studying the gain
and phase shift of the response in the total power $P$. For the initial
conditions, the steady-state solution from Step 1.4 with
$U_{lid} = 0.5$ m$\cdot$s$^{-1}$ and $P = 1$ GW is used. This initial
configuration is made exactly critical by scaling the neutron source terms,
from fission and \gls{DNP} decay, by the inverse of the criticality eigenvalue
solution from Step 1.4.

$\gamma$ is uniformly perturbed according to small-amplitude sine waves given
as:
%
\begin{align}
    \gamma =& \gamma_0 \left[ 1 + 0.1\sin\left(2 \pi f \right) \right]
    \shortintertext{where}
    \gamma_0 =& 10^6 \mbox{ W$\cdot$m$^{-3}\cdot$K$^{-1}$}, \nonumber \\
    f \in& \left\lbrace 0.0125, 0.025, 0.05, 0.1, 0.2, 0.4, 0.8 \right\rbrace 
    \mbox{ Hz.} \nonumber
\end{align}

The benchmark defines power gain as:
%
\begin{align}
    \mbox{Power gain} =& \frac{\left(P_{max} - P_{avg}\right)/P_{avg}}{
    \left(\gamma_{max} - \gamma_{avg}\right)/\gamma_{avg}}
\end{align}
%
The subscripts denote the maximum and time-averaged values of $P$ and $\gamma$.

\FloatBarrier

\section{Modeling Approach with Moltres} \label{sec:model}

This section describes the specific modeling approach for
simulating the CNRS Benchmark cases in Moltres.

For this work\footnote{The input files for
all benchmark
cases are available on the Moltres GitHub repository at 
\url{https://github.com/arfc/moltres/tree/devel/problems/2021-cnrs-benchmark}.
}, I ran the benchmark cases on a uniformly-spaced mesh
of 200$\times$200 elements. Thus, the dimensions of each mesh element are
0.01m$\times$0.01m. I adopted the group constant data
provided by Tiberga et al. \cite{tiberga_results_2020}. Next, I
discretized most variables, i.e., neutron fluxes, velocity
components, pressure, and temperature, using continuous, first-order, Lagrange
shape functions. The only exception is the precursor concentration variables,
which I discretized using zeroth-order monomial shape functions and solved
using a \gls{DFEM}. I interpolated the resulting discontinuous,
cell-centered precursor values to obtain the nodal values for results
analysis.

The
\texttt{Navier-Stokes} and \texttt{Heat} \texttt{Conduction} modules from
\gls{MOOSE} provide some of the capabilities for
modeling incompressible flow and heat transfer. In particular, I stabilized
the incompressible flow and temperature governing equations using the
\gls{SUPG} stabilization method implemented in \gls{MOOSE}
\cite{peterson_overview_2018}. Without \gls{SUPG} stabilization, I
observed spurious numerical oscillations in the velocity and temperature near
the top boundary due to the singularity on the top left corner where different
velocity boundary conditions meet. I also applied the \gls{PSPG} stabilization
scheme \cite{hughes_new_1986} from the Navier-Stokes module
\cite{peterson_overview_2018},
which enables equal-order discretizations in the velocity and pressure
variables. Equal-order discretizations with \gls{PSPG} are computationally
cheaper and more convenient than implementing higher-order
velocity discretizations for stability without \gls{PSPG}
\cite{chapelle_inf-sup_1993}.

Using the inverse power method solver in \gls{MOOSE}, I ran all eigenvalue calculations in
Steps 0.2, 1.1, 1.2, 1.3, and 1.4. I ran all other steps
using the Preconditioned Newton-Krylov solver
\cite{gaston_physics-based_2015}. The coupled steady-state problems in
Steps 1.2, 1.3, and 1.4 required segregated solvers for the neutronics
and the thermal-hydraulics due to the unique problem setups involving a
criticality search problem for the neutron multiplication factor
and a steady-state problem in thermal-hydraulics simultaneously.

\begin{table}[tb]
    \caption{Timestep sizes used for the time-dependent cases in
    Step 2.1, corresponding to 1/200th of the perturbation period.}
	\centering
	\setlength\tabcolsep{2.5pt}
	\begin{tabular}{l l l l l l l l}
	    \toprule
	    Frequency [Hz] & 0.0125 & 0.025 & 0.05 & 0.1 & 0.2 & 0.4 & 0.8 \\
	    \midrule
	    Timestep size [s] & 0.2 & 0.2 & 0.1 & 0.05 & 0.025 & 0.0125 & 0.00625
	    \\
	    \bottomrule
	\end{tabular}
	\label{table:timestep}
\end{table}

For the time-dependent cases in Step 2.1, I employed full coupling with
a second-order implicit Backward Differential Formula (BDF2) time-stepping
scheme. I set the timestep sizes for each driving frequency in the heat transfer
coefficient to 1/200th of the perturbation period. Table
\ref{table:timestep} shows the timestep sizes. I assumed the
systems reached asymptotic behavior when the magnitudes of neighboring power
peaks differed by less than 0.001\% for at least ten wavelengths. Under this
assumption, the phase shift measurements between adjacent waves always
converged before the magnitude measurements of the power peaks.

Table \ref{table:software} compares the numerical methods, meshing schemes, and
neutronics models of Moltres and the four participating software packages in
the CNRS benchmark paper \cite{tiberga_results_2020}. The $SP_N$ and
$S_N$ neutronics models refer to the simplified $P_N$ spherical harmonics and
$S_N$ discrete ordinates neutron transport models, respectively. Based on the
solvers and methods of solution, Moltres is most similar to the
PHANTOM-$S_N$ + DGFlows \cite{tiberga_discontinuous_2019} multiphysics package
from \gls{TUD} with the $S_2$ neutron transport model. Participants from
\gls{CNRS} and \gls{PSI}
employed non-uniform meshes which were refined near the boundaries. In contrast,
we and the \gls{PoliMi} and \gls{TUD} participants employed uniform meshes.

\FloatBarrier

\begin{landscape}
\begin{table*}[p]
    \caption{List of software packages and their corresponding model
    specifications for the CNRS Benchmark simulations
    \cite{tiberga_results_2020}.}
    \centering
    \begin{tabular}{p{4.2cm} p{7cm} p{3.3cm} p{2cm} p{2.7cm}}
        \toprule
        Software & Institute & Numerical method & Mesh & Neutronics model \\
        \midrule
        OpenFOAM & Centre national de la recherche scientifique (CNRS) & Finite volume & 200$\times$200 \newline Non-uniform & $SP_1$ \& $SP_3$ \\
        OpenFOAM & Politecnico di Milano (PoliMi) & Finite volume & 400$\times$400 \newline Uniform & Neutron diffusion \\
        GeN-Foam & Paul Scherrer Institute (PSI) & Finite volume & 200$\times$200 \newline Non-uniform & Neutron diffusion \\
        PHANTOM-$S_N$+DGFlows & Delft University of Technology (TUD) & Discontinuous finite \newline element & 50$\times$50 \newline Uniform & $S_2$ \& $S_6$ \\
        Moltres (This work) & University of Illinois at Urbana-Champaign (UIUC) & Continuous \& discontinuous finite element & 200$\times$200 \newline Uniform & Neutron diffusion \\
        \bottomrule
    \end{tabular}
    \label{table:software}
\end{table*}
\end{landscape}

\FloatBarrier

\input{cnrs-benchmark/results}
\section{Conclusion}

\glspl{MSR} feature significant multiphysics interactions which present
computational challenges for many existing multiphysics reactor analysis
software. This chapter presents code-to-code verification of Moltres
capabilities in modeling such multiphysics phenomena in fast-spectrum
\glspl{MSR} based on the CNRS benchmark \cite{tiberga_results_2020}.
The CNRS benchmark assesses multiphysics \gls{MSR} simulation
software through several steps involving single-physics and coupled
neutronics/thermal-hydraulics problems.

The results showed that Moltres is consistent with the participating software
presented in the CNRS benchmark paper for the modeling of important phenomena
in fast-spectrum \glspl{MSR}. The percentage discrepancies in the various
neutronics, velocity, and temperature quantities mostly fall below or within
one standard deviation of the average of the benchmark participants.
Minor deviations in the temperature in Steps 0.3 and 1.2 
stem from the discontinuous velocity
boundaries on the top corners in the lid-driven cavity flow. We have shown that
these deviations are limited to the top boundary of the domain and do not
affect the rest of the physical parameters. The results from
Moltres agree closest with the TUD-S$_2$ software package, which implements the
$S_2$ discrete ordinates method for
neutron transport on a uniform structured mesh with a \gls{DFEM}-based solver.
These features make Moltres the most similar to the TUD-$S_2$ model as compared
to the other models which employ different neutron transport models,
non-uniform meshes, and/or finite volume-based solvers.

This work verifies Moltres' capabilities for future work involving modeling and
simulation of fast-spectrum \glspl{MSR} such as the European \gls{MSFR} and
TerraPower's \gls{MCFR} \cite{terrapower_terrapower_2021}. Notably, the CNRS
benchmark does not assess modeling capabilities for complex physics phenomena
such as turbulent flow in \glspl{MSR}. This worked in favor of verifying
existing capabilities in Moltres since Moltres does not currently support
turbulence modeling. However, we expect coolant loops in many \gls{MSR} designs
will experience turbulent flow under normal operation or accident scenarios.
These expectations, alongside the subpar results of pump-initiated accidents
reported in Section \ref{sec:msfr}, call for the implementation and
verification of a turbulence model in Moltres for accurate modeling of
\glspl{MSR}.

\FloatBarrier
