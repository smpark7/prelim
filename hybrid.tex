Section \ref{sec:challenges-control-rod} highlighted the poor performance of neutron diffusion
methods for calculating neutron fluxes near control rods. Strong neutron absorption in the control
rod region produces highly anisotropic neutron flux extending some distance outside the control
rod. Neutron transport methods, which retain angular dependence of the neutron flux in one way or
other, generally fare better than neutron diffusion methods with isotropic diffusion coefficients.
However, neutron transport methods are also generally more computationally expensive given the
increased dimensionality of the problem from the angular component. Piling this extra dimension on
top of the existing geometric and neutron energy group dimensions greatly multiplies the unknowns
to be solved in a system. Many past efforts have tried introducing transport correction techniques
to improve neutron flux and multiplication factor estimates with diffusion-based methods. Other
than control rod regions, these techniques also correct for homogenization error introduced from
spatial homogenization of fuel assemblies and other structures within a reactor core. They
invariably rely on neutron transport methods to generate transport corrections in the form of
corrected diffusion coefficients, boundary conditions, Eddington factors or discontinuity factors.

In this chapter, I propose a novel hybrid method for improving control rod modeling in neutron
diffusion solvers without spatial homogenization. In essence, the hybrid method involves applying
the $S_N$ discrete ordinates neutron transport method on subregions containing the control rod to
obtain transport corrected diffusion coefficients for the diffusion method on the entire problem
domain. 

Section \ref{sec:hybrid-theory} discusses the theoretical background for the Hybrid $S_N$-Diffusion
method. Thereafter, Section \ref{sec:preliminary} discusses preliminary results of a 1-D hybrid
method implemented with the finite difference method on several example cases based off the
graphite-moderated \gls{MSRE} reactor design.

\section{Theory} \label{sec:hybrid-theory}

The proposed Hybrid $S_N$-Diffusion method is a two-level, iterative method to improve the
accuracy of neutron diffusion solutions in reactor systems with highly neutron-absorbing control
rod regions. In this discussion, I focus on the 1-D, $k$-eigenvalue implementation of the
hybrid method.

The discrete ordinates ($S_N$) method for solving the multigroup neutron transport equation
(Eq. \ref{eq:mg-bte}) discretizes the continuous angular directional phase space into a finite
number of discrete angular intervals (ordinates). The set of distinct direction variables
$\hat{\Omega}_n$ representing the discrete ordinates is typically chosen in conjunction with a
compatible quadrature set to replace the integrals across the continuous solid angles
$\hat{\Omega}$ with quadrature approximations. The multigroup, discrete ordinates ($S_N$
approximation) form of the 1-D, $k$-eigenvalue neutron transport equation with the Gauss-Legendre
quadrature set is given as:
%
\begin{align}
  \mu_n \frac{d}{dx}\Psi_g(x, \mu_n) + \Sigma_{t,g}(x)&\Psi_g(x, \mu_n) -
\sum^G_{g'=1} \sum^N_{n'=1} \sum^L_{l=0} \frac{\left(2l+1\right)}{2}
\Sigma^{g'\rightarrow g}_{s,l}(x) P_l(\mu_{n'} - \mu_n)
w_{n'}\Psi_{g'}(x,\mu_{n'}) \nonumber \\
  &= \sum^G_{g'=1} \frac{\chi_g}{2} \frac{\nu\Sigma_{f,g}(x)}{k} \phi_{g'}(x) + S_g(x,\mu_n)
  \label{eq:1d-sn}
  \shortintertext{where}
  \mu_n &= \mbox{ cosine of $\hat{\Omega}_n$ relative to the $x$-axis,} \nonumber \\
  \Psi_g(x,\mu_n) &= \mbox{ neutron angular flux along $\mu_n$ in group $g$,} \nonumber \\
%  g &= \mbox{ neutron energy group index, ranging from 1 to G,} \nonumber \\
%  \Sigma_{t,g}(x) &= \mbox{ macroscopic total cross section of neutrons in group $g$,} \nonumber \\
%  G &= \mbox{ total number of energy groups,} \nonumber \\
%  n &= \mbox{ discrete ordinate index,} \nonumber \\
%  N &= \mbox{ total number of discrete ordinates,} \nonumber \\
  l &= \mbox{ Legendre order,} \nonumber \\
  L &= \mbox{ highest Legendre order,} \nonumber \\
  \Sigma^{g'\rightarrow g}_{s,l}(x) &= \mbox{ $l$-th Legendre expansion of the macroscopic
scattering} \nonumber \\
  &\ \ \ \ \mbox{ cross section from group $g'$ to $g$,} \nonumber \\
  P_l &= \mbox{ $l$-th Legendre polynomial,} \nonumber \\
  w_{n'} &= \mbox{ $n'$-th quadrature weight,} \nonumber \\
%  \chi_g &= \mbox{ fission neutron spectrum in group $g$,} \nonumber \\
%  \nu &= \mbox{ neutrons produced per fission reaction,} \nonumber \\
%  \Sigma_{f,g}(x) &= \mbox{ macroscopic fission cross section of neutrons in group $g$,}
%  \nonumber \\
  k &= \mbox{ neutron multiplication factor.} \nonumber
%  \phi_{g'}(x) &= \mbox{ scalar neutron flux in group $g$,} \nonumber \\
%  S_g(x,\mu_n) &= \mbox{ external neutron source in group $g$.} \nonumber
\end{align}
%
The neutron scalar flux and current can be retrieved by calculating the 0th and 1st Legendre moments
as follows:
%
\begin{align}
  \phi_g(x) &= \frac{1}{2} \int^1_{-1} d\mu\ \Psi_g(x,\mu) = \frac{1}{2} \sum^N_{n=1} w_n
\Psi_g(x,\mu_n)
  \shortintertext{and}
  J_g(x) &= \frac{1}{2} \int^1_{-1} d\mu\ \mu\Psi_g(x,\mu) = \frac{1}{2} \sum^N_{n=1} w_n
\mu_n \Psi_g(x,\mu_n)
\end{align}

The 1-D form of the multigroup $k$-eigenvalue neutron diffusion equations (Eq. \ref{eq:mg-diff})
is given as:
%
\begin{align}
  -\frac{d}{dx} D_g(x) \frac{d}{dx} \phi_g(x) + \Sigma_{t,g}(x) \phi_g(x) &= \sum^G_{g'=1}\left[
  \Sigma_s^{g'\rightarrow g}(x)\phi_{g'}(x) + \chi_g\frac{\nu\Sigma_{f,g'}(x)}{k}
  \phi_{g'}(x)\right] + S_g(x)
  \label{eq:1d-diff}
  \shortintertext{where}
    D_g(x) &= \frac{1}{3 \Sigma_{tr}(x)} = \mbox{ neutron diffusion coefficient for group }g.
  \nonumber
\end{align}

\subsection{Spatially Varying Diffusion Coefficients} \label{sec:svdc}

The conventional approach for determining diffusion coefficients based on the $P_1$ approximation
for each subregion involves running a high-fidelity neutron transport to tally region-wide
estimates of the neutron transport cross section. In essence, a single value represents the
isotropic diffusion coefficient for the entire (e.g. fuel, moderator, reflector) subregion.
However, as discussed in Section \ref{sec:challenges-control-rod}, the neutron diffusion equation
is only valid in regions of high scattering-to-absorption ratios and away from interfaces to
neighboring media with highly dissimilar neutronic properties.

In the Hybrid $S_N$-Diffusion method, I propose replacing the conventional $P_1$-based
diffusion coefficients with an alternate formulation which incorporates point-wise corrections
to the neutron diffusion flux solution from the $S_N$-derived flux solution as follows:
%
\begin{align}
  D^s_g(x) &= -J^{tr}_g(x)\bigg/\frac{d\phi^{tr}_g(x)}{dx}. \label{eq:svdc}
\end{align}
%
where $D^s$ is the \glspl{SVDC}, and the $tr$ superscript denotes neutron current and scalar flux
solutions from the $S_N$ method.
The basic form of this equation is Fick's first law of diffusion. The end result is a diffusion
coefficient variable that varies in space even within a subregion and provides pointwise
corrections to closely match the diffusion flux solution to the $S_N$ flux solution.

Pounders \& Rahnema \cite{pounders_diffusion_2009} demonstrated the effectiveness of applying
point-wise corrections derived from analytical or Monte Carlo reference flux solutions. Compared
with conventional $P_1$-based out-scatter and flux-limited approximations of the diffusion
coefficient, their \textit{high-order empirical diffusion coefficient} showed superior agreement
with the reference flux solutions. Their formulation for these piecewise-constant empirical
diffusion coefficients in Eq. \ref{eq:emp} is similar to the formulation for \glspl{SVDC} in Eq.
\ref{eq:svdc} after finite difference discretization. They recognized that the volume
averaging for piecewise-constant coefficients introduce some truncation error if the flux is
non-linear within each mesh element. This design choice may be due to an intention to retain the
diffusion coefficient as a constant in the $\frac{d}{dx}D\frac{d\phi}{dx}$ term of the neutron
diffusion equation (Eq. \ref{eq:1d-diff}).

Unlike their approach, the \gls{SVDC} formulation in Eq. \ref{eq:svdc} allows for continuously
varying diffusion coefficients to reduce truncation error. In practice, the discretization order of
\gls{SVDC} variables in a numerical calculation would follow the discretization order of the
reference flux solution. This formulation introduces a minor change to the finite difference
implementation of the second-order diffusion term to specifically handle spatial derivatives of the
diffusion coefficient, but the finite element implementation sees no change as the second-order
diffusion term is commonly treated with integration-by-parts to reduce it to a first-order
differential term with no derivatives of the diffusion coefficient. Refer to Section
\ref{sec:implementation} for the numerical implementation of a finite difference neutron diffusion
solver with \glspl{SVDC}.

\begin{figure}[htb!]
  \centering
  \includegraphics[width=.7\columnwidth]{case-0-geometry}
  \caption{A 1-D, two-region system containing a 0.5-cm thick control rod and a 24.5-cm thick
    homogeneous mixture of molten fuel salt and graphite moderator. Reflective boundary conditions
    are applied on both ends.}
  \label{fig:case-0-geom}
\end{figure}

\subsubsection{\gls{SVDC} Verification with Case 0}

To facilitate the following demonstration of a neutron diffusion calculation with \glspl{SVDC},
consider a 1-D, two-region system (Case 0) consisting of a highly neutron absorbing material and a
neutron multiplying region with reflective boundary conditions on both ends (Figure
\ref{fig:case-0-geom}). The material specifications are taken from the control rod, molten fuel
salt, and graphite moderator compositions of the \gls{MSRE} \cite{robertson_msre_1965}. The molten
fuel salt and graphite regions are homogenized to minimize discrepancies arising from geometrical
heterogeneity. The group constant input data for the diffusion and $S_N$ solvers (e.g. cross
sections, fission spectra, etc.) were sampled at 900 K the OpenMC Monte Carlo neutronics software.
The neutron energy spectrum is condensed into two discrete groups bounded at $10^{-5}$, $10^0$, and
$10^8$ eV.

I solved for the neutron flux in this system using the following set of numerical solvers:
%
\begin{enumerate}
  \item Diffusion solver with $P_1$ flux-limited diffusion coefficients generated directly from the
    OpenMC calculation.
  \item $S_4$ solver with up to 1st-order Legendre expansions of the group-to-group neutron
    scattering cross sections.
  \item Diffusion solver with \glspl{SVDC} generated from the prior $S_4$ flux solution.
\end{enumerate}
%
All other relevant group constants are identical and unchanged from the original dataset generated
using OpenMC. I implemented the diffusion solvers using the finite difference method and the $S_N$
solver using the diamond-difference transport sweep method in the Python programming language. For
brevity, further numerical implementation details of these solvers are deferred to Section
\ref{sec:implementation}.

\begin{figure}[htb!]
  \centering
  \begin{subfigure}[b]{.49\textwidth}
    \centering
    \includegraphics[width=\textwidth]{case-0-group-1-flux}
    \caption{Group 1 flux}
    \label{fig:c0g1flux}
  \end{subfigure}
  \hfill
  \begin{subfigure}[b]{.49\textwidth}
    \centering
    \includegraphics[width=\textwidth]{case-0-group-2-flux}
    \caption{Group 2 flux}
    \label{fig:c0g2flux}
  \end{subfigure}
  \caption{Neutron group 1 and 2 flux distributions from the diffusion, $S_4$, and
  diffusion-\gls{SVDC} solvers for Case 0.}
  \label{fig:c0flux}
\end{figure}
%
\begin{figure}[htb!]
  \centering
  \begin{subfigure}[b]{.49\textwidth}
    \centering
    \includegraphics[width=\textwidth]{case-0-group-1-diffcoef}
    \caption{Group 1 diffusion coefficients}
    \label{fig:c0g1diffcoef}
  \end{subfigure}
  \hfill
  \begin{subfigure}[b]{.49\textwidth}
    \centering
    \includegraphics[width=\textwidth]{case-0-group-2-diffcoef}
    \caption{Group 2 diffusion coefficients}
    \label{fig:c0g2diffcoef}
  \end{subfigure}
  \caption{$P_1$-based diffusion coefficient and \gls{SVDC} spatial distributions
  for Case 0.}
  \label{fig:c0diffcoef}
\end{figure}

Figure \ref{fig:c0flux} shows the group 1 and 2 neutron fluxes from the diffusion-$P_1$, $S_4$, and
diffusion-\gls{SVDC} solvers with a fixed mesh size of 0.005 cm. The OpenMC flux solution is
omitted because the purpose of this exercise is to demonstrate the effectiveness of \glspl{SVDC}
in reproducing a $S_N$ flux solution. As expected, the diffusion-$P_1$ flux solution deviates from
the $S_4$ flux solution while the diffusion-\gls{SVDC} flux solution shows significantly better
agreement with the $S_4$ flux solution. As shown in Table \ref{table:c0k}, $k$ estimate from the
diffusion-\gls{SVDC} solver (0.62418) is also closer to the $k$ estimate from the $S_4$ solver
(0.62467) than the diffusion-$P_1$ solver (0.60833).

\begin{table}[tb!]
  \centering
  \caption{Multiplication factor $k$ estimates from the diffusion-$P_1$, $S_4$, and
  diffusion-\gls{SVDC} solvers and the absolute difference relative to the $S_4$ estimate.}
  \begin{tabular}{l S S}
    \toprule
    Solver type & {$k$} & {$k-k_{S4}$} \\
    \midrule
    Diffusion-$P_1$ & 0.60833 & -0.01634 \\
    $S_4$ & 0.62467 & {-} \\
    Diffusion-\gls{SVDC} & 0.62418 & -0.00050 \\
    % OpenMC & {0.64604 +/- 0.00051} & {-} \\
    \bottomrule
  \end{tabular}
  \label{table:c0k}
\end{table}

Comparing the $P_1$-based diffusion coefficients and \glspl{SVDC} in Figure \ref{fig:c0diffcoef},
both quantities agree closely in the bulk fuel salt region far away from the control rod region.
This observation supports the validity of the neutron diffusion method with $P_1$ approximations
in a homogeneous medium with high scattering-to-absorption ratios. Moving towards the control rod
region, the \glspl{SVDC} deviate from the $P_1$-based diffusion coefficients as the prerequisite
assumptions for diffusion theory do not hold anymore.

Another key finding from this study is that the neutron diffusion method with $P_1$-based
diffusion coefficients fails to reproduce the steep $S_4$ flux gradient in the control rod region
for group 1 and in the homogeneous fuel-graphite region adjacent to the control rod region for both
neutron groups. The neutron diffusion method requires small \gls{SVDC} values in those regions to
induce steeper flux gradients and match the $S_4$ flux solution. Given that the presence of highly
neutron-absorbing materials tend to induce steep flux gradients, we can expect similar trends in
the \glspl{SVDC} relative to $P_1$-based diffusion coefficients for other similar systems.

Thus far, this method is trivial and completely redundant because it requires a priori
knowledge of the true flux solution or at least a highly accurate solution calculated using neutron
transport methods. While existing workflows for diffusion-based methods already require
computationally intensive neutron transport simulations to generate input data for the neutron
diffusion equation, this preprocessing step (known as \textit{group constant generation}) requires
a fixed number of neutron transport simulations. For instance, $P_1$-based diffusion coefficients
may be generated at various reactor temperatures. In the subsequent multiphysics reactor analysis,
these diffusion coefficient values may be interpolated for diffusion coefficient estimates at other
temperatures within the validity range. On the other hand, \glspl{SVDC} will likely need to be
dynamically generated at every timestep, such as with a two-level iterative scheme consisting of a
high-level $S_N$ neutron transport calculation and a low-level neutron diffusion calculation.

The nature of \glspl{SVDC} as empirical, pointwise corrections for
the diffusion equation makes it highly dependent on the neutron flux gradient, and
susceptible to greater variations than region-wide estimates for $P_1$-based diffusion
coefficients. Compared to \glspl{SVDC}, $P_1$ diffusion coefficients behave much more like
intrinsic material properties as their definitions are largely tied to material cross sections.
Variations in $P_1$ diffusion coefficients largely arise from changes in the neutron energy
spectrum with no direct contribution from proximity to material interfaces (geometrical
heterogeneity).

Another significant challenge of \glspl{SVDC} and the high-order empirical diffusion coefficients
involves their determination near neutron flux peaks. The neutron currents and
flux gradients in the numerator and denominator of Eq. \ref{eq:svdc} generally do not go to zero at
the same point, resulting in very large positive or negative diffusion coefficient values when the
flux gradient is close or equal to zero. Pounders \& Rahnema tackled this issue by using
larger mesh sizes with which to calculate their empirical diffusion coefficients. However, their
remedy significantly worsens flux accuracy in regions with steep, non-linear flux such as in
control rods. I plan to investigate and find an alternative solution for this issue in the
implementation of \glspl{SVDC} in the proposed work.

\subsection{Hybrid $S_N$-Diffusion Method} \label{sec:hybrid-method}

In order to reduce the computational cost of the high-level $S_N$ calculation, I propose reducing
the problem domain of the $S_N$ method to the control rod and its vicinity. Consequently, this
Hybrid $S_N$-Diffusion method can retain accurate neutron flux estimates around the control rod
region from the $S_N$ method while making significant savings in computational cost by treating the
majority of the reactor geometry with the neutron diffusion method. Henceforth, I will refer to
the $S_N$ calculation or method on the reduced problem domain as the $S_N$ \textit{subproblem} or
\textit{subsolver} accordingly. The problem domains of the neutron diffusion and $S_N$ calculations
are defined as $\Omega^d_0$ and $\Omega^d_1$, respectively, where $\Omega^d_1 \subseteq
\Omega^d_0$. The algorithm for the Hybrid $S_N$-Diffusion method is as follows:
%
%\begin{algorithm}
%  \caption{Hybrid $S_N$-Diffusion algorithm \label{alg:hybrid}}
%  \DontPrintSemicolon
%  \KwData{Material group constants and mesh of problem domain $\Omega^d$.}
%  \KwResult{Improved flux $\phi$ and multiplication factor $k$ estimates.}
%  \BlankLine
%  Initialize $\phi_g^0$, $k^0$, $D_g^0=D_g^{P_1}$; $m=0$\;
%  \While{$\epsilon_\phi > tol_\phi$ or $\epsilon_k > tol_k$}{
%    $m = m+1$\;
%    Solve the neutron diffusion equations for $\phi_g^m$ and $k^m$ using $D_g$ in domain
%    $\Omega^d$.\;
%    Update $\epsilon_\phi = ||\phi_g^m-\phi_g^{m-1}||_2/G/n$ where $G=$ total no. of groups and
%    $n=$ total no. of mesh elements.\;
%    Calculate neutron current boundary conditions along subdomain boundaries $\partial\Omega^d_1$
%    using $\phi_g^m$ and $D_g$, where $\Omega^d_1 \subset \Omega^d$
%  }
%\end{algorithm}
%
\begin{enumerate}
  \item Start with an initial neutron diffusion calculation in $\Omega^d_0$ with conventional $P_1$
    diffusion coefficients and other standard group constants (e.g. neutron cross sections).
  \item Use the neutron diffusion flux estimates in $\Omega^d_1$ and current estimates along
    $\partial \Omega^d_1$ as initial and boundary conditions for the $S_N$ subsolver.
  \item With the $S_N$ subsolver, calculate an improved neutron flux solution in $\Omega^d_1$ which
    contains the control rod region and its immediate vicinity.
  \item Calculate \glspl{SVDC} using the $S_N$ flux solution and Eq. \ref{eq:svdc}.
  \item Pass the \glspl{SVDC} to the neutron diffusion solver to replace the conventional
    $P_1$ diffusion coefficients within the regions that overlap with the $S_N$ solver problem
    domain.
  \item Start a new iteration by running neutron diffusion calculation with the \glspl{SVDC}
    replacing $P_1$ diffusion coefficients in part or all of $\Omega^d_1$.
  \item Pass the iterative neutron diffusion flux and current solutions to the $S_N$ subsolver.
  \item Iterate until convergence is reached by meeting pre-defined convergence tolerance values.
\end{enumerate}

\subsubsection{$S_N$ Subsolver Boundary Conditions \& Dead Zone Assumptions}

The main challenge lies in determining appropriate boundary conditions for the $S_N$ subproblem.
Given that we want to limit the coverage of $\Omega^d_1$ to the control rod region and its
immediate vicinity, the boundaries $\partial\Omega^d_1$ should lie well within $\Omega^d_0$.
However, there is currently no feasible method of generating accurate boundary fluxes for an $S_N$
solver from a neutron diffusion flux solution. In 1-D, the standard $S_N$ method requires N/2
incoming flux boundary parameters per boundary mesh point for the N/2 neutron angular fluxes
flowing into $\Omega^d_1$. The neutron diffusion method can produce at most one independent
incoming flux parameter per mesh point; this parameter is the neutron forward/backward current in
the $P_1$ approximation defined as:
%
\begin{align}
  J_{g,\pm} &= \frac{\phi_g}{4} \mp \frac{D_g}{2}\frac{d\phi_g}{dx} \label{eq:p1-j}
  \shortintertext{where}
  J_{g,\pm} &= \mbox{ neutron forward/backward current of group }g. \nonumber \\
\end{align}
%
In the absence of additional information, I will assume that an isotropic angular flux
distribution, i.e. all N/2 incoming angular fluxes are equal in magnitude. This concept is similar
to white boundary conditions which describe isotropic reflection of particles at a boundary except
no reflection occurs in this case. The isotropic angular flux can be expressed mathematically for
forward angular fluxes as:
%
\begin{align}
  \sum^N_{n=N/2+1}w_n\mu_n\Psi(x,\mu_n) =& J_{+}(x) && (\mu_n>0) \nonumber \\
  \Psi(x,\mu_n)\sum^N_{n=N/2+1}w_n\mu_n =& J_{+}(x) && (\because \mbox{isotropic hemisphere})
  \nonumber \\
  \Psi(x,\mu_n) =& J_{+}(x)\Bigg/\sum^N_{n=N/2+1}w_n\mu_n
\end{align}
%
and backward angular fluxes as:
%
\begin{align}
  \sum^{N/2}_{n=1}w_n\mu_n\Psi(x,\mu_n) =& J_{-}(x) && (\mu_n<0) \nonumber \\
  \Psi(x,\mu_n)\sum^{N/2}_{n=1}w_n\mu_n =& J_{-}(x) && (\because \mbox{isotropic hemisphere})
  \nonumber \\
  \Psi(x,\mu_n) =& J_{-}(x)\Bigg/\sum^{N/2}_{n=1}w_n\mu_n \label{eq:sn-psi-j}
\end{align}
%
These boundary conditions for the $S_N$ subsolver generally yield inaccurate flux solutions since
most reactor systems do not exhibit perfectly isotropic neutron fluxes.

To resolve this issue, I posit the following hypothesis: \textit{If there exists a highly
neutron-absorbing control rod region within the $S_N$ subproblem domain $\Omega^d_1$ and the $S_N$
subproblem boundary $\partial\Omega^d_1$ lies several neutron mean free paths away from this
region, the \gls{SVDC} values calculated near the control rod region (using the $S_N$ subsolver
with isotropic hemisphere boundary conditions) will tend to the actual flux gradient solution
(using a reference $S_N$ calculation across the entire problem domain $\Omega^d$).} In other words,
suboptimal boundary conditions may induce inaccurate \gls{SVDC} values near the $\partial
\Omega^d_1$ subdomain boundaries, but the \gls{SVDC} values further from $\partial\Omega^d_1$ are
accurate and very weakly dependent on the boundary conditions. These inaccurate \gls{SVDC} values
close to $\Omega^d_1$ may be discarded in favor of the default $P_1$-based diffusion coefficients.
I will refer to the region containing these discarded values as the \textit{dead zone} or
$\Omega^d_2$, where $\Omega^d_2 \subset \Omega^d_1 \subseteq \Omega^d_0$.

\subsubsection{Hybrid Method Verification with Case 0}

Here, we revisit Case 0, the 1-D two-region system described in Section \ref{sec:svdc} (Figure
\ref{fig:case-0-geom}), to test my hypothesis and demonstrate the Hybrid $S_N$-Diffusion method.
I defined $\Omega^d_1$ to span from $x=0$ cm to $x=17.5$ cm. Note that $\Omega^d_1$
is a subdomain of the entire domain spanning from $x=0$ cm to $x=20$ cm.
Once again, the numerical implementation details are deferred to Section \ref{sec:implementation}.

The size of $\Omega^d_2$
is dynamically determined based on the \gls{SVDC} distribution generated at every iteration during
the hybrid method calculation. In the earlier study with \glspl{SVDC}, the \gls{SVDC} distribution
in the homogeneous fuel-graphite region approaches the $P_1$-based diffusion coefficient value as
$x$ is further from the control rod region. Therefore, starting from the control rod and
fuel-graphite region interface and sweeping right, I set the cutoff point for the $\Omega^d_2$ to
be \textit{the point at which the \gls{SVDC} distribution approaches within 1\% of the $P_1$-based
diffusion coefficient value}. The hybrid method converged after two outer iterations comprising of
three neutron diffusion and two $S_4$ calculations in total as outlined by the hybrid method
algorithm.
%
\begin{figure}[htb!]
  \centering
  \begin{subfigure}[b]{.49\textwidth}
    \centering
    \includegraphics[width=\textwidth]{case-0-group-1-hybrid-diffcoef}
    \caption{Group 1 diffusion coefficients}
    \label{fig:c0g1hd}
  \end{subfigure}
  \hfill
  \begin{subfigure}[b]{.49\textwidth}
    \centering
    \includegraphics[width=\textwidth]{case-0-group-2-hybrid-diffcoef}
    \caption{Group 2 diffusion coefficients}
    \label{fig:c0g2hd}
  \end{subfigure}
  \caption{$P_1$-based flux-limited diffusion coefficient and \gls{SVDC} spatial distribution for
  Case 0. The \gls{SVDC} distributions were generated from the reference $S_4$ (Full) and the
  hybrid (Hybrid) calculations.}
  \label{fig:c0hd}
\end{figure}

For a visual illustration, refer to Figures \ref{fig:c0g1hd} and \ref{fig:c0g2hd} which plot the
group 1 and 2 $P_1$-based diffusion coefficients, \glspl{SVDC} derived from the reference $S_4$
calculation as demonstrated in Section \ref{sec:svdc}, and \glspl{SVDC} derived from the Hybrid
$S_N$-Diffusion method discussed here. The first set of \glspl{SVDC} are taken to be the reference
because it was generated from a $S_4$ calculation across the entire domain $\Omega^d_0$. The
reference \glspl{SVDC} are labelled ``Full'' in the figures to avoid confusion with the reference
flux distribution from the $S_4$ calculation. The \glspl{SVDC} from the hybrid method agrees well
with the reference \glspl{SVDC} in the control rod region and for most of the fuel-graphite region
up to around $x=12.5$ cm. Both sets of \glspl{SVDC} approximately coincide with the $P_1$ diffusion
coefficient values at $x=14.370$ cm and $x=4.015$ cm for group 1 and group 2, respectively. The
dead zone spans from $x=14.370$ cm to $x=17.500$ cm for group 1 and $x=4.015$ cm to $x=17.500$ cm
for group 2, in which the hybrid method defaults to the $P_1$-based diffusion coefficients by
definition. Given that the reference \glspl{SVDC} values will generally not be known in real-world
problems, the hybrid method relies on the matching of \gls{SVDC} and $P_1$ diffusion coefficient
values for determining the dead zone cutoff point. Another important consideration is the fact that
the neutron flux gradient will be discontinuous if the \gls{SVDC} and $P_1$ diffusion coefficient
values are not sufficiently continuous at the dead zone cutoff point. A flux gradient discontinuity
in a homogeneous bulk region is non-physical and therefore unacceptable. In Section
\ref{sec:prelim-results}, I study \gls{SVDC} distribution trends in more heterogeneous systems and
their implications on determining the dead zone cutoff point.
%
\begin{figure}[htb!]
  \centering
  \begin{subfigure}[b]{.49\textwidth}
    \centering
    \includegraphics[width=\textwidth]{case-0-group-1-hybrid-flux}
    \caption{Group 1 flux}
    \label{fig:c0g1hf}
  \end{subfigure}
  \hfill
  \begin{subfigure}[b]{.49\textwidth}
    \centering
    \includegraphics[width=\textwidth]{case-0-group-2-hybrid-flux}
    \caption{Group 2 flux}
    \label{fig:c0g2hf}
  \end{subfigure}
  \caption{Neutron group 1 and 2 flux distributions from the diffusion, $S_4$, reference
  \gls{SVDC}, and hybrid solvers for Case 0.}
  \label{fig:c0hf}
\end{figure}

Figures \ref{fig:c0g1hf} and \ref{fig:c0g2hf} show the group 1 and 2 neutron flux distributions
from the neutron diffusion calculations with $P_1$ diffusion coefficients and \glspl{SVDC}, and
$S_4$ calculation. As expected following the diffusion coefficient discussion, the hybrid method
flux distribution matches the $S_4$ flux distribution well. The $k$ estimate from the hybrid
method is 0.62444 which is only 0.00026 higher than the diffusion calculation with reference
\glspl{SVDC} (0.62418), and 0.00023 lower than the $S_4$ calculation (0.62467). By comparison, the
$k$ estimate from the diffusion calculation with $P_1$ diffusion coefficients is 0.01634 lower than
the $S_4$ calculation as discussed in Section \ref{sec:svdc}.

In Case 0, $\Omega^d_1$ covers more than half of $\Omega^d_0$ due to
the control rod region being the only significant influence on the neutron flux distribution in the
otherwise homogeneous system with reflective boundary conditions. I chose Case 0 as a simple test
case to aid in introducing the Hybrid $S_N$-Diffusion method. In the next section, I
present numerical implementation details of the Hybrid $S_N$-Diffusion method and results on more
complicated geometries with neutron reflector, air, and heterogeneous fuel-graphite lattice
regions and vacuum boundary conditions. On these test cases, the hybrid method yields
smaller $\Omega^d_1$-to-$\Omega^d_0$ ratios than the corresponding ratio in Case 0.

\section{Numerical Implementation} \label{sec:implementation}

I implemented all numerical solvers discussed in this work in the Python programming language.
First, I discuss the material group constant data generation and postprocessing steps. Next, I
present the implementation details of the neutron diffusion and $S_N$ numerical methods
individually. Thereafter I present how they are coupled to form the Hybrid $S_N$-Diffusion method.

\subsection{Group Constant Data Generation}

The group constants required by either or both neutron diffusion and $S_N$ neutron transport
methods are
%
\begin{itemize}
  \item $\Sigma_{t,g}$: Macroscopic total cross section in group $g$
  \item $\Sigma_{r,g}$: Macroscopic removal cross section in group $g$
  \item $\Sigma_s^{g'\rightarrow g}$: Macroscopic group-to-group scattering cross section matrix
  \item $\Sigma_{s,l}^{g'\rightarrow g}$: $l$-th Legendre expansion of the macroscopic
    group-to-group scattering cross section matrix
  \item $\Sigma_{sp,l}^{g'\rightarrow g}$: $l$-th Legendre expansion of the macroscopic
    group-to-group scattering production cross section matrix
  \item $D_g$: $P_1$-based diffusion coefficient in group $g$
  \item $\nu\Sigma_{f,g}$: Product of the average number of neutrons produced per fission and the
    macroscopic fission cross section in group $g$
  \item $\chi_g$: Neutron fission spectrum in group $g$
\end{itemize}
%
These group constants are generated using OpenMC's multigroup cross section generation capability
and postprocessed using a Python script into a JSON file format. $\Sigma_{r,g}$ is the only
quantity that is not directly provided by OpenMC. It is calculated as
%
\begin{align}
  \Sigma_{r,g} =& \sum^G_{g'\neq g}\Sigma_s^{g\rightarrow g'}+\Sigma_{a,g}-\left(\Sigma_{sp}^{g
    \rightarrow g} - \Sigma_s^{g\rightarrow g}\right)
  \shortintertext{where}
      \Sigma_{a,g} =& \mbox{ macroscopic absorption cross section in group $g$,} \nonumber \\
      \Sigma_{sp}^{g\rightarrow g} =& \mbox{ macroscopic scattering production cross section from
      group $g$ to $g$.} \nonumber
\end{align}
%
$\Sigma_{r,g}$ primarily represents the loss of neutrons from group $g$ through outscattering and
absorption. $\Sigma_{sp}^{g\rightarrow g}$ incorporates neutron multiplication effects from neutron
knockout reactions into the scattering cross section. Neutron knockout reactions are commonly
tallied as scattering reactions, and $\Sigma_{r,g}$ is a convenient term through which to
incorporate neutron knockout effects into the neutron diffusion equations. I ran all test cases in
this work with up to 1st-order Legendre expanded scattering cross sections.

OpenMC uses the $P_1$ flux-limited formulation \cite{pomraning_flux-limited_1984} for calculating
$D_g$ as follows
%
\begin{align}
  D_g =& \frac{1}{3\Sigma_{tr,g}} \\
  \Sigma_{tr,g} =& \frac{\langle\Sigma_{t,g}\phi_g\rangle-\langle\Sigma_{s1,g}\phi_g\rangle}
  {\langle\phi_g\rangle}
  \shortintertext{where}
  \langle\Sigma_{t,g}\phi_g\rangle =& \int_{r\in V}dr \int_{4\pi}d\Omega\int^{E_{g-1}}_{E_g}dE\
  \Sigma_{t,g}(r,E)\Psi(r,E,\Omega) \nonumber \\
  \langle\Sigma_{s1,g}\phi_g\rangle =& \int_{r\in V}dr \int_{4\pi}d\Omega\int^{E_{g-1}}_{E_g}dE
  \int_{4\pi}d\Omega'\int^{\infty}_0dE'\int^1_{-1}d\mu\ \mu\Sigma_s(r,E'\rightarrow E,\Omega'\cdot
  \Omega)\phi(r,E',\Omega') \nonumber \\
  \langle \phi \rangle =& \int_{r\in V}dr\int_{4\pi}d\Omega\int^{E_{g-1}}_{E_g}dE\ \Psi(r,E,\Omega)
  .\nonumber
\end{align}

\subsection{Neutron Diffusion Method}

On a 1-D uniform spatial grid with $I+1$ mesh points, the neutron scalar flux variables
$\phi_{g,i}$ are defined on the mesh points $x_i$. Theoretically, group constants are volumetric
material properties which should be defined on the cell-centered half-integer mesh points
$x_{i+\sfrac{1}{2}}$. In practice, all material properties except diffusion coefficients are
uniform in each subregion and are sampled at $x_i$. To avoid ambiguity concerning diffusion
coefficient sampling, I formulated all test cases such that all material interfaces fall on $x_i$.
A ghost point is added to the end of the spatial grid if the reflective boundary conditions are
imposed at that boundary.

Discretizing the multigroup $k$-eigenvalue neutron diffusion equations in Eq. \ref{eq:1d-diff}
and reformulating the scattering term using neutron balance in the control volume bounded by
$x_{i-\sfrac{1}{2}}$ and $x_{i+\sfrac{1}{2}}$ yields
%
\begin{align}
  J_{g,i+\sfrac{1}{2}} - J_{g,i-\sfrac{1}{2}} + \Sigma_{t,g,i} \phi_{g,i} \Delta x = \sum^G_{g'=1}\left[
  \Sigma_{s,i}^{g'\rightarrow g}\phi_{g',i} + \chi_{g,i}\frac{\nu\Sigma_{f,g',i}}{k} \phi_{g',i}
\right]\Delta x. \label{eq:diff-j}
\end{align}
%
Using the diamond difference scheme to replace the $J$ terms with the discretized form of
Fick's first law of diffusion,
%
\begin{align}
  J_{g,i+\sfrac{1}{2}} = -D_{g,i+\sfrac{1}{2}}\frac{d\phi_{g,i+\sfrac{1}{2}}}{dx} =
  -D_{g,i+\sfrac{1}{2}} \frac{\phi_{g,i+1}-\phi_{g,i}}{\Delta x},
\end{align}
%
and rearranging the terms in Eq. \ref{eq:diff-j} yields
%
\begin{align}
  -\frac{D_{g,i-\sfrac{1}{2}}}{\Delta x} \phi_{g,i-1} + &\left[\frac{D_{g,i-\sfrac{1}{2}}+
  D_{g,i+\sfrac{1}{2}}}{\Delta x} + \Delta x\ \Sigma_{r,g,i} \right]\phi_{g,i} -
  \frac{D_{g,i+\sfrac{1}{2}}}{\Delta x}\phi_{g,i+1} -\Delta x\sum^G_{g'\neq g}
  \Sigma_{s,i}^{g'\rightarrow g}\phi_{g',i} \nonumber \\
  =& \Delta x\sum^G_{g'=1}
  \chi_{g,i} \frac{\nu\Sigma_{f,g',i}}{k} \phi_{g',i}, \label{eq:diff-fd}
  \shortintertext{where}
  \Sigma_{r,g} =& \Sigma_{t,g} - \Sigma_s^{g\rightarrow g} \nonumber \\
  =& \mbox{ macroscopic removal cross section for neutron group }g. \nonumber
\end{align}
%
The diamond difference scheme is 2nd-order accurate, and it can be easily shown that this form is
equivalent to applying 2nd-order finite differencing to the original neutron diffusion equation in
Eq. \ref{eq:1d-diff} with diamond differencing for the cell-centered group constants.
The fixed neutron source $S_g$ is ignored here since the test cases are all neutron-multiplying
systems with no fixed source.

I implemented two types of boundary conditions: vacuum and reflective boundary conditions. The
\textbf{vacuum boundary conditions} are imposed by setting the incoming flux in the $P_1$
approximation to zero and applying 2nd-order finite differencing as follows
%
\begin{align}
  \mbox{Left boundary: } \frac{\phi_{g,0}}{4}-\frac{D_{g,\sfrac{1}{2}}}{2}
  \frac{\left(-\phi_{g,2}+4\phi_{g,1}-3\phi_{g,0}\right)}{2\Delta x} = 0 \\
  \mbox{Right boundary: } \frac{\phi_{g,I}}{4}+\frac{D_{g,I-\sfrac{1}{2}}}{2}
  \frac{\left(\phi_{g,I-2}-4\phi_{g,I-1}+3\phi_{g,I}\right)}{2\Delta x} = 0.
\end{align}
%
The \textbf{reflective boundary conditions} are imposed by setting equating the flux at the last physical
mesh point to the flux at the ghost point as follows
%
\begin{align}
  \mbox{Left boundary: } \phi_{g,-1} =& \phi_{g,0} \\
  \mbox{Right boundary: } \phi_{g,I} =& \phi_{g,I+1}
\end{align}
%
where the $-1$ and $I+1$ indexes correspond to ghost point indexes. At material interfaces,
the continuity condition requires that the net neutron current on either side of the interface be
equal as follows
%
\begin{align}
  -\frac{3D_{g,i-\sfrac{1}{2}} - D_{g,i-\sfrac{3}{2}} }{2}
  \frac{\phi_{g,i-2}-4\phi_{g,i-1}+3\phi_{g,i}}{2\Delta x} =&
  -\frac{3D_{g,i+\sfrac{1}{2}} - D_{g,i+\sfrac{3}{2}} }{2}
  \frac{-3\phi_{g,i}+4\phi_{g,i+1}-\phi_{g,i}}{2\Delta x} \label{eq:itf-bc}
\end{align}
%
for a material interface at $x_i$.

Altogether, they form a system of equations of the form $\bm{A\overline{\phi}}=\bm{\frac{1}{k}
B\overline{\phi}}$, where $\bm{\overline{\phi}}$ is a flattened vector representation of
$\phi_{g,i}$, and $\bm{A}$ and $\bm{B}$ are matrices of the coefficients of $\phi_{g,i}$ as given
by Eqs. \ref{eq:diff-fd} to \ref{eq:itf-bc}. I implemented the inverse power method to find $k$ and
$\overline{\phi}$. The inverse power method algorithm is as follows
%
\begin{align}
  \shortintertext{1. Initialize $k^0$ and $\bm{\overline{\phi}}^0$}
  \shortintertext{2. Update $\bm{\overline{\phi}}$ and $k$}
  \bm{\overline{\phi}}^m =& \frac{1}{k^{m-1}}\bm{A}^{-1}\bm{B\overline{\phi}}^{m-1} \\
  k^m =& k^{m-1}\frac{|\bm{B\overline{\phi}}^m|}{|\bm{B\overline{\phi}}^{m-1}|}
  \shortintertext{3. Check whether convergence is reached}
  \frac{|\bm{\overline{\phi}}^m-\bm{\overline{\phi}}^{m-1}|}{|\bm{\overline{\phi}}^m|} <& \
  tol_{\bm{\overline{\phi}}} \\
  \frac{|k^m-k^{m-1}|}{|k^m|} <& \ tol_k
  \shortintertext{4. Return to Step 2 if either expression is false, otherwise exit.} \nonumber
\end{align}
%
$k^m$ and $\overline{\phi}^m$ denote estimates of $k$ and $\overline{\phi}$ after the $m$-th
iteration. Matrix $\bm{A}$ is a primarily tridiagonal matrix with at most $G-1$ off-diagonal terms
from the 4th term in Eq. \ref{eq:diff-fd}. Thus, $\bm{A}$ is initialized as a sparse matrix to take
advantage of the computationally efficient sparse matrix solver functions from the \texttt{sparse}
class of the \texttt{SciPy} Python library for scientific and technical computing. Matrix $\bm{B}$
is never initialized explicitly as a matrix. Instead, the vector $\bm{b}=\bm{B\overline{\phi}}$ is
updated directly in every iteration. In this system of equations, $|\bm{b}|$ corresponds to the
total number of fission neutrons produced in the system for a given $\bm{\overline{\phi}}$. This
quantity is calculated using the \texttt{trapezoid} numerical integration function from
\texttt{SciPy} to estimate the integral value of $\nu\Sigma_{g,f}\phi_{g}$ in $x$ from the discrete
flux values in $\bm{x}$. Finally, the final $\phi_{g,i}$ is normalized by a factor of $|\bm{b}|/k$
(no. of source neutrons) to obtain the neutron scalar flux per source neutron.

\subsection{$S_N$ Neutron Transport Method}

For the 1-D $S_N$ neutron transport method on the same uniform spatial grid with $I+1$ mesh points,
the neutron angular flux variables $\Psi_{g,i,n}=\Psi_g(x_i,\mu_n)$ are defined on the mesh points
$x_i$ while neutron scalar flux variables $\phi_{g,i\pm\sfrac{1}{2}}=\phi_g(x_{i\pm\sfrac{1}{2}})$
are defined on the half-integer mesh points $x_{i\pm\sfrac{1}{2}}$. All group constants are sampled
at $x_{i\pm\sfrac{1}{2}}$.

The $S_N$ equations are solved using the transport sweep method in which the algorithm ``sweeps''
through the spatial grid and sequentially updates $\Psi_{g,i,n}$. In 1-D, the algorithm sweep
direction follows the direction of neutron travel, i.e. it sweeps in the positive direction for
$\Psi_{g,i,n}$ with $\mu_n>0$ and vice versa.

Discretizing the multigroup $S_N$ neutron transport equations in Eq. \ref{eq:1d-sn} about
$x_{i+\sfrac{1}{2}}$ yields
%
\begin{align}
  \mu_n\frac{\Psi_{g,i+1,n}-\Psi_{g,i,n}}{x_{i+1}-x_i} + \Sigma_{t,g,i+\sfrac{1}{2}}
  \Psi_{g,i+\sfrac{1}{2},n} = q_{g,i+\sfrac{1}{2},n}
\end{align}
%
where $q_{g,i+\sfrac{1}{2},n}$ represents the combined scattering and fission neutron source term.
After expressing $\Psi_{g,i+\sfrac{1}{2}}$ as the average of $\Psi_{g,i+1,n}$ and $\Psi_{g,i,n}$
in the diamond difference scheme and rearranging the terms, we obtain
%
\begin{align}
  \Psi_{g,i+1,n} =& \frac{1-\Sigma_{t,g}\Delta x/2\mu_n}{1+
    \Sigma_{t,g}\Delta x/2\mu_n}\Psi_{g,i,n} +
    q\frac{\Delta x}{\mu_n\left(1+\Sigma_{t,g}\Delta x/2\mu_n\right)} \label{eq:sweep-right} &&
    (\mbox{for } \mu_n > 0)
\shortintertext{and}
  \Psi_{g,i,n} =& \frac{1+\Sigma_{t,g}\Delta x/2\mu_n}{1-
    \Sigma_{t,g}\Delta x/2\mu_n}\Psi_{g,i+1,n} -
    q\frac{\Delta x}{\mu_n\left(1-\Sigma_{t,g}\Delta x/2\mu_n\right)}. \label{eq:sweep-left} &&
    (\mbox{for } \mu_n < 0)
\end{align}
%
The remaining $i+\sfrac{1}{2}$ indexes on the group constants are dropped to reduce visual clutter.
These expressions are used to update $\Psi_{g,i,n}$ in the forward and backward transport sweeps.

The discretized scattering and fission terms in $q_{g,i+\sfrac{1}{2},n}$ are given as
%
\begin{align}
  q_{g,i+\sfrac{1}{2},n} =& \sum^G_{g'=1}\sum^L_{l=0}\frac{\left(2l+1\right)}
  {2}\Sigma_{s,l}^{g'\rightarrow g}P_l(\mu_n)\phi_{l,g',i+\sfrac{1}{2}} \nonumber \\
  &+\frac{\chi_g}{2}\sum^G_{g'=1}\frac{\nu\Sigma_{f,g'}}{k}\phi_{0,g',i+\sfrac{1}{2}}
  \label{eq:sn-q}
  \shortintertext{where}
  \phi_{l,g',i+\sfrac{1}{2}} =& \sum^N_{n'=1}w_{n'}P_l(\mu_{n'})\frac{\Psi_{g',i,n'}+
  \Psi_{g',i+1,n'}}{2}. \label{eq:phi-l}
\end{align}
%
$\phi_{l,g,i}$ are $l$-th Legendre expansions of the neutron flux evaluated using Gauss-Legendre
quadrature over $\mu_{n'}=[-1,1]$. $\phi_{0,g,i}$ and $\phi_{1,g,i}$ also correspond to the neutron
scalar flux $\phi_{g,i}$ and net current $J_{g,i}$, respectively.

For vacuum boundary conditions, $\Psi_{g,0,n}$ is zero for all positive $\mu_n$ while
$\Psi_{g,I,n}$ is zero for all negative $\mu_n$ as follows
%
\begin{align}
  \mbox{Left boundary: } \Psi_{g,0,n} =& 0 && (\mbox{for } \mu_n > 0) \\
  \mbox{Right boundary: } \Psi_{g,I,n} =& 0 && (\mbox{for } \mu_n < 0)
\end{align}
%
Reflective boundary conditions are imposed by equating the incoming angular flux to the
outgoing angular flux in the opposite direction as follows
%
\begin{align}
  \mbox{Left boundary: } \Psi_{g,0,n} =& \Psi_{g,0,n'} && (\mbox{for } \mu_n > 0, \mu_n =
  -\mu_{n'}) \\
  \mbox{Right boundary: } \Psi_{g,I,n} =& \Psi_{g,I,n'} && (\mbox{for } \mu_n < 0, \mu_n =
  -\mu_{n'})
\end{align}

The power iteration algorithm for the $S_N$ method is similar to the inverse power method algorithm
applied in the neutron diffusion method. The matrix solving step for updating $\overline{\phi}$ is
replaced with the transport sweep step along with its associated steps as follows
%
\begin{align}
  \shortintertext{1. Initialize $k^0$, $\phi_{l,g,i+\sfrac{1}{2}}^0$, and $q^0$}
  \shortintertext{2. Apply transport sweeps to solve for $\Psi^m$ using Eqs. \ref{eq:sweep-right},
  \ref{eq:sweep-left}, and \ref{eq:sn-q}}
  \shortintertext{3. Update $\phi^m$ and $k^m$ using Eqs. \ref{eq:phi-l} and \ref{eq:sn-k}}
  k^m =& k^{m-1}\frac{\sum^I_{i=0}\sum^G_{g=1}\nu\Sigma_{f,g}\phi^m_{g,i+\sfrac{1}{2}}}
  {\sum^I_{i=0}\sum^G_{g=1}\nu\Sigma_{f,g}\phi^{m-1}_{g,i+\sfrac{1}{2}}} \label{eq:sn-k}
  \shortintertext{4. Check whether convergence is reached}
  \frac{|\bm{\overline{\phi}}^m-\bm{\overline{\phi}}^{m-1}|}{|\bm{\overline{\phi}}^m|} <& \
  tol_{\bm{\overline{\phi}}} \\
  \frac{|k^m-k^{m-1}|}{|k^m|} <& \ tol_k
  \shortintertext{5. Return to Step 2 if either expression is false, otherwise exit.} \nonumber
\end{align}

The transport sweep and $\phi^m$-update algorithms are parallelized using the \texttt{joblib}
parallel computing Python library across all available CPU threads. The total computational work is
subdivided into several smaller tasks by unique combinations of the $G$ and $N$ indexes; each
thread computes all $\Psi^m$ or $\phi^m$ values on the mesh for a given set of indexes $g$ and
$n$.

\subsection{Hybrid $S_N$-Diffusion Method}

Without loss of generality, consider a 1-D system symmetric about the left boundary at $x_0=0$ cm.
Like Case 0, the control rod region is the left-most region followed by other types of non-control
rod regions. The $S_N$ subproblem domain $\Omega^d_1$ is bounded by $x_0$ and $x_j$ for some $x_j$
located several mean free paths to the right of the control rod region as governed by the relevant
discussion in Section \ref{sec:hybrid-method}.

Excluding the initial neutron diffusion calculation to initialize $k^0$ and $\phi^0$, each outer
iteration in the hybrid method consists of one $S_N$ neutron transport calculation and one neutron
diffusion calculation. The $\phi$ estimates from these $S_N$ transport and neutron diffusion
calculations are labeled as $\phi^{m+\sfrac{1}{2}}$ and $\phi^{m+1}$, respectively, during the
$(m+1)$-th outer iteration. $\phi^{m+\sfrac{1}{2}}$ spans $\Omega^d_1$ while $\phi^{m+1}$ spans
the entire domain $\Omega^d_0$.

For the $S_N$ subproblem boundary conditions, we discretize Eqs. \ref{eq:p1-j} and
\ref{eq:sn-psi-j} as follows
%
\begin{align}
  \Psi_{g,j,n} =& \frac{J_{g,-}(x_j)}{\sum^{N/2}_{n'=1}w_{n'}\mu_{n'}} \nonumber \\
  =& \frac{\frac{\phi_g(x_j)}{4}+
  \frac{D_g(x_{j-\sfrac{1}{2}})}{2}\frac{d\phi_g(x_j)}{dx}}{\sum^{N/2}_{n'=1}w_{n'}\mu_{n'}}
  \nonumber \\
  =& \frac{\frac{\phi_{g,j}}{4}+
  \frac{D_{g,j-\sfrac{1}{2}}}{2}\frac{\phi_{g,j}-\phi_{g,j-1}}{\Delta x}}
    {\sum^{N/2}_{n'=1}w_{n'}\mu_{n'}} && (\mu_n < 0) \label{eq:sn-bc}
\end{align}
%
where $D_g$ is the $P_1$-based diffusion coefficient value.

The \gls{SVDC} formulation in Eq. \ref{eq:svdc} is discretized using the diamond difference scheme
as follows
%
\begin{align}
  D^s_{g,i+\sfrac{1}{2}} =& -\frac{J^{tr}_{g,i+1}+J^{tr}_{g,i}}{2} \left(
  \frac{\phi^{tr}_{g,i+1}-\phi^{tr}_{g,i}}{\Delta x} \right)^{-1} = -\frac{\Delta x}{2}
  \frac{J^{tr}_{g,i+1}+J^{tr}_{g,i}}{\phi^{tr}_{g,i+1}-\phi^{tr}_{g,i}}. \label{eq:svdc-num}
\end{align}
%
From Eq. \ref{eq:svdc-num}, we note that numerical instabilities may occur near flux peaks where
the flux gradient and thus the denominator of Eq. \ref{eq:svdc-num} approach zero. On the other
hand, this observation also implies that diffusion coefficient values have negligible influence on
the flux solution near flux peaks due to the small flux gradient values. Therefore, we can avoid
numerical instabilities by applying the following logic after calculating $D^s_g$: If the ratio of
$|\frac{d\phi_g}{dx}|$ to $|\phi_g|$ is sufficiently small and $|D^s_g-D_g| > D_g$, the hybrid
method defaults to using $D_g$ instead of $|D^s_g-D_g|$. The first conditional statement locates
regions of relatively flat flux, and of these regions, the second conditional statement excludes
regions which naturally experience flat flux such as air-filled regions. In all test cases,
applying the first conditional as $|\frac{1}{\phi_g}\frac{d\phi_g}{dx}|<5\times 10^{-2}$ was
sufficient for this task.

The Hybrid $S_N$-Diffusion algorithm is as follows
%
\begin{align}
  \shortintertext{1. Initialize $k^0$ and $\phi^0$ with an initial neutron diffusion calculation on
  $\Omega^d_0$}
  \shortintertext{2. Generate estimates for $\Psi_{g,j,n}$ at $x_j$ for the $S_N$ transport
  calculation boundary conditions using Eq. \ref{eq:sn-bc}}
  \shortintertext{3. Calculate $\phi^{m+\sfrac{1}{2}}$ with the $S_N$ transport subsolver on
  $\Omega^d_1$}
  \shortintertext{4. Generate \glspl{SVDC} with $\phi^{m+\sfrac{1}{2}}$ and $J^{m+\sfrac{1}{2}}$
  using Eq. \ref{eq:svdc-num}}
  \shortintertext{5. Calculate $\phi^{m+1}$ with the newly generated \glspl{SVDC} and the neutron
  diffusion solver on $\Omega^d_0$}
  \shortintertext{6. Check whether convergence is reached}
  \frac{|\bm{\overline{\phi}}^m-\bm{\overline{\phi}}^{m-1}|}{|\bm{\overline{\phi}}^m|} <& \
  tol_{\bm{\overline{\phi}}} \\
  \frac{|k^m-k^{m-1}|}{|k^m|} <& \ tol_k
  \shortintertext{7. Return to Step 2 if either expression is false, otherwise exit.} \nonumber
\end{align}
%
In general, the convergence tolerance values for the outer hybrid method iteration must be smaller
than the tolerance values for the neutron diffusion and $S_N$ transport inner iterations. Using
$\phi^m$ from the neutron diffusion calculation as initial conditions for the $S_N$ transport
calculation in the $(m+1)$-th iteration helps to significantly reduce the number of transport
sweeps required.

\section{Description of 1-D Test Cases}

I designed ten 1-D test cases with increasing complexity to test the performance of the
Hybrid $S_N$-Diffusion method in response to various geometrical features. The latter cases
resemble the reference \gls{MSRE} design which has centrally located control rods.and air-filled
guide tubes. Figure \ref{fig:case-geom} shows the geometries of the Cases 0 to 5b.
All geometries have reflective boundary conditions at $x=0$ cm to reduce computational costs by
creating half-core or repeating unit cell models. Cases 1a, 2a, 3a, and 4a are repeating unit
cell models with reflecting boundaries on the right-side boundaries. Cases 1b, 2b, 3b, 4b, 5a, and
5b are half-core models with vacuum boundaries on the right-side boundaries.

Starting with an infinite, homogeneous region in Case 1a, I systematically added geometric
complexities to each successive test case to identify how each feature impacts the neutronics
results. Case 1b introduces a reflector region and vacuum boundary conditions relative to Case 1a.
Cases 2a and 2b introduce a control rod region between $x=0$ cm and $x=0.5$ cm relative to Cases 1a
and 1b. Cases 3a and 3b introduce a 0.5 cm-thick air gap between the control rod and the
fuel-graphite mixture regions relative to Cases 2a and 2b. For Cases 4a and 4b, I replaced the
homogeneous fuel-graphite mixture in Cases 1a and 2b with explicit fuel-graphite lattices as shown
in Figure \ref{fig:case-geom}. Case 5a introduces an air gap between the control rod and the
fuel-graphite lattice regions. Lastly, Case 5b replaces the control rod region with an extended
air gap region to provide a base case for control rod worth calculations relative to Case 5a.

\begin{figure}[htb!]
  \centering
  \includegraphics[width=\columnwidth]{case-geometry}
  \caption{Geometries of the 1-D test cases. The material labelled ``mixture'' represents a
    homogeneous mixture of fuel and graphite at a ratio of 22.5\%-77.5\% by volume. All geometries
    have reflective boundary conditions on the boundary at $x=0$ cm. The right-side boundaries are
    reflecting for Cases 1a, 2a, 3a, and 4a, and vacuum for Cases 1b, 2b, 3b, 4b, 5a, and 5b.}
  \label{fig:case-geom}
\end{figure}

%\begin{table}[tb!]
%  \centering
%  \caption{Description of the 1-D Test Case Geometries.}
%  \begin{tabular}{l c c c}
%    \toprule
%    Cases & Ordered list of regions & Ordered list of interface x-coordinates & Left \& Right BCs\\
%    \midrule
%    Case 0 & Control rod, homogenized fuel-graphite lattice &
%    \bottomrule
%  \end{tabular}
%  \label{table:c0k}
%\end{table}

I ran all cases with four neutron energy groups bounded at $E=10^{-5}, 10^0, 10^2, 10^5, 10^8$
eV. Figures \ref{fig:spectrum} and \ref{fig:reaction} show how this energy group structure
partitions the neutron flux energy spectrum and neutron reaction rates. For this preliminary work,
the four-group structure is sufficient for showing different scattering and absorption trends of
slow, intermediate, and fast neutrons. I also ran all cases on OpenMC to generate the required
group constants and to assess the accuracy of my deterministic methods against OpenMC in
continuous-energy (OpenMC-CE) and multigroup (OpenMC-MG) modes. Table \ref{table:var} shows how the
position, direction of travel, neutron energy, and angle-dependence in $\Sigma_s$ are handled by
OpenMC and the deterministic methods. Comparing OpenMC-MG results with OpenMC-CE results allows us
to quantify errors arising from neutron energy group discretization and the scattering cross
section simplifications.

\begin{figure}[htb!]
  \centering
  \begin{subfigure}[t]{.49\textwidth}
    \centering
    \includegraphics[width=\textwidth]{spectrum}
    \caption{Neutron flux energy spectrum per source neutron}
    \label{fig:spectrum}
  \end{subfigure}
  \hfill
  \begin{subfigure}[t]{.49\textwidth}
    \centering
    \includegraphics[width=\textwidth]{reaction}
    \caption{Flux-normalized scattering and absorption reaction rates}
    \label{fig:reaction}
  \end{subfigure}
  \caption{Neutron flux energy spectrum and reaction rates for Case 5a. The dotted vertical lines
  correspond to the discrete neutron energy group boundaries.}
  \label{fig:spec-reac}
\end{figure}
%
\begin{table}[tb!]
  \centering
  \footnotesize
  \caption{Variable handling in OpenMC under continuous-energy (OpenMC-CE) and multigroup
  (OpenMC-MG) modes, and in the $S_4$ neutron transport, neutron diffusion, and Hybrid
  $S_N$-Diffusion methods. }
  \begin{tabular}{c c c c c c}
    \toprule
    Variable & OpenMC-CE & OpenMC-MG & $S_4$ & Diffusion & Hybrid \\
    \midrule
    Position, $\bm{r}$ & Continuous & Continuous & Discrete & Discrete & Discrete \\
    Direction of travel, $\bm{\hat{\Omega}}$ & Continuous & Continuous & Discrete & N/A & N/A \\
    Energy, $E$ & Continuous & Discrete & Discrete & Discrete & Discrete \\
    Angle-dependence in $\Sigma_s$ & Continuous & 1st-order Legendre & 1st-order Legendre
    & N/A & N/A \\
    \bottomrule
  \end{tabular}
  \label{table:var}
\end{table}

Based on a mesh convergence study on Case 2b, all three methods reach reasonable convergence
with a mesh size of $\Delta x=0.0125$ cm; further mesh refinement results in less than 0.00030
change in $k$.

\section{Results \& Discussion} \label{sec:prelim-results}

In this section, I will compare various neutronics parameters for all test cases calculated from
the OpenMC and deterministic methods. I refer to the OpenMC calculations under continuous-energy
and multigroup modes as OpenMC-CE and OpenMC-MG, respectively.

\subsection{Comparison of Multiplication factors, $k$}

Tables \ref{table:ck1} and \ref{table:ck2} show the $k$ estimates from the OpenMC and deterministic
methods for all test cases. The tables also include statistical uncertainties for the OpenMC
$k$ estimates. I did not apply the hybrid method for cases which do not contain control rod
regions. 

All five methods generally show loose agreement with one another within each test case. The
differences between the OpenMC-CE and OpenMC-MG $k$ estimates vary significantly from 0.00011 in
Case 1a to 0.01667 in Case 4b. These Monte Carlo methods exhibit greater differences in cases
containing reflector regions (e.g. 1b, 2b, 3b, 4b, 5a, 5b). The reflector region consists of an
equal volume mixture of steel and water. The hydrogen in water is a strong neutron moderator, and
it induces highly anisotropic scattering due to its small atomic mass. Therefore, the differences
imply that the neutron energy group discretization scheme and/or the Legendre expansion orders of
scattering cross sections are insufficient for accurately modeling neutron flux in the reflector
regions.

\begin{table}[htb!]
  \centering
  \footnotesize
  \caption{Multiplication factor $k$ estimates for Cases 1a, 1b, 2a, 2b, 3a, and 3b from the
    OpenMC-CE, OpenMC-MG, $S_4$ neutron transport, neutron diffusion, and Hybrid $S_N$-Diffusion
    methods.}
  \begin{tabular}{c S[table-format=1.7] S[table-format=1.7] S[table-format=1.7] S[table-format=1.7]
  S[table-format=1.7] S[table-format=1.7]}
    \toprule
    \multirow{2}{*}{\textbf{Method}} &
    \multicolumn{6}{c}{\textbf{Multiplication factor,} $\bm{k}$} \\
    \cmidrule{2-7}
    & {\textbf{Case 1a}} & {\textbf{Case 1b}} & {\textbf{Case 2a}} &
    {\textbf{Case 2b}} & {\textbf{Case 3a}} & {\textbf{Case 3b}} \\
    \midrule
    OpenMC-CE & 1.60033(56) & 1.16749(60) & 1.20666(66) & 0.69499(47) & 1.19908(63) & 0.68565(53)\\
    OpenMC-MG & 1.60044(33) & 1.18123(60) & 1.20675(62) & 0.70622(44) & 1.20043(48) & 0.69751(50)\\
    $S_4$     & 1.60036     & 1.18050     & 1.20474     & 0.70302     & 1.19844     & 0.69483    \\
    Diffusion & 1.60036     & 1.17940     & 1.19666     & 0.69210     & 1.19020     & 0.68386    \\
    Hybrid    & {N/A}       & {N/A}       & 1.20446     & 0.70156     & 1.19816     & 0.69336    \\
    \bottomrule
  \end{tabular}
  \label{table:ck1}
\end{table}
%
\begin{table}[htb!]
  \centering
  \footnotesize
  \caption{Multiplication factor $k$ estimates for Cases 4a, 4b, 5a, and 5b from the OpenMC-CE,
    OpenMC-MG, $S_4$ neutron transport, neutron diffusion, and Hybrid $S_N$-Diffusion methods.}
  \begin{tabular}{c S[table-format=1.7] S[table-format=1.7] S[table-format=1.7]
    S[table-format=1.7]}
    \toprule
    \multirow{2}{*}{\textbf{Method}} &
    \multicolumn{4}{c}{\textbf{Multiplication factor,} $\bm{k}$} \\
    \cmidrule{2-5}
    & {\textbf{Case 4a}} & {\textbf{Case 4b}} & {\textbf{Case 5a}} &
    {\textbf{Case 5b}} \\
    \midrule
    OpenMC-CE & 1.64506(53) & 1.19116(63) & 0.70352(64) & 1.16832(58) \\
    OpenMC-MG & 1.64434(33) & 1.20783(59) & 0.71513(55) & 1.18279(59) \\
    $S_4$     & 1.64365     & 1.20790     & 0.71280     & 1.18252     \\
    Diffusion & 1.64395     & 1.20819     & 0.70428     & 1.18274     \\
    Hybrid    & {N/A}       & {N/A}       & 0.71396     & {N/A}       \\
    \bottomrule
  \end{tabular}
  \label{table:ck2}
\end{table}
%
\begin{table}[htb!]
  \centering
  \footnotesize
  \caption{Differences in $k$ estimates for Cases 1a, 1b, 2a, 2b, 3a, and 3b for the $S_4$ neutron
    transport, neutron diffusion, and Hybrid $S_N$-Diffusion methods relative to OpenMC-MG.}
  \begin{tabular}{c S[table-format=1.7] S[table-format=1.7] S[table-format=1.7] S[table-format=1.7]
  S[table-format=1.7] S[table-format=1.7]}
    \toprule
    \multirow{2}{*}{\textbf{Method}} & \multicolumn{6}{c}{$\bm{k-k_{MG}}$} \\
    \cmidrule{2-7}
    & {\textbf{Case 1a}} & {\textbf{Case 1b}} & {\textbf{Case 2a}} &
    {\textbf{Case 2b}} & {\textbf{Case 3a}} & {\textbf{Case 3b}} \\
    \midrule
    $S_4$     & -0.00008 & -0.00073 & -0.00201 & -0.00320 & -0.00199 & -0.00268 \\
    Diffusion & -0.00008 & -0.00183 & -0.01009 & -0.01412 & -0.01023 & -0.01365 \\
    Hybrid    & {N/A}    & {N/A}    & -0.00229 & -0.00466 & -0.00227 & -0.00415 \\
    \bottomrule
  \end{tabular}
  \label{table:ckdiff1}
\end{table}
%
\begin{table}[htb!]
  \centering
  \footnotesize
  \caption{Differences in $k$ estimates for Cases 4a, 4b, 5a, and 5b for the $S_4$ neutron
    transport, neutron diffusion, and Hybrid $S_N$-Diffusion methods relative to OpenMC-MG.}
  \begin{tabular}{c S[table-format=1.7] S[table-format=1.7] S[table-format=1.7]
    S[table-format=1.7]}
    \toprule
    \multirow{2}{*}{\textbf{Method}} & \multicolumn{4}{c}{$\bm{k-k_{MG}}$} \\
    \cmidrule{2-5}
    & {\textbf{Case 4a}} & {\textbf{Case 4b}} & {\textbf{Case 5a}} &
    {\textbf{Case 5b}} \\
    \midrule
    $S_4$     & -0.00069 & +0.00007 & -0.00233 & -0.00027 \\
    Diffusion & -0.00039 & +0.00036 & -0.01085 & -0.00005 \\
    Hybrid    & {N/A}    & {N/A}    & -0.00117 & {N/A}    \\
    \bottomrule
  \end{tabular}
  \label{table:ckdiff2}
\end{table}
%
\begin{table}[tb!]
  \centering
  \footnotesize
  \caption{Control rod worths calculated as changes in the reactivity between Case 5a and 5b for
    the OpenMC continuous-energy and multigroup mode calculations, and the $S_4$ neutron transport,
  neutron diffusion, and Hybrid $S_N$-Diffusion methods.}
  \begin{tabular}{c S}
    \toprule
    \multirow{2}{*}{\textbf{Method}} & {\textbf{Control rod worth}} \\
                                     & {$\bm{\rho_{(5b)}-\rho_{(5a)}}$} \\
    \midrule
    OpenMC-CE & 0.56549 \\
    OpenMC-MG & 0.55289 \\
    $S_4$     & 0.55727 \\
    Diffusion & 0.57439 \\
    Hybrid    & 0.55515 \\
    \bottomrule
  \end{tabular}
  \label{table:rod-worth}
\end{table}

In contrast with OpenMC-CE, the $k$ estimates from the $S_4$ neutron transport method show closer
agreement with the estimates from OpenMC-MG. Tables \ref{table:ckdiff1} and \ref{table:ckdiff2}
shows the differences in $k$ between OpenMC-MG and the deterministic methods. The differences in
$k$ range from 0.00008 in Case 1a to 0.00320 in Case 2b. Given that the mesh is sufficiently fine
for the $S_4$ method, the only other significant difference between these methods is the
discretization of the direction of travel variable $\hat{\Omega}$. The flux distributions exhibit
more apparent differences as I will discuss in the next subsection. Nevertheless, the $S_4$
method is sufficiently accurate and consistent for assessing control rod neutronics modeling with
the Hybrid $S_N$-Diffusion method.

As expected, the neutron diffusion method performs significantly worse than the $S_4$ method in
Cases 2, 3, and 5a which all contain control rod regions. It deviates from the OpenMC-MG $k$
estimate by up to 0.01412 in Case 2b compared to 0.00320 for the $S_4$ method. The hybrid method
proves to be effective as it improves the $k$ estimate to almost match the $S_4$ method results for
the cases with control rods.

Finally, I calculated the control rod worth by taking the change in reactivity $\rho$ between Case
5a and 5b for all five methods as shown in Table \ref{table:rod-worth}. The formula for $\rho$ is
given as
%
\begin{align}
  \rho =& \frac{k-1}{k}.
\end{align}
%
Since I did not run the hybrid method for Case 5b, I took the $k$ estimate from the neutron
diffusion method as the reference non-rodded for the control rod worth calculation for the hybrid
method. Compared with OpenMC-MG, the control rod worth from the neutron diffusion method deviates
the most by 3.9\% as opposed to 0.8\% and 0.4\% by the $S_4$ and hybrid methods, respectively.

\subsection{Comparison of Neutron Flux Distributions, $\phi$}

I focus my discussion in this section on Cases 3b and 5a which represent the most complex
geometries with homogeneous and lattice fuel-graphite regions. I sampled the flux distributions in
OpenMC-CE and OpenMC-MG across coarser $0.1$-cm intervals to reduce statistical uncertainties of
each individual flux reading.

\begin{figure}[htb!]
  \centering
  \begin{subfigure}[t]{.49\textwidth}
    \centering
    \includegraphics[width=\textwidth]{case-3b-group-1-flux}
    \caption{Group 1}
    \label{fig:c3bg1}
  \end{subfigure}
  \hfill
  \begin{subfigure}[t]{.49\textwidth}
    \centering
    \includegraphics[width=\textwidth]{case-3b-group-2-flux}
    \caption{Group 2}
    \label{fig:c3bg2}
  \end{subfigure}
  \hfill
  \begin{subfigure}[t]{.49\textwidth}
    \centering
    \includegraphics[width=\textwidth]{case-3b-group-3-flux}
    \caption{Group 3}
    \label{fig:c3bg3}
  \end{subfigure}
  \hfill
  \begin{subfigure}[t]{.49\textwidth}
    \centering
    \includegraphics[width=\textwidth]{case-3b-group-4-flux}
    \caption{Group 4}
    \label{fig:c3bg4}
  \end{subfigure}
  \caption{Neutron group flux distributions for Case 3b from OpenMC-CE, OpenMC-MG, $S_4$, neutron
  diffusion, and hybrid methods.}
  \label{fig:c3bflux}
\end{figure}

Figure \ref{fig:c3bflux} shows the neutron flux distributions for groups 1 to 4. Notably, the
OpenMC-CE flux distribution is the most dissimilar to the rest of the distributions likely due to
the inadequate neutron energy group discretization scheme. Consequently, I compared the
deterministic methods with OpenMC-MG to eliminate energy discretization errors. Figure
\ref{fig:c3bfluxe} shows the relative differences in the flux distributions from $S_4$, neutron
diffusion, and hybrid methods with respect to OpenMC-MG. The neutron diffusion method
performs worse than $S_4$, especially near the control rod region between $x=0$ cm and $x=20$ cm
for the slower group 3 and 4 neutron flux. Although the control rod region spans between $x=0$ cm
and $x=0.5$ cm only, it induces strong flux gradients in its vicinity and thus renders the neutron
diffusion method less valid. The hybrid method brings about significant improvements to all four
group fluxes in this region. Beyond $x=40$ cm, both neutron diffusion and hybrid methods produce
similar flux error distributions due to the influence of the material interface between the
fuel-graphite and reflector regions.

\begin{figure}[htb!]
  \centering
  \begin{subfigure}[t]{.49\textwidth}
    \centering
    \includegraphics[width=\textwidth]{case-3b-group-1-flux-error}
    \caption{Group 1}
    \label{fig:c3bg1e}
  \end{subfigure}
  \hfill
  \begin{subfigure}[t]{.49\textwidth}
    \centering
    \includegraphics[width=\textwidth]{case-3b-group-2-flux-error}
    \caption{Group 2}
    \label{fig:c3bg2e}
  \end{subfigure}
  \hfill
  \begin{subfigure}[t]{.49\textwidth}
    \centering
    \includegraphics[width=\textwidth]{case-3b-group-3-flux-error}
    \caption{Group 3}
    \label{fig:c3bg3e}
  \end{subfigure}
  \hfill
  \begin{subfigure}[t]{.49\textwidth}
    \centering
    \includegraphics[width=\textwidth]{case-3b-group-4-flux-error}
    \caption{Group 4}
    \label{fig:c3bg4e}
  \end{subfigure}
  \caption{Relative differences of the neutron group flux distributions for Case 3b from the $S_4$,
    neutron diffusion, and hybrid methods with respect to OpenMC-MG.}
  \label{fig:c3bfluxe}
\end{figure}

For a quantitative assessment, I defined the normalized flux error $\epsilon$ for each energy group
with respect to the OpenMC-MG flux distribution using a normalized Frobenius/Euclidean norm formula
given as
%
\begin{align}
  \epsilon_g =& \frac{\lVert\sum^I_{i=1}\phi_{g,i} - \phi^{MG}_{g,i}\rVert_F}
  {\lVert\sum^I_{i=1}\phi^{MG}_{g,i}\rVert_F} =
  \frac{\left[\sum^I_{i=1}\lvert\phi_{g,i} - \phi^{MG}_{g,i}\rvert^2 \right]^{\sfrac{1}{2}}}
  {\left[\sum^I_{i=1}\lvert\phi^{MG}_{g,i}\rvert^2\right]^{\sfrac{1}{2}}}
\end{align}
%
I integrated on the fine flux distributions over 0.1-cm intervals from $S_4$, neutron diffusion,
and hybrid methods to match the 0.1-cm intervals of the OpenMC-MG flux distribution.
Table \ref{table:c3berror} shows the $\epsilon$ values for the deterministic methods. As expected,
the hybrid method produces smaller flux error norms than the neutron diffusion method. The
improvements are more pronounced for group 3 and 4 fluxes.
%
\begin{table}[tb!]
  \centering
  \footnotesize
  \caption{Normalized flux error $\epsilon$ for Case 3b from the $S_4$, neutron diffusion, and
    hybrid methods with respect to OpenMC-MG.}
  \begin{tabular}{c S S S S}
    \toprule
    {\multirow{2}{*}{\textbf{Method}}} &
    \multicolumn{4}{c}{\textbf{Normalized flux error,} $\bm{\epsilon_g}$} \\
    \cmidrule{2-5}
    & {\textbf{Group 1}} & {\textbf{Group 2}} & {\textbf{Group 3}} &
    {\textbf{Group 4}} \\
    \midrule
    $S_4$     & 0.0070 & 0.0080 & 0.0092 & 0.0082 \\
    Diffusion & 0.0141 & 0.0172 & 0.0275 & 0.0272 \\
    Hybrid    & 0.0116 & 0.0128 & 0.0136 & 0.0107 \\
    \bottomrule
  \end{tabular}
  \label{table:c3berror}
\end{table}

Case 5a introduces significant geometrical heterogeneity from the fuel-graphite lattice region.
This is very apparent from the flux distributions in Figure \ref{fig:c5aflux}. The OpenMC Monte
Carlo methods captured the fluctuations in the flux with their continuous angle variable as opposed
to the $S_4$, neutron diffusion, and hybrid methods. The group 1 flux peaks are notably pronounced
since all fission neutrons are born in the fuel region with most of them in group 1. Figure
\ref{fig:c5afluxe} shows the relative differences in flux for the $S_4$, neutron diffusion, and
hybrid methods relative to OpenMC-MG. All three methods show regular fluctuations in the relative
flux differences corresponding to the fuel-graphite lattice. The $S_4$ method generally deviates
the least from OpenMC-MG while the neutron diffusion method exhibits characteristically
significant flux deviations near the control rod region prior to $x=20$ cm and the material
interface between the fuel and reflector regions beyond $x=40$ cm.

\begin{figure}[htb!]
  \centering
  \begin{subfigure}[t]{.49\textwidth}
    \centering
    \includegraphics[width=\textwidth]{case-5a-group-1-flux}
    \caption{Group 1}
    \label{fig:c5ag1}
  \end{subfigure}
  \hfill
  \begin{subfigure}[t]{.49\textwidth}
    \centering
    \includegraphics[width=\textwidth]{case-5a-group-2-flux}
    \caption{Group 2}
    \label{fig:c5ag2}
  \end{subfigure}
  \hfill
  \begin{subfigure}[t]{.49\textwidth}
    \centering
    \includegraphics[width=\textwidth]{case-5a-group-3-flux}
    \caption{Group 3}
    \label{fig:c5ag3}
  \end{subfigure}
  \hfill
  \begin{subfigure}[t]{.49\textwidth}
    \centering
    \includegraphics[width=\textwidth]{case-5a-group-4-flux}
    \caption{Group 4}
    \label{fig:c5ag4}
  \end{subfigure}
  \caption{Neutron group flux distributions for Case 5a from OpenMC-CE, OpenMC-MG, $S_4$, neutron
  diffusion, and hybrid methods.}
  \label{fig:c5aflux}
\end{figure}
%
\begin{figure}[htb!]
  \centering
  \begin{subfigure}[t]{.49\textwidth}
    \centering
    \includegraphics[width=\textwidth]{case-5a-group-1-flux-error}
    \caption{Group 1}
    \label{fig:c5ag1e}
  \end{subfigure}
  \hfill
  \begin{subfigure}[t]{.49\textwidth}
    \centering
    \includegraphics[width=\textwidth]{case-5a-group-2-flux-error}
    \caption{Group 2}
    \label{fig:c5ag2e}
  \end{subfigure}
  \hfill
  \begin{subfigure}[t]{.49\textwidth}
    \centering
    \includegraphics[width=\textwidth]{case-5a-group-3-flux-error}
    \caption{Group 3}
    \label{fig:c5ag3e}
  \end{subfigure}
  \hfill
  \begin{subfigure}[t]{.49\textwidth}
    \centering
    \includegraphics[width=\textwidth]{case-5a-group-4-flux-error}
    \caption{Group 4}
    \label{fig:c5ag4e}
  \end{subfigure}
  \caption{Relative differences of the neutron group flux distributions for Case 5a from the $S_4$,
    neutron diffusion, and hybrid methods with respect to OpenMC-MG.}
  \label{fig:c5afluxe}
\end{figure}
%
\begin{table}[tb!]
  \centering
  \footnotesize
  \caption{Normalized flux error $\epsilon$ for Case 5a from the $S_4$, neutron diffusion, and
    hybrid methods with respect to OpenMC-MG.}
  \begin{tabular}{c S S S S}
    \toprule
    {\multirow{2}{*}{\textbf{Method}}} &
    \multicolumn{4}{c}{\textbf{Normalized flux error,} $\bm{\epsilon_g}$} \\
    \cmidrule{2-5}
    & {\textbf{Group 1}} & {\textbf{Group 2}} & {\textbf{Group 3}} &
    {\textbf{Group 4}} \\
    \midrule
    $S_4$     & 0.0149 & 0.0079 & 0.0127 & 0.0187 \\
    Diffusion & 0.0300 & 0.0197 & 0.0336 & 0.0445 \\
    Hybrid    & 0.0283 & 0.0145 & 0.0188 & 0.0285 \\
    \bottomrule
  \end{tabular}
  \label{table:c5aerror}
\end{table}



\section{Summary}

The Hybrid $S_N$-Diffusion method is promising for studying reactor systems such as \glspl{MSR},
which consist of mostly highly scattering regions (small neutron mean free path), because the $S_N$
subproblem domain $\Omega^d_1$ covers a small fraction of the entire reactor geometry. This
reduces the computational costs of the iterative $S_N$ calculations in the hybrid method.

I plan to perform further investigations to replace the approximate energy group discretization
used here and to use higher-order Legendre expansions of the scattering cross sections.
