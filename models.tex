Chapter \ref{chap:lit} briefly introduced unique characteristics of liquid-fuel
\glspl{MSR}
and highlighted the challenges of modeling liquid-fuel \glspl{MSR} particularly
with older software designed for solid-fuel reactors. In addition, Chapter
\ref{chap:lit} discussed existing \gls{MSR} multiphysics simulation tools and
their capabilities and the noted lack of capabilities for modeling control rod
movement in \glspl{MSR}. Chapter \ref{chap:moltres} illustrated general
features and physics models in Moltres and summarized previous work done with
Moltres for multiphysics modeling of \glspl{MSR}. The latter section outlined
several outstanding limitations in \gls{MSR} multiphysics modeling with
Moltres including but not limited to the need for rigorous verification of
Moltres' capabilities, a turbulence model for simulating turbulent flow, and a
control rod modeling capability for transient simulations involving control rod
movement. In turn, Chapter \ref{chap:benchmark} verified the fast-spectrum
\gls{MSR} modeling capabilities in Moltres by evaluating its performance
through the CNRS benchmark and comparing its results against results from other
similar \gls{MSR} simulation tools.

This chapter proposes further enhancements to Moltres to address the
need for a turbulence model and control rod model in Moltres. The proposed work
contributes towards the overall goal of improving on Moltres' capabilities for
multiphysics modeling of \glspl{MSR} and, by extension, of advanced reactors.
The turbulence model will allow for improved simulations of turbulent flow
commonly observed in \glspl{MSR} during normal operation and transient accident
scenarioes. The following sections outline the methods and simulations I will
employ to implement and verify the new capabilities.

\section{Turbulence Model Implementation}

I will implement the $k$-$\tau$ turbulence model which falls under the class of
\gls{RANS}-based turbulence models discussed in Chapter \ref{chap:lit}. The
$k$-$\tau$ model will be implemented within the \gls{MOOSE} framework and
designed to be compatible with the fluid dynamics modeling infrastructure in
the existing \texttt{Navier-Stokes} module. This approach leverages on the
advanced finite-element solver and multiphysics coupling capabilities in
\gls{MOOSE}.

A comprehensive validation of the $k$-$\tau$ model under various types of
turbulent flow falls outside the scope of the proposed work. Instead, I will
focus on validating its performance under specific flow conditions expected in
liquid-fuel \glspl{MSR}---wall-bounded turbulent flow with flow separation past
sharp changes in the flow channel geometry. The backward facing step is a
widely known fluid dynamics problem commonly used to assess the accuracy of
turbulence model solvers \cite{lasher_computation_1992}. As shown in Figure *,
the problem domain features a straight duct
on the left followed by a sudden back step in the lower wall which causes flow
separation. The flow eventually reattaches and recovers downstream of the step.
I will simulate and validate against experimental data from Driver \&
Seegmiller \cite{driver_features_1985}. Quantities of interest from this
exercise are the spatial distributions of the velocity components, turbulent
kinetic energy, and eddy viscosity, and the reattachment length of the
turbulent shear layer.

\section{Control Rod Model Implementation}

\section{Conclusion}
