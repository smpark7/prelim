Compared with solid-fueled reactors, liquid-fueled \glspl{MSR} feature additional multiphysics
phenomena such as flow-induced DNP drift and temperature advection-diffusion in the molten fuel
salt. Modeling coupled neutronics and thermal–hydraulics in
liquid-fueled \glspl{MSR} requires robust and flexible multiphysics software for accurate
simulations at reasonable computational costs. Moltres is a multiphysics reactor software that
is built on the \gls{MOOSE} framework and is designed
to meet that goal. In my preliminary work, I completed a verification study of Moltres based on a
numerical benchmark developed at \gls{CNRS} for the verification of \gls{MSR} simulation tools
designed for fast-spectrum \gls{MSR} modeling. I propose running an additional verification and
validation study based on pump-initiated experiments performed on the \gls{MSRE} in the 1960s.
This study will be performed in collaboration with the developer of another \gls{MSR} simulation
tool, QuasiMolto, for code-to-code verification. Beyond the verification and validation of Moltres’
existing capabilities, I propose implementing a turbulent flow modeling
capability in Moltres. Various \gls{MSR} designs exhibit turbulent flow in high Reynolds number
flow regions. Instead of high-fidelity turbulence models whose high computational costs are at odds
with the intermediate-fidelity simulations with Moltres, I propose implementing the
Spalart-Allmaras turbulence model for acceptably accurate estimates of turbulence and its effects
on temperature and \gls{DNP} advection.
Lastly, I aim to address a gap in the existing literature for multiphysics simulations of MSRs with
control rods. While many time-independent neutronics studies of MSRs
include control rod elements in their models, most multiphysics studies omit control rod
elements and instead rely on neutron source scaling to model the effects of control rod insertion
and withdrawal. For the few multiphysics studies that retain control
rod elements in their models, they introduce modeling discrepancies such as material homogenization
or geometrical changes. They do not demonstrate time-dependent control rod insertion/withdrawal
modeling capabilities. Therefore, I propose to develop a novel hybrid neutronics method to improve
neutron diffusion calculations within and near highly absorbing regions. The hybrid method explores
using discrete ordinates ($S_N$) neutron transport sub-solvers to generate spatially-dependent
diffusion coefficients within/near highly absorbing regions for the global neutron diffusion
solver. In this preliminary report, I present several 1D test cases to demonstrate the
implementation and performance of the hybrid method.
