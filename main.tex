%% Package and Class "uiucthesis2021" for use with LaTeX2e.
\documentclass[edeposit,fullpage,12pt]{uiucthesis2021}
\doublespacing

\usepackage[acronym,toc]{glossaries}
\makeglossaries
\include{acros}

\usepackage{xspace}
\usepackage{graphicx}
\graphicspath{{images/}}
\usepackage[title,titletoc]{appendix}

\usepackage{placeins}
\usepackage{booktabs} % nice rules (thick lines) for tables
\usepackage{array}
\usepackage{microtype} % improves typography for PDF

\usepackage[hyphens]{url}
\usepackage[hidelinks]{hyperref}
\usepackage{caption}
\usepackage{subcaption}
\usepackage{hhline}
\usepackage{amsmath}
\usepackage{amssymb}
\usepackage{mathtools}
\allowdisplaybreaks
\usepackage{color}
\usepackage{multirow}
\usepackage{siunitx}
\usepackage{xfrac}
\usepackage{bm}

\usepackage{threeparttable, tablefootnote}

\usepackage{environ}
\makeatletter

\usepackage{tabularx}
\usepackage{float}
\usepackage{enumitem}
\usepackage{diagbox}
\usepackage{courier}
\usepackage{pdflscape}

\usepackage{cleveref}
\usepackage{datatool}
\usepackage[numbers]{natbib}
\usepackage{notoccite}

\usepackage{tikz}
\usepackage{tkz-euclide}
\usetikzlibrary{positioning, arrows, decorations, shapes}
\usetikzlibrary{shapes.geometric,arrows}
\definecolor{illiniblue}{HTML}{B1C6E2}
\definecolor{illiniorange}{HTML}{f8c2a2}
\definecolor{green}{HTML}{c2e2b1}
\tikzstyle{process} = [rectangle, rounded corners, thick, minimum width=7cm, minimum height=1cm, text centered, draw=black, fill=illiniblue, text width=18em]
\tikzstyle{object} = [ellipse, minimum width=2cm, thick, minimum height=2.2cm, text centered, draw=black, fill=illiniorange, text width=12em]
\tikzstyle{decision} = [diamond, thick, aspect=2, minimum width=2cm, minimum height=2cm, text centered, draw=black, fill=green, text width=6em]
\tikzstyle{arrow} = [thick,->,>=stealth]

\usepackage[ruled,linesnumbered,lined]{algorithm2e}
\SetArgSty{textrm}


\title{Advancement of Moltres for Multiphysics Molten Salt Reactor Modeling}
\author{Sun Myung Park}
\department{Nuclear, Plasma \& Radiological Engineering}
\prelim
\degreeyear{2023}
\committee{Dr Madicken Munk, Chair \\
           Professor Rizwan Uddin \\
           Associate Professor Tomasz Kozlowski \\
           Professor Paul Fischer}


\begin{document}
\maketitle

\frontmatter
%% Create an abstract that can also be used for the ProQuest abstract.
%% Note that ProQuest truncates their abstracts at 350 words.
\begin{abstract}

Compared with solid-fueled reactors, liquid-fueled \glspl{MSR} feature additional multiphysics
phenomena such as flow-induced DNP drift and temperature advection-diffusion in the molten fuel
salt. Modeling strongly coupled neutronics and thermal–hydraulics in
liquid-fueled \glspl{MSR} requires robust and flexible multiphysics software for accurate
simulations at reasonable computational costs. Moltres is a multiphysics reactor software designed
to meet that goal. In my preliminary work, I completed a verification study of Moltres based on a
numerical benchmark developed at \gls{CNRS} for the verification of \gls{MSR} simulation tools
designed for fast-spectrum \gls{MSR} modeling. I propose running an additional verification and
validation study based on pump-initiated experiments performed on the \gls{MSRE} in the 1960s.
This study will be performed in collaboration with the developer of another \gls{MSR} simulation
tool, QuasiMolto, for code-to-code verification. Beyond the verification and validation of Moltres’
existing capabilities, I propose implementing a turbulent flow modeling
capability in Moltres. Various \gls{MSR} designs exhibit turbulent flow in high Reynolds number
flow regions. Instead of high-fidelity turbulence models whose high computational costs are at odds
with the intermediate-fidelity simulations with Moltres, I propose implementing the
Spalart-Allmaras turbulence model for acceptably accurate estimates of turbulence and its effects
on temperature and \gls{DNP} advection.
Lastly, I aim to address a gap in the existing literature for multiphysics simulations of MSRs with
control rods. While many time-independent neutronics studies of MSRs
include control rod elements in their models, most multiphysics studies omit control rod
elements and instead rely on neutron source scaling to model the effects of control rod insertion
and withdrawal. For the few multiphysics studies that retain control
rod elements in their models, they introduce modeling discrepancies such as material homogenization
or geometrical changes. They do not demonstrate time-dependent control rod insertion/withdrawal
modeling capabilities. Therefore, I propose to develop a novel hybrid neutronics method to improve
neutron diffusion calculations within and near highly absorbing regions. The hybrid method explores
using discrete ordinates ($S_N$) neutron transport sub-solvers to generate spatially-dependent
diffusion coefficients within/near highly absorbing regions for the global neutron diffusion
solver. In this preliminary report, I present several 1D test cases to demonstrate the
implementation and performance of the hybrid method.


\end{abstract}
\addcontentsline{toc}{chapter}{Abstract}

%\begin{dedication}
%
%\end{dedication}

%\begin{acknowledgments}
%
%%\input{acks}
%
%\end{acknowledgments}
%\addcontentsline{toc}{chapter}{Acknowledgments}

%% The thesis format requires the Table of Contents to come
%% before any other major sections, all of these sections after
%% the Table of Contents must be listed therein (i.e., use \chapter,
%% not \chapter*).  Common sections to have between the Table of
%% Contents and the main text are:
%%
%% List of Tables
%% List of Figures
%% List Symbols and/or Abbreviations
%% etc.

{
  \hypersetup{linkcolor=black}
  \tableofcontents
}

%% Create a List of Abbreviations. The left column
%% is 1 inch wide and left-justified
\printglossary[title=List of Abbreviations,type=\acronymtype,nonumberlist,
nogroupskip=true]
%% Create a List of Symbols. The left column
%% is 0.7 inch wide and centered

\pagebreak
\mainmatter
\glsresetall

\chapter{Introduction}
\label{chap:intro}
\section{Motivation}

A liquid-fueled \gls{MSR} is a class of advanced nuclear reactor which has fissile material
dissolved in a molten salt mixture. This fuel salt mixture doubles as the primary reactor coolant
for transferring heat from the reactor core to the primary heat exchangers. Figure \ref{fig:msr}
shows a schematic diagram of a thermal-spectrum, channel-type \gls{MSR}. In channel-type
\glspl{MSR}, the fuel salt flows through vertical channels in the reactor core next to neutron
moderators (e.g., graphite, heavy water, molten sodium hydroxide). 
%
\begin{figure}[htb!]
	\centering
	\includegraphics[width=.7\columnwidth]{msr}
	\caption{Schematic diagram of the \gls{MSR} concept. Retrieved from
	\cite{doe_technology_2002}.}
	\label{fig:msr}
\end{figure}

The \gls{MSR} is one of six advanced reactor designs selected at the \gls{GIF} for improved safety,
sustainability, efficiency, and cost over the current generation of predominantly \glspl{LWR}.
Due to the high thermal expansion coefficient, \glspl{MSR} possess an inherently robust
safety feature in the strong negative fuel temperature coefficient of
reactivity \cite{elsheikh_safety_2013}. This reactivity coefficient limits the
maximum temperature that the reactor core would experience in an accident
scenario such as an unprotected reactivity insertion because the subsequent
rise in core temperatures induces a significant drop in reactivity which
quickly neutralizes the initial reactivity insertion. \glspl{MSR} also
operate at a large thermal margin to boiling and can rely on natural
circulation in the event of a pump failure. As a last resort, many \gls{MSR}
designs incorporate a drain plug consisting of actively-cooled frozen salt, which
melts when the core temperatures exceed safety thresholds. The hot molten salt
in the core would then flow down into a drain tank designed to hold the fuel salt in
a subcritical configuration to disrupt any further chain fission reactions.

Some \glspl{MSR} like the \gls{MSBR} or the \gls{MSFR} can
incorporate the thorium fuel cycle for improved sustainability arising from the
use of abundant natural thorium resources and reduced transuranic waste
\cite{heuer_towards_2014}. The latter consequence also reduces costs
associated with long-term nuclear waste storage. In addition, the ability to
operate near atmospheric pressures eliminates the need for a thick pressure
vessel and drives down construction costs, while online fuel reprocessing
reduces reactor downtime during reactor operation \cite{dolan_1_2017}.
We can make further economic arguments supporting \glspl{MSR} in the context of the
carbon-constrained future envisioned in \gls{IEA}'s \gls{NZE} roadmap
\cite{iea_net_2021}. The roadmap for minimizing carbon emissions requires solar \gls{PV}- and
wind-dominated energy markets, which can bring about highly variable electricity generation. The
resulting volatility in electricity prices
encourages the construction of heat storage and peak power
production plants. At the same time, demand for carbon-neutral
fuel will rise as electrification is economically unfeasible
for some industries such as the aviation and marine sectors, which depend on
energy-dense fuels for propulsion. As described by Forsberg
\cite{forsberg_market_2020}, the most cost-efficient options for the
aforementioned resources (heat storage, peak power
production, and hydrogen fuel) all require high-temperature heat.
This requirement favors \glspl{MSR} which are expected to deliver heat at higher average
temperatures than \glspl{LWR} and other high-temperature advanced reactors.
\glspl{MSR} may also possess an edge over other high-temperature advanced reactors from
synergistic benefits of siting molten salt heat storage facilities near the power generation
facility.

\subsection{Past \& Present \gls{MSR} Research \& Development}

\gls{ORNL} researchers first conceived the \gls{MSR} concept in pursuit of a high-temperature
liquid fuel reactor for the US Aircraft Nuclear Propulsion program in
the 1950s \cite{rosenthal_molten-salt_1970}. They
built the first ever operational \gls{MSR}, the 2.5 MW$_{\text{th}}$
\gls{ARE} reactor at \gls{ORNL}. The successful demonstration of the \gls{ARE} spurred further
research into adapting \glspl{MSR} for civilian power generation
\cite{rosenthal_molten-salt_1970}. Continued \gls{MSR} research efforts culminated in the design,
construction, and successful operation of the 8-MW$_{\text{th}}$, thermal-spectrum \gls{MSRE} with
graphite channels and a LiF-BeF$_2$-ZrF$_4$-UF$_4$ fuel salt mixture
\cite{haubenreich_experience_1970}. In addition to other operational achievements, the
\gls{MSRE} became the first reactor to run on $^{233}$U fuel bred from $^{232}$Th. Building on
their experience with the \gls{MSRE}, \gls{ORNL} proposed a new program for the construction and
opeartion of a demonstration reactor based on the \gls{MSBR} concept that they had developed
\cite{macpherson_molten_1985}. The \gls{MSBR} is a thermal-spectrum \gls{MSR} with fertile
$^{232}$Th isotopes mixed directly into the \gls{FLiBe} molten salt for $^{233}$U breeding. Like
the \gls{MSRE}, the \gls{MSBR} was to rely on continuous online salt reprocessing to add fertile
material and remove fission product neutron poisons.
However, the \gls{MSBR} project was canceled prior to the demonstration stage in
favor of the \gls{LMFBR}, which had benefited from a head start in development and stronger
political backing \cite{macpherson_molten_1985}.

Following a relative lull lasting until the late 1990s, renewed research efforts and the \gls{GIF}
provided new impetus for \gls{MSR} research and development. As of the end of 2022, numerous
\gls{MSR} designs exist at various stages of development. Leading \gls{MSR} designs, in terms of
development, licensing, and/or demonstration, include the \gls{MSFR} \cite{merle_optimized_2007},
\gls{MCFR} \cite{terrapower_terrapower_2021}, TMSR-LF1 \cite{zhang_review_2018}, and \gls{IMSR}
\cite{leblanc_18_2017}. The \gls{MSFR} is a fast-spectrum breeder reactor developed through
collaborative efforts from European institutes with funding support from the
European Union. Figure \ref{fig:msfr} shows a schematic diagram of the \gls{MSFR}. As opposed to
the multi-channel design of the \gls{MSRE} and \gls{MSBR}, the
\gls{MSFR} core consists of a large molten salt pool without graphite moderators to avoid
frequent graphite replacements and positive graphite temperature reactivity feedback. The
\gls{MCFR} is a similar pool-type reactor under active development at TerraPower. TerraPower and
Southern Company embarked on a joint project to design, construct, and operate a prototype
\gls{MCRE} design with funding support from \gls{DOE}'s \gls{ARDP}. \gls{CAS} launched the
\gls{TMSR} program in 2011 to develop and construct both solid-fueled and liquid-fueled \gls{TMSR}
designs \cite{zou_research_2019}. They finished construction of TMSR-LF1, a 2-MW$_{\text{th}}$
liquid-fueled prototype, in August 2021 and received approval for reactor commissioning in August
2022. Lastly, Canada-based Terrestrial Energy is also developing their \gls{IMSR}, a small modular
\gls{MSR} based on the \gls{MSRE}, which passed a joint technical review carried out by the
Canadian and US nuclear regulators in July 2022.
%
\begin{figure}[htb!]
	\centering
	\includegraphics[width=.7\columnwidth]{msfr}
	\caption{Schematic diagram of the \gls{MSFR}. Retrieved from 
	\cite{allibert_7_2016}.}
	\label{fig:msfr}
\end{figure}

\subsection{\gls{MSR} Modeling \& Simulation}

With the renewed global interest in \glspl{MSR}, \gls{MSR} modeling software play an important role
in supporting \gls{MSR} development.
Accurate reactor modeling capabilities are important because they
accelerate reactor design and optimization by enabling quicker iteration through numerous design
changes. \gls{MSR} modeling software are also essential tools in reactor safety analysis and
licensing efforts as reactor developers must demonstrate and verify the \gls{MSR} systems perform
as designed and remain safe under various accident scenarios.

While modeling \glspl{MSR} is not necessarily more difficult than modeling
solid-fueled reactors, we must adapt our software tools to accurately model the
unique phenomena found in these circulating-fuel reactors. The differences in
the challenges of simulating \glspl{MSR} compared to solid-fueled reactors stem
mainly from the liquid fuel form of the fuel salt \cite{diamond_phenomena_2018,
huff_identifying_2019}.

Liquids generally exhibit greater thermal
expansion per unit change in temperature than solids. A decrease in density of
the fuel medium increases the likelihood of neutrons escaping the fuel region
and being absorbed by non-fissile material elsewhere in the reactor.
Consequently, combined with the temperature-dependent Doppler broadening of
resonance capture cross sections, \glspl{MSR} possess stronger negative fuel
temperature reactivity feedback than their solid-fueled counterparts
\cite{elsheikh_safety_2013}. These
phenomena ultimately result in strong interactions between the neutron fluxes
and core temperatures given that neutron fluxes affect core temperatures
through fission heat generation and core temperatures in turn affect neutron
fluxes through the mechanisms as described prior.

With the fuel
salt also serving the role of providing cooling in the core, velocity flow
profiles in the fuel salts strongly impact the temperature distribution via
advection-dominated heat transfer \cite{diamond_phenomena_2018}. This contrasts
with the relatively static temperature profiles in fuel pins and
other forms of solid fuel matrixes, which are physically separate from the coolant.

\glspl{DNP} flow freely within the primary coolant loop as opposed to
being held in place as in solid-fueled reactors. Thus, the delayed neutron
source distribution varies significantly depending on the flow profile and
velocity. In addition, the reactor loses some delayed neutrons from out-of-core
\gls{DNP} decay. These delayed neutrons are considered lost as they're emitted
in subcritical regions and are unlikely to contribute to further fission
reactions in the active core. The reduced delayed neutron fraction in the core
contributes to a greater prompt power spike following a reactivity insertion
event compared to solid-fueled reactors, absent any temperature reactivity
feedback.

Molten-salt flow along various parts of the coolant loop may fall within the turbulent flow
regime, characterized by chaotic eddies, vortices, and other flow instabilities.
Turbulent flow effects further complicate multiphysics interactions of flow with the temperature
and \gls{DNP} distribution. Turbulent flow effects contribute significantly to advection-dominated
heat and particle transfer in molten salt systems, thereby causing enhanced mixing. Therefore,
multiphysics software for \gls{MSR} analysis require adequate flow modeling capabilities with
support for \gls{DNP} drift and some form of turbulence modeling.

\subsection{Moltres for Multiphysics \gls{MSR} Analysis}

Moltres \cite{lindsay_moltres_2017} is an open-source multiphysics reactor simulation software
developed specifically with the considerations for \gls{MSR} characteristics in mind. Moltres is
built on the \gls{MOOSE} \cite{permann_moose_2020} open-source finite-element framework,
which facilitates multiphysics coupling between different
\gls{MOOSE}-based and \gls{MOOSE}-wrapped applications. The \gls{MOOSE} framework also provides
Moltres with advanced meshing and numerical solver capabilities through interfacing with libMesh
\cite{kirk_libmesh_2006} and PETSc \cite{satish_petsc_2019} open-source libraries. Therefore,
Moltres supports up to 3-D unstructured meshes, scales well on high-performance computing systems,
and provides a flexible multiphysics coupling system, which can be tailored for the type of
system being analyzed.

Moltres models coupled neutronics and thermal-hydraulics in reactors. While
generally applicable to most reactor concepts, much of
Moltres' development focuses on meeting the needs of \gls{MSR} multiphysics simulations.
Together with \gls{MOOSE}'s \texttt{Heat}
\texttt{Conduction} and \texttt{Navier-Stokes} \cite{peterson_overview_2018}
modules, Moltres solves the multigroup neutron diffusion
equations, for an arbitrary number of energy and precursor groups, and
thermal-hydraulics equations simultaneously on the same mesh (or separately solved and coupled
through fixed-point iterations if desired).

Lindsay et al. \cite{lindsay_introduction_2018}
demonstrated Moltres' multiphysics \gls{MSR} modeling capabilities with 1D salt
flow in 2D-axisymmetric and 3D models of the \gls{MSRE}. The neutron flux and
temperature distributions agreed qualitatively with legacy
\gls{MSRE} data albeit with some minor quantitative discrepancies due to
simplifications and assumptions in the reactor geometry. I demonstrated Moltres' capabilities for
1) looping of \gls{DNP} drift back into the reactor core, 2) coupling the \gls{DNP}
drift to numerically calculated salt flow profiles within the reactor core,
and 3) a decay heat model to simulate decay heat from fission products, with a 2-D axisymmetric
design of the \gls{MSFR} in my Master's thesis \cite{park_advancement_2020}.

While the Moltres
model of the \gls{MSFR} showed good agreement with other studies in most steady-state and transient
simulation cases, the Moltres model showed significant discrepancies during pump-initiated
transient scenarios in the absence of a proper turbulence model. Instead, the model applied a
uniform eddy viscosity assumption, which proved to be inadequate under non-steady flow. In order to
advance Moltres as a multiphysics simulation software for \gls{MSR} analysis, Moltres requires a
turbulence modeling capability to capture adequate turbulent flow phenomena and its interactions
with other physics present in \glspl{MSR}.

\subsection{\gls{VV} of Multiphysics \gls{MSR} Models and Software}

\gls{VV} of simulation software and models are important steps in simulation software development
\cite{sargent_verification_2010}. Verification is the process of checking whether a software
and its implementation accurately represents the conceptual description and specifications.
Validation is the process of checking whether a model is an accurate representations of the real
world within the range of its intended uses. For reactor software, verification is commonly
performed by comparison with other reactor software designed to run the same type of reactor
simulations. On the other hand, validation is performed by comparing numerical results from
a simulation model to experimental data from the corresponding live test. The validity of a model,
and the results derived from it, depends on the outcome of both verification and validation.

The important multiphysics phenomena in \glspl{MSR} for \gls{VV} are salt flow-induced
\gls{DNP} drift and the strong coupling between neutron flux and temperature advection-diffusion.
Notable \gls{VV} studies on \gls{MSR} modeling include work by Delpech et al.
\cite{delpech_benchmark_2003}, Tiberga et al. \cite{tiberga_results_2020}, Fratoni et al.
\cite{fratoni_molten_2020}, and Brovchenko et al. \cite{brovchenko_neutronic_2019} in relation to
the \gls{MSRE} and the \gls{MSFR} designs.



Delpech et al. \cite{delpech_benchmark_2003}
published one of the first modern \gls{VV} work for \gls{MSR} multiphysics modeling. Collaborators
from six institutions modeled \gls{MSRE} pump start-up, pump coast-down, and natural
circulation transients to assess and validate their models and codes for studying the effects of
salt flow on the reactivity and power. Given the wide range of neutronics methods from
multidimensional Monte Carlo methods to zero-dimensional point reactor kinetics, some deviations
were observed between different codes. Neverthless, all results showed generally good agreement
with \gls{MSRE} experimental data from \gls{ORNL}.

As mentioned in Subsection \ref{sec:msr-tools}, Tiberga et al. \cite{tiberga_results_2020}
published the CNRS benchmark for the verification of \gls{MSR} simulation tools designed for
fast-spectrum \gls{MSR} modeling. In contrast with the multi-channel \gls{MSRE} and its derivative
designs, the CNRS benchmark has a 2 m$\times$2 m problem domain of homogeneous fuel salt mimicking
the large salt pool in fast-spectrum \gls{MSR} designs. The CNRS benchmark consists of three phases
starting with single-physics calculations in Phase 0, followed by problems which gradually
introduce multiphysics coupling in Phase 1, and lastly time-dependent pertubation problems in Phase
2. Thus, the benchmark provides a systematic approach aimed towards helping code developers
identify sources of discrepancies which may otherwise be masked by error cancellation or other
dominant sources of discrepancies. The final steady-state and time-dependent problems involve
studying the effects of natural circulation and lid-driven flow on the reactivity and power output.
Aside from the problem specifications, Tiberga et al. also published the associated group
constant cross section data required by deterministic neutronics solvers to perform neutronics
calculations. Four institutions participated in the benchmarking exercise with neutron diffusion,
$SP_N$, and $S_N$-based solvers for the neutronics calculations. Their measured neutronics and
\gls{TH} parameters showed excellent agreement within up to 2.5\% discrepancy from their combined
average.

Neutronic benchmark studies of the \gls{MSRE} and the \gls{MSFR} by Fratoni et al.
\cite{fratoni_molten_2020} and Brovchenko et al. \cite{brovchenko_neutronic_2019} measured the
delayed neutron losses due to the decay of \glspl{DNP} flowing out of the active core region.
Fratoni et al. sought to establish a standard validation platform for \gls{MSR} neutronics
simulation tools with \gls{MSRE} experimental data for inclusion in the \gls{IRPhEP} handbook.
They characterized and validated a model of the \gls{MSRE} in the Monte Carlo particle transport
code Serpent. On the other hand, the \gls{MSFR} benchmark by Brovchenko et al. featured results
from multiple \gls{MSR} simulation tools by several collaborators. Their assessment found that
the choice of nuclear database for the cross sections and decay data has the most significant
impact on the neutronics results.

While these publications have plugged significant technical gaps, more can be done to develop
open \gls{VV} procedures for \gls{MSR} multiphysics modeling. For instance, the
CNRS benchmark does not assess the loss of delayed neutrons due to the decay of \glspl{DNP} flowing
out of the active core region. This phenomenon is important as the delayed neutron fraction in the
core directly impacts the transient power response in unprotected accident scenarios. Meanwhile,
the neutronic benchmark studies by Fratoni et al. \cite{fratoni_molten_2020} and Brovchenko et al.
\cite{brovchenko_neutronic_2019} did not provide standardized group constant data required by most
deterministic multiphysics \gls{MSR} simulation tools. Therefore, it is difficult to isolate
discrepancies arising from code implementations as opposed to discrepancies from using different
nuclear databases or stochastic uncertainties in Monte Carlo simulations. A good model verification
procedure for delayed neutron loss should ideally provide well-defined problems and the necessary
input data. It is also helpful to perform model verification studies on simpler problems like the
bare homogeneous problem domain in the CNRS benchmark before embarking on more complicated
validation studies which require accurate models of the reference experiments.

\section{Objectives}

This thesis demonstrates latest capabilities of Moltres
\cite{lindsay_introduction_2018}.
In particular, this thesis presents two more recent
developments in Moltres, namely fully integrating \gls{MOOSE}'s incompressible
Navier-Stokes module into Moltres, and introducing a
decay heat model.
The main objective of this thesis is to verify Moltres'
latest capabilities in modeling multiphysics, steady-state, and transient
behavior of fast-spectrum \glspl{MSR} through the study of the \gls{MSFR}
concept. Code-to-code verification is an important exercise in software
development for ensuring that the application produces accurate and reliable
results. This thesis covers the \gls{MSFR} concept mainly because it has been
studied extensively with readily available data in the literature to verify
against. The \gls{MSFR} design also features interesting flow
patterns that greatly affect the steady-state and transient behavior. This
present work will first present a verification of Moltres' \gls{MSFR}
diffusion neutronics against the Monte Carlo neutron transport software
Serpent 2, followed by a verification of
the coupled neutronics/thermal-hydraulics steady-state and accident transient
results against two sets of results published by
Fiorina et al. \cite{fiorina_modelling_2014}. The two sets of results arose
from a collaborative benchmarking exercise by researchers at Politecnico di
Milano and Technical University of Delft with two separate \gls{MSR}
simulation tools. Section \ref{chap:lit} discusses these tools
in greater detail. The
secondary objective is to identify areas of improvement in Moltres for future
development.

\section{Outline}

The outline of this thesis is as follows. Chapter 2 discusses the history and
features of \glspl{MSR}, and a literature review of existing \gls{MSR}
simulation tools. The chapter also covers the \gls{MSFR} concept in greater
detail. Chapter 3 details the software and the general modeling
approach for generating the results in this thesis. Chapter 4 provides a
neutronics assessment by comparing key neutronics parameters from Moltres'
eigenvalue calculations to Serpent's Monte Carlo calculations. Chapter 5
presents steady-state results of coupled neutronics/thermal-hydraulics
\gls{MSFR} simulations in Moltres. Chapter 6 presents accident transient
simulation results for unprotected reactivity insertions, unprotected loss of
heat sink, unprotected loss of flow, and unprotected pump overspeed. Lastly,
Chapter 7 summarizes the key findings in this thesis
and posits some potential avenues for future work.

\glsresetall

\chapter{Molten Salt Reactor Modeling}
\label{chap:lit}
\glspl{MSR} possess unique characteristics which render existing \gls{LWR}
analysis software inappropriate for \gls{MSR} analysis. Legacy \gls{LWR}
software typically scale poorly on modern high-performance computing
clusters and do not support complex geometries beyond regular \gls{LWR} fuel
assembly lattices. Furthermore, \glspl{MSR} feature strong multiphysics
coupling which force segregated solvers into taking smaller timesteps to
maintain accuracy. This chapter provides a brief history of \glspl{MSR}
development and operation, followed by a discussion of the challenges in
\gls{MSR} multiphysics modeling
for reactor accident analysis. Next, this chapter presents a literature
review of existing multiphysics simulation software developed for \glspl{MSR}
analysis. This work focuses on software for analysing short-term reactor
dynamics which requires the ability to accurately simulate various transient
scenarios such as reactor start-up and coast-down, load-following operations,
steady-state operation, and accident analysis. Long-term dynamics such as fuel
burnup and structural corrosion fall outside the scope of this work. Lastly,
this chapter provides a literature review of turbulence modeling and control
rod modeling, and their relevance in \glspl{MSR} modeling.

\section{Molten Salt Reactors}

\subsection{Historical Molten Salt Reactor Development and Operation}

\gls{ORNL} researchers first conceived the \gls{MSR} concept in pursuit of a
liquid fuel reactor for the US Aircraft Nuclear Propulsion program in
the 1950s \cite{rosenthal_molten-salt_1970}. Due to high temperature
requirements, water-cooled reactors were not suitable as aircraft jet engines
\cite{dolan_1_2017}. Instead, the researchers selected molten fluoride
salts in particular for high uranium solubility, chemical stability, low vapor
pressure even at high temperatures, good heat transfer properties,
resistance against radiation damage, and reduced corrosive effects on some
common structural material \cite{rosenthal_molten-salt_1970}. They
subsequently built the first ever operational \gls{MSR}, 2.5 MW$_{\text{th}}$
\gls{ARE} reactor at \gls{ORNL}. The \gls{ARE}
achieved criticality on November 1954 and generated 100 MWh over nine days.
The reactor ran on enriched uranium in a molten salt mixture of NaF,
ZrF$_4$, and UF$_4$ with BeO neutron moderators. The aircraft program
ultimately never came to fruition as the development of intercontinental
ballistic missiles effectively eliminated the need for long-range
nuclear-powered bomber aircraft.

However, the successful demonstration of the \gls{ARE} spurred further
research into adapting \glspl{MSR} for civilian power generation
\cite{rosenthal_molten-salt_1970}. One key finding from the
research was that the thorium fuel cycle had a better breeding ratio than the
$^{238}$U-to-$^{239}$Pu fuel cycle in thermal-spectrum reactors.
Ultimately, these efforts culminated in the design, construction, and
successful operation of the \gls{MSRE}. The \gls{MSRE} had a 8 MW$_{\text{th}}$,
graphite-moderated design with a LiF-BeF$_2$-ZrF$_4$-UF$_4$ fuel salt mixture
\cite{haubenreich_experience_1970}. In January 1969, the \gls{MSRE} became the
first reactor to run on $^{233}$U fuel bred from $^{232}$Th.

Building on their experience with the \gls{MSRE}, \gls{ORNL} proposed a
new program for the construction and operation of a demonstration reactor
based on the \gls{MSBR} concept that they had
developed \cite{macpherson_molten_1985}. The \gls{MSBR} is a thermal-spectrum,
single fluid reactor with fertile $^{232}$Th isotopes mixed directly into the
FLiBe molten salt for $^{233}$U breeding \cite{robertson_conceptual_1971}. Like the
\gls{MSRE}, the \gls{MSBR} relies on continuous online reprocessing to add
fertile material and remove fission product neutron poisons. Researchers
estimated the doubling time (the minimum amount of time required to produce
enough fissile material to start up another \gls{MSBR}) to be
approximately 22 years. While \gls{ORNL} failed to secure funding for the
\gls{MSBR} program, two independent
technology evaluation and design studies of the \gls{MSR} had reported
favorably on the promise of the system \cite{macpherson_molten_1985}.

Following a relative lull lasting until the late 20th century, researchers at \gls{CNRS} began
research into \glspl{MSR} in 1997 \cite{heuer_simulation_2010}. Starting from the \gls{MSBR}
design, they performed parametric studies on reactor safety, breeding, and other performance
metrics \cite{mathieu_thorium_2006}. They found that graphite-moderated designs required careful
consideration of the fuel-to-moderator ratio as some designs exhibited positive temperature
feedback coefficients. Safer reactor configurations generally operated at very thermalized spectra
owing to large neutron losses in the graphite, or at epithermal and fast spectra owing to
$^{232}$Th resonance capture cross sections. Combined with findings on poor breeding performance
with thermal spectra and accelerated graphite irradiation damage with fast spectra, the authors
suggested eliminating the graphite moderator to optimize both reactor safety and breeding.

Their efforts culminated in the \gls{MSFR} concept, a fast-spectrum breeder \gls{MSR}
designed to run on the thorium fuel cycle \cite{merle_optimized_2007}. As opposed to the
multi-channel design of the \gls{MSRE} and \gls{MSBR}, the \gls{MSFR} reactor core consists of a
single, large channel through which the salt flows as shown in Figure \ref{fig:msfr}. In 2008, the
Generation IV International Forum highlighted the \gls{MSFR} among other \gls{MSR} designs for
further development \cite{gif_generation_2008}. The \gls{MSFR} has also benefited from
collaborative research through three projects, the \gls{EVOL} \cite{euratom_final_2015},
\gls{SAMOFAR} \cite{kloosterman_20_2017}, and \gls{SAMOSAFER} \cite{cordis_severe_nodate} projects.
Under the \gls{EVOL} project, researchers further optimized the \gls{MSFR} design based on
neutronic and thermal-hydraulic safety analyses. One significant modification involved a transition
from the original straight cylinder core design to curved walls in the reactor core to facilitate
smoother fuel salt flow and prevent large eddies from forming in the core
\cite{rouch_preliminary_2014}. Large eddies create hot, swirling pockets of fuel salt which could
complicate reactor operation. For instance, changes in flow conditions, intended or otherwise,
could cause the eddies to collapse and in turn introduce a localized pocket of hot fuel salt to the
circulating fuel loop. Coupled with the strong temperature reactivity feedback in the \gls{MSFR},
reactor could experience large power fluctuations every time the hot fuel salt passes through the
core. The \gls{SAMOFAR} project, which started approximately two years after the end of the
\gls{EVOL} project, supported more comprehensive safety assessments of the reactor and the
reprocessing plant, and funded a number of experiments for validation of the \gls{MSFR}'s safety
features. The ongoing \gls{SAMOSAFER} project funds further research activities with the goal of
achieving modeling, analysis, and design improvements on various aspects of \gls{MSR} operation and
safety.
%
\begin{figure}[htb!]
	\centering
	\includegraphics[width=.7\columnwidth]{msfr}
	\caption{Schematic diagram of the \gls{MSFR}. Retrieved from 
	\cite{allibert_7_2016}.}
	\label{fig:msfr}
\end{figure}

China and India have also started national programs supporting \gls{MSR}
development. China launched their \gls{TMSR} program in 2011 to develop and
construct both solid-fueled and liquid-fueled \gls{TMSR} designs
\cite{zou_research_2019}. Latest updates at the time of writing indicate that a
2-MW liquid-fueled prototype will complete construction by August
2021, with tests expected to start the following month. India also signaled
their interest especially in thorium-based
reactors given their vast thorium reserves
\cite{jayaram_overview_1987}. They have developed conceptual designs of the
\gls{IMSBR}, expected to run on a fast/epithermal neutron spectrum to avoid
using graphite moderators given their tendency to deform under high neutron
fluence.

In the US, Southern Company received funding from \gls{DOE}'s
\gls{ARDP} \cite{doe_office_2021} to design, construct, and operate the
\gls{MCRE}, a prototype chloride salt-based \gls{MSR} relevant to TerraPower's
\gls{MCFR} \cite{terrapower_mcfr_2020}. This project builds on a prior
five-year cost-sharing project for the development of the \gls{MCFR} involving
TerraPower, Southern Company, Oak Ridge National Laboratory, Idaho National
Laboratory, the Electric Power Research Institute, and Vanderbilt University.
The \gls{MCFR} is a fast-spectrum \gls{MSR} similar to the \gls{MSFR}.
Canada-based Terrestrial Energy is also developing their \gls{IMSR}
\cite{leblanc_18_2017}, a small modular \gls{MSR} based on the \gls{MSRE}. The
replaceable \gls{IMSR} core-unit which holds the reactor core, pumps, heat
exchangers, and control rods, runs for approximately 7 years before it is shut
down and replaced after a cool-down period.

With the renewed global interest in \glspl{MSR}, \gls{MSR} modeling software play an important role
in supporting \gls{MSR} development. Reactor modeling can accelerate reactor design and
optimization by enabling engineers to iterate through numerous design changes. \gls{MSR} modeling
software are also essential tools in reactor safety analysis and licensing efforts as engineers
must demonstrate and verify their \gls{MSR} designs remain safe under various accident scenarios.
However, many reactor software were tailored for modeling solid-fueled reactors such as \glspl{LWR}
which make up most of the world's operating reactor fleet. \glspl{MSR} possess unique features and
physics which are typically not modeled in other reactor types. The \gls{ARE} and \gls{MSRE}
projects first demonstrated the feasibility of a liquid-fueled reactor with fissile fuel dissolved
in the molten salt coolant. The \gls{MSBR} design features $^{233}$U breeding from $^{232}$Th with
online fuel reprocessing as opposed to batch-wise reprocessing of solid-fueled breeder reactors.
The various modern \gls{MSR} designs feature numerous changes relative to the older \gls{MSRE} and
\gls{MSBR} designs such as running on epithermal and fast neutron spectrums, different core
structures, and different molten salt compositions.

\gls{MSR} modeling software play an important role in supporting
\gls{MSR} development. They accelerate reactor design and optimization by
enabling engineers to iterate through numerous design changes. \gls{MSR}
modeling software are also essential tools in reactor safety analysis and
licensing efforts as engineers must demonstrate and verify their \gls{MSR}
designs remain safe under various accident scenarios. The next section
identifies challenges in \gls{MSR} multiphysics modeling with respect to
reactor accident analysis.

\section{Challenges in MSR Multiphysics Modeling} \label{sec:challenges}

While modeling \glspl{MSR} is not necessarily more difficult than modeling
solid-fueled reactors, we must adapt our software tools to accurately model the
unique phenomena found in these circulating-fuel reactors. The differences in
the challenges of simulating \glspl{MSR} compared to solid-fueled reactors stem
mainly from the liquid fuel form of the fuel salt \cite{huff_identifying_2019,
diamond_phenomena_2018}.

Firstly, liquids generally exhibit greater thermal
expansion per unit change in temperature than solids. A decrease in density of
the fuel medium increases the likelihood of neutrons escaping the fuel region
and being absorbed by non-fissile material elsewhere in the reactor.
Consequently, combined with the temperature-dependent Doppler broadening of
resonance capture cross sections, \glspl{MSR} possess stronger negative fuel
temperature reactivity feedback than their solid-fueled counterparts
\cite{elsheikh_safety_2013}. These
phenomena ultimately result in strong interactions between the neutron fluxes
and core temperatures given that neutron fluxes affect core temperatures
through fission heat generation and core temperatures in turn affect neutron
fluxes through the mechanisms as described prior.

Secondly, with the fuel
salt also serving the role of providing cooling in the core, velocity flow
profiles in the fuel salts strongly impact the temperature distribution via
advection-dominated heat transfer \cite{diamond_phenomena_2018}. This contrasts
with the relatively static temperature distribution shapes in fuel pins and
other forms of solid fuel matrixes physically separated from the coolant; 
changes in coolant and the resultant changes in heat transfer rates in
solid-fueled reactors are often reduced to empirical correlations governing
convective heat transfer between the fuel cladding and the coolant. On the
other hand, the velocity flow profile has a more direct effect on the
temperature distribution and the temperature-dependent neutron cross sections
in \glspl{MSR}.

Lastly, \glspl{DNP} flow freely within the primary coolant loop as opposed to
being held in place as in solid-fueled reactors. Thus, the delayed neutron
source distribution varies significantly depending on the flow profile and
velocity. In addition, the reactor loses some delayed neutrons from out-of-core
\gls{DNP} decay. These delayed neutrons are considered lost as they're emitted
in subcritical regions and are unlikely to contribute to further fission
reactions in the active core. The reduced delayed neutron fraction in the core
contributes to a greater prompt power spike following a reactivity insertion
event compared to solid-fueled reactors, absent any temperature reactivity
feedback. Thus, accurate \gls{MSR} transient simulations require accurate
modeling of \gls{DNP} drift.

As a result, multiphysics
software must employ \textit{tight coupling schemes} to couple the neutronics
and thermal-hydraulics governing equations to accurately capture the strong
multiphysics interactions in transient \gls{MSR} simulations. Tightly coupled
numerical models handle multiphysics interactions by either updating all state
variables simultaneously in one monolithic solve (\textit{full coupling}) or
iteratively updating all state variables (\textit{fixed point iterations})
until the solution converges in every timestep \cite{keyes_multiphysics_2013}.
Full coupling tends to be more computationally expensive because it combines
all physics equations into a single large system of equations to be solved
simultaneously, whereas fixed point iterations involve operator splitting to
separate the system of equations into smaller systems based on their associated
physics, solving smaller systems separately, and iteratively updating the state
variables until convergence. Fixed point iterative methods are often less
stable, less accurate, and have poorer
convergence rates since these methods make iterative corrections
without any regard to potentially destabilizing modes introduced by the
multiphysics coupling \cite{keyes_multiphysics_2013}. Notably, proven
techniques exist for improving the performance of fixed point iteration
coupling schemes for many relevant computational multiphysics research fields,
including reactor analysis \cite{ragusa_consistent_2009}. While fully coupled
schemes deal with solving a large system of equations, they can outperform
fixed point iterative methods in some multiphysics problems through superior
stability and convergence rates. 

In contrast to tight coupling schemes, \textit{loose coupling schemes}
solve each set of single-physics equations using state variable
data from the previous timestep without iterative corrections within every
timestep. Loosely coupled schemes are inappropriate for modeling \glspl{MSR}
given the strong coupling between the neutronics and thermal-hydraulics.
Aufiero et al. \cite{aufiero_development_2014} demonstrated a loose coupling
approach that failed to reproduce the expected increase in reactor power
in an \gls{MSR} in response to a 150 pcm reactivity insertion.

Consequently, most MSR multiphysics simulation tools employ, at
the very least, tight coupling schemes through fixed point iterations. The
next section explores various numerical coupling approaches in existing
multiphysics \gls{MSR} simulation tools.

\section{MSR Multiphysics Simulation Tools} \label{sec:msr-tools}

In recent years, several simulation tools have been developed for full-core
modeling of fast-spectrum \glspl{MSR}. \textit{Tightly coupled} approaches,
through segregated solvers, involve coupling separate single-physics neutronics
and thermal-hydraulics software. For example, researchers at
the \gls{TUD} coupled the 3D neutron diffusion software DALTON
\cite{boer_validation_2010} and the \gls{CFD} software HEAT
\cite{de_zwaan_static_2007} to perform a safety analysis of the \gls{MSFR}
\cite{fiorina_modelling_2014}. In a later effort from the same institute,
Tiberga et al. \cite{tiberga_discontinuous_2019} coupled PHANTOM-$S_N$ and
DGFlows in their participation in the CNRS benchmark study
\cite{tiberga_results_2020}. The CNRS benchmark, named after the \gls{CNRS}
where it was originally developed, facilitates code-to-code verification of
\gls{MSR} multiphysics software \cite{aufiero_testing_2018}. Another
multiphysics package was developed at the Paul Scherrer Institute (PSI)
coupling the thermal-hydraulics system software \gls{TRACE}
\cite{nrc_trace_2007} with the
nodal neutron diffusion software \gls{PARCS} \cite{downar_parcs_2010} for the
safety analysis of the \gls{MSFR} \cite{pettersen_coupled_2016}. Coupling
single-physics software to form an integrated multiphysics tool allows
researchers to leverage on older, well-validated, single-physics software.
These single-physics software are also highly optimized for solving specific
types of \glspl{PDE} relevant to the investigated system.

With modern advancements in computing hardware and growing access to
high-performance computing systems, others have developed multiphysics tools by
coupling the \gls{CFD} software OpenFOAM
\cite{the_openfoam_foundation_ltd_openfoam_2021} with the Monte Carlo particle
transport software
Serpent \cite{leppanen_serpent_2014}, thus achieving high-fidelity neutronics
calculations in transient reactor analyses. Laureau et al.
\cite{laureau_transient_2017} developed an innovative technique called the
\gls{TFM} method through the introduction of additional time-dependence
operators to conventional fission matrices typically used to accelerate source
convergence in Monte Carlo neutronics calculations. The \gls{TFM} method
pre-calculates three \glspl{TFM} of the reactor system in Serpent and
interpolates the matrix values during the actual transient calculations to
incorporate the effects of temperature-induced density change and Doppler
effect on the neutron cross sections and ultimately the neutron flux. Blanco
\cite{blanco_neutronic_2020} took a more integrated approach by
compiling Serpent as an internal \texttt{C}-based function within OpenFOAM's
\texttt{C++}-based framework. This approach reduced the amount of required data
transfers between Serpent and OpenFOAM as both software have access to shared
memory during runtime. Their integrated tool employs the Quasi-Static
method for transient neutronics calculations and runs Serpent Monte Carlo
calculations several times per timestep until convergence is reached.

Another \gls{MSR} simulation approach involves developing ``all-in-one''
multiphysics software which handle all multiphysics calculations and data
transfer internally. Among earlier efforts, Nicolino et al.
\cite{nicolino_coupled_2008} and Zhang et al. \cite{zhang_development_2009}
recognized the
need for more robust multiphysics coupling techniques and higher-fidelity
thermal hydraulics solutions to accurately capture complex flow profiles in
pool-type \glspl{MSR}. They each independently developed unnamed multiphysics
simulation tools and demonstrated their tools with non-moderated \gls{MSR}
designs. Later, Li et al. \cite{li_transient_2015} demonstrated the
steady-state and transient analysis capabilities of COUPLE, a neutronics and
thermal-hydraulics software developed at the Karlsruhe Institute of Technology.
Others adopted extensible software frameworks for developing numerical solvers
to develop multiphysics reactor analysis software. Examples of these software
frameworks include the commercial COMSOL
Multiphysics\textsuperscript{\textregistered} software
\cite{comsol_ab_comsol_nodate}, the aforementioned open-source CFD toolbox
OpenFOAM, and the open-source finite-element
framework \gls{MOOSE} \cite{gaston_physics-based_2015}. Researchers at
\gls{PoliMi} developed a \gls{MSR} simulation tool in COMSOL and
modeled the \gls{MSBR} as a single axisymmetric fuel channel with a uniform
flow profile \cite{cammi_multi-physics_2011}, followed by the \gls{MSRE} core
also as a single axisymmetric fuel channel with parabola-shaped laminar flow
\cite{cammi_dimensional_2012}. They later expanded on their approach by
modeling the \gls{MSRE} upper plenum, downcomer and lower plenum, primary heat
exchanger, and secondary heat exchanger as 0D systems (lumped-parameter model),
and substituting the 2D fuel channel with a 3D fuel channel which more closely
resembled the actual fuel channels in the \gls{MSRE}
\cite{zanetti_geometric_2015}. Beyond graphite-moderated \glspl{MSR}, they
also modeled the \gls{MSFR} in the same publication which featured \gls{TUD}'s
DALTON + HEAT coupled multiphysics tool.
More recently, several European institutes (\gls{CNRS}, \gls{PoliMi},
and the \gls{PSI}) have also developed coupled neutronics and
thermal-hydraulics tools in OpenFOAM. Their tools share some
similarities such as implementing the $SP_N$ simplified $P_N$ neutron transport
model and leveraging OpenFOAM's turbulent flow modeling capabilities.
Differences include fuel compressibility modeling and helium bubble tracking
capabilities from \gls{PoliMi} \cite{cervi_development_2019}, fuel
performance analysis capability from \gls{PSI} \cite{fiorina_creation_2018},
and the aforementioned external coupling capability with Serpent from
\gls{CNRS} \cite{blanco_neutronic_2020}.

Finally, within the MOOSE framework, simulation tools capable of modeling
\glspl{MSR} include: Rattlesnake \cite{wang_rattlesnake_2021}; and Moltres
\cite{lindsay_moltres_2017}\textemdash the subject of this work.
Rattlesnake primarily tackles radiation transport problems, but the MOOSE
framework facilitates multiphysics coupling
with MOOSE-based applications for other physics
such that all applications share the same data structure. This eliminates
computational costs from external data transfers and optionally allowing for
\textit{fully coupled} solves in which the application solves all physics
simultaneously. Similarly, Moltres benefits from the highly-integrated
cross-compatibility
within the ecosystem of MOOSE-based applications. Abou-Jaoude et al.
\cite{abou-jaoude_coupled_2020} coupled Rattlesnake with Pronghorn, another
MOOSE-based application for advanced reactor thermal-hydraulics modeling, to
demonstrate several steady-state \gls{MSR} simulation capabilities defined in
the CNRS benchmark. Lindsay et al.
\cite{lindsay_introduction_2018} first demonstrated Moltres' \gls{MSR} modeling
capabilities on 2D axisymmetric and 3D Cartesian models of the \gls{MSRE} with
fixed velocity flow on a fully coupled neutronics and thermal-hydraulics solve.
We later demonstrated some of Moltres' more recent developments through
modeling a 2D axisymmetric model of the \gls{MSFR} for steady-state operation
and transient accident analysis \cite{park_advancement_2020}. The latter study
introduced looped \gls{DNP} flow, coupling the \gls{DNP} drift and temperature 
advection-diffusion to incompressible flow, and decay heat modeling
capabilities. The proposed work aims to continue Moltres development for
multiphysics \gls{MSR} analysis. Chapter \ref{chap:moltres} describes Moltres
and its existing capabilities for reactor analysis.

\section{Turbulence Modeling in MSRs}

In fluid dynamics, turbulent flow is characterized by unsteady, irregular, and
chaotic fluid motion as opposed to neat, parallel flow layers in laminar flow
\cite{pope_turbulent_2000}. The transition from laminar to turbulent flow
typically occur at Reynolds numbers between 2000 and 4000, depending on the
setup \cite{pope_turbulent_2000}. Turbulent flows are expected in \glspl{MSR}.
Kedl \cite{kedl_fluid_1970} reports expected Reynolds numbers in the \gls{MSRE}
ranging from 1000 in the regular fuel coolant channels to over 10000 in the
flow distributor volute and core wall cooling annulus regions. For the
\gls{MSFR}, salt flow in the central core region is highly turbulent and
reaches Reynolds numbers on the order of $10^5$.

\subsection{Turbulence Models}

Numerous types of turbulence models exist for various turbulent flow
applications. The most common turbulence models can be classified into the
following categories by order of increasing computational complexity:
%
\begin{itemize}
    \item \gls{RANS}-based models
    \begin{itemize}
        \item Eddy viscosity models
        \begin{itemize}
            \item Algebraic models
            \item One- and two-equation models
        \end{itemize}
        \item \gls{RSM}
    \end{itemize}
    \item \gls{DES}
    \item \gls{LES}
    \item \gls{DNS}
\end{itemize}

\gls{RANS}-based models are based on the \gls{RANS} equations obtained from
applying time-averaging to the equations of fluid flow. The \gls{RANS}
equations separate flow into time-averaged $U$ and fluctuating $u$ components
and can be writtin in Einstein notation and Cartesian coordinates as:
%
\begin{align}
    \frac{\partial U_i}{\partial t} + U_j \frac{\partial u_i}{\partial x_j} =&
    -\frac{1}{\rho} \frac{\partial P}{\partial x_i} + \nu \nabla^2 U_i -
    \frac{\partial \langle u_i u_j \rangle}{x_j}
    \shortintertext{where}
    \langle \cdot \rangle =& \mbox{ time-averaging operator,} \nonumber \\
    \rho =& \mbox{ fluid density,} \nonumber \\
    P =& \mbox{ time-averaged pressure field,} \nonumber \\
    \nu =& \mbox{ kinematic viscosity.} \nonumber
\end{align}

Eddy viscosity models, which comprise of the most widely used turbulence models
in use today \cite{rodi_turbulence_2017}, operate on the eddy viscosity
hypothesis which states that the Reynolds stresses in the \gls{RANS} equations
are given by:
%
\begin{align}
    \langle u_iu_j \rangle =& \frac{2}{3}k \delta_{ij} - \nu_T \left(
    \frac{\partial U_i}{\partial x_j} + \frac{\partial U_j}{\partial x_i}
    \right)
    \shortintertext{where}
    k =& \mbox{ mean turbulent kinetic energy,} \nonumber \\
    \delta_{ij} =& \mbox{ Kronecker delta,} \nonumber \\
    \nu_T =& \mbox{ eddy viscosity.} \nonumber \\
\end{align}

The various eddy viscosity models mainly differ in their approach towards
the closure problem of calculating the eddy viscosity. As the name suggests,
algebraic models rely on algebraic equations to calculate the eddy viscosity
distribution directly from flow variables. As a result, algebraic models are
the least computationally intensive models for turbulence. Algebraic models
can be further categorized into two types: uniform eddy viscosity models
and mixing length models. Uniform eddy viscosity models apply a uniform eddy
viscosity throughout the problem domain. The uniform eddy viscosity is
calculated from flow parameters such as the characteristic velocity, the
characteristic flow width, and empirically determined turbulent Reynolds
number. Given that eddy viscosities usually vary significantly in most types of
flow, uniform eddy viscosity models have a very limited range of applicability
\cite{pope_turbulent_2000}. Mixing length models add a level of complexity by
relating the eddy viscosity to spatially-varying flow parameters such as the
mean velocity gradient (Prandtl \cite{prandtl_7_1925} and Cebeci-Smith
\cite{smith_numerical_1967} models) or the mean rate of strain (Baldwin-Lomax
\cite{baldwin_thin-layer_1978} model) and an empirical mixing length parameter.
Combined with empirical data for the mixing length parameter, these
models provide better approximations of free shear flows, but still
underperform for more complex flows involving flow separation and significant streamline curvature.

One- and two-equation turbulence models introduce differential equations to
describing turbulence quantities such as the turbulence kinetic energy and the
turbulence rate of dissipation to obtain the eddy viscosity distribution. The
most common and best performing one-equation model is the Spalart-Allmaras
model which provides an equation for the eddy viscosity directly with several
closure coefficients and functions \cite{wilcox_turbulence_2006}. The
Spalart-Allmaras model is considered ``complete'' as it does not involve any
adjustable coefficients or functions. Calibrated for free shear flows in
aeronautical applications, the model performs modestly better than algebraic
models in these applications \cite{pope_turbulent_2000}, but it still deviates
significantly from experimental data
for separated flows \cite{wilcox_turbulence_2006}.

Investigations with
one-equation models reveal the need for an extra equation to account for
turbulent length scales separately from turbulent velocity. Thus, two-equation
models became the most widely adopted turbulence model in the late 20th century
\cite{pope_turbulent_2000}. Two-equation models include the $k$-$\epsilon$,
$k$-$\omega$, and $k$-$\tau$ models. The variables $k$, $\epsilon$, $\omega$,
and $\tau$ correspond to turbulent kinetic energy, turbulent dissipation,
specific turbulent dissipation rate, and turbulent time scale, respectively.
While none of these models perform universally well, they are generally more
accurate than the algebraic and one-equation models. Successive contributions
and modifications to the two-equation models through the years have also
improved their performance in predicting various types of turbulent flow. Their
moderate computational expense compared to expensive, high-fidelity turbulence
models favor their adoption in most commercial \gls{CFD} software for
engineering applications \cite{pope_turbulent_2000}.

\glspl{RSM} directly computes the individual components $\langle u_i u_j
\rangle$ of the Reynolds stress tensor instead of approximating it with a
single, isotropic eddy viscosity term. As a consequence, \glspl{RSM} provide
more realistic predictions for flows with significant rotational motion and
sudden changes in the mean strain rate, albeit at greater computational
expense, compared to the one- and two-equation models
\cite{wilcox_turbulence_2006}. Smaller improvements are observed in modeling
free shear flows and backward-facing step flows \cite{wilcox_turbulence_2006}.

Due to the much higher computational cost for \gls{DES}, \gls{LES}, and
\gls{DNS}, these models have limited applicability in routine, high-Reynolds
number engineering problems today. However, given their high accuracy, these
models are useful for flow problems with relatively simple geometries and at
low Reynolds numbers and validating the lower-fidelity turbulence models
\cite{zhiyin_large-eddy_2015}.

\subsection{Turbulence Modeling in MSR Simulation Tools}

For MSR modeling, the $k$-$\epsilon$ and $k$-$\omega$ turbulence models are the
most commonly used models as shown in published work with COMSOL
\cite{fiorina_modelling_2014}, OpenFOAM \cite{aufiero_development_2014}, and
\gls{TUD}'s in-house codes \cite{fiorina_modelling_2014,tiberga_results_2020}.
Podila et al. \cite{podila_cfd_2019} performed \gls{CFD} simulations of the
\gls{MSRE} core with six different turbulence models, namely a Spalart-Allmaras
model, two variants of the $k$-$\epsilon$ model, a $k$-$\omega$ model, and two
variants of \glspl{RSM}. Their results showed relatively small differences
in graphite and fuel temperatures among different turbulence models. However,
they observed significant differences in the turbulence intensities near the
wall. Given the lack of experimental data for model validation, the authors
could not make a clear assessment of the models' accuracies. Nevertheless, the
close agreement of the fuel temperatures imply that the discrepancies in the
turbulent intensities near the wall have a negligible impact on the overall
distribution of advected quantities in the \gls{MSRE}. Podila et al.
\cite{podila_cfd_2019} opted to use a $k$-$\epsilon$ model for subsequent
calculations in their work given its lowest computational cost and the close
agreement in the temperature distributions.

Amongst other \gls{MSR} simulation tools, the $k$-$\epsilon$ and $k$-$\omega$
turbulence models are the most commonly used models as shown in published work
with COMSOL \cite{fiorina_modelling_2014}, OpenFOAM
\cite{aufiero_development_2014}, and \gls{TUD}'s in-house codes
\cite{fiorina_modelling_2014,tiberga_results_2020}. Fiorina et al.
\cite{fiorina_modelling_2014} compared the flow distribution from both models
in a 2D axisymmetric \gls{MSFR} geometry and observed that the $k$-$\omega$
model produced a wider recirculation zone near the outer wall, an additional
recirculation zone near the top wall, and significantly higher maximum
temperatures within the former recirculation zone. Thus, their work highlights
the difficulties of modeling separated flows and calls for extra attention
towards the choice of turbulence model in \glspl{MSR}.

\section{Control Rod Modeling}

Most reactor designs have control rods to control the fission rate or completely shut down the
reactor by inserting the rods further into the reactor core. As such, control rods consist of
strong neutron absorbers like boron, cadmium, and gadolinium to reduce the number of fission
neutrons available to contribute to further fission chain reactions. All reactors start operation
with excess reactivity to ensure that they have enough fissile material to last through to the next
scheduled refueling. Control rods, along with burnable poisons, allow reactors to operate at a
multiplication factor of unity by canceling out the excess reactivity. Control rods are also useful
for ramping the power up or down during start-up, shut-down, or load-following operations. Most
importantly, control rods provide a safety mechanism for quickly shutting down a reactor under
dangerous conditions. Rapid control rod insertion is the main mechanism of a reactor scram to
reliably introduce a large negative reactivity insertion in response to an unintended reactivity
insertion or other events which may threaten the safe operation of the reactor. 

The number and locations of control rods in a reactor vary significantly for different reactor
types, depending on their maximum operating power, refueling frequency, and other factors. For
instance, as shown in Figure *, a single fuel assembly in a \gls{PWR} can accommodate 24 control
rods in close proximity to the fuel pins. The total number of fuel assemblies vary from 37
assemblies in the NuScale VOYGR reactor to 157 assemblies in the Westinghouse AP1000 reactor. On
the other hand, the entire MHTGR-350 high-temperature gas reactor, shown in Figure *, has 30
control rods in total. The control rods are located in graphite reflector blocks, thereby placing
them further from fuel elements. Lastly, the \gls{MSRE} design has three control rods located
centrally in graphite-moderated core with fuel salt channels. While the objective of the proposed
work is to improve diffusion-based control rod modeling in Moltres for \gls{MSR}-like designs, this
literature review also explores existing methods developed for other reactor types.

[Insert control rod illustration]

\subsection{Challenges of Control Rod Modeling} \label{sec:challenges-control-rod}

In reactor design studies, we aim to determine control rod worth which is measured as the reduction
in the neutron multiplication factor $k_{eff}$ due to the presence of the control rod in the
reactor. We are also interested in the accompanying change in flux shape in the vicinity of the
control rod. Neutrons entering the control rods have a higher chance of being absorbed than
adjacent regions in the reactor core. Therefore, control rods induce highly anisotropic neutron
angular fluxes and sharp gradients in the neutron flux in their vicinity. Control rods also cause
shifts in the neutron energy spectrum because their absorption cross sections are much higher in
the lower neutron energy range; more energetic neutrons generally have a higher probability of
escaping the control rod region. As a consequence, modeling control rods accurately requires
high-fidelity computational methods to capture the angular- and energy-dependence in the neutron
flux near the highly absorbing medium.

High-fidelity \textit{neutron transport methods} fall under two categories: stochastic Monte Carlo
methods and deterministic methods. Monte Carlo methods involve simulating a finite number of
neutron histories. Randomly generated numbers determine the outcome of various probabilistic events
such as travel distances between particle collisions, types of collisions, and scattering angles in
each history until it is terminated by a neutron capture event. Deterministic methods for neutron
transport, such as the discrete ordinates $S_N$ and spherical harmonics $P_N$ methods, solve the
\gls{BTE} for neutron transport with some approximations for handling the angular and energy
dependence. The \gls{BTE} for neutron transport is given as:
%
\begin{align}
  \frac{1}{v(E)} \frac{\partial}{\partial t} \Psi(\vec{r},E,\hat{\Omega},&t) + \hat{\Omega}\cdot
  \nabla\Psi(\vec{r},E,\hat{\Omega},t) + \Sigma_t(\vec{r},E,t)\Psi(\vec{r},E,\hat{\Omega},t) 
  \nonumber \\
  - \int^\infty_0 dE'& \int_{4\pi} d\hat{\Omega}' \Sigma_s(\vec{r},E'\rightarrow E,\hat{\Omega}'
  \rightarrow \hat{\Omega},t) \Psi(\vec{r},E',\hat{\Omega}',t) \nonumber \\
  =& \frac{\chi(E)}{4\pi}
  \int^\infty_0 dE' \int_{4\pi} d\hat{\Omega}' \nu\Sigma_f(\vec{r},E',t) \Psi(\vec{r},E',
  \hat{\Omega}',t)+S(\vec{r},E,\hat{\Omega}',t) \label{eq:bte}
  \shortintertext{where}
  v(E) =& \mbox{ neutron velocity,} \nonumber \\
  \Psi(\vec{r},E,\hat{\Omega},t) =& \mbox{ neutron angular flux,} \nonumber \\
  \vec{r} =& \mbox{ spatial coordinates,} \nonumber \\
  E =& \mbox{ neutron energy,} \nonumber \\
  \hat{\Omega} =& \mbox{ direction of neutron travel,} \nonumber \\
  t =& \mbox{ time,} \nonumber \\
  \Sigma_t(\vec{r},E,t) =& \mbox{ macroscopic total cross section,} \nonumber \\
  \Sigma_s(\vec{r},E'\rightarrow E,\hat{\Omega}'\rightarrow \hat{\Omega},t) =&
  \mbox{ macroscopic scattering cross section,} \nonumber \\
  \chi(E) =& \mbox{ fission neutron spectrum,} \nonumber \\
  \nu =& \mbox{ number of neutrons produced per fission reaction,} \nonumber \\
  \Sigma_f(\vec{r},E',t) =& \mbox{ macroscopic fission cross section,} \nonumber \\
  S(\vec{r},E,\hat{\Omega}',t) =& \mbox{ external neutron source.} \nonumber
\end{align}

The $S_N$ method discretizes the continuous angular directional phase space into a few discrete
angular directions (ordinates) and uses quadrature rules to replace the integrals over
$\hat{\Omega}$ with summations over the ordinates. On the other hand, the $P_N$ method introduces
Legendre polynomial expansions of the angular flux to approximate the angular dependence. Both
methods require higher-order approximations, through more discrete ordinates or higher-order
Legendre expansions, to produce accurate flux solutions. Both methods also discretize the
continuous energy dependence into discrete neutron energy groups which cover non-overlapping,
finite energy ranges across the entire energy spectrum to form multigroup equations as follows:
%
\begin{align}
  \frac{1}{v_g} \frac{\partial}{\partial t} \Psi_g(\vec{r},\hat{\Omega},&t) + \hat{\Omega}\cdot
  \nabla\Psi_g(\vec{r},\hat{\Omega},t) + \Sigma_{t,g}(\vec{r},t)\Psi_g(\vec{r},\hat{\Omega},t)
  \nonumber \\
  - \sum^G_{g'=1}& \int_{4\pi} d\hat{\Omega}' \Sigma_s^{g'\rightarrow g}(\vec{r},\hat{\Omega}'
  \rightarrow \hat{\Omega},t) \Psi_{g'}(\vec{r},\hat{\Omega}',t) \nonumber \\
  =& \frac{\chi_g}{4\pi}
  \sum^G_{g'=1} \int_{4\pi} d\hat{\Omega}' \nu\Sigma_{f,g'}(\vec{r},t) \Psi_{g'}(\vec{r},
  \hat{\Omega}',t)+S_g(\vec{r},\hat{\Omega}',t) \label{eq:mg-bte}
  \shortintertext{where}
  G =& \mbox{ total number of energy groups,} \nonumber \\
  g =& \mbox{ neutron energy group index (in decreasing energy order)} = 1,2,...,G. \nonumber
\end{align}

The subscript $g$ denotes the corresponding quantity for neutrons in energy group $g$.
Overall, neutron transport methods are very computationally expensive and thus are mainly used for
time-independent neutronic analyses. For time-dependent multiphysics simulations coupling
neutronics to thermal-hydraulics and other physics present in nuclear reactors, most reactor codes
rely on the neutron diffusion equation for modeling neutronics. The multigroup neutron diffusion
equations are \glspl{PDE} derived from Eq. \ref{eq:mg-bte} by making
simplifying assumptions on the angular dependence in the scattering cross section and integrating
the equations over all solid angles to eliminate angular dependence as follows:
%
\begin{align}
  \frac{1}{v_g} \frac{\partial}{\partial t} \phi_g&(\vec{r},t) + \nabla\cdot J_g(\vec{r},t)
  +\Sigma_{t,g}(\vec{r},t) \phi_g(\vec{r},t) \nonumber \\
  =& \sum^G_{g'=1}\left[\Sigma_s^{g'\rightarrow g}
  \phi_{g'}(\vec{r},t) + \chi_g \nu\Sigma_{f,g'} \phi_{g'}(\vec{r},t)\right] + S_g(\vec{r},t)
  \label{eq:mg-diff} \\
  \shortintertext{where}
  J_g(\vec{r},t) =& \mbox{ neutron current for neutron group $g$.} \nonumber
\end{align}

Fick's first law of diffusion provides a closure relation for Eq. \ref{eq:mg-diff} by relating the
current to the flux:
%
\begin{align}
  J_g(\vec{r},t) =& -D_g(\vec{r},t)\nabla\phi_g(\vec{r},t) \label{eq:fick}
  \shortintertext{where}
  D_g(\vec{r},t) =& \mbox{ neutron diffusion coefficient for neutron group $g$.} \nonumber
\end{align}

The diffusion coefficient itself is estimated from the total or transport cross sections, depending
on whether we take scattering to be isotropic or linearly anisotropic in the $P_1$ approximation of
the \gls{BTE} \cite{lamarsh_introduction_1975}:
%
\begin{align}
  D(\vec{r},t) =& \frac{1}{3\Sigma_t(\vec{r},t)} \quad \mbox{(isotropic)} \\
  D(\vec{r},t) =& \frac{1}{3\Sigma_{tr}(\vec{r},t)} = \frac{1}{3\left(\Sigma_t(\vec{r},t)-
  \bar{\mu}\Sigma_s(\vec{r},t)\right)}
  \quad \mbox{(linearly anisotropic)}
  \shortintertext{where}
  \Sigma_{tr} =& \mbox{ macroscopic transport cross section,} \nonumber \\
  \bar{\mu} =& \mbox{ average cosine of the scattering angles.} \nonumber
\end{align}

This treatment significantly reduces the number of coupled \glspl{PDE} to be solved and the
computational costs of modeling neutronics in reactors. Yet, the simplifications limit the validity
of the neutron diffusion equation to regions of high scattering-to-absorption ratios at least a
few mean free paths away from shared interfaces to neighboring media with highly dissimilar
neutronic properties \cite{shultis_chapter_2016}. A single diffusion constant cannot capture the
strongly anisotropic flux within and near control rod regions.

Many legacy diffusion solvers are based on coarse-mesh or nodal methods. These methods
primarily involve replacing heterogeneous lattices of materials of differing properties with
equivalent homogeneous mixtures of the same materials in each coarse mesh (referred to as nodes in
nodal methods) \cite{stacey_nuclear_2007}. Reducing the heterogeneity of the geometry reduces
computational complexity and circumvents poor diffusion performance in highly heterogeneous
interfaces. Each coarse mesh typically corresponds to a subregion comprising of repeating
substructures, e.g. a single fuel assembly or randomly distributed spherical fuel pebbles. The
homogenization procedure consists of two main steps: a transport calculation to obtain detailed
heterogeneous flux distribution within each subregion, followed by the calculation of homogenized
\textit{group constants} from the detailed flux distribution. \textit{Group constants} refer to
macroscopic neutron cross section values for various neutron interactions such as neutron
scattering, absorption, and fission. Macroscopic cross sections represent the probability that a
neutron, in a given energy range, will undergo the associated interaction per unit distance
traveled in the material. Group constants also broadly include neutron diffusion
coefficients/constants and delayed neutron precursor data. While advanced coarse-mesh and nodal
methods provide reasonable flux solutions for calculating global and intermediate-scale quantities,
they do not capture detailed heterogeneous flux distribution within each coarse mesh subregion.

\subsection{Transport Correction Techniques With Neutron Diffusion-Based Solvers}

Transport correction techniques for diffusion-based solvers require additional information beyond
conventional homogenized group constants from transport-based solvers to ameliorate diffusion
solution accuracy in highly absorbing and near-interface regions.
Therefore, this literature review focuses on hybrid diffusion-transport methods developed to relay
transport corrections in the form of more accurate neutron flux and current estimates to
diffusion-based solvers. The methods differ mainly in how they incorporate corrections into a
diffusion-based solver.

\subsubsection{Absorber Blackness and Linear Extrapolation Length}

Methods based on absorber blackness \cite{davison_influence_1951, spinks_extrapolation_1965,
pellaud_extrapolation_1968, mendelson_two-dimensional_1969} encompass a
broad class of procedures for generating boundary conditions to match approximate solutions of
low-order methods (e.g. diffusion) to more accurate solutions from high-order methods (e.g.
transport). The internal boundary conditions replace absorber regions and mimic their presence in
the problem domain. The boundary conditions are generalizations of the Marshak boundary condition
\cite{marshak_note_1947} which in 1D are of the form:
%
\begin{align}
  \frac{\phi(x)}{d\phi(x)/dx} =& \lambda \label{eq:marshak}
  \shortintertext{where}
  \lambda =& \mbox{ linear extrapolation length.} \nonumber
\end{align}
%
The gradient of the flux in the denominator is taken in the outward direction at the surface of the
absorber or black body. In Cartesian coordinates, it is common to approximate Eq. \ref{eq:marshak}
in the form:
%
\begin{align}
  \phi(x+\lambda) =& 0
\end{align}
because Dirichlet boundary conditions are easier to work with analytically and numerically.
The corresponding relations in cylindrical and spherical coordinates depend on the radius of the
absorber rod. Various forms for $\lambda$ exist for specific absorber geometries such as slabs
\cite{maynard_blackness_1959} and cylinders \cite{spinks_extrapolation_1965,
pellaud_extrapolation_1968} which were dependent
on the size of the absorber region and coefficients quantifying the escape probability of neutrons
entering the absorber region. The $\lambda$ and the associated coefficients were derived
analytically from the \gls{BTE} with some simplifying assumptions as such having uniform or
cosine-shaped incident neutron currents, isotropic scattering within the absorber, and mathematical
approximations in ignoring higher-order terms in intermediate steps. Blackness theory emerged in
the mid-20th century when computational resources were limited. Later, as computational
resources became more readily available and powerful enough for more complicated transport
calculations, appropriate boundary conditions could be calculated numerically
\cite{bretscher_computing_1997}. Alternatively, effective group constants can be calculated for
diffusion codes which are not programmed to handle internal boundary conditions. For instance,
Bretscher \cite{bretscher_computing_1997} provided formulae for effective diffusion coefficient and
absorption cross section for thin absorber slabs as a function of blackness coefficients, absorber
thickness, and mesh size.

Diffusion calculations with corrections from blackness theory can generally provide control rod
worths which are in reasonable agreement with more accurate transport calculations. However, the
diffusion flux distributions still deviate significantly from transport calculations because no
corrections are introduced in the adjacent regions near the absorber where the angular flux can be
strongly anisotropic.

\subsubsection{Method of Equivalent Cross Sections}

Scherer \& Neef developed the \gls{MECS} \cite{scherer_determination_1976} to improve control rod
modeling in \gls{HTGR} with mesh-centered nodal diffusion methods. The \gls{MECS} involves running
a 1D neutron transport calculation on a representative \textit{super cell} of the heterogeneous
absorber region and its vicinity. The super cell is a volume-preserved model of the absorber region
approximated from its cylindrical or Cartesian geometry in the diffusion solve. The transport
calculation is typically performed using a fine group structure, fine spatial discretization, and
high-order angular discretization and scattering moments (e.g. $S_8$ method with $P_3$ scattering
matrix) \cite{fen_modelling_1992}. The calculated net neutron leakage rates from the tranport
solver are matched with the calculated leakage rates from the diffusion solver according to an
analytic calculation to obtain equivalent diffusion coefficients or cross sections for the absorber
region. The diffusion coefficient formula for this analytic calculation depends on the solution
method of the diffusion solver. In the most widely used implementation of \gls{MECS} used in
conjunction with the mesh-centered diffusion code CITATION \cite{teuchert_vsop94_1994}, the
diffusion coefficient formula assumes there is only one inner mesh point in the absorber region as
using more inner mesh points greatly increases the difficulty of finding the proper analytic
formula \cite{fen_modelling_1992}. Other macroscopic cross sections for the absorber region are
calculated using conventional group constant generation techniques from the transport solution
using the super cell average flux and reaction rates. To facilitate further discussion of the
\gls{MECS}, here is an example of applying the \gls{MECS} on a control rod in x-y geometry (Figure
\ref{fig:mecs-geometry}) in the CITATION code as provided by Fen et al. \cite{fen_modelling_1992}.
%
\begin{figure}[htb!]
    \centering
    \includegraphics[width=.6\columnwidth]{mecs-geometry}
    \caption{Geometries of the absorber region and its vicinity in the super
        cell (top) and the diffusion solver mesh (bottom).
        The shaded and unshaded regions represent the homogenized absorber
        region and the adjacent non-absorber region, respectively.
        Retrieved from Fen et al. \cite{fen_modelling_1992}.}
    \label{fig:mecs-geometry}
\end{figure}

Figure \ref{fig:mecs-geometry} shows the 1D cylindrical super cell for the
transport solver on the top and the corresponding 2D Cartesian geometry for the
diffusion solver on the bottom. The four adjacent nodes in the Cartesian
geometry and their corresponding flux values are pairwise identical.

According to the mesh-centered geometry in CITATION, the leakage $L$ is
calculated as:
%
\begin{align}
  L =& \frac{F_y\left(\phi_x-\phi_a\right)}{\frac{\delta_x}{D_a}+
    \frac{\Delta_x}{D_o}} + \frac{F_x\left(\phi_y-\phi_a\right)}{
  \frac{\delta_y}{D_a}+\frac{\Delta_y}{D_o}} \label{eq:mecs}
  \shortintertext{where}
  \phi_x =& \mbox{ flux in the $x$-neighbor nodes,} \nonumber \\
  \phi_y =& \mbox{ flux in the $y$-neighbor nodes,} \nonumber \\
  \phi_a =& \mbox{ flux in the absorber node,} \nonumber \\
  F_y =& \mbox{ surface area to $y$-neighbor nodes} = 2pa, \nonumber \\
  F_x =& \mbox{ surface area to $x$-neighbor nodes} = 2a, \nonumber \\
  \delta_x =& \mbox{ distance from the absorber node center to the
    $x$-surface} = a/2, \nonumber \\
  \delta_y =& \mbox{ distance from the absorber node center to the
    $y$-surface} = pa/2, \nonumber \\
  \Delta_x =& \mbox{ distance from the $x$-surface to the $x$-neighbor node
    center} = p_x a/2, \nonumber \\
  \Delta_y =& \mbox{ distance from the $y$-surface to the $y$-neighbor node
    center} = p p_y a/2, \nonumber \\
  a, p, p_x, p_y =& \mbox{ geometric parameters of the CITATION mesh geometry
    (Figure \ref{fig:mecs-geometry}),} \nonumber \\
  D_a =& \mbox{ diffusion coefficient in the absorber node,} \nonumber \\
  D_o =& \mbox{ diffusion coefficient in the neighboring nodes.} \nonumber
\end{align}

In the original formulation by Scherer \& Neef \cite{scherer_determination_1976}, they chose to let
$D_a=D_o$. Taking leakage and flux values at $R_x$ and $R_y$ (figure \ref{fig:mecs-geometry}) from
the transport solver to be equal to the corresponding values in equation \ref{eq:mecs} yields a
value for $\phi_a$ which also represents the average flux in the absorber region. The equivalent
cross sections for each reaction type $i$ is then determined by matching reaction rates from the
transport calculation to the reaction rate as governed by diffusion theory as follows:
%
\begin{align}
  \Sigma_i =& \frac{A_i}{\phi_a V}
  \shortintertext{where}
  \Sigma_i =& \mbox{ macroscopic cross section of reaction type $i$,} \nonumber \\
  A_i =& \mbox{ reaction rate of reaction type $i$ from the transport calculation,} \nonumber \\
  V =& \mbox{ volume of the absorber region} = a^2 p.
\end{align}

Fen et al. \cite{fen_modelling_1992} later provided an updated formulation by assuming that the
cross sections are accurate and instead determined the equivalent diffusion coefficient from Eq.
\ref{eq:mecs} and the transport leakage and flux values as follows:
%
\begin{align}
  \frac{1}{D_a} =& 2p\frac{\phi_x-\phi_a}{L}+\frac{2}{p}\frac{\phi_y-\phi_a}{L}
    -\frac{p_x+p_y}{2D_o}+\sqrt{R}
  \shortintertext{where}
  R =& \left(2p\frac{\phi_x-\phi_a}{L}+\frac{2}{p}\frac{\phi_y-\phi_a}{L}+
  \frac{p_x+p_y}{2D_o}\right)^2+4p\left(p_y-p_x\right)\frac{\phi_x-\phi_a}{L}
  \frac{1}{D_o} \nonumber
\end{align}

As illustrated by the implementation in CITATION, \gls{MECS} requires considerable user input on
the super cell configuration to ensure that leakage rates of the transport solution are equivalent
to the leakage rates of the subsequent diffusion solution. This procedure also places some
constraints on the location and size of the absorber node that can be inconsistent with the
nodalization of the rest of the reactor geometry \cite{ougouag_transport_2010}. \gls{MECS} is also
incompatible with reactor geometries which contain control rods that are too close to each other
such that there is not enough distance in between to define neighboring nodes where
flux-equivalence is assumed. Nevertheless, \gls{MECS} has been widely used within the \gls{VSOP}
suite of codes (which contains CITATION) and has been shown to be effective in a number of
\gls{HTGR} studies \cite{fen_modelling_1992, reitsma_evaluating_2003, mulder_neutronics_2020}.

\subsubsection{Response-Based Methods}

Similar to the \gls{MECS}, \textit{response-based methods} rely on response-function-based
transport methods to resolve the flux around absorber regions. In this context, coarse mesh/nodal
response functions relate quantities of interest of an individual node to input values from its
neighboring nodes. For instance, a response function may provide the average nodal flux and the
outgoing partial currents of a node in response to a given set of incident
partial fluxes from its neighboring nodes \cite{ougouag_transport_2010}. Transport solutions are
used to generate sets of response functions characterizing individual coarse meshes which contain
absorber regions. These response functions can be used directly or indirectly as modified boundary
conditions to accurately capture the effects of control rods on the global flux solution.

%% Delete paragraph below if not relevant
%The response-based methods described here are closely related to response-function-based transport
%methods \cite{mosher_incident_2006}, in which every node is characterized by a set of response
%functions as opposed to only generating response functions for nodes that contain absorber regions.
%The main difference lies in the accuracy and/or presence of higher order expansions of the neutron
%phase space distribution in response-function-based transport methods compared to similar
%diffusion methods. Accordingly, response-function-based transport calculations are more accurate as
%they capture more information from the reference heterogeneous transport solution for each node.

Fen et al. \cite{fen_modelling_1992} developed the \gls{RMM} which generates modified boundary
conditions from response functions to treat absorber nodes in whole core diffusion calculations.
The response functions relate the incident partial current on one face of the node to the resultant
outgoing partial currents on all four faces of the same node. To be more precise, each incident
partial current of each neutron energy group on each face may contribute to the outgoing partial
current of any energy group on any face of the same node. The following equation for the response
matrix $A$ encapsulates the response values to be generated from the \gls{RMM}:
%
\begin{align}
  A^{jk}_{nm} =& \frac{J^{+j}_n}{J^{-k}_m} \mbox{ for } j,k=1...G \mbox{ and } n,m=1...N
  \label{eq:rmm} \\
  \shortintertext{where}
  A =& \mbox{ response matrix,} \nonumber \\
  N =& \mbox{ number of spatial intervals along the perimeter of the absorber node,} \nonumber \\
  J^{-k}_m =& \mbox{ incident partial current in energy group $k$ at spatial interval $m$,}
    \nonumber \\
  J^{+j}_n =& \mbox{ outgoing partial current in energy group $j$ at spatial interval $n$}
    \nonumber \\
  &\mbox{ in response to $J^{-k}_m$.} \nonumber
\end{align}

The \gls{RMM} compares favorably against the \gls{MECS} because it captures non-isotropic flux
effects arising from the non-central control rod location within the reactor core
\cite{fen_modelling_1992}. The \gls{RMM} also does not require
meticulously tuning of thick adjacent nodes to obtain equivalent fluxes to apply the \gls{MECS}.
However, both \gls{RMM} and \gls{MECS} involve precalculation procedures which must be rerun if the
absorber region is subjected to significant changes.

Rahnema et al. \cite{rahnema_advanced_2011} later developed an \gls{IDT} method which embeds the
transport correction for absorber regions in the full-core nodal diffusion calculation. Instead of
modified boundary conditions, the \gls{IDT} method generates coupling coefficients which are
morphologically identical to those used in nodal diffusion methods. The response region, which
contains the absorber region, is further subdivided into several nodes. The transport correction
relies on higher order spatial and angular response moments to maintain detailed responses between
adjacent response nodes. Changes in the response region can be modeled by swapping out response
nodes without having to rerun the transport solver to generate new coupling coefficients. The
intra-response region calculations were iteratively coupled to the full-core diffusion calculations
to avoid introducing extra off-diagonal terms which would increase the solve time of an otherwise
tri-diagonal system of nodal diffusion equations. In several verification studies of static
\gls{HTGR} core configurations, the \gls{IDT} method produced similar eigenvalue and flux
distribution results \cite{rahnema_advanced_2011} as the \gls{RMM}. However, they did not
demonstrate the response region swapping that the \gls{IDT} was designed for.

\subsubsection{Ronen Method}

Ronen \cite{ronen_accurate_2004} postulated an alternative formulation for diffusion coefficients
based on neutron currents from transport calculations.
Fick's law of diffusion is valid under three assumptions: the neutron flux gradient is small, the
absorption-to-scattering ratio is small, and scattering sources are isotropic. Therefore, diffusion
theory fails for anisotropic fluxes in and near absorber regions. The Ronen method proposes using
the integral form of the transport equation to derive transport operators for the neutron current
and substituting the values into Fick's first law of diffusion (Eq. \ref{eq:fick}) to obtain
space-dependent diffusion coefficients as follows:
%
\begin{align}
  D(\vec{r},E) =& -\frac{J_{tr}(\vec{r},E)}{\nabla \phi(\vec{r},E)}
  \label{eq:ronen}
  \shortintertext{where}
  J_{tr} =& \mbox{ transport-derived neutron current.} \nonumber
\end{align}

In doing so, the Ronen method provides pointwise corrections to the diffusion equation which
overcome the small flux gradient limitation. Tomatis \& Dall'Osso \cite{tomatis_application_2011}
numerically implemented the Ronen method for a 1D homogeneous slab with two energy groups and
isotropic scattering. Instead of using Eq. \ref{eq:ronen} which showed instabilities near flat flux
regions where the denominator approaches zero, they calculated corrections to the diffusion
coefficients at cell interfaces using the difference between the transport- and diffusion-derived
currents as follows:
%
\begin{align}
  \delta D(x_{i+1/2},E) =& -\delta J(x_{i+1/2},E) \frac{(\Delta x_{i+1}+\Delta x_i)/2}{
  \phi(x_{i+1},E)-\phi(x_i,E)}
  \shortintertext{where}
  x_i =& \mbox{ $i$-th spatial interval,} \nonumber \\
  \delta J(x,E) =& J_{tr}(x,E) - J_D(x,E), \nonumber \\
  \Delta x_i =& \mbox{ size of $i$-th spatial interval.} \nonumber
\end{align}

They derived expressions for the transport operators for $J_{tr}$ in terms of exponential integral
functions and Legendre expansions of the angular flux.
Gross et al. \cite{gross_high-accuracy_2020} extended the derivation of the transport operators to
handle 1D heterogeneous problems in the form of fuel assemblies with fuel, water, and fuel+absorber
regions. Tomatis et al. \cite{tomatis_ronen_2021} developed new numerical implementations for 1D
slab, cylindrical, and spherical geometries by employing probabilistic treatments from the
\gls{CPM} \cite{lewis_computational_1984} for the transport operators. The authors also implemented
a solver acceleration scheme which helped with the poor convergence rate observed in the earlier
studies \cite{tomatis_application_2011, gross_high-accuracy_2020}.

The results from all three studies showed improvements in flux solutions with the Ronen method over
pure diffusion solvers in all of the test cases, particularly for a 1D heterogeneous
\gls{BWR}-based problem with isotropic scattering and strong absorption cross sections
\cite{gross_high-accuracy_2020}. The error in reactivity values for three different \gls{BWR}
configurations differed from the reference $S_16$ transport calculations by at most 62 pcm after
one hundred iterations ($1$ pcm $=10^{-5}$) . However, the demonstrations were limited to simple
1D geometries. Although the authors provided derivations for anisotropic scattering, their test
cases incorporated only isotropic scattering. The derivation of semi-analytic transport operators
for more complex reactor geometries would be a much more complicated endeavor. Furthermore, without
an accompanying solver acceleration scheme, poorly converged solutions retained significant
discrepancies near material interfaces and vacuum boundaries.

\subsubsection{Averaged Eddington Factors and High-Order Empirical Diffusion Coefficients}

In a similar vein, Pounders \& Rahnema developed two separate methods for generating
space-dependent diffusion coefficients \cite{pounders_diffusion_2009}. Both methods come with the
same caveat in requiring a priori knowledge of the flux and current. The first method, called the
\gls{AEF} method, relies on the Eddington factor, which is defined as the second angular moment of
the angular flux normalized by its zeroth moment. In 1D, the Eddington factor is given as:
%
\begin{align}
  E_g(z) =& \frac{\int^1_{-1} \mu^2\psi(z,\mu)d\mu}{\int^1_{-1} \psi(z,\mu)d\mu}
\end{align}
%
The $g$ subscript denotes the discrete neutron group index of the multigroup diffusion equations
obtained from discretizing the continuous energy variable as described in Section
\ref{sec:challenges-control-rod}. The Eddington factor features in the fist angular moment of the
1D multigroup transport equations, obtained by multiplying the transport equation throughout by
$\mu=\cos\theta$ and integrating over $\mu=-1$ to $\mu=1$:
%
\begin{align}
  \frac{d\left[E_g(z)\phi_g(z)\right]}{dz} + \Sigma_{t,g}(z)J_g(z) =& \sum^G_{g'=1}
  \Sigma^{g'\rightarrow g}_{s1}(z)J_{g'}(z) \label{eq:bte-1st-mom}
  \shortintertext{where}
  \Sigma^{g'\rightarrow g}_{s1} =& \int^1_{-1} \mu \Sigma_s d\mu \nonumber
\end{align}
%
Assuming the Eddington factor varies slowly in space, we can approximate Eq. \ref{eq:bte-1st-mom}
as:
%
\begin{align}
  E_g(z)\frac{d\bar{\phi}_g(z)}{dz} + \Sigma_{t,g}(z)\bar{J}_g(z) =& \sum^G_{g'=1}
  \Sigma^{g'\rightarrow g}_{s1}(z)\bar{J}_{g'}(z) \mbox{ for } z \in V_i \label{eq:bte-1st-est}
  \shortintertext{where}
  V_i =& \mbox{ a subvolume of the system domain.}
\end{align}
The overbars distinguish the approximate solutions of Eq. \ref{eq:bte-1st-est} from the true
solution of Eq. \ref{eq:bte-1st-mom}. From Eq. \ref{eq:bte-1st-est}, the diffusion coefficient can
be defined as:
%
\begin{align}
  D_g(z) = E_g(z)\left[\Sigma_{t,g}(z)-\sum^g_{g'=1}\Sigma^{g'\rightarrow g}_{s1}(z)
  \frac{\bar{J}_{g'}(z)}{\bar{J}_g(z)}\right]^{-1} \label{eq:diffcoef-edd}
\end{align}
By further assuming that the Eddington factors are piecewise constant in each subvolume $V_i$, the
averaged Eddington factor $E^i_g$ can be evaluated as:
%
\begin{align}
  E^i_g =& \frac{E_g(z_{i+1})\phi_g(z_{i+1})-E_g(z_i)\phi_g(z_i)}{\bar{\phi}_g(z_{i+1})-
  \bar{\phi}_g(z_i)}
  \shortintertext{where}
  z_{i+1} =& \mbox{ upper bound of subvolume $V_i$,} \nonumber \\
  z_i =& \mbox{ lower bound of subvolume $V_i$.} \nonumber
\end{align}
%
The diffusion coefficient from Eq. \ref{eq:diffcoef-edd} can then be calculated as:
%
\begin{align}
  D^{AEF}_g(z) =& E^i_g\left[\hat{\Sigma}_{t,g}-\sum^G_{g'=1}\hat{\Sigma}^{g'\rightarrow g}_{s1}
  \frac{\hat{J}_{g'}}{\hat{J}_g}\right]^{-1}
  \shortintertext{where}
  \hat{\Sigma}_{t,g} =& \frac{\int_{V_i}\Sigma_{t,g}(z)J_g(z)dz}{\int_{V_i}\bar{J}_g(z)dz},
  \nonumber \\
  \hat{\Sigma}^{g'\rightarrow g}_{s1} =& \frac{\int_{V_i}\Sigma^{g'\rightarrow g}_{s1}(z)J_{g'}(z)
  dz}{\int_{V_i}\bar{J}_{g'}(z)dz}, \nonumber \\
  \hat{J}_g =& \int_{V_i} \bar{J}_g(z)dz. \nonumber
\end{align}

The second method employs a much simpler premise in that Fick's law is assumed to be accurate with
high-order empirical diffusion coefficients that are to be determined. Given a known transport
solution, the following integration holds for a small homogeneous volume $V_i$:
%
\begin{align}
  \frac{1}{V_i}\int_{V_i}J_g(z)dV =& -\frac{1}{V_i}\int_{V_i}D_g(z)\frac{d\phi_g(z)}{dz}dV.
\end{align}
%
Taking $D_g$ to be constant in $V_i$, we may apply divergence theorem to obtain
%
\begin{align}
  \bar{J}_gV_i =& -D^i_g\int_{\partial V_i} \phi_g(z)dA
  \shortintertext{where}
  \bar{J}_g =& \mbox{ average current in $V_i$,} \nonumber \\
  \partial V_i =& \mbox{ bounding surface of $V_i$.} \nonumber
\end{align}
%
Rearranging the terms, we may obtain the empirical diffusion coefficient formulation from the
transport-derived flux and current as follows:
%
\begin{align}
  D^i_g =& -\frac{\left(z_{i+1}-z_i\right) \bar{J}_g}{\left[\phi_g(z_{i+1})-\phi_g(z_i)\right]}
\end{align}
%
Both \gls{AEF} and empirical methods require the respective subvolumes $V_i$ to be small enough
for the assumptions of constant Eddington factors and diffusion coefficients to hold within $V_i$.

Both methods performed much better than with conventional diffusion coefficients derived using the
$P_1$ approximation method. For the same 1D heterogeneous \gls{BWR} problem demonstrating the prior
Ronen method \cite{gross_high-accuracy_2020} but with anisotropic scattering, the reactivity errors
were around 10 pcm after eliminating errors associated with energy group condensation. The error
values compare favorably with the 16-68 pcm error with the $P_1$ method. The \gls{AEF} and
empirical methods also reproduced the flux distributions better with a maximum pointwise error of
2\% as opposed to 5\% from the $P_1$ method. Similar to the Ronen method, the \gls{AEF} and
empirical methods introduce pointwise corrections with information from transport methods. However,
the present methods require a priori knowledge of the true solution or otherwise accurate estimates
from transport methods, whereas the Ronen method relies on analytical transport operators which use
diffusion flux estimates from the previous iteration to update the flux solution. Nevertheless, the
\gls{AEF} and empirical methods provide the foundation for further development of practical,
self-closing transport correction techniques.

\subsubsection{General Equivalence Theory and Superhomogenization Method}

As mentioned in Section \ref{sec:challenges-control-rod}, coarse-mesh and nodal methods homogenize
heterogeneities in reactor geometries to reduce computational costs of running full-core
simulations. Equivalence techniques which reduce spatial homogenization error are also effective
for accurately modeling the worths of control rods within fuel assemblies. The \gls{GET}
\cite{koebke_new_1980, smith_nodal_1983} and \gls{SPH} \cite{kavenoky_sph_1978,
hebert_consistent_1991} methods represent the most widely used equivalence methods for improving
the performance of diffusion calculations in homogenized \gls{LWR} models. Both methods involve
deriving additional homogenization parameters from single-assembly transport calculations with
reflective boundary conditions. Since the transport calculation step is already a prerequisite step
for generating homogenized group constants, the equivalence methods are simple to implement and
impose reasonably small additional computational costs.

Koebke \cite{koebke_new_1980} first proposed abandoning continuous surface fluxes in favor of
preserving net surface currents through discontinuity factors. Smith \cite{smith_assembly_1986}
later extended this concept for assembly-homogenized calculations. The discontinuity factors are
calculated for each face of the homogenized region to preserve volumetric reaction rates and
surface neutron currents. The discontinuity factors are calculated as follows:
%
\begin{align}
  DF =& \frac{\phi^{het,sur}}{\phi^{hom,sur}}
  \shortintertext{where}
  DF =& \mbox{ discontinuity factor,} \nonumber \\
  \phi^{het,sur} =& \mbox{ surface flux of the region from the heterogeneous calculation,}
  \nonumber \\
  \phi^{hom,sur} =& \mbox{ surface flux of the region from the homogeneous calculation.}
  \nonumber
\end{align}
%
\gls{GET} was later extended for fine mesh calculations \cite{yamamoto_cell_2004} which refers to
cell-level homogenization; each fuel cell in an assembly, consisting of the fuel pellet, cladding,
and moderator, is individually homogenized as opposed to lumping together the entire assembly.

On the other hand, the \gls{SPH} method, first proposed by Kavenoky \cite{kavenoky_sph_1978} for
irregular lattices and later applied as a cell-homogenization technique by Hebert
\cite{hebert_consistent_1991}, introduces correction factors to homogeneous cross sections as
follows:
%
\begin{align}
  \tilde{\Sigma}^{hom}_k =& \mu_k \Sigma^{hom}_k
  \shortintertext{where}
  \tilde{\Sigma}^{hom}_k =& \mbox{ \gls{SPH}-corrected homogeneous cross section for region $k$,}
  \nonumber \\
  \mu_k =& \mbox{ \gls{SPH} correction factor for region $k$,} \nonumber \\
  \Sigma^{hom}_k =& \mbox{ uncorrected homogeneous cross section for region $k$.} \nonumber
\end{align}
%
The \gls{SPH} factor is calculated as follows:
%
\begin{align}
  \mu_k =& \frac{\bar{\phi}^{het}_k}{\phi^{hom}_k}
  \shortintertext{where}
  \bar{\phi}^{het}_k =& \mbox{ average flux in region $k$ from the heterogeneous calculation,}
  \nonumber \\
  \phi^{hom}_k =& \mbox{ average flux in region $k$ from the homogeneous calculation}
  \nonumber \\
                & \mbox{ with \gls{SPH}-corrected cross sections.} \nonumber
\end{align}

Both \gls{GET} and the \gls{SPH} method are designed to preserve reaction rates at the assembly
level \cite{yamamoto_cell_2004}. Given that the \gls{SPH} method introduces only one correction
factor per cell, it only preserves the average net current across all surfaces as opposed
to the net current at each surface with \gls{GET}. Similarly, the isotropic nature of \gls{SPH}
factors lead to worse pin-power estimates near control rods and reflectors compared to \gls{GET}.
However, the \gls{SPH} method is simpler to implement in the diffusion solver as the \gls{SPH}
factors can be precombined with the cross sections to generate \gls{SPH}-corrected cross sections.
\gls{GET} requires diffusion solvers which allow for flux discontinuities at the interfaces.
Furthermore, \gls{GET} requires more memory to store up to six discontinuity factors, one for each
mesh surface, in 3D calculations.

Modeling control rods in full-core calculations with the \gls{SPH} method or \gls{GET} provides
better estimates of the multiplication factor and the flux distribution than reference diffusion
solutions. In this regard, the correction factors do not distinguish between errors arising from
homogenization or the diffusion approximation. However, the improved solutions, especially with the
\gls{SPH} method, can still deviate significantly from reference transport calculations. A super
cell, similar to the one adopted by \gls{MECS}, can be used to reduce errors within and near
control rod cells \cite{ortensi_newton_2018}.

\glsresetall

\chapter{Description of Moltres and Assessment of its Current Capabilities}
\label{chap:moltres}
This chapter provides an abridged overview of Moltres as a multiphysics
simulation software for molten salt reactors. 
Section \ref{sec:moltres-features}
describes general software features of Moltres, Section
\ref{sec:moltres-physics} expands on the physics models in Moltres, and Section
\ref{sec:moltres-previous} describes the current capabilities in Moltres.

\section{General Features} \label{sec:moltres-features}

This section discusses the general software features of Moltres. The
discussion specifically focuses on robustness, extensibility, and ease of use
as these characteristics represent the hallmarks of a good multiphysics
software \cite{keyes_multiphysics_2013}. These criteria also apply for
assessing \gls{MSR} simulation tools since the nonlinearity of \gls{MSR}
multiphysics analysis and the complexity of advanced reactor designs
necessitate robust, scalable, and flexible computational tools.

Moltres draws many advantages from being developed on MOOSE
\cite{permann_moose_2020}. MOOSE is an open-source, finite-element,
multiphysics framework developed at \gls{INL}. The framework provides a
user-friendly interface for developing multiphysics software through
\gls{OOP} in \texttt{C++} to modularize various
functions relevant to finite-element, multiphysics solvers. In this approach,
MOOSE and MOOSE-based applications break down \glspl{PDE} into individual terms
and store them as individual \texttt{C++ objects} referred to as
\texttt{Kernels}. These \texttt{Kernels} contain functions for calculating
their weak form residual and Jacobian
contributions and other relevant functions required to solve a given
\gls{PDE}. \gls{OOP} in MOOSE simplifies software development
since developers can write new \texttt{Kernels} as child classes in
\texttt{C++} derived from existing \texttt{Kernels} (base classes) which share
similar physics to inherit common functions.
The same philosophy applies for all other systems in MOOSE such as
the \texttt{BoundaryCondition}, \texttt{Materials}, and \texttt{Postprocessor}
systems for handling relevant boundary conditions, material properties, and
postprocessing calculations, respectively. Overall, this approach also saves
researchers time and effort as they are unencumbered by the technical details
and complexities involved in programming efficient computational tools for
numerical analysis.

\begin{figure}[htb!]
	\centering
	\includegraphics[width=.7\columnwidth]{moose}
	\caption{Structure of MOOSE and its dependencies.}
	\label{fig:moose}
\end{figure}

MOOSE relies on two other open-source libraries: libMesh
\cite{kirk_libmesh_2006} for its \gls{FEM} capabilities on
unstructured mesh, and PETSc \cite{satish_petsc_2019} for its non-linear
solvers and preconditioning routines. By extension, Moltres gains access to
these libraries for their \gls{PDE} solving capabilities. Figure
\ref{fig:moose} shows how MOOSE serves as an interface between physics
applications and libMesh/PETSc. MOOSE supports
modeling on up to three-dimensional (3-D) unstructured meshes for a wide range
of mesh file formats, including the commonly used Exodus II file format. For
2-D meshes, users can opt between Cartesian and polar RZ coordinates. MOOSE
also supports parallel computing through the \gls{MPI} library to leverage
modern high-performance computing for large multiphysics simulations.

Moltres benefits from the highly-integrated cross compatibility within the
ecosystem of MOOSE-based applications. MOOSE facilitates multiphysics coupling
among all MOOSE-based applications by providing a common framework for shared
data access and file input/output, thus eliminating computational costs from
data transfers and allowing for fully coupled solves. For example, Moltres
couples with the \texttt{Navier-Stokes} module \cite{peterson_overview_2018}
from MOOSE for fully coupled reactor simulations modeling neutronics and
thermal-hydraulics with incompressible flow. Section
\ref{sec:challenges} highlighted the advantages of using fully coupled schemes
for modeling strongly coupled systems such as the coupled
neutronics and thermal-hydraulics in \glspl{MSR}. Moltres can also
couple to other MOOSE-based applications in a similar fashion with ease. In
addition, MOOSE
provides the option for either tight or loose coupling through the
\texttt{MultiApp} system \cite{gaston_physics-based_2015}. Tight coupling
schemes can outperform fully coupled schemes in weakly coupled systems in which
the computational expenses of fully coupled schemes outweigh the savings from
running fewer Newton iterations due to the superior convergence rate. Loose
coupling schemes are useful for accelerating time-dependent simulations of
stable systems towards steady state in which only the steady-state
configuration is of interest to the user. This typically occurs in the later
stages \gls{MSR} simulations when the delayed neutron precursor concentrations
converge slowly due to their relatively large decay half-lives. Furthermore,
segregated solves through the \texttt{MultiApp} system enables Moltres to
introduce delayed neutron precursor drift and non-uniform temperature
distributions into criticality search simulations. Regardless of which
coupling scheme is best, MOOSE-based applications provide the flexibility
to switch among the schemes as users see fit. For time-dependent simulations,
MOOSE provides more than ten different implicit and explicit timestepping
schemes. The default, most common scheme is the first-order backward Euler
method which offers excellent solver stability for stiff \glspl{PDE}.

Lastly, Moltres is an open-source \gls{LGPL} software hosted on
GitHub \cite{github_build_2017}. Open-sourcing software provides ease of access
and expands the userbase. These characteristics promote software quality
through increased feedback on users' needs and transparency for peer review.
Open-source software accelerate research progress by supporting research
collaboration and sharing best software practices. Supporting Moltres'
continued development, Moltres relies on GitHub for online version control with
continuous integrated testing to protect its existing capabilities.

In summary, Moltres provides robust, yet flexible coupling capabilities to
model strongly coupled neutronics and thermal-hydraulics in \glspl{MSR}. As a
MOOSE-based application, Moltres is highly extensible by means of coupling to
other MOOSE-based applications and benefits from MOOSE's user-friendly
interface for software development and general ease of use.

\section{Physics Models} \label{sec:moltres-physics}

This section describes the various physics models available in Moltres to model
coupled neutronics and thermal-hydraulics in \glspl{MSR}.

\subsection{Neutronics}

\subsubsection{Multigroup Neutron Diffusion Equations}

Moltres solves the multigroup neutron diffusion equations for the neutron
flux solution within the problem domain. These equations are derived from the
neutron transport
equation in the diffusion-dominated limit with Fick's law of diffusion and
further simplified by discretizing the continuous neutron energy variable into
a finite number of energy groups. The time-dependent multigroup neutron
diffusion equations with $G$ energy groups and $I$ delayed neutron precursor
groups are given by:
%
\begin{align}
    \frac{1}{v_g} \frac{\partial \phi_g}{\partial t} =& \nabla \cdot D_g
    \nabla \phi_g - \Sigma^r_g \phi_g +
    \sum^G_{g' \neq g} \Sigma^s_{g' \rightarrow g} \phi_{g'} \nonumber \\
    &+ \chi^p_g \sum^G_{g'=1} \left( 1-\beta \right) \nu \Sigma^f_{g'}
    \phi_{g'} + \chi^d_g \sum^I_i \lambda_i C_i \label{eq:neutron} \\
    %
    \shortintertext{where}
    v_g =& \text{ average speed of neutrons in group $g$,} 
    \nonumber \\
    \phi_g =& \text{ neutron flux in group $g$,}
    \nonumber \\
    t =& \text{ time,} \nonumber \\
    D_g =& \text{ diffusion coefficient of neutrons in} \nonumber \\
    &\text{ group $g$,} \nonumber \\
    \Sigma^r_g =& \text{ macroscopic cross section for removal of} \nonumber \\
    &\text{ neutrons from group $g$,} \nonumber \\
    \Sigma^s_{g' \rightarrow g} =& \text{ macroscopic cross section of
    scattering from} \nonumber \\
    &\text{ groups $g'$ to $g$,} \nonumber \\
    \chi^p_g =& \text{ prompt fission spectrum for neutrons in} \nonumber \\
    &\text{ group $g$,} \nonumber \\
    G =& \text{ total number of discrete neutron groups,} \nonumber \\
    \nu_g =& \text{ average number of neutrons produced per} \nonumber \\
    &\text{ fission,} \nonumber \\
    \Sigma^f_{g} =& \text{ macroscopic fission cross section for neutron}
    \nonumber \\
    &\text{ in group $g$,} \nonumber \\
    \chi^d_g =& \text{ delayed fission spectrum for neutrons in} \nonumber \\
    &\text{ group $g$,} \nonumber \\
    I =& \text{ total number of delayed neutron precursor} \nonumber \\
    &\text{ groups,} \nonumber \\
    \beta =& \text{ total delayed neutron fraction.} \nonumber
\end{align}

In spite of forming only around 0.7\% of all neutrons emitted, delayed neutrons
play outsized roles in reactor kinetics. The relatively long half-lives of
delayed neutron precursors gives us ample time in adequately designed reactors
to control reactor power output and intervene in case of power excursions.
The precursor concentration balance equations for $I$ precursor
groups are given by:
%
\begin{align}
    \frac{\partial C_i}{\partial t} =& \beta_i \sum^G_{g'=1} \nu \Sigma^f_{g'}
    \phi_{g'} - \lambda_i C_i - \vec{u} \cdot \nabla C_i + \nabla \cdot
    D_{\text{P}} \nabla C_i \label{eq:precursor} \\
    %
    \shortintertext{where}
    \beta_i =& \text{ delayed neutron fraction of precursor group $i$,}
    \nonumber \\
    \lambda_i =& \text{ average decay constant of delayed neutron} \nonumber \\
    &\text{ precursors in precursor group $i$,} \nonumber \\
    C_i =& \text{ concentration of delayed neutron precursors in}
    \nonumber \\
    &\text{ precursor group $i$,} \nonumber \\
    \vec{u} =& \text{ molten salt flow velocity vector,}
    \nonumber \\
    D_{\text{P}} =& \text{ effective diffusion coefficient of the delayed}
    \nonumber \\
    &\text{ neutron precursors.} \nonumber
\end{align}

These two equations are largely similar to conventional formulations of the
multigroup neutron diffusion equations with delayed neutrons for most reactor
types. The only differences are in the last two terms in Equation \ref{eq:dnp}
which represent the advection and diffusion terms, respectively, to model the
movement of \gls{DNP} in liquid-fuel \glspl{MSR}.

As shown in Equations \ref{eq:neutron} and \ref{eq:precursor}, Moltres requires
group constant data from dedicated high-fidelity neutronics software such as
the NEWT module in SCALE \cite{dehart_reactor_2011}, Serpent
\cite{leppanen_serpent_2014}, or OpenMC \cite{romano_openmc:_2015}. These group
constant data are the neutron energy group $g$ values for $v_g$, $D_g$,
$\Sigma^r_g$, $\Sigma^s_{g' \rightarrow g}$, $\chi^p_g$, $\chi^d_g$,
$\Sigma^f_{g}$, and $\nu\Sigma^f_{g}$, and precursor group $i$ values for
$\beta_i$ and $\lambda_i$. Users
can run a Python script in Moltres' Github repository which automatically reads
user-provided SCALE/Serpent/OpenMC output data files and creates
Moltres-compatible JSON format files containing all required group constant
data. Moltres allows for an arbitrary number of neutron energy groups $G$ and
precursor groups $I$ as long as the user provides the necessary group constant
data. In practice, $I$ depends on the nuclear data library used to generate
group constants---the JEFF \cite{plompen_joint_2020} and ENDF
\cite{brown_endfb-viii0_2018} data libraries define eight precursor groups
and six precursor groups, respectively.

In multiphysics reactor simulations, we model the coupling between neutronics
and thermodynamics through temperature-dependent group constants. To sample
group constants at different temperatures in Moltres, users must provide group
constant data measured at more than one temperature (e.g. 800K--1500K at 100K
intervals). Users can then choose from linear spline, cubic spline, or monotone
cubic interpolation methods available in Moltres to interpolate the group
constant data for values falling within the provided temperature range. 

\subsubsection{Boundary Conditions}

Moltres provides two types of boundary conditions for neutron fluxes; these are
conventionally known as the vacuum and reflective boundary conditions given,
respectively, as:

\begin{align}
    \frac{\partial \phi}{\partial x_i} + \frac{\phi}{2} =& 0
    \shortintertext{and}
    \frac{\partial \phi}{\partial x_i} =& 0
    \shortintertext{where}
    x_i =& \mbox{ spatial coordinate perpendicular to the boundary.} \nonumber
\end{align}

The vacuum boundary condition typically applies to the external boundaries of
the reactor beyond which lies low-interaction media such as air, while the
reflective boundary condition is useful for exploiting symmetries in the
model geometry such as the axial boundary in axisymmetric geometries. The
reflective boundary condition is also equivalent to the more generally known
homogeneous Neumann boundary condition. Relevant boundary conditions for
delayed neutron precursors include the homogeneous Neumann boundary condition
along fuel salt-structural interfaces; and outflow/inflow boundary
conditions along the outlet/inlet boundaries through which the precursors
flow as they circulate the fuel salt loop.

\subsection{Thermal-Hydraulics}

\subsubsection{Temperature Advection-Diffusion}

\subsubsection{Incompressible Navier-Stokes Flow}

\subsubsection{Boundary Conditions}

\subsection{Precursor Loop}

\section{Previous Work} \label{sec:moltres-previous}

\subsection{MSRE}

\subsection{MSFR}

\subsection{CNRS Benchmark}

%\section{Motivation}

A liquid-fueled \gls{MSR} is a class of advanced nuclear reactor which has fissile material
dissolved in a molten salt mixture. This fuel salt mixture doubles as the primary reactor coolant
for transferring heat from the reactor core to the primary heat exchangers. Figure \ref{fig:msr}
shows a schematic diagram of a thermal-spectrum, channel-type \gls{MSR}. In channel-type
\glspl{MSR}, the fuel salt flows through vertical channels in the reactor core next to neutron
moderators (e.g., graphite, heavy water, molten sodium hydroxide). 
%
\begin{figure}[htb!]
	\centering
	\includegraphics[width=.7\columnwidth]{msr}
	\caption{Schematic diagram of the \gls{MSR} concept. Retrieved from
	\cite{doe_technology_2002}.}
	\label{fig:msr}
\end{figure}

The \gls{MSR} is one of six advanced reactor designs selected at the \gls{GIF} for improved safety,
sustainability, efficiency, and cost over the current generation of predominantly \glspl{LWR}.
Due to the high thermal expansion coefficient, \glspl{MSR} possess an inherently robust
safety feature in the strong negative fuel temperature coefficient of
reactivity \cite{elsheikh_safety_2013}. This reactivity coefficient limits the
maximum temperature that the reactor core would experience in an accident
scenario such as an unprotected reactivity insertion because the subsequent
rise in core temperatures induces a significant drop in reactivity which
quickly neutralizes the initial reactivity insertion. \glspl{MSR} also
operate at a large thermal margin to boiling and can rely on natural
circulation in the event of a pump failure. As a last resort, many \gls{MSR}
designs incorporate a drain plug consisting of actively-cooled frozen salt, which
melts when the core temperatures exceed safety thresholds. The hot molten salt
in the core would then flow down into a drain tank designed to hold the fuel salt in
a subcritical configuration to disrupt any further chain fission reactions.

Some \glspl{MSR} like the \gls{MSBR} or the \gls{MSFR} can
incorporate the thorium fuel cycle for improved sustainability arising from the
use of abundant natural thorium resources and reduced transuranic waste
\cite{heuer_towards_2014}. The latter consequence also reduces costs
associated with long-term nuclear waste storage. In addition, the ability to
operate near atmospheric pressures eliminates the need for a thick pressure
vessel and drives down construction costs, while online fuel reprocessing
reduces reactor downtime during reactor operation \cite{dolan_1_2017}.
We can make further economic arguments supporting \glspl{MSR} in the context of the
carbon-constrained future envisioned in \gls{IEA}'s \gls{NZE} roadmap
\cite{iea_net_2021}. The roadmap for minimizing carbon emissions requires solar \gls{PV}- and
wind-dominated energy markets, which can bring about highly variable electricity generation. The
resulting volatility in electricity prices
encourages the construction of heat storage and peak power
production plants. At the same time, demand for carbon-neutral
fuel will rise as electrification is economically unfeasible
for some industries such as the aviation and marine sectors, which depend on
energy-dense fuels for propulsion. As described by Forsberg
\cite{forsberg_market_2020}, the most cost-efficient options for the
aforementioned resources (heat storage, peak power
production, and hydrogen fuel) all require high-temperature heat.
This requirement favors \glspl{MSR} which are expected to deliver heat at higher average
temperatures than \glspl{LWR} and other high-temperature advanced reactors.
\glspl{MSR} may also possess an edge over other high-temperature advanced reactors from
synergistic benefits of siting molten salt heat storage facilities near the power generation
facility.

\subsection{Past \& Present \gls{MSR} Research \& Development}

\gls{ORNL} researchers first conceived the \gls{MSR} concept in pursuit of a high-temperature
liquid fuel reactor for the US Aircraft Nuclear Propulsion program in
the 1950s \cite{rosenthal_molten-salt_1970}. They
built the first ever operational \gls{MSR}, the 2.5 MW$_{\text{th}}$
\gls{ARE} reactor at \gls{ORNL}. The successful demonstration of the \gls{ARE} spurred further
research into adapting \glspl{MSR} for civilian power generation
\cite{rosenthal_molten-salt_1970}. Continued \gls{MSR} research efforts culminated in the design,
construction, and successful operation of the 8-MW$_{\text{th}}$, thermal-spectrum \gls{MSRE} with
graphite channels and a LiF-BeF$_2$-ZrF$_4$-UF$_4$ fuel salt mixture
\cite{haubenreich_experience_1970}. In addition to other operational achievements, the
\gls{MSRE} became the first reactor to run on $^{233}$U fuel bred from $^{232}$Th. Building on
their experience with the \gls{MSRE}, \gls{ORNL} proposed a new program for the construction and
opeartion of a demonstration reactor based on the \gls{MSBR} concept that they had developed
\cite{macpherson_molten_1985}. The \gls{MSBR} is a thermal-spectrum \gls{MSR} with fertile
$^{232}$Th isotopes mixed directly into the \gls{FLiBe} molten salt for $^{233}$U breeding. Like
the \gls{MSRE}, the \gls{MSBR} was to rely on continuous online salt reprocessing to add fertile
material and remove fission product neutron poisons.
However, the \gls{MSBR} project was canceled prior to the demonstration stage in
favor of the \gls{LMFBR}, which had benefited from a head start in development and stronger
political backing \cite{macpherson_molten_1985}.

Following a relative lull lasting until the late 1990s, renewed research efforts and the \gls{GIF}
provided new impetus for \gls{MSR} research and development. As of the end of 2022, numerous
\gls{MSR} designs exist at various stages of development. Leading \gls{MSR} designs, in terms of
development, licensing, and/or demonstration, include the \gls{MSFR} \cite{merle_optimized_2007},
\gls{MCFR} \cite{terrapower_terrapower_2021}, TMSR-LF1 \cite{zhang_review_2018}, and \gls{IMSR}
\cite{leblanc_18_2017}. The \gls{MSFR} is a fast-spectrum breeder reactor developed through
collaborative efforts from European institutes with funding support from the
European Union. Figure \ref{fig:msfr} shows a schematic diagram of the \gls{MSFR}. As opposed to
the multi-channel design of the \gls{MSRE} and \gls{MSBR}, the
\gls{MSFR} core consists of a large molten salt pool without graphite moderators to avoid
frequent graphite replacements and positive graphite temperature reactivity feedback. The
\gls{MCFR} is a similar pool-type reactor under active development at TerraPower. TerraPower and
Southern Company embarked on a joint project to design, construct, and operate a prototype
\gls{MCRE} design with funding support from \gls{DOE}'s \gls{ARDP}. \gls{CAS} launched the
\gls{TMSR} program in 2011 to develop and construct both solid-fueled and liquid-fueled \gls{TMSR}
designs \cite{zou_research_2019}. They finished construction of TMSR-LF1, a 2-MW$_{\text{th}}$
liquid-fueled prototype, in August 2021 and received approval for reactor commissioning in August
2022. Lastly, Canada-based Terrestrial Energy is also developing their \gls{IMSR}, a small modular
\gls{MSR} based on the \gls{MSRE}, which passed a joint technical review carried out by the
Canadian and US nuclear regulators in July 2022.
%
\begin{figure}[htb!]
	\centering
	\includegraphics[width=.7\columnwidth]{msfr}
	\caption{Schematic diagram of the \gls{MSFR}. Retrieved from 
	\cite{allibert_7_2016}.}
	\label{fig:msfr}
\end{figure}

\subsection{\gls{MSR} Modeling \& Simulation}

With the renewed global interest in \glspl{MSR}, \gls{MSR} modeling software play an important role
in supporting \gls{MSR} development.
Accurate reactor modeling capabilities are important because they
accelerate reactor design and optimization by enabling quicker iteration through numerous design
changes. \gls{MSR} modeling software are also essential tools in reactor safety analysis and
licensing efforts as reactor developers must demonstrate and verify the \gls{MSR} systems perform
as designed and remain safe under various accident scenarios.

While modeling \glspl{MSR} is not necessarily more difficult than modeling
solid-fueled reactors, we must adapt our software tools to accurately model the
unique phenomena found in these circulating-fuel reactors. The differences in
the challenges of simulating \glspl{MSR} compared to solid-fueled reactors stem
mainly from the liquid fuel form of the fuel salt \cite{diamond_phenomena_2018,
huff_identifying_2019}.

Liquids generally exhibit greater thermal
expansion per unit change in temperature than solids. A decrease in density of
the fuel medium increases the likelihood of neutrons escaping the fuel region
and being absorbed by non-fissile material elsewhere in the reactor.
Consequently, combined with the temperature-dependent Doppler broadening of
resonance capture cross sections, \glspl{MSR} possess stronger negative fuel
temperature reactivity feedback than their solid-fueled counterparts
\cite{elsheikh_safety_2013}. These
phenomena ultimately result in strong interactions between the neutron fluxes
and core temperatures given that neutron fluxes affect core temperatures
through fission heat generation and core temperatures in turn affect neutron
fluxes through the mechanisms as described prior.

With the fuel
salt also serving the role of providing cooling in the core, velocity flow
profiles in the fuel salts strongly impact the temperature distribution via
advection-dominated heat transfer \cite{diamond_phenomena_2018}. This contrasts
with the relatively static temperature profiles in fuel pins and
other forms of solid fuel matrixes, which are physically separate from the coolant.

\glspl{DNP} flow freely within the primary coolant loop as opposed to
being held in place as in solid-fueled reactors. Thus, the delayed neutron
source distribution varies significantly depending on the flow profile and
velocity. In addition, the reactor loses some delayed neutrons from out-of-core
\gls{DNP} decay. These delayed neutrons are considered lost as they're emitted
in subcritical regions and are unlikely to contribute to further fission
reactions in the active core. The reduced delayed neutron fraction in the core
contributes to a greater prompt power spike following a reactivity insertion
event compared to solid-fueled reactors, absent any temperature reactivity
feedback.

Molten-salt flow along various parts of the coolant loop may fall within the turbulent flow
regime, characterized by chaotic eddies, vortices, and other flow instabilities.
Turbulent flow effects further complicate multiphysics interactions of flow with the temperature
and \gls{DNP} distribution. Turbulent flow effects contribute significantly to advection-dominated
heat and particle transfer in molten salt systems, thereby causing enhanced mixing. Therefore,
multiphysics software for \gls{MSR} analysis require adequate flow modeling capabilities with
support for \gls{DNP} drift and some form of turbulence modeling.

\subsection{Moltres for Multiphysics \gls{MSR} Analysis}

Moltres \cite{lindsay_moltres_2017} is an open-source multiphysics reactor simulation software
developed specifically with the considerations for \gls{MSR} characteristics in mind. Moltres is
built on the \gls{MOOSE} \cite{permann_moose_2020} open-source finite-element framework,
which facilitates multiphysics coupling between different
\gls{MOOSE}-based and \gls{MOOSE}-wrapped applications. The \gls{MOOSE} framework also provides
Moltres with advanced meshing and numerical solver capabilities through interfacing with libMesh
\cite{kirk_libmesh_2006} and PETSc \cite{satish_petsc_2019} open-source libraries. Therefore,
Moltres supports up to 3-D unstructured meshes, scales well on high-performance computing systems,
and provides a flexible multiphysics coupling system, which can be tailored for the type of
system being analyzed.

Moltres models coupled neutronics and thermal-hydraulics in reactors. While
generally applicable to most reactor concepts, much of
Moltres' development focuses on meeting the needs of \gls{MSR} multiphysics simulations.
Together with \gls{MOOSE}'s \texttt{Heat}
\texttt{Conduction} and \texttt{Navier-Stokes} \cite{peterson_overview_2018}
modules, Moltres solves the multigroup neutron diffusion
equations, for an arbitrary number of energy and precursor groups, and
thermal-hydraulics equations simultaneously on the same mesh (or separately solved and coupled
through fixed-point iterations if desired).

Lindsay et al. \cite{lindsay_introduction_2018}
demonstrated Moltres' multiphysics \gls{MSR} modeling capabilities with 1D salt
flow in 2D-axisymmetric and 3D models of the \gls{MSRE}. The neutron flux and
temperature distributions agreed qualitatively with legacy
\gls{MSRE} data albeit with some minor quantitative discrepancies due to
simplifications and assumptions in the reactor geometry. I demonstrated Moltres' capabilities for
1) looping of \gls{DNP} drift back into the reactor core, 2) coupling the \gls{DNP}
drift to numerically calculated salt flow profiles within the reactor core,
and 3) a decay heat model to simulate decay heat from fission products, with a 2-D axisymmetric
design of the \gls{MSFR} in my Master's thesis \cite{park_advancement_2020}.

While the Moltres
model of the \gls{MSFR} showed good agreement with other studies in most steady-state and transient
simulation cases, the Moltres model showed significant discrepancies during pump-initiated
transient scenarios in the absence of a proper turbulence model. Instead, the model applied a
uniform eddy viscosity assumption, which proved to be inadequate under non-steady flow. In order to
advance Moltres as a multiphysics simulation software for \gls{MSR} analysis, Moltres requires a
turbulence modeling capability to capture adequate turbulent flow phenomena and its interactions
with other physics present in \glspl{MSR}.

\subsection{\gls{VV} of Multiphysics \gls{MSR} Models and Software}

\gls{VV} of simulation software and models are important steps in simulation software development
\cite{sargent_verification_2010}. Verification is the process of checking whether a software
and its implementation accurately represents the conceptual description and specifications.
Validation is the process of checking whether a model is an accurate representations of the real
world within the range of its intended uses. For reactor software, verification is commonly
performed by comparison with other reactor software designed to run the same type of reactor
simulations. On the other hand, validation is performed by comparing numerical results from
a simulation model to experimental data from the corresponding live test. The validity of a model,
and the results derived from it, depends on the outcome of both verification and validation.

The important multiphysics phenomena in \glspl{MSR} for \gls{VV} are salt flow-induced
\gls{DNP} drift and the strong coupling between neutron flux and temperature advection-diffusion.
Notable \gls{VV} studies on \gls{MSR} modeling include work by Delpech et al.
\cite{delpech_benchmark_2003}, Tiberga et al. \cite{tiberga_results_2020}, Fratoni et al.
\cite{fratoni_molten_2020}, and Brovchenko et al. \cite{brovchenko_neutronic_2019} in relation to
the \gls{MSRE} and the \gls{MSFR} designs.



Delpech et al. \cite{delpech_benchmark_2003}
published one of the first modern \gls{VV} work for \gls{MSR} multiphysics modeling. Collaborators
from six institutions modeled \gls{MSRE} pump start-up, pump coast-down, and natural
circulation transients to assess and validate their models and codes for studying the effects of
salt flow on the reactivity and power. Given the wide range of neutronics methods from
multidimensional Monte Carlo methods to zero-dimensional point reactor kinetics, some deviations
were observed between different codes. Neverthless, all results showed generally good agreement
with \gls{MSRE} experimental data from \gls{ORNL}.

As mentioned in Subsection \ref{sec:msr-tools}, Tiberga et al. \cite{tiberga_results_2020}
published the CNRS benchmark for the verification of \gls{MSR} simulation tools designed for
fast-spectrum \gls{MSR} modeling. In contrast with the multi-channel \gls{MSRE} and its derivative
designs, the CNRS benchmark has a 2 m$\times$2 m problem domain of homogeneous fuel salt mimicking
the large salt pool in fast-spectrum \gls{MSR} designs. The CNRS benchmark consists of three phases
starting with single-physics calculations in Phase 0, followed by problems which gradually
introduce multiphysics coupling in Phase 1, and lastly time-dependent pertubation problems in Phase
2. Thus, the benchmark provides a systematic approach aimed towards helping code developers
identify sources of discrepancies which may otherwise be masked by error cancellation or other
dominant sources of discrepancies. The final steady-state and time-dependent problems involve
studying the effects of natural circulation and lid-driven flow on the reactivity and power output.
Aside from the problem specifications, Tiberga et al. also published the associated group
constant cross section data required by deterministic neutronics solvers to perform neutronics
calculations. Four institutions participated in the benchmarking exercise with neutron diffusion,
$SP_N$, and $S_N$-based solvers for the neutronics calculations. Their measured neutronics and
\gls{TH} parameters showed excellent agreement within up to 2.5\% discrepancy from their combined
average.

Neutronic benchmark studies of the \gls{MSRE} and the \gls{MSFR} by Fratoni et al.
\cite{fratoni_molten_2020} and Brovchenko et al. \cite{brovchenko_neutronic_2019} measured the
delayed neutron losses due to the decay of \glspl{DNP} flowing out of the active core region.
Fratoni et al. sought to establish a standard validation platform for \gls{MSR} neutronics
simulation tools with \gls{MSRE} experimental data for inclusion in the \gls{IRPhEP} handbook.
They characterized and validated a model of the \gls{MSRE} in the Monte Carlo particle transport
code Serpent. On the other hand, the \gls{MSFR} benchmark by Brovchenko et al. featured results
from multiple \gls{MSR} simulation tools by several collaborators. Their assessment found that
the choice of nuclear database for the cross sections and decay data has the most significant
impact on the neutronics results.

While these publications have plugged significant technical gaps, more can be done to develop
open \gls{VV} procedures for \gls{MSR} multiphysics modeling. For instance, the
CNRS benchmark does not assess the loss of delayed neutrons due to the decay of \glspl{DNP} flowing
out of the active core region. This phenomenon is important as the delayed neutron fraction in the
core directly impacts the transient power response in unprotected accident scenarios. Meanwhile,
the neutronic benchmark studies by Fratoni et al. \cite{fratoni_molten_2020} and Brovchenko et al.
\cite{brovchenko_neutronic_2019} did not provide standardized group constant data required by most
deterministic multiphysics \gls{MSR} simulation tools. Therefore, it is difficult to isolate
discrepancies arising from code implementations as opposed to discrepancies from using different
nuclear databases or stochastic uncertainties in Monte Carlo simulations. A good model verification
procedure for delayed neutron loss should ideally provide well-defined problems and the necessary
input data. It is also helpful to perform model verification studies on simpler problems like the
bare homogeneous problem domain in the CNRS benchmark before embarking on more complicated
validation studies which require accurate models of the reference experiments.

\section{Objectives}

This thesis demonstrates latest capabilities of Moltres
\cite{lindsay_introduction_2018}.
In particular, this thesis presents two more recent
developments in Moltres, namely fully integrating \gls{MOOSE}'s incompressible
Navier-Stokes module into Moltres, and introducing a
decay heat model.
The main objective of this thesis is to verify Moltres'
latest capabilities in modeling multiphysics, steady-state, and transient
behavior of fast-spectrum \glspl{MSR} through the study of the \gls{MSFR}
concept. Code-to-code verification is an important exercise in software
development for ensuring that the application produces accurate and reliable
results. This thesis covers the \gls{MSFR} concept mainly because it has been
studied extensively with readily available data in the literature to verify
against. The \gls{MSFR} design also features interesting flow
patterns that greatly affect the steady-state and transient behavior. This
present work will first present a verification of Moltres' \gls{MSFR}
diffusion neutronics against the Monte Carlo neutron transport software
Serpent 2, followed by a verification of
the coupled neutronics/thermal-hydraulics steady-state and accident transient
results against two sets of results published by
Fiorina et al. \cite{fiorina_modelling_2014}. The two sets of results arose
from a collaborative benchmarking exercise by researchers at Politecnico di
Milano and Technical University of Delft with two separate \gls{MSR}
simulation tools. Section \ref{chap:lit} discusses these tools
in greater detail. The
secondary objective is to identify areas of improvement in Moltres for future
development.

\section{Outline}

The outline of this thesis is as follows. Chapter 2 discusses the history and
features of \glspl{MSR}, and a literature review of existing \gls{MSR}
simulation tools. The chapter also covers the \gls{MSFR} concept in greater
detail. Chapter 3 details the software and the general modeling
approach for generating the results in this thesis. Chapter 4 provides a
neutronics assessment by comparing key neutronics parameters from Moltres'
eigenvalue calculations to Serpent's Monte Carlo calculations. Chapter 5
presents steady-state results of coupled neutronics/thermal-hydraulics
\gls{MSFR} simulations in Moltres. Chapter 6 presents accident transient
simulation results for unprotected reactivity insertions, unprotected loss of
heat sink, unprotected loss of flow, and unprotected pump overspeed. Lastly,
Chapter 7 summarizes the key findings in this thesis
and posits some potential avenues for future work.

\section{Description of the CNRS Benchmark} \label{sec:benchmark}

The CNRS Benchmark \cite{tiberga_results_2020} is a numerical
benchmark for multiphysics software dedicated to modeling \glspl{MSR}. It
consists of three phases and eight steps in total. Each
step is a well-defined subproblem for systematically assessing the
capabilities of \gls{MSR} software and pinpointing sources of discrepancies
between software. Phase 0 consists of three single-physics problems in fluid
dynamics, neutronics, and temperature. Phase 1 consists
of four coupled steady-state problems. Lastly, Phase 2 consists of one
coupled, time-dependent problem.

\begin{figure}[htb!]
	\centering
	\includegraphics[width=.6\columnwidth]{cnrs-geometry}
	\caption{2m$\times$2m 2D domain of the CNRS Benchmark. $U_{lid}$
	represents the velocity along the top boundary. For comparison, various quantities are
	measured along the centerlines AA' and BB'. From Tiberga et
	al. \cite{tiberga_results_2020}.}
	\label{fig:cnrs-geometry}
\end{figure}

As shown in Figure \ref{fig:cnrs-geometry}, the domain geometry is a
2m$\times$2m square cavity filled with LiF-BeF$_2$-UF$_4$ molten salt at an
initial temperature of 900K \cite{tiberga_results_2020}.
Standard vacuum boundary conditions apply for neutron flux along all
boundaries whereby outgoing neutrons are considered lost, while homogeneous
boundary conditions apply for delayed neutron precursors. No-slip boundary
conditions apply for velocity variables in the cavity, except along the top
boundary for Steps 0.1, 0.3, 1.1, 1.2, and 1.4, which impose forced flow in the
form of lid-driven
cavity flow. For the temperature variable, all boundaries are insulated, and we
simulate salt cooling with the following volumetric heat sink equation:
%
\begin{align}
    q'''(\vec{r}) &= \gamma \left(900 - T(\vec{r})\right) \label{eq:cnrs-heat}
    \shortintertext{where}
    q''' &= \mbox{volumetric heat sink [W$\cdot$m$^{-3}$],}
    \nonumber \\
    \gamma &= \mbox{heat transfer coefficient [W$\cdot$m$^{-3}\cdot$K$^{-1}$],}
    \nonumber \\
    T(\vec{r}) &= \mbox{temperature at point $\vec{r}$ [K].} \nonumber
\end{align}

Tiberga et al. \cite{tiberga_results_2020} used Serpent 2
\cite{leppanen_serpent_2014} with the JEFF-3.1 library
\cite{koning_jeff-31_2006} to generate multigroup neutronics data for the
LiF-BeF$_2$-UF$_4$ salt in the domain at 900K, which they condensed into six
energy groups and eight precursor groups. We direct readers to their paper for
the group constant data \cite{tiberga_results_2020}. In addition, the
benchmark prescribes the following equations to govern the temperature
dependence in the cross sections and the neutron diffusion coefficients:
%
\begin{align}
    \Sigma_i (T) &= \Sigma_i(T_{ref})
    \frac{\rho_{fuel}(T)}{\rho_{fuel}(T_{ref})}
    \shortintertext{and}
    D (T) &= D(T_{ref})
    \frac{\rho_{fuel}(T_{ref})}{\rho_{fuel}(T)}
    \shortintertext{where}
    \Sigma_i &= \mbox{relevant macroscopic cross section [cm${-1}$],}
    \nonumber \\
    D &= \mbox{neutron diffusion coefficient [cm$^2\cdot$s$^{-1}$],}   
    \nonumber \\
    \rho_{fuel} &= \mbox{density of the fuel salt [kg$\cdot$m$^{-3}$],}
    \nonumber \\
    T_{ref} &= \mbox{reference temperature} = 900\mbox{ K}. \nonumber
\end{align}

The benchmark also prescribes incompressible Navier-Stokes flow with the
Boussinesq approximation for evaluating the salt flow in the
domain but does not restrict the type of neutronics model.
Table \ref{table:benchmark} lists the relevant input parameters and observables.

\begin{table*}[tp!]
	\caption{Input parameters and observables of each benchmark step.}
	\centering
	\footnotesize
    \begin{tabular}{p{.05\textwidth} p{.1\textwidth} p{.3\textwidth} p{.45\textwidth}}
		\toprule
        \textbf{Step} & \textbf{Name} & \textbf{Input parameters} & \textbf{Observables} \\
		\midrule
        0.1 & Velocity field &
		\begin{itemize}[nosep,noitemsep,left=0pt,
		                before={\begin{minipage}[t]{\hsize}},
                        after ={\end{minipage}}]
		    \item $U_{lid} = 0.5$ m$\cdot$s$^{-1}$
		\end{itemize}\vspace*{-\baselineskip}\mbox{} &
		\begin{itemize}[nosep,noitemsep,left=0pt,
		                before={\begin{minipage}[t]{\hsize}},
                        after ={\end{minipage}}]
		    \item Velocity components $(u_x,u_y)$ along AA' and BB'
		\end{itemize}\vspace*{-\baselineskip}\mbox{} \\
        \midrule
        0.2 & Neutronics &
        \begin{itemize}[nosep,noitemsep,left=0pt,
		                before={\begin{minipage}[t]{\hsize}},
                        after ={\end{minipage}}]
		    \item $U_{lid} = 0$ m$\cdot$s$^{-1}$
		    \item $T = 900$ K
		    \item $P = 1$ GW
		\end{itemize} &
		\begin{itemize}[nosep,noitemsep,left=0pt,
		                before={\begin{minipage}[t]{\hsize}},
                        after ={\end{minipage}}]
		    \item Fission rate density $\sum^6_g \Sigma_{f,g} \phi_g(\vec{r})$ along AA'
            \item Reactivity $\rho$
		\end{itemize}\vspace*{-\baselineskip}\mbox{} \\
        \midrule
        0.3 & Temperature &
        \begin{itemize}[nosep,noitemsep,left=0pt,
		                before={\begin{minipage}[t]{\hsize}},
                        after ={\end{minipage}}]
		    \item Fixed flow field from Step 0.1 for
		    $U_{lid} = 0.5$ m$\cdot$s$^{-1}$
		    \item Fixed heat source distribution
		    $\sum^6_{g} \epsilon_g \Sigma_{f,g} \phi_g(\vec{r})$ from Step 0.2
		    \item $\gamma = 10^6$ W$\cdot$m$^{-3}\cdot$K$^{-1}$
		\end{itemize} &
		\begin{itemize}[nosep,noitemsep,left=0pt,
		                before={\begin{minipage}[t]{\hsize}},
                        after ={\end{minipage}}]
		    \item Temperature $T$ along AA' and BB'
		\end{itemize}\vspace*{-\baselineskip}\mbox{} \\
        \midrule
        1.1 & Circulating fuel &
        \begin{itemize}[nosep,noitemsep,left=0pt,
		                before={\begin{minipage}[t]{\hsize}},
                        after ={\end{minipage}}]
		    \item Fixed flow field from Step 0.1 for
		    $U_{lid} = 0.5$ m$\cdot$s$^{-1}$
		    \item $T = 900$ K
		    \item $P = 1$ GW
		\end{itemize} &
		\begin{itemize}[nosep,noitemsep,left=0pt,
		                before={\begin{minipage}[t]{\hsize}},
                        after ={\end{minipage}}]
		    \item Delayed neutron source $\sum^8_i \lambda_i C_i$ along AA' and BB'
		    \item Reactivity change between Step 1.1 and Step 0.2,
		    $\Delta \rho = \rho - \rho_{s_{0.2}}$
		\end{itemize}\vspace*{-\baselineskip}\mbox{} \\
        \midrule
        1.2 & Power coupling &
        \begin{itemize}[nosep,noitemsep,left=0pt,
		                before={\begin{minipage}[t]{\hsize}},
                        after ={\end{minipage}}]
		    \item Fixed flow field from Step 0.1 for
		    $U_{lid} = 0.5$ m$\cdot$s$^{-1}$
		    \item $P = 1$ GW
		    \item $\gamma = 10^6$ W$\cdot$m$^{-3}\cdot$K$^{-1}$
		\end{itemize}\vspace*{-\baselineskip}\mbox{} &
		\begin{itemize}[nosep,noitemsep,left=0pt,
		                before={\begin{minipage}[t]{\hsize}},
                        after ={\end{minipage}}]
		    \item Temperature $T$ along AA' and BB'
            \item Reactivity change between Step 1.2 and Step 1.1,
            $\Delta\rho = \rho - \rho_{s_{1.1}}$
            \item Change in fission rate density
            $\sum^6_g \Sigma_{f,g} \phi_g(\vec{r}) -
            \left[\sum^6_g \Sigma_{f,g} \phi_g(\vec{r})\right]_{s_{0.2}}$
		\end{itemize} \\
        \midrule
        1.3 & Buoyancy &
        \begin{itemize}[nosep,noitemsep,left=0pt,
		                before={\begin{minipage}[t]{\hsize}},
                        after ={\end{minipage}}]
		    \item $P = 1$ GW
		    \item $U_{lid} = 0$ m$\cdot$s$^{-1}$
		    \item $\gamma = 10^6$ W$\cdot$m$^{-3}\cdot$K$^{-1}$
		\end{itemize}\vspace*{-\baselineskip}\mbox{} &
		\begin{itemize}[nosep,noitemsep,left=0pt,
		                before={\begin{minipage}[t]{\hsize}},
                        after ={\end{minipage}}]
		    \item Velocity components $(u_x, u_y)$ along AA' and BB'
            \item Temperature $T$ along AA' and BB'
            \item Delayed neutron source $\sum^8_i \lambda_i C_i$ along AA' and BB'
            \item Reactivity change from Step 0.2
        $\Delta\rho = \rho - \rho_{s_{0.2}}$
		\end{itemize} \\
        \midrule
        1.4 & Full coupling &
        \begin{itemize}[nosep,noitemsep,left=0pt,
		                before={\begin{minipage}[t]{\hsize}},
                        after ={\end{minipage}}]
		    \item $\gamma = 10^6$ W$\cdot$m$^{-3}\cdot$K$^{-1}$
		    \item $P$ variable in the range $[0,1]$ GW with a step of 0.2 GW
		    \item $U_{lid}$ variable in the range $[0,0.5]$ m$\cdot$s$^{-1}$
		    with a step of 0.1 m$\cdot$s$^{-1}$
		\end{itemize} &
		\begin{itemize}[nosep,noitemsep,left=0pt,
		                before={\begin{minipage}[t]{\hsize}},
                        after ={\end{minipage}}]
		    \item Reactivity change between Step 1.4 and Step 0.2,
		    $\Delta\rho = \rho - \rho_{s_{0.2}}$, for all permutations of $P$
		    and $U_{lid}$ values
		\end{itemize}\vspace*{-\baselineskip}\mbox{} \\
        \midrule
        2.1 & Forced convection transient &
        \begin{itemize}[nosep,noitemsep,left=0pt,
		                before={\begin{minipage}[t]{\hsize}},
                        after ={\end{minipage}}]
		    \item $\gamma = 10^6$ W$\cdot$m$^{-3}\cdot$K$^{-1}$
            \item Steady-state solution from Step 1.4 for $U_{lid} = 0.5$
        m$\cdot$s$^{-1}$ and $P = 1.0$ GW
		\end{itemize} &
		\begin{itemize}[nosep,noitemsep,left=0pt,
		                before={\begin{minipage}[t]{\hsize}},
                        after ={\end{minipage}}]
		    \item Power gain and shift as a function of the perturbation frequency
		\end{itemize}\vspace*{-\baselineskip}\mbox{} \\
		\bottomrule
	\end{tabular}
	\label{table:benchmark}
\end{table*}

Step 2.1 studies the transient response of the fully coupled nonlinear system.
Linear perturbation analyses are performed by introducing periodic
perturbations to the heat transfer coefficient $\gamma$ and studying the gain
and phase shift of the response in the total power $P$. For the initial
conditions, the steady-state solution from Step 1.4 with
$U_{lid} = 0.5$ m$\cdot$s$^{-1}$ and $P = 1$ GW is used. This initial
configuration is made exactly critical by scaling the neutron source terms,
from fission and \gls{DNP} decay, by the inverse of the criticality eigenvalue
solution from Step 1.4.

$\gamma$ is uniformly perturbed according to small-amplitude sine waves given
as:
%
\begin{align}
    \gamma =& \gamma_0 \left[ 1 + 0.1\sin\left(2 \pi f \right) \right]
    \shortintertext{where}
    \gamma_0 =& 10^6 \mbox{ W$\cdot$m$^{-3}\cdot$K$^{-1}$}, \nonumber \\
    f \in& \left\lbrace 0.0125, 0.025, 0.05, 0.1, 0.2, 0.4, 0.8 \right\rbrace 
    \mbox{ Hz.} \nonumber
\end{align}

The benchmark defines power gain as:
%
\begin{align}
    \mbox{Power gain} =& \frac{\left(P_{max} - P_{avg}\right)/P_{avg}}{
    \left(\gamma_{max} - \gamma_{avg}\right)/\gamma_{avg}}
\end{align}
%
The subscripts denote the maximum and time-averaged values of $P$ and $\gamma$.

\FloatBarrier

\section{Modeling Approach with Moltres} \label{sec:model}

This section describes the specific modeling approach for
simulating the CNRS Benchmark cases in Moltres.

For this work\footnote{The input files for
all benchmark
cases are available on the Moltres GitHub repository at 
\url{https://github.com/arfc/moltres/tree/devel/problems/2021-cnrs-benchmark}.
}, I ran the benchmark cases on a uniformly-spaced mesh
of 200$\times$200 elements. Thus, the dimensions of each mesh element are
0.01m$\times$0.01m. I adopted the group constant data
provided by Tiberga et al. \cite{tiberga_results_2020}. Next, I
discretized most variables, i.e., neutron fluxes, velocity
components, pressure, and temperature, using continuous, first-order, Lagrange
shape functions. The only exception is the precursor concentration variables,
which I discretized using zeroth-order monomial shape functions and solved
using a \gls{DFEM}. I interpolated the resulting discontinuous,
cell-centered precursor values to obtain the nodal values for results
analysis.

The
\texttt{Navier-Stokes} and \texttt{Heat} \texttt{Conduction} modules from
\gls{MOOSE} provide some of the capabilities for
modeling incompressible flow and heat transfer. In particular, I stabilized
the incompressible flow and temperature governing equations using the
\gls{SUPG} stabilization method implemented in \gls{MOOSE}
\cite{peterson_overview_2018}. Without \gls{SUPG} stabilization, I
observed spurious numerical oscillations in the velocity and temperature near
the top boundary due to the singularity on the top left corner where different
velocity boundary conditions meet. I also applied the \gls{PSPG} stabilization
scheme \cite{hughes_new_1986} from the Navier-Stokes module
\cite{peterson_overview_2018},
which enables equal-order discretizations in the velocity and pressure
variables. Equal-order discretizations with \gls{PSPG} are computationally
cheaper and more convenient than implementing higher-order
velocity discretizations for stability without \gls{PSPG}
\cite{chapelle_inf-sup_1993}.

Using the inverse power method solver in \gls{MOOSE}, I ran all eigenvalue calculations in
Steps 0.2, 1.1, 1.2, 1.3, and 1.4. I ran all other steps
using the Preconditioned Newton-Krylov solver
\cite{gaston_physics-based_2015}. The coupled steady-state problems in
Steps 1.2, 1.3, and 1.4 required segregated solvers for the neutronics
and the thermal-hydraulics due to the unique problem setups involving a
criticality search problem for the neutron multiplication factor
and a steady-state problem in thermal-hydraulics simultaneously.

\begin{table}[tb]
    \caption{Timestep sizes used for the time-dependent cases in
    Step 2.1, corresponding to 1/200th of the perturbation period.}
	\centering
	\setlength\tabcolsep{2.5pt}
	\begin{tabular}{l l l l l l l l}
	    \toprule
	    Frequency [Hz] & 0.0125 & 0.025 & 0.05 & 0.1 & 0.2 & 0.4 & 0.8 \\
	    \midrule
	    Timestep size [s] & 0.2 & 0.2 & 0.1 & 0.05 & 0.025 & 0.0125 & 0.00625
	    \\
	    \bottomrule
	\end{tabular}
	\label{table:timestep}
\end{table}

For the time-dependent cases in Step 2.1, I employed full coupling with
a second-order implicit Backward Differential Formula (BDF2) time-stepping
scheme. I set the timestep sizes for each driving frequency in the heat transfer
coefficient to 1/200th of the perturbation period. Table
\ref{table:timestep} shows the timestep sizes. I assumed the
systems reached asymptotic behavior when the magnitudes of neighboring power
peaks differed by less than 0.001\% for at least ten wavelengths. Under this
assumption, the phase shift measurements between adjacent waves always
converged before the magnitude measurements of the power peaks.

Table \ref{table:software} compares the numerical methods, meshing schemes, and
neutronics models of Moltres and the four participating software packages in
the CNRS benchmark paper \cite{tiberga_results_2020}. The $SP_N$ and
$S_N$ neutronics models refer to the simplified $P_N$ spherical harmonics and
$S_N$ discrete ordinates neutron transport models, respectively. Based on the
solvers and methods of solution, Moltres is most similar to the
PHANTOM-$S_N$ + DGFlows \cite{tiberga_discontinuous_2019} multiphysics package
from \gls{TUD} with the $S_2$ neutron transport model. Participants from
\gls{CNRS} and \gls{PSI}
employed non-uniform meshes which were refined near the boundaries. In contrast,
we and the \gls{PoliMi} and \gls{TUD} participants employed uniform meshes.

\FloatBarrier

\begin{landscape}
\begin{table*}[p]
    \caption{List of software packages and their corresponding model
    specifications for the CNRS Benchmark simulations
    \cite{tiberga_results_2020}.}
    \centering
    \begin{tabular}{p{4.2cm} p{7cm} p{3.3cm} p{2cm} p{2.7cm}}
        \toprule
        Software & Institute & Numerical method & Mesh & Neutronics model \\
        \midrule
        OpenFOAM & Centre national de la recherche scientifique (CNRS) & Finite volume & 200$\times$200 \newline Non-uniform & $SP_1$ \& $SP_3$ \\
        OpenFOAM & Politecnico di Milano (PoliMi) & Finite volume & 400$\times$400 \newline Uniform & Neutron diffusion \\
        GeN-Foam & Paul Scherrer Institute (PSI) & Finite volume & 200$\times$200 \newline Non-uniform & Neutron diffusion \\
        PHANTOM-$S_N$+DGFlows & Delft University of Technology (TUD) & Discontinuous finite \newline element & 50$\times$50 \newline Uniform & $S_2$ \& $S_6$ \\
        Moltres (This work) & University of Illinois at Urbana-Champaign (UIUC) & Continuous \& discontinuous finite element & 200$\times$200 \newline Uniform & Neutron diffusion \\
        \bottomrule
    \end{tabular}
    \label{table:software}
\end{table*}
\end{landscape}

\FloatBarrier

\input{cnrs-benchmark/results}
\section{Conclusion}

\glspl{MSR} feature significant multiphysics interactions which present
computational challenges for many existing multiphysics reactor analysis
software. This chapter presents code-to-code verification of Moltres
capabilities in modeling such multiphysics phenomena in fast-spectrum
\glspl{MSR} based on the CNRS benchmark \cite{tiberga_results_2020}.
The CNRS benchmark assesses multiphysics \gls{MSR} simulation
software through several steps involving single-physics and coupled
neutronics/thermal-hydraulics problems.

The results showed that Moltres is consistent with the participating software
presented in the CNRS benchmark paper for the modeling of important phenomena
in fast-spectrum \glspl{MSR}. The percentage discrepancies in the various
neutronics, velocity, and temperature quantities mostly fall below or within
one standard deviation of the average of the benchmark participants.
Minor deviations in the temperature in Steps 0.3 and 1.2 
stem from the discontinuous velocity
boundaries on the top corners in the lid-driven cavity flow. We have shown that
these deviations are limited to the top boundary of the domain and do not
affect the rest of the physical parameters. The results from
Moltres agree closest with the TUD-S$_2$ software package, which implements the
$S_2$ discrete ordinates method for
neutron transport on a uniform structured mesh with a \gls{DFEM}-based solver.
These features make Moltres the most similar to the TUD-$S_2$ model as compared
to the other models which employ different neutron transport models,
non-uniform meshes, and/or finite volume-based solvers.

This work verifies Moltres' capabilities for future work involving modeling and
simulation of fast-spectrum \glspl{MSR} such as the European \gls{MSFR} and
TerraPower's \gls{MCFR} \cite{terrapower_terrapower_2021}. Notably, the CNRS
benchmark does not assess modeling capabilities for complex physics phenomena
such as turbulent flow in \glspl{MSR}. This worked in favor of verifying
existing capabilities in Moltres since Moltres does not currently support
turbulence modeling. However, we expect coolant loops in many \gls{MSR} designs
will experience turbulent flow under normal operation or accident scenarios.
These expectations, alongside the subpar results of pump-initiated accidents
reported in Section \ref{sec:msfr}, call for the implementation and
verification of a turbulence model in Moltres for accurate modeling of
\glspl{MSR}.

\FloatBarrier

\glsresetall

\chapter{Verification of Moltres for Multiphysics Molten Salt
Reactor Modeling}
\label{chap:benchmark}
\section{Description of the CNRS Benchmark} \label{sec:benchmark}

The CNRS Benchmark \cite{tiberga_results_2020} is a numerical
benchmark for multiphysics software dedicated to modeling \glspl{MSR}. It
consists of three phases and eight steps in total. Each
step is a well-defined subproblem for systematically assessing the
capabilities of \gls{MSR} software and pinpointing sources of discrepancies
between software. Phase 0 consists of three single-physics problems in fluid
dynamics, neutronics, and temperature. Phase 1 consists
of four coupled steady-state problems. Lastly, Phase 2 consists of one
coupled, time-dependent problem.

\begin{figure}[htb!]
	\centering
	\includegraphics[width=.6\columnwidth]{cnrs-geometry}
	\caption{2m$\times$2m 2D domain of the CNRS Benchmark. $U_{lid}$
	represents the velocity along the top boundary. For comparison, various quantities are
	measured along the centerlines AA' and BB'. From Tiberga et
	al. \cite{tiberga_results_2020}.}
	\label{fig:cnrs-geometry}
\end{figure}

As shown in Figure \ref{fig:cnrs-geometry}, the domain geometry is a
2m$\times$2m square cavity filled with LiF-BeF$_2$-UF$_4$ molten salt at an
initial temperature of 900K \cite{tiberga_results_2020}.
Standard vacuum boundary conditions apply for neutron flux along all
boundaries whereby outgoing neutrons are considered lost, while homogeneous
boundary conditions apply for delayed neutron precursors. No-slip boundary
conditions apply for velocity variables in the cavity, except along the top
boundary for Steps 0.1, 0.3, 1.1, 1.2, and 1.4, which impose forced flow in the
form of lid-driven
cavity flow. For the temperature variable, all boundaries are insulated, and we
simulate salt cooling with the following volumetric heat sink equation:
%
\begin{align}
    q'''(\vec{r}) &= \gamma \left(900 - T(\vec{r})\right) \label{eq:cnrs-heat}
    \shortintertext{where}
    q''' &= \mbox{volumetric heat sink [W$\cdot$m$^{-3}$],}
    \nonumber \\
    \gamma &= \mbox{heat transfer coefficient [W$\cdot$m$^{-3}\cdot$K$^{-1}$],}
    \nonumber \\
    T(\vec{r}) &= \mbox{temperature at point $\vec{r}$ [K].} \nonumber
\end{align}

Tiberga et al. \cite{tiberga_results_2020} used Serpent 2
\cite{leppanen_serpent_2014} with the JEFF-3.1 library
\cite{koning_jeff-31_2006} to generate multigroup neutronics data for the
LiF-BeF$_2$-UF$_4$ salt in the domain at 900K, which they condensed into six
energy groups and eight precursor groups. We direct readers to their paper for
the group constant data \cite{tiberga_results_2020}. In addition, the
benchmark prescribes the following equations to govern the temperature
dependence in the cross sections and the neutron diffusion coefficients:
%
\begin{align}
    \Sigma_i (T) &= \Sigma_i(T_{ref})
    \frac{\rho_{fuel}(T)}{\rho_{fuel}(T_{ref})}
    \shortintertext{and}
    D (T) &= D(T_{ref})
    \frac{\rho_{fuel}(T_{ref})}{\rho_{fuel}(T)}
    \shortintertext{where}
    \Sigma_i &= \mbox{relevant macroscopic cross section [cm${-1}$],}
    \nonumber \\
    D &= \mbox{neutron diffusion coefficient [cm$^2\cdot$s$^{-1}$],}   
    \nonumber \\
    \rho_{fuel} &= \mbox{density of the fuel salt [kg$\cdot$m$^{-3}$],}
    \nonumber \\
    T_{ref} &= \mbox{reference temperature} = 900\mbox{ K}. \nonumber
\end{align}

The benchmark also prescribes incompressible Navier-Stokes flow with the
Boussinesq approximation for evaluating the salt flow in the
domain but does not restrict the type of neutronics model.
Table \ref{table:benchmark} lists the relevant input parameters and observables.

\begin{table*}[tp!]
	\caption{Input parameters and observables of each benchmark step.}
	\centering
	\footnotesize
    \begin{tabular}{p{.05\textwidth} p{.1\textwidth} p{.3\textwidth} p{.45\textwidth}}
		\toprule
        \textbf{Step} & \textbf{Name} & \textbf{Input parameters} & \textbf{Observables} \\
		\midrule
        0.1 & Velocity field &
		\begin{itemize}[nosep,noitemsep,left=0pt,
		                before={\begin{minipage}[t]{\hsize}},
                        after ={\end{minipage}}]
		    \item $U_{lid} = 0.5$ m$\cdot$s$^{-1}$
		\end{itemize}\vspace*{-\baselineskip}\mbox{} &
		\begin{itemize}[nosep,noitemsep,left=0pt,
		                before={\begin{minipage}[t]{\hsize}},
                        after ={\end{minipage}}]
		    \item Velocity components $(u_x,u_y)$ along AA' and BB'
		\end{itemize}\vspace*{-\baselineskip}\mbox{} \\
        \midrule
        0.2 & Neutronics &
        \begin{itemize}[nosep,noitemsep,left=0pt,
		                before={\begin{minipage}[t]{\hsize}},
                        after ={\end{minipage}}]
		    \item $U_{lid} = 0$ m$\cdot$s$^{-1}$
		    \item $T = 900$ K
		    \item $P = 1$ GW
		\end{itemize} &
		\begin{itemize}[nosep,noitemsep,left=0pt,
		                before={\begin{minipage}[t]{\hsize}},
                        after ={\end{minipage}}]
		    \item Fission rate density $\sum^6_g \Sigma_{f,g} \phi_g(\vec{r})$ along AA'
            \item Reactivity $\rho$
		\end{itemize}\vspace*{-\baselineskip}\mbox{} \\
        \midrule
        0.3 & Temperature &
        \begin{itemize}[nosep,noitemsep,left=0pt,
		                before={\begin{minipage}[t]{\hsize}},
                        after ={\end{minipage}}]
		    \item Fixed flow field from Step 0.1 for
		    $U_{lid} = 0.5$ m$\cdot$s$^{-1}$
		    \item Fixed heat source distribution
		    $\sum^6_{g} \epsilon_g \Sigma_{f,g} \phi_g(\vec{r})$ from Step 0.2
		    \item $\gamma = 10^6$ W$\cdot$m$^{-3}\cdot$K$^{-1}$
		\end{itemize} &
		\begin{itemize}[nosep,noitemsep,left=0pt,
		                before={\begin{minipage}[t]{\hsize}},
                        after ={\end{minipage}}]
		    \item Temperature $T$ along AA' and BB'
		\end{itemize}\vspace*{-\baselineskip}\mbox{} \\
        \midrule
        1.1 & Circulating fuel &
        \begin{itemize}[nosep,noitemsep,left=0pt,
		                before={\begin{minipage}[t]{\hsize}},
                        after ={\end{minipage}}]
		    \item Fixed flow field from Step 0.1 for
		    $U_{lid} = 0.5$ m$\cdot$s$^{-1}$
		    \item $T = 900$ K
		    \item $P = 1$ GW
		\end{itemize} &
		\begin{itemize}[nosep,noitemsep,left=0pt,
		                before={\begin{minipage}[t]{\hsize}},
                        after ={\end{minipage}}]
		    \item Delayed neutron source $\sum^8_i \lambda_i C_i$ along AA' and BB'
		    \item Reactivity change between Step 1.1 and Step 0.2,
		    $\Delta \rho = \rho - \rho_{s_{0.2}}$
		\end{itemize}\vspace*{-\baselineskip}\mbox{} \\
        \midrule
        1.2 & Power coupling &
        \begin{itemize}[nosep,noitemsep,left=0pt,
		                before={\begin{minipage}[t]{\hsize}},
                        after ={\end{minipage}}]
		    \item Fixed flow field from Step 0.1 for
		    $U_{lid} = 0.5$ m$\cdot$s$^{-1}$
		    \item $P = 1$ GW
		    \item $\gamma = 10^6$ W$\cdot$m$^{-3}\cdot$K$^{-1}$
		\end{itemize}\vspace*{-\baselineskip}\mbox{} &
		\begin{itemize}[nosep,noitemsep,left=0pt,
		                before={\begin{minipage}[t]{\hsize}},
                        after ={\end{minipage}}]
		    \item Temperature $T$ along AA' and BB'
            \item Reactivity change between Step 1.2 and Step 1.1,
            $\Delta\rho = \rho - \rho_{s_{1.1}}$
            \item Change in fission rate density
            $\sum^6_g \Sigma_{f,g} \phi_g(\vec{r}) -
            \left[\sum^6_g \Sigma_{f,g} \phi_g(\vec{r})\right]_{s_{0.2}}$
		\end{itemize} \\
        \midrule
        1.3 & Buoyancy &
        \begin{itemize}[nosep,noitemsep,left=0pt,
		                before={\begin{minipage}[t]{\hsize}},
                        after ={\end{minipage}}]
		    \item $P = 1$ GW
		    \item $U_{lid} = 0$ m$\cdot$s$^{-1}$
		    \item $\gamma = 10^6$ W$\cdot$m$^{-3}\cdot$K$^{-1}$
		\end{itemize}\vspace*{-\baselineskip}\mbox{} &
		\begin{itemize}[nosep,noitemsep,left=0pt,
		                before={\begin{minipage}[t]{\hsize}},
                        after ={\end{minipage}}]
		    \item Velocity components $(u_x, u_y)$ along AA' and BB'
            \item Temperature $T$ along AA' and BB'
            \item Delayed neutron source $\sum^8_i \lambda_i C_i$ along AA' and BB'
            \item Reactivity change from Step 0.2
        $\Delta\rho = \rho - \rho_{s_{0.2}}$
		\end{itemize} \\
        \midrule
        1.4 & Full coupling &
        \begin{itemize}[nosep,noitemsep,left=0pt,
		                before={\begin{minipage}[t]{\hsize}},
                        after ={\end{minipage}}]
		    \item $\gamma = 10^6$ W$\cdot$m$^{-3}\cdot$K$^{-1}$
		    \item $P$ variable in the range $[0,1]$ GW with a step of 0.2 GW
		    \item $U_{lid}$ variable in the range $[0,0.5]$ m$\cdot$s$^{-1}$
		    with a step of 0.1 m$\cdot$s$^{-1}$
		\end{itemize} &
		\begin{itemize}[nosep,noitemsep,left=0pt,
		                before={\begin{minipage}[t]{\hsize}},
                        after ={\end{minipage}}]
		    \item Reactivity change between Step 1.4 and Step 0.2,
		    $\Delta\rho = \rho - \rho_{s_{0.2}}$, for all permutations of $P$
		    and $U_{lid}$ values
		\end{itemize}\vspace*{-\baselineskip}\mbox{} \\
        \midrule
        2.1 & Forced convection transient &
        \begin{itemize}[nosep,noitemsep,left=0pt,
		                before={\begin{minipage}[t]{\hsize}},
                        after ={\end{minipage}}]
		    \item $\gamma = 10^6$ W$\cdot$m$^{-3}\cdot$K$^{-1}$
            \item Steady-state solution from Step 1.4 for $U_{lid} = 0.5$
        m$\cdot$s$^{-1}$ and $P = 1.0$ GW
		\end{itemize} &
		\begin{itemize}[nosep,noitemsep,left=0pt,
		                before={\begin{minipage}[t]{\hsize}},
                        after ={\end{minipage}}]
		    \item Power gain and shift as a function of the perturbation frequency
		\end{itemize}\vspace*{-\baselineskip}\mbox{} \\
		\bottomrule
	\end{tabular}
	\label{table:benchmark}
\end{table*}

Step 2.1 studies the transient response of the fully coupled nonlinear system.
Linear perturbation analyses are performed by introducing periodic
perturbations to the heat transfer coefficient $\gamma$ and studying the gain
and phase shift of the response in the total power $P$. For the initial
conditions, the steady-state solution from Step 1.4 with
$U_{lid} = 0.5$ m$\cdot$s$^{-1}$ and $P = 1$ GW is used. This initial
configuration is made exactly critical by scaling the neutron source terms,
from fission and \gls{DNP} decay, by the inverse of the criticality eigenvalue
solution from Step 1.4.

$\gamma$ is uniformly perturbed according to small-amplitude sine waves given
as:
%
\begin{align}
    \gamma =& \gamma_0 \left[ 1 + 0.1\sin\left(2 \pi f \right) \right]
    \shortintertext{where}
    \gamma_0 =& 10^6 \mbox{ W$\cdot$m$^{-3}\cdot$K$^{-1}$}, \nonumber \\
    f \in& \left\lbrace 0.0125, 0.025, 0.05, 0.1, 0.2, 0.4, 0.8 \right\rbrace 
    \mbox{ Hz.} \nonumber
\end{align}

The benchmark defines power gain as:
%
\begin{align}
    \mbox{Power gain} =& \frac{\left(P_{max} - P_{avg}\right)/P_{avg}}{
    \left(\gamma_{max} - \gamma_{avg}\right)/\gamma_{avg}}
\end{align}
%
The subscripts denote the maximum and time-averaged values of $P$ and $\gamma$.

\FloatBarrier

\glsresetall

\chapter{Hybrid S$_\text{N}$-Diffusion Method for Improved Control Rod Modeling
in Neutron Diffusion Methods}
\label{chap:hybrid}
Section \ref{sec:challenges-control-rod} highlighted the poor performance of neutron diffusion
methods for calculating neutron fluxes near control rods. Strong neutron absorption in the control
rod region produces highly anisotropic neutron flux extending some distance outside the control
rod. Neutron transport methods, which retain angular dependence of the neutron flux in one way or
other, generally fare better than neutron diffusion methods with isotropic diffusion coefficients.
However, neutron transport methods are also generally more computationally expensive given the
increased dimensionality of the problem from the angular component. Piling this extra dimension on
top of the existing geometric and neutron energy group dimensions greatly multiplies the unknowns
to be solved in a system. Many past efforts have tried introducing transport correction techniques
to improve neutron flux and multiplication factor estimates with diffusion-based methods. Other
than control rod regions, these techniques also correct for homogenization error introduced from
spatial homogenization of fuel assemblies and other structures within a reactor core. They
invariably rely on neutron transport methods to generate transport corrections in the form of
corrected diffusion coefficients, boundary conditions, Eddington factors or discontinuity factors.

In this chapter, I propose a novel hybrid method for improving control rod modeling in neutron
diffusion solvers without spatial homogenization. In essence, the hybrid method involves applying
the $S_N$ discrete ordinates neutron transport method on subregions containing the control rod to
obtain transport corrected diffusion coefficients for the diffusion method on the entire problem
domain. 

Section \ref{sec:hybrid-theory} discusses the theoretical background for the Hybrid $S_N$-Diffusion
method. Thereafter, Section \ref{sec:preliminary} discusses preliminary results of a 1-D hybrid
method implemented with the finite difference method on several example cases based off the
graphite-moderated \gls{MSRE} reactor design.

\section{Theory} \label{sec:hybrid-theory}

The proposed Hybrid $S_N$-Diffusion method is a two-level, iterative method to improve the
accuracy of neutron diffusion solutions in reactor systems with highly neutron-absorbing control
rod regions. In this discussion, I focus on the 1-D, $k$-eigenvalue implementation of the
hybrid method.

The discrete ordinates ($S_N$) method for solving the multigroup neutron transport equation
(Eq. \ref{eq:mg-bte}) discretizes the continuous angular directional phase space into a finite
number of discrete angular intervals (ordinates). The set of distinct direction variables
$\hat{\Omega}_n$ representing the discrete ordinates is typically chosen in conjunction with a
compatible quadrature set to replace the integrals across the continuous solid angles
$\hat{\Omega}$ with quadrature approximations. The multigroup, discrete ordinates ($S_N$
approximation) form of the 1-D, $k$-eigenvalue neutron transport equation with the Gauss-Legendre
quadrature set is given as:
%
\begin{align}
  \mu_n \frac{d}{dx}\Psi_g(x, \mu_n) + \Sigma_{t,g}(x)&\Psi_g(x, \mu_n) -
\sum^G_{g'=1} \sum^N_{n'=1} \sum^L_{l=0} \frac{\left(2l+1\right)}{2}
\Sigma^{g'\rightarrow g}_{s,l}(x) P_l(\mu_{n'} - \mu_n)
w_{n'}\Psi_{g'}(x,\mu_{n'}) \nonumber \\
  &= \sum^G_{g'=1} \frac{\chi_g}{2} \frac{\nu\Sigma_{f,g}(x)}{k} \phi_{g'}(x) + S_g(x,\mu_n)
  \label{eq:1d-sn}
  \shortintertext{where}
  \mu_n &= \mbox{ cosine of $\hat{\Omega}_n$ relative to the $x$-axis,} \nonumber \\
  \Psi_g(x,\mu_n) &= \mbox{ neutron angular flux along $\mu_n$ in group $g$,} \nonumber \\
%  g &= \mbox{ neutron energy group index, ranging from 1 to G,} \nonumber \\
%  \Sigma_{t,g}(x) &= \mbox{ macroscopic total cross section of neutrons in group $g$,} \nonumber \\
%  G &= \mbox{ total number of energy groups,} \nonumber \\
%  n &= \mbox{ discrete ordinate index,} \nonumber \\
%  N &= \mbox{ total number of discrete ordinates,} \nonumber \\
  l &= \mbox{ Legendre order,} \nonumber \\
  L &= \mbox{ highest Legendre order,} \nonumber \\
  \Sigma^{g'\rightarrow g}_{s,l}(x) &= \mbox{ $l$-th Legendre expansion of the macroscopic
scattering} \nonumber \\
  &\ \ \ \ \mbox{ cross section from group $g'$ to $g$,} \nonumber \\
  P_l &= \mbox{ $l$-th Legendre polynomial,} \nonumber \\
  w_{n'} &= \mbox{ $n'$-th quadrature weight,} \nonumber \\
%  \chi_g &= \mbox{ fission neutron spectrum in group $g$,} \nonumber \\
%  \nu &= \mbox{ neutrons produced per fission reaction,} \nonumber \\
%  \Sigma_{f,g}(x) &= \mbox{ macroscopic fission cross section of neutrons in group $g$,}
%  \nonumber \\
  k &= \mbox{ neutron multiplication factor.} \nonumber
%  \phi_{g'}(x) &= \mbox{ scalar neutron flux in group $g$,} \nonumber \\
%  S_g(x,\mu_n) &= \mbox{ external neutron source in group $g$.} \nonumber
\end{align}
%
The neutron scalar flux and current can be retrieved by calculating the 0th and 1st Legendre moments
as follows:
%
\begin{align}
  \phi_g(x) &= \frac{1}{2} \int^1_{-1} d\mu\ \Psi_g(x,\mu) = \frac{1}{2} \sum^N_{n=1} w_n
\Psi_g(x,\mu_n)
  \shortintertext{and}
  J_g(x) &= \frac{1}{2} \int^1_{-1} d\mu\ \mu\Psi_g(x,\mu) = \frac{1}{2} \sum^N_{n=1} w_n
\mu_n \Psi_g(x,\mu_n)
\end{align}

The 1-D form of the multigroup $k$-eigenvalue neutron diffusion equations (Eq. \ref{eq:mg-diff})
is given as:
%
\begin{align}
  -\frac{d}{dx} D_g(x) \frac{d}{dx} \phi_g(x) + \Sigma_{t,g}(x) \phi_g(x) &= \sum^G_{g'=1}\left[
  \Sigma_s^{g'\rightarrow g}(x)\phi_{g'}(x) + \chi_g\frac{\nu\Sigma_{f,g'}(x)}{k}
  \phi_{g'}(x)\right] + S_g(x)
  \label{eq:1d-diff}
  \shortintertext{where}
    D_g(x) &= \frac{1}{3 \Sigma_{tr}(x)} = \mbox{ neutron diffusion coefficient for group }g.
  \nonumber
\end{align}

\subsection{Spatially Varying Diffusion Coefficients} \label{sec:svdc}

The conventional approach for determining diffusion coefficients based on the $P_1$ approximation
for each subregion involves running a high-fidelity neutron transport to tally region-wide
estimates of the neutron transport cross section. In essence, a single value represents the
isotropic diffusion coefficient for the entire (e.g. fuel, moderator, reflector) subregion.
However, as discussed in Section \ref{sec:challenges-control-rod}, the neutron diffusion equation
is only valid in regions of high scattering-to-absorption ratios and away from interfaces to
neighboring media with highly dissimilar neutronic properties.

In the Hybrid $S_N$-Diffusion method, I propose replacing the conventional $P_1$-based
diffusion coefficients with an alternate formulation which incorporates point-wise corrections
to the neutron diffusion flux solution from the $S_N$-derived flux solution as follows:
%
\begin{align}
  D^s_g(x) &= -J^{tr}_g(x)\bigg/\frac{d\phi^{tr}_g(x)}{dx}. \label{eq:svdc}
\end{align}
%
where $D^s$ is the \glspl{SVDC}, and the $tr$ superscript denotes neutron current and scalar flux
solutions from the $S_N$ method.
The basic form of this equation is Fick's first law of diffusion. The end result is a diffusion
coefficient variable that varies in space even within a subregion and provides pointwise
corrections to closely match the diffusion flux solution to the $S_N$ flux solution.

Pounders \& Rahnema \cite{pounders_diffusion_2009} demonstrated the effectiveness of applying
point-wise corrections derived from analytical or Monte Carlo reference flux solutions. Compared
with conventional $P_1$-based out-scatter and flux-limited approximations of the diffusion
coefficient, their \textit{high-order empirical diffusion coefficient} showed superior agreement
with the reference flux solutions. Their formulation for these piecewise-constant empirical
diffusion coefficients in Eq. \ref{eq:emp} is similar to the formulation for \glspl{SVDC} in Eq.
\ref{eq:svdc} after finite difference discretization. They recognized that the volume
averaging for piecewise-constant coefficients introduce some truncation error if the flux is
non-linear within each mesh element. This design choice may be due to an intention to retain the
diffusion coefficient as a constant in the $\frac{d}{dx}D\frac{d\phi}{dx}$ term of the neutron
diffusion equation (Eq. \ref{eq:1d-diff}).

Unlike their approach, the \gls{SVDC} formulation in Eq. \ref{eq:svdc} allows for continuously
varying diffusion coefficients to reduce truncation error. In practice, the discretization order of
\gls{SVDC} variables in a numerical calculation would follow the discretization order of the
reference flux solution. This formulation introduces a minor change to the finite difference
implementation of the second-order diffusion term to specifically handle spatial derivatives of the
diffusion coefficient, but the finite element implementation sees no change as the second-order
diffusion term is commonly treated with integration-by-parts to reduce it to a first-order
differential term with no derivatives of the diffusion coefficient. Refer to Section
\ref{sec:implementation} for the numerical implementation of a finite difference neutron diffusion
solver with \glspl{SVDC}.

\begin{figure}[htb!]
  \centering
  \includegraphics[width=.7\columnwidth]{case-0-geometry}
  \caption{A 1-D, two-region system containing a 0.5-cm thick control rod and a 24.5-cm thick
    homogeneous mixture of molten fuel salt and graphite moderator. Reflective boundary conditions
    are applied on both ends.}
  \label{fig:case-0-geom}
\end{figure}

\subsubsection{\gls{SVDC} Verification with Case 0}

To facilitate the following demonstration of a neutron diffusion calculation with \glspl{SVDC},
consider a 1-D, two-region system (Case 0) consisting of a highly neutron absorbing material and a
neutron multiplying region with reflective boundary conditions on both ends (Figure
\ref{fig:case-0-geom}). The material specifications are taken from the control rod, molten fuel
salt, and graphite moderator compositions of the \gls{MSRE} \cite{robertson_msre_1965}. The molten
fuel salt and graphite regions are homogenized to minimize discrepancies arising from geometrical
heterogeneity. The group constant input data for the diffusion and $S_N$ solvers (e.g. cross
sections, fission spectra, etc.) were sampled at 900 K the OpenMC Monte Carlo neutronics software.
The neutron energy spectrum is condensed into two discrete groups bounded at $10^{-5}$, $10^0$, and
$10^8$ eV.

I solved for the neutron flux in this system using the following set of numerical solvers:
%
\begin{enumerate}
  \item Diffusion solver with $P_1$ flux-limited diffusion coefficients generated directly from the
    OpenMC calculation.
  \item $S_4$ solver with up to 1st-order Legendre expansions of the group-to-group neutron
    scattering cross sections.
  \item Diffusion solver with \glspl{SVDC} generated from the prior $S_4$ flux solution.
\end{enumerate}
%
All other relevant group constants are identical and unchanged from the original dataset generated
using OpenMC. I implemented the diffusion solvers using the finite difference method and the $S_N$
solver using the diamond-difference transport sweep method in the Python programming language. For
brevity, further numerical implementation details of these solvers are deferred to Section
\ref{sec:implementation}.

\begin{figure}[htb!]
  \centering
  \begin{subfigure}[b]{.49\textwidth}
    \centering
    \includegraphics[width=\textwidth]{case-0-group-1-flux}
    \caption{Group 1 flux}
    \label{fig:c0g1flux}
  \end{subfigure}
  \hfill
  \begin{subfigure}[b]{.49\textwidth}
    \centering
    \includegraphics[width=\textwidth]{case-0-group-2-flux}
    \caption{Group 2 flux}
    \label{fig:c0g2flux}
  \end{subfigure}
  \caption{Neutron group 1 and 2 flux distributions from the diffusion, $S_4$, and
  diffusion-\gls{SVDC} solvers for Case 0.}
  \label{fig:c0flux}
\end{figure}
%
\begin{figure}[htb!]
  \centering
  \begin{subfigure}[b]{.49\textwidth}
    \centering
    \includegraphics[width=\textwidth]{case-0-group-1-diffcoef}
    \caption{Group 1 diffusion coefficients}
    \label{fig:c0g1diffcoef}
  \end{subfigure}
  \hfill
  \begin{subfigure}[b]{.49\textwidth}
    \centering
    \includegraphics[width=\textwidth]{case-0-group-2-diffcoef}
    \caption{Group 2 diffusion coefficients}
    \label{fig:c0g2diffcoef}
  \end{subfigure}
  \caption{$P_1$-based diffusion coefficient and \gls{SVDC} spatial distributions
  for Case 0.}
  \label{fig:c0diffcoef}
\end{figure}

Figure \ref{fig:c0flux} shows the group 1 and 2 neutron fluxes from the diffusion-$P_1$, $S_4$, and
diffusion-\gls{SVDC} solvers with a fixed mesh size of 0.005 cm. The OpenMC flux solution is
omitted because the purpose of this exercise is to demonstrate the effectiveness of \glspl{SVDC}
in reproducing a $S_N$ flux solution. As expected, the diffusion-$P_1$ flux solution deviates from
the $S_4$ flux solution while the diffusion-\gls{SVDC} flux solution shows significantly better
agreement with the $S_4$ flux solution. As shown in Table \ref{table:c0k}, $k$ estimate from the
diffusion-\gls{SVDC} solver (0.62418) is also closer to the $k$ estimate from the $S_4$ solver
(0.62467) than the diffusion-$P_1$ solver (0.60833).

\begin{table}[tb!]
  \centering
  \caption{Multiplication factor $k$ estimates from the diffusion-$P_1$, $S_4$, and
  diffusion-\gls{SVDC} solvers and the absolute difference relative to the $S_4$ estimate.}
  \begin{tabular}{l S S}
    \toprule
    Solver type & {$k$} & {$k-k_{S4}$} \\
    \midrule
    Diffusion-$P_1$ & 0.60833 & -0.01634 \\
    $S_4$ & 0.62467 & {-} \\
    Diffusion-\gls{SVDC} & 0.62418 & -0.00050 \\
    % OpenMC & {0.64604 +/- 0.00051} & {-} \\
    \bottomrule
  \end{tabular}
  \label{table:c0k}
\end{table}

Comparing the $P_1$-based diffusion coefficients and \glspl{SVDC} in Figure \ref{fig:c0diffcoef},
both quantities agree closely in the bulk fuel salt region far away from the control rod region.
This observation supports the validity of the neutron diffusion method with $P_1$ approximations
in a homogeneous medium with high scattering-to-absorption ratios. Moving towards the control rod
region, the \glspl{SVDC} deviate from the $P_1$-based diffusion coefficients as the prerequisite
assumptions for diffusion theory do not hold anymore.

Another key finding from this study is that the neutron diffusion method with $P_1$-based
diffusion coefficients fails to reproduce the steep $S_4$ flux gradient in the control rod region
for group 1 and in the homogeneous fuel-graphite region adjacent to the control rod region for both
neutron groups. The neutron diffusion method requires small \gls{SVDC} values in those regions to
induce steeper flux gradients and match the $S_4$ flux solution. Given that the presence of highly
neutron-absorbing materials tend to induce steep flux gradients, we can expect similar trends in
the \glspl{SVDC} relative to $P_1$-based diffusion coefficients for other similar systems.

Thus far, this method is trivial and completely redundant because it requires a priori
knowledge of the true flux solution or at least a highly accurate solution calculated using neutron
transport methods. While existing workflows for diffusion-based methods already require
computationally intensive neutron transport simulations to generate input data for the neutron
diffusion equation, this preprocessing step (known as \textit{group constant generation}) requires
a fixed number of neutron transport simulations. For instance, $P_1$-based diffusion coefficients
may be generated at various reactor temperatures. In the subsequent multiphysics reactor analysis,
these diffusion coefficient values may be interpolated for diffusion coefficient estimates at other
temperatures within the validity range. On the other hand, \glspl{SVDC} will likely need to be
dynamically generated at every timestep, such as with a two-level iterative scheme consisting of a
high-level $S_N$ neutron transport calculation and a low-level neutron diffusion calculation.

The nature of \glspl{SVDC} as empirical, pointwise corrections for
the diffusion equation makes it highly dependent on the neutron flux gradient, and
susceptible to greater variations than region-wide estimates for $P_1$-based diffusion
coefficients. Compared to \glspl{SVDC}, $P_1$ diffusion coefficients behave much more like
intrinsic material properties as their definitions are largely tied to material cross sections.
Variations in $P_1$ diffusion coefficients largely arise from changes in the neutron energy
spectrum with no direct contribution from proximity to material interfaces (geometrical
heterogeneity).

Another significant challenge of \glspl{SVDC} and the high-order empirical diffusion coefficients
involves their determination near neutron flux peaks. The neutron currents and
flux gradients in the numerator and denominator of Eq. \ref{eq:svdc} generally do not go to zero at
the same point, resulting in very large positive or negative diffusion coefficient values when the
flux gradient is close or equal to zero. Pounders \& Rahnema tackled this issue by using
larger mesh sizes with which to calculate their empirical diffusion coefficients. However, their
remedy significantly worsens flux accuracy in regions with steep, non-linear flux such as in
control rods. I plan to investigate and find an alternative solution for this issue in the
implementation of \glspl{SVDC} in the proposed work.

\subsection{Hybrid $S_N$-Diffusion Method} \label{sec:hybrid-method}

In order to reduce the computational cost of the high-level $S_N$ calculation, I propose reducing
the problem domain of the $S_N$ method to the control rod and its vicinity. Consequently, this
Hybrid $S_N$-Diffusion method can retain accurate neutron flux estimates around the control rod
region from the $S_N$ method while making significant savings in computational cost by treating the
majority of the reactor geometry with the neutron diffusion method. Henceforth, I will refer to
the $S_N$ calculation or method on the reduced problem domain as the $S_N$ \textit{subproblem} or
\textit{subsolver} accordingly. The problem domains of the neutron diffusion and $S_N$ calculations
are defined as $\Omega^d_0$ and $\Omega^d_1$, respectively, where $\Omega^d_1 \subseteq
\Omega^d_0$. The algorithm for the Hybrid $S_N$-Diffusion method is as follows:
%
%\begin{algorithm}
%  \caption{Hybrid $S_N$-Diffusion algorithm \label{alg:hybrid}}
%  \DontPrintSemicolon
%  \KwData{Material group constants and mesh of problem domain $\Omega^d$.}
%  \KwResult{Improved flux $\phi$ and multiplication factor $k$ estimates.}
%  \BlankLine
%  Initialize $\phi_g^0$, $k^0$, $D_g^0=D_g^{P_1}$; $m=0$\;
%  \While{$\epsilon_\phi > tol_\phi$ or $\epsilon_k > tol_k$}{
%    $m = m+1$\;
%    Solve the neutron diffusion equations for $\phi_g^m$ and $k^m$ using $D_g$ in domain
%    $\Omega^d$.\;
%    Update $\epsilon_\phi = ||\phi_g^m-\phi_g^{m-1}||_2/G/n$ where $G=$ total no. of groups and
%    $n=$ total no. of mesh elements.\;
%    Calculate neutron current boundary conditions along subdomain boundaries $\partial\Omega^d_1$
%    using $\phi_g^m$ and $D_g$, where $\Omega^d_1 \subset \Omega^d$
%  }
%\end{algorithm}
%
\begin{enumerate}
  \item Start with an initial neutron diffusion calculation in $\Omega^d_0$ with conventional $P_1$
    diffusion coefficients and other standard group constants (e.g. neutron cross sections).
  \item Use the neutron diffusion flux estimates in $\Omega^d_1$ and current estimates along
    $\partial \Omega^d_1$ as initial and boundary conditions for the $S_N$ subsolver.
  \item With the $S_N$ subsolver, calculate an improved neutron flux solution in $\Omega^d_1$ which
    contains the control rod region and its immediate vicinity.
  \item Calculate \glspl{SVDC} using the $S_N$ flux solution and Eq. \ref{eq:svdc}.
  \item Pass the \glspl{SVDC} to the neutron diffusion solver to replace the conventional
    $P_1$ diffusion coefficients within the regions that overlap with the $S_N$ solver problem
    domain.
  \item Start a new iteration by running neutron diffusion calculation with the \glspl{SVDC}
    replacing $P_1$ diffusion coefficients in part or all of $\Omega^d_1$.
  \item Pass the iterative neutron diffusion flux and current solutions to the $S_N$ subsolver.
  \item Iterate until convergence is reached by meeting pre-defined convergence tolerance values.
\end{enumerate}

\subsubsection{$S_N$ Subsolver Boundary Conditions \& Dead Zone Assumptions}

The main challenge lies in determining appropriate boundary conditions for the $S_N$ subproblem.
Given that we want to limit the coverage of $\Omega^d_1$ to the control rod region and its
immediate vicinity, the boundaries $\partial\Omega^d_1$ should lie well within $\Omega^d_0$.
However, there is currently no feasible method of generating accurate boundary fluxes for an $S_N$
solver from a neutron diffusion flux solution. In 1-D, the standard $S_N$ method requires N/2
incoming flux boundary parameters per boundary mesh point for the N/2 neutron angular fluxes
flowing into $\Omega^d_1$. The neutron diffusion method can produce at most one independent
incoming flux parameter per mesh point; this parameter is the neutron forward/backward current in
the $P_1$ approximation defined as:
%
\begin{align}
  J_{g,\pm} &= \frac{\phi_g}{4} \mp \frac{D_g}{2}\frac{d\phi_g}{dx} \label{eq:p1-j}
  \shortintertext{where}
  J_{g,\pm} &= \mbox{ neutron forward/backward current of group }g. \nonumber \\
\end{align}
%
In the absence of additional information, I will assume that an isotropic angular flux
distribution, i.e. all N/2 incoming angular fluxes are equal in magnitude. This concept is similar
to white boundary conditions which describe isotropic reflection of particles at a boundary except
no reflection occurs in this case. The isotropic angular flux can be expressed mathematically for
forward angular fluxes as:
%
\begin{align}
  \sum^N_{n=N/2+1}w_n\mu_n\Psi(x,\mu_n) =& J_{+}(x) && (\mu_n>0) \nonumber \\
  \Psi(x,\mu_n)\sum^N_{n=N/2+1}w_n\mu_n =& J_{+}(x) && (\because \mbox{isotropic hemisphere})
  \nonumber \\
  \Psi(x,\mu_n) =& J_{+}(x)\Bigg/\sum^N_{n=N/2+1}w_n\mu_n
\end{align}
%
and backward angular fluxes as:
%
\begin{align}
  \sum^{N/2}_{n=1}w_n\mu_n\Psi(x,\mu_n) =& J_{-}(x) && (\mu_n<0) \nonumber \\
  \Psi(x,\mu_n)\sum^{N/2}_{n=1}w_n\mu_n =& J_{-}(x) && (\because \mbox{isotropic hemisphere})
  \nonumber \\
  \Psi(x,\mu_n) =& J_{-}(x)\Bigg/\sum^{N/2}_{n=1}w_n\mu_n \label{eq:sn-psi-j}
\end{align}
%
These boundary conditions for the $S_N$ subsolver generally yield inaccurate flux solutions since
most reactor systems do not exhibit perfectly isotropic neutron fluxes.

To resolve this issue, I posit the following hypothesis: \textit{If there exists a highly
neutron-absorbing control rod region within the $S_N$ subproblem domain $\Omega^d_1$ and the $S_N$
subproblem boundary $\partial\Omega^d_1$ lies several neutron mean free paths away from this
region, the \gls{SVDC} values calculated near the control rod region (using the $S_N$ subsolver
with isotropic hemisphere boundary conditions) will tend to the actual flux gradient solution
(using a reference $S_N$ calculation across the entire problem domain $\Omega^d$).} In other words,
suboptimal boundary conditions may induce inaccurate \gls{SVDC} values near the $\partial
\Omega^d_1$ subdomain boundaries, but the \gls{SVDC} values further from $\partial\Omega^d_1$ are
accurate and very weakly dependent on the boundary conditions. These inaccurate \gls{SVDC} values
close to $\Omega^d_1$ may be discarded in favor of the default $P_1$-based diffusion coefficients.
I will refer to the region containing these discarded values as the \textit{dead zone} or
$\Omega^d_2$, where $\Omega^d_2 \subset \Omega^d_1 \subseteq \Omega^d_0$.

\subsubsection{Hybrid Method Verification with Case 0}

Here, we revisit Case 0, the 1-D two-region system described in Section \ref{sec:svdc} (Figure
\ref{fig:case-0-geom}), to test my hypothesis and demonstrate the Hybrid $S_N$-Diffusion method.
I defined $\Omega^d_1$ to span from $x=0$ cm to $x=17.5$ cm. Note that $\Omega^d_1$
is a subdomain of the entire domain spanning from $x=0$ cm to $x=20$ cm.
Once again, the numerical implementation details are deferred to Section \ref{sec:implementation}.

The size of $\Omega^d_2$
is dynamically determined based on the \gls{SVDC} distribution generated at every iteration during
the hybrid method calculation. In the earlier study with \glspl{SVDC}, the \gls{SVDC} distribution
in the homogeneous fuel-graphite region approaches the $P_1$-based diffusion coefficient value as
$x$ is further from the control rod region. Therefore, starting from the control rod and
fuel-graphite region interface and sweeping right, I set the cutoff point for the $\Omega^d_2$ to
be \textit{the point at which the \gls{SVDC} distribution approaches within 1\% of the $P_1$-based
diffusion coefficient value}. The hybrid method converged after two outer iterations comprising of
three neutron diffusion and two $S_4$ calculations in total as outlined by the hybrid method
algorithm.
%
\begin{figure}[htb!]
  \centering
  \begin{subfigure}[b]{.49\textwidth}
    \centering
    \includegraphics[width=\textwidth]{case-0-group-1-hybrid-diffcoef}
    \caption{Group 1 diffusion coefficients}
    \label{fig:c0g1hd}
  \end{subfigure}
  \hfill
  \begin{subfigure}[b]{.49\textwidth}
    \centering
    \includegraphics[width=\textwidth]{case-0-group-2-hybrid-diffcoef}
    \caption{Group 2 diffusion coefficients}
    \label{fig:c0g2hd}
  \end{subfigure}
  \caption{$P_1$-based flux-limited diffusion coefficient and \gls{SVDC} spatial distribution for
  Case 0. The \gls{SVDC} distributions were generated from the reference $S_4$ (Full) and the
  hybrid (Hybrid) calculations.}
  \label{fig:c0hd}
\end{figure}

For a visual illustration, refer to Figures \ref{fig:c0g1hd} and \ref{fig:c0g2hd} which plot the
group 1 and 2 $P_1$-based diffusion coefficients, \glspl{SVDC} derived from the reference $S_4$
calculation as demonstrated in Section \ref{sec:svdc}, and \glspl{SVDC} derived from the Hybrid
$S_N$-Diffusion method discussed here. The first set of \glspl{SVDC} are taken to be the reference
because it was generated from a $S_4$ calculation across the entire domain $\Omega^d_0$. The
reference \glspl{SVDC} are labelled ``Full'' in the figures to avoid confusion with the reference
flux distribution from the $S_4$ calculation. The \glspl{SVDC} from the hybrid method agrees well
with the reference \glspl{SVDC} in the control rod region and for most of the fuel-graphite region
up to around $x=12.5$ cm. Both sets of \glspl{SVDC} approximately coincide with the $P_1$ diffusion
coefficient values at $x=14.370$ cm and $x=4.015$ cm for group 1 and group 2, respectively. The
dead zone spans from $x=14.370$ cm to $x=17.500$ cm for group 1 and $x=4.015$ cm to $x=17.500$ cm
for group 2, in which the hybrid method defaults to the $P_1$-based diffusion coefficients by
definition. Given that the reference \glspl{SVDC} values will generally not be known in real-world
problems, the hybrid method relies on the matching of \gls{SVDC} and $P_1$ diffusion coefficient
values for determining the dead zone cutoff point. Another important consideration is the fact that
the neutron flux gradient will be discontinuous if the \gls{SVDC} and $P_1$ diffusion coefficient
values are not sufficiently continuous at the dead zone cutoff point. A flux gradient discontinuity
in a homogeneous bulk region is non-physical and therefore unacceptable. In Section
\ref{sec:prelim-results}, I study \gls{SVDC} distribution trends in more heterogeneous systems and
their implications on determining the dead zone cutoff point.
%
\begin{figure}[htb!]
  \centering
  \begin{subfigure}[b]{.49\textwidth}
    \centering
    \includegraphics[width=\textwidth]{case-0-group-1-hybrid-flux}
    \caption{Group 1 flux}
    \label{fig:c0g1hf}
  \end{subfigure}
  \hfill
  \begin{subfigure}[b]{.49\textwidth}
    \centering
    \includegraphics[width=\textwidth]{case-0-group-2-hybrid-flux}
    \caption{Group 2 flux}
    \label{fig:c0g2hf}
  \end{subfigure}
  \caption{Neutron group 1 and 2 flux distributions from the diffusion, $S_4$, reference
  \gls{SVDC}, and hybrid solvers for Case 0.}
  \label{fig:c0hf}
\end{figure}

Figures \ref{fig:c0g1hf} and \ref{fig:c0g2hf} show the group 1 and 2 neutron flux distributions
from the neutron diffusion calculations with $P_1$ diffusion coefficients and \glspl{SVDC}, and
$S_4$ calculation. As expected following the diffusion coefficient discussion, the hybrid method
flux distribution matches the $S_4$ flux distribution well. The $k$ estimate from the hybrid
method is 0.62444 which is only 0.00026 higher than the diffusion calculation with reference
\glspl{SVDC} (0.62418), and 0.00023 lower than the $S_4$ calculation (0.62467). By comparison, the
$k$ estimate from the diffusion calculation with $P_1$ diffusion coefficients is 0.01634 lower than
the $S_4$ calculation as discussed in Section \ref{sec:svdc}.

In Case 0, $\Omega^d_1$ covers more than half of $\Omega^d_0$ due to
the control rod region being the only significant influence on the neutron flux distribution in the
otherwise homogeneous system with reflective boundary conditions. I chose Case 0 as a simple test
case to aid in introducing the Hybrid $S_N$-Diffusion method. In the next section, I
present numerical implementation details of the Hybrid $S_N$-Diffusion method and results on more
complicated geometries with neutron reflector, air, and heterogeneous fuel-graphite lattice
regions and vacuum boundary conditions. On these test cases, the hybrid method yields
smaller $\Omega^d_1$-to-$\Omega^d_0$ ratios than the corresponding ratio in Case 0.

\section{Numerical Implementation} \label{sec:implementation}

I implemented all numerical solvers discussed in this work in the Python programming language.
First, I discuss the material group constant data generation and postprocessing steps. Next, I
present the implementation details of the neutron diffusion and $S_N$ numerical methods
individually. Thereafter I present how they are coupled to form the Hybrid $S_N$-Diffusion method.

\subsection{Group Constant Data Generation}

The group constants required by either or both neutron diffusion and $S_N$ neutron transport
methods are
%
\begin{itemize}
  \item $\Sigma_{t,g}$: Macroscopic total cross section in group $g$
  \item $\Sigma_{r,g}$: Macroscopic removal cross section in group $g$
  \item $\Sigma_s^{g'\rightarrow g}$: Macroscopic group-to-group scattering cross section matrix
  \item $\Sigma_{s,l}^{g'\rightarrow g}$: $l$-th Legendre expansion of the macroscopic
    group-to-group scattering cross section matrix
  \item $\Sigma_{sp,l}^{g'\rightarrow g}$: $l$-th Legendre expansion of the macroscopic
    group-to-group scattering production cross section matrix
  \item $D_g$: $P_1$-based diffusion coefficient in group $g$
  \item $\nu\Sigma_{f,g}$: Product of the average number of neutrons produced per fission and the
    macroscopic fission cross section in group $g$
  \item $\chi_g$: Neutron fission spectrum in group $g$
\end{itemize}
%
These group constants are generated using OpenMC's multigroup cross section generation capability
and postprocessed using a Python script into a JSON file format. $\Sigma_{r,g}$ is the only
quantity that is not directly provided by OpenMC. It is calculated as
%
\begin{align}
  \Sigma_{r,g} =& \sum^G_{g'\neq g}\Sigma_s^{g\rightarrow g'}+\Sigma_{a,g}-\left(\Sigma_{sp}^{g
    \rightarrow g} - \Sigma_s^{g\rightarrow g}\right)
  \shortintertext{where}
      \Sigma_{a,g} =& \mbox{ macroscopic absorption cross section in group $g$,} \nonumber \\
      \Sigma_{sp}^{g\rightarrow g} =& \mbox{ macroscopic scattering production cross section from
      group $g$ to $g$.} \nonumber
\end{align}
%
$\Sigma_{r,g}$ primarily represents the loss of neutrons from group $g$ through outscattering and
absorption. $\Sigma_{sp}^{g\rightarrow g}$ incorporates neutron multiplication effects from neutron
knockout reactions into the scattering cross section. Neutron knockout reactions are commonly
tallied as scattering reactions, and $\Sigma_{r,g}$ is a convenient term through which to
incorporate neutron knockout effects into the neutron diffusion equations. I ran all test cases in
this work with up to 1st-order Legendre expanded scattering cross sections.

OpenMC uses the $P_1$ flux-limited formulation \cite{pomraning_flux-limited_1984} for calculating
$D_g$ as follows
%
\begin{align}
  D_g =& \frac{1}{3\Sigma_{tr,g}} \\
  \Sigma_{tr,g} =& \frac{\langle\Sigma_{t,g}\phi_g\rangle-\langle\Sigma_{s1,g}\phi_g\rangle}
  {\langle\phi_g\rangle}
  \shortintertext{where}
  \langle\Sigma_{t,g}\phi_g\rangle =& \int_{r\in V}dr \int_{4\pi}d\Omega\int^{E_{g-1}}_{E_g}dE\
  \Sigma_{t,g}(r,E)\Psi(r,E,\Omega) \nonumber \\
  \langle\Sigma_{s1,g}\phi_g\rangle =& \int_{r\in V}dr \int_{4\pi}d\Omega\int^{E_{g-1}}_{E_g}dE
  \int_{4\pi}d\Omega'\int^{\infty}_0dE'\int^1_{-1}d\mu\ \mu\Sigma_s(r,E'\rightarrow E,\Omega'\cdot
  \Omega)\phi(r,E',\Omega') \nonumber \\
  \langle \phi \rangle =& \int_{r\in V}dr\int_{4\pi}d\Omega\int^{E_{g-1}}_{E_g}dE\ \Psi(r,E,\Omega)
  .\nonumber
\end{align}

\subsection{Neutron Diffusion Method}

On a 1-D uniform spatial grid with $I+1$ mesh points, the neutron scalar flux variables
$\phi_{g,i}$ are defined on the mesh points $x_i$. Theoretically, group constants are volumetric
material properties which should be defined on the cell-centered half-integer mesh points
$x_{i+\sfrac{1}{2}}$. In practice, all material properties except diffusion coefficients are
uniform in each subregion and are sampled at $x_i$. To avoid ambiguity concerning diffusion
coefficient sampling, I formulated all test cases such that all material interfaces fall on $x_i$.
A ghost point is added to the end of the spatial grid if the reflective boundary conditions are
imposed at that boundary.

Discretizing the multigroup $k$-eigenvalue neutron diffusion equations in Eq. \ref{eq:1d-diff}
and reformulating the scattering term using neutron balance in the control volume bounded by
$x_{i-\sfrac{1}{2}}$ and $x_{i+\sfrac{1}{2}}$ yields
%
\begin{align}
  J_{g,i+\sfrac{1}{2}} - J_{g,i-\sfrac{1}{2}} + \Sigma_{t,g,i} \phi_{g,i} \Delta x = \sum^G_{g'=1}\left[
  \Sigma_{s,i}^{g'\rightarrow g}\phi_{g',i} + \chi_{g,i}\frac{\nu\Sigma_{f,g',i}}{k} \phi_{g',i}
\right]\Delta x. \label{eq:diff-j}
\end{align}
%
Using the diamond difference scheme to replace the $J$ terms with the discretized form of
Fick's first law of diffusion,
%
\begin{align}
  J_{g,i+\sfrac{1}{2}} = -D_{g,i+\sfrac{1}{2}}\frac{d\phi_{g,i+\sfrac{1}{2}}}{dx} =
  -D_{g,i+\sfrac{1}{2}} \frac{\phi_{g,i+1}-\phi_{g,i}}{\Delta x},
\end{align}
%
and rearranging the terms in Eq. \ref{eq:diff-j} yields
%
\begin{align}
  -\frac{D_{g,i-\sfrac{1}{2}}}{\Delta x} \phi_{g,i-1} + &\left[\frac{D_{g,i-\sfrac{1}{2}}+
  D_{g,i+\sfrac{1}{2}}}{\Delta x} + \Delta x\ \Sigma_{r,g,i} \right]\phi_{g,i} -
  \frac{D_{g,i+\sfrac{1}{2}}}{\Delta x}\phi_{g,i+1} -\Delta x\sum^G_{g'\neq g}
  \Sigma_{s,i}^{g'\rightarrow g}\phi_{g',i} \nonumber \\
  =& \Delta x\sum^G_{g'=1}
  \chi_{g,i} \frac{\nu\Sigma_{f,g',i}}{k} \phi_{g',i}, \label{eq:diff-fd}
  \shortintertext{where}
  \Sigma_{r,g} =& \Sigma_{t,g} - \Sigma_s^{g\rightarrow g} \nonumber \\
  =& \mbox{ macroscopic removal cross section for neutron group }g. \nonumber
\end{align}
%
The diamond difference scheme is 2nd-order accurate, and it can be easily shown that this form is
equivalent to applying 2nd-order finite differencing to the original neutron diffusion equation in
Eq. \ref{eq:1d-diff} with diamond differencing for the cell-centered group constants.
The fixed neutron source $S_g$ is ignored here since the test cases are all neutron-multiplying
systems with no fixed source.

I implemented two types of boundary conditions: vacuum and reflective boundary conditions. The
\textbf{vacuum boundary conditions} are imposed by setting the incoming flux in the $P_1$
approximation to zero and applying 2nd-order finite differencing as follows
%
\begin{align}
  \mbox{Left boundary: } \frac{\phi_{g,0}}{4}-\frac{D_{g,\sfrac{1}{2}}}{2}
  \frac{\left(-\phi_{g,2}+4\phi_{g,1}-3\phi_{g,0}\right)}{2\Delta x} = 0 \\
  \mbox{Right boundary: } \frac{\phi_{g,I}}{4}+\frac{D_{g,I-\sfrac{1}{2}}}{2}
  \frac{\left(\phi_{g,I-2}-4\phi_{g,I-1}+3\phi_{g,I}\right)}{2\Delta x} = 0.
\end{align}
%
The \textbf{reflective boundary conditions} are imposed by setting equating the flux at the last physical
mesh point to the flux at the ghost point as follows
%
\begin{align}
  \mbox{Left boundary: } \phi_{g,-1} =& \phi_{g,0} \\
  \mbox{Right boundary: } \phi_{g,I} =& \phi_{g,I+1}
\end{align}
%
where the $-1$ and $I+1$ indexes correspond to ghost point indexes. At material interfaces,
the continuity condition requires that the net neutron current on either side of the interface be
equal as follows
%
\begin{align}
  -\frac{3D_{g,i-\sfrac{1}{2}} - D_{g,i-\sfrac{3}{2}} }{2}
  \frac{\phi_{g,i-2}-4\phi_{g,i-1}+3\phi_{g,i}}{2\Delta x} =&
  -\frac{3D_{g,i+\sfrac{1}{2}} - D_{g,i+\sfrac{3}{2}} }{2}
  \frac{-3\phi_{g,i}+4\phi_{g,i+1}-\phi_{g,i}}{2\Delta x} \label{eq:itf-bc}
\end{align}
%
for a material interface at $x_i$.

Altogether, they form a system of equations of the form $\bm{A\overline{\phi}}=\bm{\frac{1}{k}
B\overline{\phi}}$, where $\bm{\overline{\phi}}$ is a flattened vector representation of
$\phi_{g,i}$, and $\bm{A}$ and $\bm{B}$ are matrices of the coefficients of $\phi_{g,i}$ as given
by Eqs. \ref{eq:diff-fd} to \ref{eq:itf-bc}. I implemented the inverse power method to find $k$ and
$\overline{\phi}$. The inverse power method algorithm is as follows
%
\begin{align}
  \shortintertext{1. Initialize $k^0$ and $\bm{\overline{\phi}}^0$}
  \shortintertext{2. Update $\bm{\overline{\phi}}$ and $k$}
  \bm{\overline{\phi}}^m =& \frac{1}{k^{m-1}}\bm{A}^{-1}\bm{B\overline{\phi}}^{m-1} \\
  k^m =& k^{m-1}\frac{|\bm{B\overline{\phi}}^m|}{|\bm{B\overline{\phi}}^{m-1}|}
  \shortintertext{3. Check whether convergence is reached}
  \frac{|\bm{\overline{\phi}}^m-\bm{\overline{\phi}}^{m-1}|}{|\bm{\overline{\phi}}^m|} <& \
  tol_{\bm{\overline{\phi}}} \\
  \frac{|k^m-k^{m-1}|}{|k^m|} <& \ tol_k
  \shortintertext{4. Return to Step 2 if either expression is false, otherwise exit.} \nonumber
\end{align}
%
$k^m$ and $\overline{\phi}^m$ denote estimates of $k$ and $\overline{\phi}$ after the $m$-th
iteration. Matrix $\bm{A}$ is a primarily tridiagonal matrix with at most $G-1$ off-diagonal terms
from the 4th term in Eq. \ref{eq:diff-fd}. Thus, $\bm{A}$ is initialized as a sparse matrix to take
advantage of the computationally efficient sparse matrix solver functions from the \texttt{sparse}
class of the \texttt{SciPy} Python library for scientific and technical computing. Matrix $\bm{B}$
is never initialized explicitly as a matrix. Instead, the vector $\bm{b}=\bm{B\overline{\phi}}$ is
updated directly in every iteration. In this system of equations, $|\bm{b}|$ corresponds to the
total number of fission neutrons produced in the system for a given $\bm{\overline{\phi}}$. This
quantity is calculated using the \texttt{trapezoid} numerical integration function from
\texttt{SciPy} to estimate the integral value of $\nu\Sigma_{g,f}\phi_{g}$ in $x$ from the discrete
flux values in $\bm{x}$. Finally, the final $\phi_{g,i}$ is normalized by a factor of $|\bm{b}|/k$
(no. of source neutrons) to obtain the neutron scalar flux per source neutron.

\subsection{$S_N$ Neutron Transport Method}

For the 1-D $S_N$ neutron transport method on the same uniform spatial grid with $I+1$ mesh points,
the neutron angular flux variables $\Psi_{g,i,n}=\Psi_g(x_i,\mu_n)$ are defined on the mesh points
$x_i$ while neutron scalar flux variables $\phi_{g,i\pm\sfrac{1}{2}}=\phi_g(x_{i\pm\sfrac{1}{2}})$
are defined on the half-integer mesh points $x_{i\pm\sfrac{1}{2}}$. All group constants are sampled
at $x_{i\pm\sfrac{1}{2}}$.

The $S_N$ equations are solved using the transport sweep method in which the algorithm ``sweeps''
through the spatial grid and sequentially updates $\Psi_{g,i,n}$. In 1-D, the algorithm sweep
direction follows the direction of neutron travel, i.e. it sweeps in the positive direction for
$\Psi_{g,i,n}$ with $\mu_n>0$ and vice versa.

Discretizing the multigroup $S_N$ neutron transport equations in Eq. \ref{eq:1d-sn} about
$x_{i+\sfrac{1}{2}}$ yields
%
\begin{align}
  \mu_n\frac{\Psi_{g,i+1,n}-\Psi_{g,i,n}}{x_{i+1}-x_i} + \Sigma_{t,g,i+\sfrac{1}{2}}
  \Psi_{g,i+\sfrac{1}{2},n} = q_{g,i+\sfrac{1}{2},n}
\end{align}
%
where $q_{g,i+\sfrac{1}{2},n}$ represents the combined scattering and fission neutron source term.
After expressing $\Psi_{g,i+\sfrac{1}{2}}$ as the average of $\Psi_{g,i+1,n}$ and $\Psi_{g,i,n}$
in the diamond difference scheme and rearranging the terms, we obtain
%
\begin{align}
  \Psi_{g,i+1,n} =& \frac{1-\Sigma_{t,g}\Delta x/2\mu_n}{1+
    \Sigma_{t,g}\Delta x/2\mu_n}\Psi_{g,i,n} +
    q\frac{\Delta x}{\mu_n\left(1+\Sigma_{t,g}\Delta x/2\mu_n\right)} \label{eq:sweep-right} &&
    (\mbox{for } \mu_n > 0)
\shortintertext{and}
  \Psi_{g,i,n} =& \frac{1+\Sigma_{t,g}\Delta x/2\mu_n}{1-
    \Sigma_{t,g}\Delta x/2\mu_n}\Psi_{g,i+1,n} -
    q\frac{\Delta x}{\mu_n\left(1-\Sigma_{t,g}\Delta x/2\mu_n\right)}. \label{eq:sweep-left} &&
    (\mbox{for } \mu_n < 0)
\end{align}
%
The remaining $i+\sfrac{1}{2}$ indexes on the group constants are dropped to reduce visual clutter.
These expressions are used to update $\Psi_{g,i,n}$ in the forward and backward transport sweeps.

The discretized scattering and fission terms in $q_{g,i+\sfrac{1}{2},n}$ are given as
%
\begin{align}
  q_{g,i+\sfrac{1}{2},n} =& \sum^G_{g'=1}\sum^L_{l=0}\frac{\left(2l+1\right)}
  {2}\Sigma_{s,l}^{g'\rightarrow g}P_l(\mu_n)\phi_{l,g',i+\sfrac{1}{2}} \nonumber \\
  &+\frac{\chi_g}{2}\sum^G_{g'=1}\frac{\nu\Sigma_{f,g'}}{k}\phi_{0,g',i+\sfrac{1}{2}}
  \label{eq:sn-q}
  \shortintertext{where}
  \phi_{l,g',i+\sfrac{1}{2}} =& \sum^N_{n'=1}w_{n'}P_l(\mu_{n'})\frac{\Psi_{g',i,n'}+
  \Psi_{g',i+1,n'}}{2}. \label{eq:phi-l}
\end{align}
%
$\phi_{l,g,i}$ are $l$-th Legendre expansions of the neutron flux evaluated using Gauss-Legendre
quadrature over $\mu_{n'}=[-1,1]$. $\phi_{0,g,i}$ and $\phi_{1,g,i}$ also correspond to the neutron
scalar flux $\phi_{g,i}$ and net current $J_{g,i}$, respectively.

For vacuum boundary conditions, $\Psi_{g,0,n}$ is zero for all positive $\mu_n$ while
$\Psi_{g,I,n}$ is zero for all negative $\mu_n$ as follows
%
\begin{align}
  \mbox{Left boundary: } \Psi_{g,0,n} =& 0 && (\mbox{for } \mu_n > 0) \\
  \mbox{Right boundary: } \Psi_{g,I,n} =& 0 && (\mbox{for } \mu_n < 0)
\end{align}
%
Reflective boundary conditions are imposed by equating the incoming angular flux to the
outgoing angular flux in the opposite direction as follows
%
\begin{align}
  \mbox{Left boundary: } \Psi_{g,0,n} =& \Psi_{g,0,n'} && (\mbox{for } \mu_n > 0, \mu_n =
  -\mu_{n'}) \\
  \mbox{Right boundary: } \Psi_{g,I,n} =& \Psi_{g,I,n'} && (\mbox{for } \mu_n < 0, \mu_n =
  -\mu_{n'})
\end{align}

The power iteration algorithm for the $S_N$ method is similar to the inverse power method algorithm
applied in the neutron diffusion method. The matrix solving step for updating $\overline{\phi}$ is
replaced with the transport sweep step along with its associated steps as follows
%
\begin{align}
  \shortintertext{1. Initialize $k^0$, $\phi_{l,g,i+\sfrac{1}{2}}^0$, and $q^0$}
  \shortintertext{2. Apply transport sweeps to solve for $\Psi^m$ using Eqs. \ref{eq:sweep-right},
  \ref{eq:sweep-left}, and \ref{eq:sn-q}}
  \shortintertext{3. Update $\phi^m$ and $k^m$ using Eqs. \ref{eq:phi-l} and \ref{eq:sn-k}}
  k^m =& k^{m-1}\frac{\sum^I_{i=0}\sum^G_{g=1}\nu\Sigma_{f,g}\phi^m_{g,i+\sfrac{1}{2}}}
  {\sum^I_{i=0}\sum^G_{g=1}\nu\Sigma_{f,g}\phi^{m-1}_{g,i+\sfrac{1}{2}}} \label{eq:sn-k}
  \shortintertext{4. Check whether convergence is reached}
  \frac{|\bm{\overline{\phi}}^m-\bm{\overline{\phi}}^{m-1}|}{|\bm{\overline{\phi}}^m|} <& \
  tol_{\bm{\overline{\phi}}} \\
  \frac{|k^m-k^{m-1}|}{|k^m|} <& \ tol_k
  \shortintertext{5. Return to Step 2 if either expression is false, otherwise exit.} \nonumber
\end{align}

The transport sweep and $\phi^m$-update algorithms are parallelized using the \texttt{joblib}
parallel computing Python library across all available CPU threads. The total computational work is
subdivided into several smaller tasks by unique combinations of the $G$ and $N$ indexes; each
thread computes all $\Psi^m$ or $\phi^m$ values on the mesh for a given set of indexes $g$ and
$n$.

\subsection{Hybrid $S_N$-Diffusion Method}

Without loss of generality, consider a 1-D system symmetric about the left boundary at $x_0=0$ cm.
Like Case 0, the control rod region is the left-most region followed by other types of non-control
rod regions. The $S_N$ subproblem domain $\Omega^d_1$ is bounded by $x_0$ and $x_j$ for some $x_j$
located several mean free paths to the right of the control rod region as governed by the relevant
discussion in Section \ref{sec:hybrid-method}.

Excluding the initial neutron diffusion calculation to initialize $k^0$ and $\phi^0$, each outer
iteration in the hybrid method consists of one $S_N$ neutron transport calculation and one neutron
diffusion calculation. The $\phi$ estimates from these $S_N$ transport and neutron diffusion
calculations are labeled as $\phi^{m+\sfrac{1}{2}}$ and $\phi^{m+1}$, respectively, during the
$(m+1)$-th outer iteration. $\phi^{m+\sfrac{1}{2}}$ spans $\Omega^d_1$ while $\phi^{m+1}$ spans
the entire domain $\Omega^d_0$.

For the $S_N$ subproblem boundary conditions, we discretize Eqs. \ref{eq:p1-j} and
\ref{eq:sn-psi-j} as follows
%
\begin{align}
  \Psi_{g,j,n} =& \frac{J_{g,-}(x_j)}{\sum^{N/2}_{n'=1}w_{n'}\mu_{n'}} \nonumber \\
  =& \frac{\frac{\phi_g(x_j)}{4}+
  \frac{D_g(x_{j-\sfrac{1}{2}})}{2}\frac{d\phi_g(x_j)}{dx}}{\sum^{N/2}_{n'=1}w_{n'}\mu_{n'}}
  \nonumber \\
  =& \frac{\frac{\phi_{g,j}}{4}+
  \frac{D_{g,j-\sfrac{1}{2}}}{2}\frac{\phi_{g,j}-\phi_{g,j-1}}{\Delta x}}
    {\sum^{N/2}_{n'=1}w_{n'}\mu_{n'}} && (\mu_n < 0) \label{eq:sn-bc}
\end{align}
%
where $D_g$ is the $P_1$-based diffusion coefficient value.

The \gls{SVDC} formulation in Eq. \ref{eq:svdc} is discretized using the diamond difference scheme
as follows
%
\begin{align}
  D^s_{g,i+\sfrac{1}{2}} =& -\frac{J^{tr}_{g,i+1}+J^{tr}_{g,i}}{2} \left(
  \frac{\phi^{tr}_{g,i+1}-\phi^{tr}_{g,i}}{\Delta x} \right)^{-1} = -\frac{\Delta x}{2}
  \frac{J^{tr}_{g,i+1}+J^{tr}_{g,i}}{\phi^{tr}_{g,i+1}-\phi^{tr}_{g,i}}. \label{eq:svdc-num}
\end{align}
%
From Eq. \ref{eq:svdc-num}, we note that numerical instabilities may occur near flux peaks where
the flux gradient and thus the denominator of Eq. \ref{eq:svdc-num} approach zero. On the other
hand, this observation also implies that diffusion coefficient values have negligible influence on
the flux solution near flux peaks due to the small flux gradient values. Therefore, we can avoid
numerical instabilities by applying the following logic after calculating $D^s_g$: If the ratio of
$|\frac{d\phi_g}{dx}|$ to $|\phi_g|$ is sufficiently small and $|D^s_g-D_g| > D_g$, the hybrid
method defaults to using $D_g$ instead of $|D^s_g-D_g|$. The first conditional statement locates
regions of relatively flat flux, and of these regions, the second conditional statement excludes
regions which naturally experience flat flux such as air-filled regions. In all test cases,
applying the first conditional as $|\frac{1}{\phi_g}\frac{d\phi_g}{dx}|<5\times 10^{-2}$ was
sufficient for this task.

The Hybrid $S_N$-Diffusion algorithm is as follows
%
\begin{align}
  \shortintertext{1. Initialize $k^0$ and $\phi^0$ with an initial neutron diffusion calculation on
  $\Omega^d_0$}
  \shortintertext{2. Generate estimates for $\Psi_{g,j,n}$ at $x_j$ for the $S_N$ transport
  calculation boundary conditions using Eq. \ref{eq:sn-bc}}
  \shortintertext{3. Calculate $\phi^{m+\sfrac{1}{2}}$ with the $S_N$ transport subsolver on
  $\Omega^d_1$}
  \shortintertext{4. Generate \glspl{SVDC} with $\phi^{m+\sfrac{1}{2}}$ and $J^{m+\sfrac{1}{2}}$
  using Eq. \ref{eq:svdc-num}}
  \shortintertext{5. Calculate $\phi^{m+1}$ with the newly generated \glspl{SVDC} and the neutron
  diffusion solver on $\Omega^d_0$}
  \shortintertext{6. Check whether convergence is reached}
  \frac{|\bm{\overline{\phi}}^m-\bm{\overline{\phi}}^{m-1}|}{|\bm{\overline{\phi}}^m|} <& \
  tol_{\bm{\overline{\phi}}} \\
  \frac{|k^m-k^{m-1}|}{|k^m|} <& \ tol_k
  \shortintertext{7. Return to Step 2 if either expression is false, otherwise exit.} \nonumber
\end{align}
%
In general, the convergence tolerance values for the outer hybrid method iteration must be smaller
than the tolerance values for the neutron diffusion and $S_N$ transport inner iterations. Using
$\phi^m$ from the neutron diffusion calculation as initial conditions for the $S_N$ transport
calculation in the $(m+1)$-th iteration helps to significantly reduce the number of transport
sweeps required.

\section{Description of 1-D Test Cases}

I designed ten 1-D test cases with increasing complexity to test the performance of the
Hybrid $S_N$-Diffusion method in response to various geometrical features. The latter cases
resemble the reference \gls{MSRE} design which has centrally located control rods.and air-filled
guide tubes. Figure \ref{fig:case-geom} shows the geometries of the Cases 0 to 5b.
All geometries have reflective boundary conditions at $x=0$ cm to reduce computational costs by
creating half-core or repeating unit cell models. Cases 1a, 2a, 3a, and 4a are repeating unit
cell models with reflecting boundaries on the right-side boundaries. Cases 1b, 2b, 3b, 4b, 5a, and
5b are half-core models with vacuum boundaries on the right-side boundaries.

Starting with an infinite, homogeneous region in Case 1a, I systematically added geometric
complexities to each successive test case to identify how each feature impacts the neutronics
results. Case 1b introduces a reflector region and vacuum boundary conditions relative to Case 1a.
Cases 2a and 2b introduce a control rod region between $x=0$ cm and $x=0.5$ cm relative to Cases 1a
and 1b. Cases 3a and 3b introduce a 0.5 cm-thick air gap between the control rod and the
fuel-graphite mixture regions relative to Cases 2a and 2b. For Cases 4a and 4b, I replaced the
homogeneous fuel-graphite mixture in Cases 1a and 2b with explicit fuel-graphite lattices as shown
in Figure \ref{fig:case-geom}. Case 5a introduces an air gap between the control rod and the
fuel-graphite lattice regions. Lastly, Case 5b replaces the control rod region with an extended
air gap region to provide a base case for control rod worth calculations relative to Case 5a.

\begin{figure}[htb!]
  \centering
  \includegraphics[width=\columnwidth]{case-geometry}
  \caption{Geometries of the 1-D test cases. The material labelled ``mixture'' represents a
    homogeneous mixture of fuel and graphite at a ratio of 22.5\%-77.5\% by volume. All geometries
    have reflective boundary conditions on the boundary at $x=0$ cm. The right-side boundaries are
    reflecting for Cases 1a, 2a, 3a, and 4a, and vacuum for Cases 1b, 2b, 3b, 4b, 5a, and 5b.}
  \label{fig:case-geom}
\end{figure}

%\begin{table}[tb!]
%  \centering
%  \caption{Description of the 1-D Test Case Geometries.}
%  \begin{tabular}{l c c c}
%    \toprule
%    Cases & Ordered list of regions & Ordered list of interface x-coordinates & Left \& Right BCs\\
%    \midrule
%    Case 0 & Control rod, homogenized fuel-graphite lattice &
%    \bottomrule
%  \end{tabular}
%  \label{table:c0k}
%\end{table}

I ran all cases with four neutron energy groups bounded at $E=10^{-5}, 10^0, 10^2, 10^5, 10^8$
eV. Figures \ref{fig:spectrum} and \ref{fig:reaction} show how this energy group structure
partitions the neutron flux energy spectrum and neutron reaction rates. For this preliminary work,
the four-group structure is sufficient for showing different scattering and absorption trends of
slow, intermediate, and fast neutrons. I also ran all cases on OpenMC to generate the required
group constants and to assess the accuracy of my deterministic methods against OpenMC in
continuous-energy (OpenMC-CE) and multigroup (OpenMC-MG) modes. Table \ref{table:var} shows how the
position, direction of travel, neutron energy, and angle-dependence in $\Sigma_s$ are handled by
OpenMC and the deterministic methods. Comparing OpenMC-MG results with OpenMC-CE results allows us
to quantify errors arising from neutron energy group discretization and the scattering cross
section simplifications.

\begin{figure}[htb!]
  \centering
  \begin{subfigure}[t]{.49\textwidth}
    \centering
    \includegraphics[width=\textwidth]{spectrum}
    \caption{Neutron flux energy spectrum per source neutron}
    \label{fig:spectrum}
  \end{subfigure}
  \hfill
  \begin{subfigure}[t]{.49\textwidth}
    \centering
    \includegraphics[width=\textwidth]{reaction}
    \caption{Flux-normalized scattering and absorption reaction rates}
    \label{fig:reaction}
  \end{subfigure}
  \caption{Neutron flux energy spectrum and reaction rates for Case 5a. The dotted vertical lines
  correspond to the discrete neutron energy group boundaries.}
  \label{fig:spec-reac}
\end{figure}
%
\begin{table}[tb!]
  \centering
  \footnotesize
  \caption{Variable handling in OpenMC under continuous-energy (OpenMC-CE) and multigroup
  (OpenMC-MG) modes, and in the $S_4$ neutron transport, neutron diffusion, and Hybrid
  $S_N$-Diffusion methods. }
  \begin{tabular}{c c c c c c}
    \toprule
    Variable & OpenMC-CE & OpenMC-MG & $S_4$ & Diffusion & Hybrid \\
    \midrule
    Position, $\bm{r}$ & Continuous & Continuous & Discrete & Discrete & Discrete \\
    Direction of travel, $\bm{\hat{\Omega}}$ & Continuous & Continuous & Discrete & N/A & N/A \\
    Energy, $E$ & Continuous & Discrete & Discrete & Discrete & Discrete \\
    Angle-dependence in $\Sigma_s$ & Continuous & 1st-order Legendre & 1st-order Legendre
    & N/A & N/A \\
    \bottomrule
  \end{tabular}
  \label{table:var}
\end{table}

Based on a mesh convergence study on Case 2b, all three methods reach reasonable convergence
with a mesh size of $\Delta x=0.0125$ cm; further mesh refinement results in less than 0.00030
change in $k$.

\section{Results \& Discussion} \label{sec:prelim-results}

In this section, I will compare various neutronics parameters for all test cases calculated from
the OpenMC and deterministic methods. I refer to the OpenMC calculations under continuous-energy
and multigroup modes as OpenMC-CE and OpenMC-MG, respectively.

\subsection{Comparison of Multiplication factors, $k$}

Tables \ref{table:ck1} and \ref{table:ck2} show the $k$ estimates from the OpenMC and deterministic
methods for all test cases. The tables also include statistical uncertainties for the OpenMC
$k$ estimates. I did not apply the hybrid method for cases which do not contain control rod
regions. 

All five methods generally show loose agreement with one another within each test case. The
differences between the OpenMC-CE and OpenMC-MG $k$ estimates vary significantly from 0.00011 in
Case 1a to 0.01667 in Case 4b. These Monte Carlo methods exhibit greater differences in cases
containing reflector regions (e.g. 1b, 2b, 3b, 4b, 5a, 5b). The reflector region consists of an
equal volume mixture of steel and water. The hydrogen in water is a strong neutron moderator, and
it induces highly anisotropic scattering due to its small atomic mass. Therefore, the differences
imply that the neutron energy group discretization scheme and/or the Legendre expansion orders of
scattering cross sections are insufficient for accurately modeling neutron flux in the reflector
regions.

\begin{table}[htb!]
  \centering
  \footnotesize
  \caption{Multiplication factor $k$ estimates for Cases 1a, 1b, 2a, 2b, 3a, and 3b from the
    OpenMC-CE, OpenMC-MG, $S_4$ neutron transport, neutron diffusion, and Hybrid $S_N$-Diffusion
    methods.}
  \begin{tabular}{c S[table-format=1.7] S[table-format=1.7] S[table-format=1.7] S[table-format=1.7]
  S[table-format=1.7] S[table-format=1.7]}
    \toprule
    \multirow{2}{*}{\textbf{Method}} &
    \multicolumn{6}{c}{\textbf{Multiplication factor,} $\bm{k}$} \\
    \cmidrule{2-7}
    & {\textbf{Case 1a}} & {\textbf{Case 1b}} & {\textbf{Case 2a}} &
    {\textbf{Case 2b}} & {\textbf{Case 3a}} & {\textbf{Case 3b}} \\
    \midrule
    OpenMC-CE & 1.60033(56) & 1.16749(60) & 1.20666(66) & 0.69499(47) & 1.19908(63) & 0.68565(53)\\
    OpenMC-MG & 1.60044(33) & 1.18123(60) & 1.20675(62) & 0.70622(44) & 1.20043(48) & 0.69751(50)\\
    $S_4$     & 1.60036     & 1.18050     & 1.20474     & 0.70302     & 1.19844     & 0.69483    \\
    Diffusion & 1.60036     & 1.17940     & 1.19666     & 0.69210     & 1.19020     & 0.68386    \\
    Hybrid    & {N/A}       & {N/A}       & 1.20446     & 0.70156     & 1.19816     & 0.69336    \\
    \bottomrule
  \end{tabular}
  \label{table:ck1}
\end{table}
%
\begin{table}[htb!]
  \centering
  \footnotesize
  \caption{Multiplication factor $k$ estimates for Cases 4a, 4b, 5a, and 5b from the OpenMC-CE,
    OpenMC-MG, $S_4$ neutron transport, neutron diffusion, and Hybrid $S_N$-Diffusion methods.}
  \begin{tabular}{c S[table-format=1.7] S[table-format=1.7] S[table-format=1.7]
    S[table-format=1.7]}
    \toprule
    \multirow{2}{*}{\textbf{Method}} &
    \multicolumn{4}{c}{\textbf{Multiplication factor,} $\bm{k}$} \\
    \cmidrule{2-5}
    & {\textbf{Case 4a}} & {\textbf{Case 4b}} & {\textbf{Case 5a}} &
    {\textbf{Case 5b}} \\
    \midrule
    OpenMC-CE & 1.64506(53) & 1.19116(63) & 0.70352(64) & 1.16832(58) \\
    OpenMC-MG & 1.64434(33) & 1.20783(59) & 0.71513(55) & 1.18279(59) \\
    $S_4$     & 1.64365     & 1.20790     & 0.71280     & 1.18252     \\
    Diffusion & 1.64395     & 1.20819     & 0.70428     & 1.18274     \\
    Hybrid    & {N/A}       & {N/A}       & 0.71396     & {N/A}       \\
    \bottomrule
  \end{tabular}
  \label{table:ck2}
\end{table}
%
\begin{table}[htb!]
  \centering
  \footnotesize
  \caption{Differences in $k$ estimates for Cases 1a, 1b, 2a, 2b, 3a, and 3b for the $S_4$ neutron
    transport, neutron diffusion, and Hybrid $S_N$-Diffusion methods relative to OpenMC-MG.}
  \begin{tabular}{c S[table-format=1.7] S[table-format=1.7] S[table-format=1.7] S[table-format=1.7]
  S[table-format=1.7] S[table-format=1.7]}
    \toprule
    \multirow{2}{*}{\textbf{Method}} & \multicolumn{6}{c}{$\bm{k-k_{MG}}$} \\
    \cmidrule{2-7}
    & {\textbf{Case 1a}} & {\textbf{Case 1b}} & {\textbf{Case 2a}} &
    {\textbf{Case 2b}} & {\textbf{Case 3a}} & {\textbf{Case 3b}} \\
    \midrule
    $S_4$     & -0.00008 & -0.00073 & -0.00201 & -0.00320 & -0.00199 & -0.00268 \\
    Diffusion & -0.00008 & -0.00183 & -0.01009 & -0.01412 & -0.01023 & -0.01365 \\
    Hybrid    & {N/A}    & {N/A}    & -0.00229 & -0.00466 & -0.00227 & -0.00415 \\
    \bottomrule
  \end{tabular}
  \label{table:ckdiff1}
\end{table}
%
\begin{table}[htb!]
  \centering
  \footnotesize
  \caption{Differences in $k$ estimates for Cases 4a, 4b, 5a, and 5b for the $S_4$ neutron
    transport, neutron diffusion, and Hybrid $S_N$-Diffusion methods relative to OpenMC-MG.}
  \begin{tabular}{c S[table-format=1.7] S[table-format=1.7] S[table-format=1.7]
    S[table-format=1.7]}
    \toprule
    \multirow{2}{*}{\textbf{Method}} & \multicolumn{4}{c}{$\bm{k-k_{MG}}$} \\
    \cmidrule{2-5}
    & {\textbf{Case 4a}} & {\textbf{Case 4b}} & {\textbf{Case 5a}} &
    {\textbf{Case 5b}} \\
    \midrule
    $S_4$     & -0.00069 & +0.00007 & -0.00233 & -0.00027 \\
    Diffusion & -0.00039 & +0.00036 & -0.01085 & -0.00005 \\
    Hybrid    & {N/A}    & {N/A}    & -0.00117 & {N/A}    \\
    \bottomrule
  \end{tabular}
  \label{table:ckdiff2}
\end{table}
%
\begin{table}[tb!]
  \centering
  \footnotesize
  \caption{Control rod worths calculated as changes in the reactivity between Case 5a and 5b for
    the OpenMC continuous-energy and multigroup mode calculations, and the $S_4$ neutron transport,
  neutron diffusion, and Hybrid $S_N$-Diffusion methods.}
  \begin{tabular}{c S}
    \toprule
    \multirow{2}{*}{\textbf{Method}} & {\textbf{Control rod worth}} \\
                                     & {$\bm{\rho_{(5b)}-\rho_{(5a)}}$} \\
    \midrule
    OpenMC-CE & 0.56549 \\
    OpenMC-MG & 0.55289 \\
    $S_4$     & 0.55727 \\
    Diffusion & 0.57439 \\
    Hybrid    & 0.55515 \\
    \bottomrule
  \end{tabular}
  \label{table:rod-worth}
\end{table}

In contrast with OpenMC-CE, the $k$ estimates from the $S_4$ neutron transport method show closer
agreement with the estimates from OpenMC-MG. Tables \ref{table:ckdiff1} and \ref{table:ckdiff2}
shows the differences in $k$ between OpenMC-MG and the deterministic methods. The differences in
$k$ range from 0.00008 in Case 1a to 0.00320 in Case 2b. Given that the mesh is sufficiently fine
for the $S_4$ method, the only other significant difference between these methods is the
discretization of the direction of travel variable $\hat{\Omega}$. The flux distributions exhibit
more apparent differences as I will discuss in the next subsection. Nevertheless, the $S_4$
method is sufficiently accurate and consistent for assessing control rod neutronics modeling with
the Hybrid $S_N$-Diffusion method.

As expected, the neutron diffusion method performs significantly worse than the $S_4$ method in
Cases 2, 3, and 5a which all contain control rod regions. It deviates from the OpenMC-MG $k$
estimate by up to 0.01412 in Case 2b compared to 0.00320 for the $S_4$ method. The hybrid method
proves to be effective as it improves the $k$ estimate to almost match the $S_4$ method results for
the cases with control rods.

Finally, I calculated the control rod worth by taking the change in reactivity $\rho$ between Case
5a and 5b for all five methods as shown in Table \ref{table:rod-worth}. The formula for $\rho$ is
given as
%
\begin{align}
  \rho =& \frac{k-1}{k}.
\end{align}
%
Since I did not run the hybrid method for Case 5b, I took the $k$ estimate from the neutron
diffusion method as the reference non-rodded for the control rod worth calculation for the hybrid
method. Compared with OpenMC-MG, the control rod worth from the neutron diffusion method deviates
the most by 3.9\% as opposed to 0.8\% and 0.4\% by the $S_4$ and hybrid methods, respectively.

\subsection{Comparison of Neutron Flux Distributions, $\phi$}

I focus my discussion in this section on Cases 3b and 5a which represent the most complex
geometries with homogeneous and lattice fuel-graphite regions. I sampled the flux distributions in
OpenMC-CE and OpenMC-MG across coarser $0.1$-cm intervals to reduce statistical uncertainties of
each individual flux reading.

\begin{figure}[htb!]
  \centering
  \begin{subfigure}[t]{.49\textwidth}
    \centering
    \includegraphics[width=\textwidth]{case-3b-group-1-flux}
    \caption{Group 1}
    \label{fig:c3bg1}
  \end{subfigure}
  \hfill
  \begin{subfigure}[t]{.49\textwidth}
    \centering
    \includegraphics[width=\textwidth]{case-3b-group-2-flux}
    \caption{Group 2}
    \label{fig:c3bg2}
  \end{subfigure}
  \hfill
  \begin{subfigure}[t]{.49\textwidth}
    \centering
    \includegraphics[width=\textwidth]{case-3b-group-3-flux}
    \caption{Group 3}
    \label{fig:c3bg3}
  \end{subfigure}
  \hfill
  \begin{subfigure}[t]{.49\textwidth}
    \centering
    \includegraphics[width=\textwidth]{case-3b-group-4-flux}
    \caption{Group 4}
    \label{fig:c3bg4}
  \end{subfigure}
  \caption{Neutron group flux distributions for Case 3b from OpenMC-CE, OpenMC-MG, $S_4$, neutron
  diffusion, and hybrid methods.}
  \label{fig:c3bflux}
\end{figure}

Figure \ref{fig:c3bflux} shows the neutron flux distributions for groups 1 to 4. Notably, the
OpenMC-CE flux distribution is the most dissimilar to the rest of the distributions likely due to
the inadequate neutron energy group discretization scheme. Consequently, I compared the
deterministic methods with OpenMC-MG to eliminate energy discretization errors. Figure
\ref{fig:c3bfluxe} shows the relative differences in the flux distributions from $S_4$, neutron
diffusion, and hybrid methods with respect to OpenMC-MG. The neutron diffusion method
performs worse than $S_4$, especially near the control rod region between $x=0$ cm and $x=20$ cm
for the slower group 3 and 4 neutron flux. Although the control rod region spans between $x=0$ cm
and $x=0.5$ cm only, it induces strong flux gradients in its vicinity and thus renders the neutron
diffusion method less valid. The hybrid method brings about significant improvements to all four
group fluxes in this region. Beyond $x=40$ cm, both neutron diffusion and hybrid methods produce
similar flux error distributions due to the influence of the material interface between the
fuel-graphite and reflector regions.

\begin{figure}[htb!]
  \centering
  \begin{subfigure}[t]{.49\textwidth}
    \centering
    \includegraphics[width=\textwidth]{case-3b-group-1-flux-error}
    \caption{Group 1}
    \label{fig:c3bg1e}
  \end{subfigure}
  \hfill
  \begin{subfigure}[t]{.49\textwidth}
    \centering
    \includegraphics[width=\textwidth]{case-3b-group-2-flux-error}
    \caption{Group 2}
    \label{fig:c3bg2e}
  \end{subfigure}
  \hfill
  \begin{subfigure}[t]{.49\textwidth}
    \centering
    \includegraphics[width=\textwidth]{case-3b-group-3-flux-error}
    \caption{Group 3}
    \label{fig:c3bg3e}
  \end{subfigure}
  \hfill
  \begin{subfigure}[t]{.49\textwidth}
    \centering
    \includegraphics[width=\textwidth]{case-3b-group-4-flux-error}
    \caption{Group 4}
    \label{fig:c3bg4e}
  \end{subfigure}
  \caption{Relative differences of the neutron group flux distributions for Case 3b from the $S_4$,
    neutron diffusion, and hybrid methods with respect to OpenMC-MG.}
  \label{fig:c3bfluxe}
\end{figure}

For a quantitative assessment, I defined the normalized flux error $\epsilon$ for each energy group
with respect to the OpenMC-MG flux distribution using a normalized Frobenius/Euclidean norm formula
given as
%
\begin{align}
  \epsilon_g =& \frac{\lVert\sum^I_{i=1}\phi_{g,i} - \phi^{MG}_{g,i}\rVert_F}
  {\lVert\sum^I_{i=1}\phi^{MG}_{g,i}\rVert_F} =
  \frac{\left[\sum^I_{i=1}\lvert\phi_{g,i} - \phi^{MG}_{g,i}\rvert^2 \right]^{\sfrac{1}{2}}}
  {\left[\sum^I_{i=1}\lvert\phi^{MG}_{g,i}\rvert^2\right]^{\sfrac{1}{2}}}
\end{align}
%
I integrated on the fine flux distributions over 0.1-cm intervals from $S_4$, neutron diffusion,
and hybrid methods to match the 0.1-cm intervals of the OpenMC-MG flux distribution.
Table \ref{table:c3berror} shows the $\epsilon$ values for the deterministic methods. As expected,
the hybrid method produces smaller flux error norms than the neutron diffusion method. The
improvements are more pronounced for group 3 and 4 fluxes.
%
\begin{table}[tb!]
  \centering
  \footnotesize
  \caption{Normalized flux error $\epsilon$ for Case 3b from the $S_4$, neutron diffusion, and
    hybrid methods with respect to OpenMC-MG.}
  \begin{tabular}{c S S S S}
    \toprule
    {\multirow{2}{*}{\textbf{Method}}} &
    \multicolumn{4}{c}{\textbf{Normalized flux error,} $\bm{\epsilon_g}$} \\
    \cmidrule{2-5}
    & {\textbf{Group 1}} & {\textbf{Group 2}} & {\textbf{Group 3}} &
    {\textbf{Group 4}} \\
    \midrule
    $S_4$     & 0.0070 & 0.0080 & 0.0092 & 0.0082 \\
    Diffusion & 0.0141 & 0.0172 & 0.0275 & 0.0272 \\
    Hybrid    & 0.0116 & 0.0128 & 0.0136 & 0.0107 \\
    \bottomrule
  \end{tabular}
  \label{table:c3berror}
\end{table}

Case 5a introduces significant geometrical heterogeneity from the fuel-graphite lattice region.
This is very apparent from the flux distributions in Figure \ref{fig:c5aflux}. The OpenMC Monte
Carlo methods captured the fluctuations in the flux with their continuous angle variable as opposed
to the $S_4$, neutron diffusion, and hybrid methods. The group 1 flux peaks are notably pronounced
since all fission neutrons are born in the fuel region with most of them in group 1. Figure
\ref{fig:c5afluxe} shows the relative differences in flux for the $S_4$, neutron diffusion, and
hybrid methods relative to OpenMC-MG. All three methods show regular fluctuations in the relative
flux differences corresponding to the fuel-graphite lattice. The $S_4$ method generally deviates
the least from OpenMC-MG while the neutron diffusion method exhibits characteristically
significant flux deviations near the control rod region prior to $x=20$ cm and the material
interface between the fuel and reflector regions beyond $x=40$ cm.

\begin{figure}[htb!]
  \centering
  \begin{subfigure}[t]{.49\textwidth}
    \centering
    \includegraphics[width=\textwidth]{case-5a-group-1-flux}
    \caption{Group 1}
    \label{fig:c5ag1}
  \end{subfigure}
  \hfill
  \begin{subfigure}[t]{.49\textwidth}
    \centering
    \includegraphics[width=\textwidth]{case-5a-group-2-flux}
    \caption{Group 2}
    \label{fig:c5ag2}
  \end{subfigure}
  \hfill
  \begin{subfigure}[t]{.49\textwidth}
    \centering
    \includegraphics[width=\textwidth]{case-5a-group-3-flux}
    \caption{Group 3}
    \label{fig:c5ag3}
  \end{subfigure}
  \hfill
  \begin{subfigure}[t]{.49\textwidth}
    \centering
    \includegraphics[width=\textwidth]{case-5a-group-4-flux}
    \caption{Group 4}
    \label{fig:c5ag4}
  \end{subfigure}
  \caption{Neutron group flux distributions for Case 5a from OpenMC-CE, OpenMC-MG, $S_4$, neutron
  diffusion, and hybrid methods.}
  \label{fig:c5aflux}
\end{figure}
%
\begin{figure}[htb!]
  \centering
  \begin{subfigure}[t]{.49\textwidth}
    \centering
    \includegraphics[width=\textwidth]{case-5a-group-1-flux-error}
    \caption{Group 1}
    \label{fig:c5ag1e}
  \end{subfigure}
  \hfill
  \begin{subfigure}[t]{.49\textwidth}
    \centering
    \includegraphics[width=\textwidth]{case-5a-group-2-flux-error}
    \caption{Group 2}
    \label{fig:c5ag2e}
  \end{subfigure}
  \hfill
  \begin{subfigure}[t]{.49\textwidth}
    \centering
    \includegraphics[width=\textwidth]{case-5a-group-3-flux-error}
    \caption{Group 3}
    \label{fig:c5ag3e}
  \end{subfigure}
  \hfill
  \begin{subfigure}[t]{.49\textwidth}
    \centering
    \includegraphics[width=\textwidth]{case-5a-group-4-flux-error}
    \caption{Group 4}
    \label{fig:c5ag4e}
  \end{subfigure}
  \caption{Relative differences of the neutron group flux distributions for Case 5a from the $S_4$,
    neutron diffusion, and hybrid methods with respect to OpenMC-MG.}
  \label{fig:c5afluxe}
\end{figure}
%
\begin{table}[tb!]
  \centering
  \footnotesize
  \caption{Normalized flux error $\epsilon$ for Case 5a from the $S_4$, neutron diffusion, and
    hybrid methods with respect to OpenMC-MG.}
  \begin{tabular}{c S S S S}
    \toprule
    {\multirow{2}{*}{\textbf{Method}}} &
    \multicolumn{4}{c}{\textbf{Normalized flux error,} $\bm{\epsilon_g}$} \\
    \cmidrule{2-5}
    & {\textbf{Group 1}} & {\textbf{Group 2}} & {\textbf{Group 3}} &
    {\textbf{Group 4}} \\
    \midrule
    $S_4$     & 0.0149 & 0.0079 & 0.0127 & 0.0187 \\
    Diffusion & 0.0300 & 0.0197 & 0.0336 & 0.0445 \\
    Hybrid    & 0.0283 & 0.0145 & 0.0188 & 0.0285 \\
    \bottomrule
  \end{tabular}
  \label{table:c5aerror}
\end{table}



\section{Summary}

The Hybrid $S_N$-Diffusion method is promising for studying reactor systems such as \glspl{MSR},
which consist of mostly highly scattering regions (small neutron mean free path), because the $S_N$
subproblem domain $\Omega^d_1$ covers a small fraction of the entire reactor geometry. This
reduces the computational costs of the iterative $S_N$ calculations in the hybrid method.

I plan to perform further investigations to replace the approximate energy group discretization
used here and to use higher-order Legendre expansions of the scattering cross sections.

\glsresetall

\chapter{Proposed Work}
\label{chap:proposedwork}
Chapter \ref{chap:intro} introduced the unique characteristics of liquid-fuel
\glspl{MSR}
and highlighted the challenges of modeling liquid-fuel \glspl{MSR}, particularly
with legacy software designed for solid-fuel reactors. In addition, Chapter
\ref{chap:lit} summarized and discussed existing \gls{MSR} multiphysics simulation tools and
their capabilities. In this discussion, I noted the lack of capabilities for modeling control rod
movement in multiphysics tools for \glspl{MSR}. Chapter \ref{chap:moltres} illustrated general
features and physics models in Moltres and summarized previous work done with
Moltres for multiphysics modeling of \glspl{MSR}. The latter sections detailed
several limitations in \gls{MSR} multiphysics modeling with
Moltres, including the need for further verification of
Moltres' capabilities, a turbulence model for simulating turbulent flow, and a
control rod modeling capability for transient simulations. In turn,
Chapter \ref{chap:benchmark} detailed a verification study of Moltres for fast-spectrum
\gls{MSR} modeling capabilities by evaluating its performance
through the CNRS benchmark and comparing its results against other
\gls{MSR} simulation tools. Lastly, Chapter \ref{chap:hybrid} presented the theory and preliminary
1-D results of the novel hybrid $S_N$-diffusion method for improved control rod modeling over
standard neutron diffusion methods.

In this chapter, I will detail the proposed work to further verify and validate Moltres' \gls{MSR}
modeling capabilities, and I will describe the further enhancements to Moltres that I will
implement to address the
need for enhanced turbulence modeling and control rod modeling in Moltres. The
proposed work contributes towards improving on Moltres' capabilities for
multiphysics modeling of \glspl{MSR} and, by extension, of advanced reactors.
Section \ref{sec:vv-study} details the proposed \gls{VV} study of Moltres based on the \gls{MSRE}
pump start-up and coast-down experiments. Section \ref{sec:turb} describes the proposed
implementation of a \gls{RANS}-based turbulence model in Moltres. Section \ref{sec:devel-hybrid}
describes the proposed development and implementation of the hybrid $S_N$-diffusion in Moltres.
Lastly, Section \ref{sec:devel-conclusion} summarizes the chapter and describes the research impact
of my proposed work on \gls{MSR} modeling.

\section{Verification \& Validation Study Based on the MSRE Pump Start-up and Coast-Down
Experiments} \label{sec:vv-study}

In addition to the completed verification study described in Chapter \ref{chap:benchmark}, I will
perform a \gls{VV} study of Moltres based on the \gls{MSRE} transient flow-rate tests,
consisting of fuel pump start-up and coast-down experiments at zero power criticality. The changing
flow rates cause changes to the \gls{DNP} distribution and the \gls{DNF} in the core. During both
experiments, the reactivity effects of \gls{DNP} drift were measured by
allowing the flux servo controller to maintain criticality. The controller maintains core
criticality by adjusting the control rod position in response to the reactivity effects. The
reactivity effect over time can be calculated by comparing the various control rod positions
(Figure \ref{fig:msre-trans}) with the control rod worth curve (Figure \ref{fig:msre-rod})
obtained during earlier experiments. I aim to
reproduce the calculated reactivity curve with a Moltres model of the \gls{MSRE}.

\begin{figure}[htb!]
  \centering
  \begin{minipage}[t]{.49\textwidth}
    \centering
    \includegraphics[width=\textwidth]{msre-transient}
    \caption{Control rod response to fuel pump start-up and coast-down
    \cite{prince_zero-power_1968}.}
    \label{fig:msre-trans}
  \end{minipage}
  \hfill
  \begin{minipage}[t]{.49\textwidth}
    \centering
    \includegraphics[width=\textwidth]{msre-rod-worth}
    \caption{Integral worth of control rod no. 1 \cite{prince_zero-power_1968}.}
    \label{fig:msre-rod}
  \end{minipage}
\end{figure}
%
\begin{figure}[htb!]
  \centering
  \includegraphics[width=.5\textwidth]{msre-2d}
  \caption{2-D axisymmetric model of the \gls{MSRE} to be used for the \gls{VV} study.}
  \label{fig:msre-2d}
\end{figure}

For this study, I will collaborate with Aaron Reynolds, the developer of QuasiMolto
\cite{reynolds_analysis_2023},
so that our software can be compared accurately. Therefore, Moltres
will be validated against \gls{MSRE} experimental data and verified against QuasiMolto simulation
results simultaneously. Both QuasiMolto and Moltres models of the \gls{MSRE} will be in R-Z
coordinates because QuasiMolto supports only 2-D R-Z \gls{MSR} modeling. Figure \ref{fig:msre-2d}
shows the 2-D axisymmetric model of the \gls{MSRE} that we intend to use
for this \gls{VV} study. The model omits the control rod in the \gls{MSRE}. Instead, we will
conduct $k$-eigenvalue calculations at every time-step to obtain the multiplication factor and
scale the \gls{DNP} source term. The salt flow speeds will be taken directly from available data in
the \gls{MSRE} report \cite{prince_zero-power_1968}. The datasets that we intend to compare between
QuasiMolto and Moltres for verification are:
%
\begin{itemize}
  \item Change in reactivity relative to the static (no flow) configuration over time during the
    pump start-up and coast-down transients
  \item Axial and radial neutron flux distributions of the static and steady-state (full flow)
    configurations
  \item Axial and radial \gls{DNP} distributions of the static and steady-state configurations
\end{itemize}

Since both QuasiMolto and Moltres will adopt the neutron diffusion method with the same flow
profile, we expect our results to be highly consistent with each other. We expect any discrepancies
between QuasiMolto and Moltres to be minor and due to differences in numerical discretization
schemes and routines. We will collaborate to eliminate errors arising from model misimplementation.

We also expect our reactivity data to be similar to the control rod response measured in Figure
\ref{fig:msre-trans}. However, our results are unlikely to match the experimental data exactly due
to known deficiencies in the physical \gls{MSRE} experimental setup and the simplifications we
adopted in our 2-D models. Among the model simplifications that we adopted, we expect the omission
of the upper and lower reactor plena to impact our results the most because the plena contain a
large volume of fuel salt above and below the fuel-graphite lattice. Therefore, as an extension, I
will consider exploring ways to include the plena in our models if our results deviate
significantly from the experimental data.

\section{Implementation of a RANS-Based Turbulence Model in Moltres} \label{sec:turb}

In Chapter \ref{chap:lit}, I presented requirements for \gls{MSR} modeling related to turbulence.
To address the current limitations, I will implement a Spalart-Allmaras turbulence model which
falls under the class of \gls{RANS}-based turbulence models. The
model will be implemented within the \gls{MOOSE} framework and
designed to be compatible with the fluid dynamics modeling infrastructure in
the existing \texttt{Navier-Stokes} module. This approach leverages the
advanced finite-element solver and multiphysics coupling capabilities in
\gls{MOOSE}.

I will
focus on verifying and validating its performance under specific flow conditions expected in
liquid-fuel \glspl{MSR}: wall-bounded turbulent flow with flow separation past
sharp changes in the flow channel geometry. The backward-facing step is a
widely known fluid dynamics problem commonly used to assess the accuracy of
turbulence model solvers \cite{lasher_computation_1992}.
The problem domain features a straight duct
on the left followed by a sudden back step in the lower wall which causes flow
separation. The flow eventually reattaches to the wall further downstream.
I will simulate and validate against experimental data from Driver \&
Seegmiller \cite{driver_features_1985}. The quantities of interest from their data that I plan to
use are the spatial distributions of the velocity components, turbulent
kinetic energy, and eddy viscosity, and the reattachment length of the
turbulent shear layer.

Potential issues that I may face include the lack of a crosswind diffusion stabilization scheme in
\gls{MOOSE} and the fact that local mass conservation is not guaranteed when modeling the
Navier-Stokes equations using the \gls{FEM} without special treatment. If these issues prove to be
insurmountable within a reasonable time frame, I will focus my efforts on adopting the mixing
length turbulence models currently available in \gls{MOOSE} or two-equation models which are in
active development with the recent \gls{FVM} implementation in \gls{MOOSE}. In this alternate
scenario, I will explore demonstrating coupled neutronics/thermal-hydraulics simulations with
turbulent flow modeling of the \gls{MSFR}.

\section{Development of a Novel Hybrid Method to Improve Control Rod Modeling in Moltres}
\label{sec:devel-hybrid}

In Chapter \ref{chap:hybrid}, I presented the theory and preliminary results of the hybrid
$S_N$-diffusion method for several 1-D \gls{MSRE}-inspired, graphite-moderated geometries. The
hybrid method generates \glspl{SVDC}, which provide pointwise corrections to the diffusion
sub-solver from
the $S_N$ sub-solver neutron current and flux gradient solutions. The hybrid method minimizes
computational costs by imposing the $S_N$ calculations on a small subdomain centered on the control
rod region and relaxing the $S_N$ sub-solver convergence tolerance since the neutron current and
flux gradient converges faster than the neutron flux. As intended,
the hybrid method provided better multiplication factor and neutron flux estimates over the
standard neutron diffusion method in systems containing strongly neutron-absorbing control rods.

In this section, I present a detailed description of the proposed work for the development and
implementation of the hybrid method in Moltres, the verification of the hybrid method against
reference OpenMC neutronics calculations, and the computational performance characterization of the
hybrid method.

\subsection{Development \& Implementation of the Hybrid Method}

The preliminary investigations in Chapter \ref{chap:hybrid} brought up three issues to be
addressed for the continued development of the hybrid $S_N$-diffusion method. The first issue
surrounds resolving \glspl{SVDC} near neutron flux peaks and troughs. The existing formulation
for \glspl{SVDC} in Eq. \ref{eq:svdc} are undefined at flux peaks and troughs due to division by
the flux gradient. I will explore alternative formulations to address this issue. For instance,
Tomatis \& Dall'Osso \cite{tomatis_application_2011} developed the formulation in Eq.
\ref{eq:tomatis} to address the same issue in their implementation of the Ronen method. Eq.
\ref{eq:tomatis} calculates corrections to the neutron streaming terms as additive
$\delta J$ terms. I will evaluate whether their formulation can be adapted to the hybrid
$S_N$-diffusion method. Otherwise, I plan to investigate similar additive correction formulations
for the hybrid $S_N$-diffusion method.

The second issue concerns the evaluation of appropriate correction regions and buffer zones for the
hybrid method. The correction region represents the problem domain of the $S_N$ sub-solver. It must
be large enough to provide sufficient transport correction in regions heavily influenced by the
control rod and to accommodate the discarding of inaccurate \glspl{SVDC} expected near the
boundary. I will investigate various test cases in 2-D and 3-D, similar to my preliminary work in
this report, to identify how various geometrical and material properties affect the minimum
required size of the correction region. For instance, my preliminary work showed that the
geometrical heterogeneity and optical thickness strongly affects the \gls{SVDC} distributions,
which are a measure of the amount of transport correction required. With my expected findings, I
will generate a set of criteria for determining the appropriate correction region size for a given
reactor geometry.

The last issue concerns the automatic evaluation of buffer zones which are regions near the
boundaries of the correction region where inaccurate \glspl{SVDC} are discarded. For the 1-D
problems, I set up an outward sweeping algorithm that compares the \glspl{SVDC} with the default
$P_1$-based diffusion coefficients. The implementation of this sweeping algorithm is less clear for
2-D and 3-D problems which feature more than two angular directions. I will tackle this issue
in conjunction with the second issue by analyzing the \gls{SVDC} distributions obtained from
2-D and 3-D test cases.

Naturally, the implementation and extension of the hybrid method in Moltres for 2-D and 3-D
modeling must precede my attempts at resolving the second and third issue. I will implement the
$S_N$ method with diffusion synthetic acceleration in Moltres or as a separate MOOSE-based
application. I will also implement supporting features for the coupling the $S_N$ solver with the
existing diffusion solver to run the hybrid method.

\subsection{Verification of the Hybrid Method}

I will verify the hybrid method against reference OpenMC calculations with several test cases
similar to the 1-D test cases in Chapter \ref{chap:hybrid}. The test cases will be various 2-D
and 3-D representations of \gls{MSRE}-inspired models with the fuel-graphite lattice, the control
rod, the air-filled rod guide tube, and reflectors. The verification study will include
permutations of the following factors:
%
\begin{itemize}
  \item Dimensionality (e.g, 2-D, 3-D)
  \item Geometrical symmetries and asymmetries related to the control rod position
  \item Static control rods at various levels of insertion
  \item Control rod material composition
\end{itemize}

\subsection{Computational Performance Characterization of the Hybrid Method}

I intend for the hybrid $S_N$-diffusion method to be tractable on small computing clusters for
time-dependent multiphysics simulations. Therefore, I will characterize the computational
performance of the hybrid method and compare it against the standard $S_N$ and neutron diffusion
methods. I expect the hybrid method to be faster than the standard $S_N$ method due to the small
correction region size relative to the full reactor geometry and the faster convergence of
\glspl{SVDC} relative to the neutron flux in the hybrid $S_N$ sub-solvers. I plan to perform the
characterization with some of the 2-D and 3-D verification study test cases.

In addition, I will evaluate the hybrid method's parallel scaling performance to identify potential
performance bottlenecks in the implementation. Good parallel scaling performance is essential for
leveraging on computational resources for large reactor simulations. I will run both weak and
strong scaling studies using some of the 2-D and 3-D test cases for this evaluation.

\section{Conclusion} \label{sec:devel-conclusion}

In this chapter, I detailed the proposed work to verify and validate Moltres' existing capabilities
for modeling \gls{DNP} drift and out-of-core decay in \glspl{MSR}, implement a \gls{RANS}-based
turbulence model in Moltres, and develop a hybrid $S_N$-diffusion method for improved control
rod modeling in Moltres. My work will extend the applicability of Moltres for a wider range of
MSR safety analyses involving turbulent flow and control rod effects. Most notably, the hybrid
method will enable Moltres users to run time-dependent multiphysics \gls{MSR} simulations with
control rod movement and address this technical gap in the existing literature. Through my work on
the hybrid method, I will also characterize and provide insights on neutron transport effects in
\gls{MSR} that are neglected by the neutron diffusion method.

\glsresetall

\backmatter

\bibliographystyle{IEEEtran}
\bibliography{bibliography}

\end{document}
\endinput
%%
