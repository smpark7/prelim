\section{Background and Motivation}

Greenhouse gas emission from human activities is the main cause of climate
change, which has dire consequences on human health and safety due to extreme
weather events and the overall impact on food production
\cite{mcmichael_global_2004}. Thus, many countries have pledged to meet the
net zero carbon emissions goal by 2050 to limit the rise in global temperatures
to 1.5$^{\circ}$C (ref iea). In this low-carbon future, we expect to see
renewable energy sources such as solar \gls{PV} and wind making up the bulk of
electricity production capacity and a mass electrification of the transport
industry through either adopting the use of battery-powered vehicles or
substituting fossil fuels with hydrogen-based fuels produced from low-carbon
energy sources or biomass. While solar \gls{PV} and wind have low levelized
cost of electricity at good geographical locations, they are non-dispatchable
energy sources since they depend on favorable weather conditions. According to
an interdisciplinary study from \gls{MIT} \cite{petti_future_2018}, without
sufficient dispatchable electricity sources to cover for periods when demand
outstrips supply, electricity generation costs will increase significantly
arising from the excess electricity storage capacity required to maintain
electricity supply when solar and wind outputs fall. Solar \gls{PV}
and wind are also ill-suited for meeting heat demand such as district heating
and industrial heat applications.

Nuclear power, especially from advanced reactor designs, can complement
renewable energy by addressing the aforementioned shortfalls. Unlike solar or
wind power, nuclear energy is a dispatchable energy source; it provides
consistent and reliable base-load power independent of weather conditions.
Beyond the electricity sector, nuclear energy also has the
potential to replace fossil fuels to meet energy demands from industrial
process heat applications and transportation (ref iaea).
Advanced reactors such as \glspl{MSR} and \glspl{HTGR} operate at sufficiently
high temperatures for chemical, petrochemical, steel, and other industries.
These include hydrogen and ammonia production, two carbon-free candidates for
replacing fossil fuel consumption in the aviation and marine sectors.

The world would have to ramp up the current rate of reactor deployments to
displace a portion of the presently large share of energy production from
fossil fuels. However, several obstacles stand in the way of mass
reactor deployments. These obstacles include perceived safety risks,
sustainability concerns, nuclear proliferation
risks, and the ability to compete economically with other sources of energy
\cite{massachusetts_institute_of_technology_future_2003}. A potential solution
to the aforementioned issues is the \gls{MSR} concept, one of six advanced
reactor designs selected by the Generation IV International Forum
\cite{gif_technology_2002} for continued research and development.

``\glspl{MSR}'' refer to all reactor designs that use molten salt as the
primary coolant. Liquid-fueled variants, which have fissile and/or fertile
materials dissolved directly in the coolants.,
possess an inherently robust safety feature in the strongly negative fuel
temperature coefficient of reactivity. Some designs can also incorporate the
thorium fuel cycle for improved sustainability arising from the use of
abundant natural thorium resources and reduced transuranic waste. The
latter also reduces economical costs
associated with long-term nuclear waste storage. In addition, the ability to
operate at near atmospheric pressures eliminates the need for a thick pressure
vessel and drives down construction costs, while online fuel reprocessing
reduces reactor downtime during reactor operation.

However, the liquid fuel form also brings about novel computational
challenges in simulating the transient behavior of \glspl{MSR}; the
neutronics and thermal-hydraulics are more strongly coupled due
to greater thermal expansion of the liquid fuel. Furthermore, fissile material
and \glspl{DNP} in \glspl{MSR} flow freely within the primary coolant
loop as opposed to being held in place in a solid fuel matrix. Therefore,
the choice of coupling method to model both sets of physics requires careful
consideration.

(Discussion on single physics codes vs multiphysics codes and coupling methods)

Most reactor analysis applications are usually reactor-specific by
design such as TRACE \cite{nrc_trace_nodate} for \glspl{LWR}, and
SAS4A/SASSYS-1 \cite{fanning_sas4a/sassys-1_2017} for
liquid metal cooled reactors. Thus, these applications would disregard
\gls{MSR}-specific phenomena and are inappropriate for \gls{MSR}
analysis without modifications to the source code. Some research efforts
do focus on adapting these applications for \gls{MSR} analysis. Examples
include the coupling of modified versions of TRACE and PARCS
\cite{pettersen_coupled_2016}, and the development of VERA-MSR from the
integrated \gls{LWR} simulation tool VERA \cite{graham_development_2019}.
Others developed their \gls{MSR} simulation tools from general
multiphysics or \gls{CFD} applications such as COMSOL
\cite{fiorina_modelling_2014} and OpenFOAM \cite{aufiero_development_2014}.

Similarly, Moltres \cite{lindsay_introduction_2018} is an open-source MSR
simulation tool built in the \gls{MOOSE} \cite{gaston_physics-based_2015}
parallel finite element framework. Lindsay et al.
\cite{lindsay_introduction_2018} first presented the tool in 2017 and
demonstrated its capabilities by simulating 2-D and 3-D models of the
\gls{MSRE}. The results showed good qualitative
agreement with the original design calculations by \gls{MSRE} researchers at
\gls{ORNL}. This thesis presents some of the new developments in Moltres
allowing for more complex and accurate \gls{MSR} simulations.

\section{Objectives}

This thesis demonstrates latest capabilities of Moltres
\cite{lindsay_introduction_2018}.
In particular, this thesis presents two more recent
developments in Moltres, namely fully integrating \gls{MOOSE}'s incompressible
Navier-Stokes module into Moltres, and introducing a
decay heat model.
The main objective of this thesis is to verify Moltres'
latest capabilities in modeling multiphysics, steady-state, and transient
behavior of fast-spectrum \glspl{MSR} through the study of the \gls{MSFR}
concept. Code-to-code verification is an important exercise in software
development for ensuring that the application produces accurate and reliable
results. This thesis covers the \gls{MSFR} concept mainly because it has been
studied extensively with readily available data in the literature to verify
against. The \gls{MSFR} design also features interesting flow
patterns that greatly affect the steady-state and transient behavior. This
present work will first present a verification of Moltres' \gls{MSFR}
diffusion neutronics against the Monte Carlo neutron transport software
Serpent 2, followed by a verification of
the coupled neutronics/thermal-hydraulics steady-state and accident transient
results against two sets of results published by
Fiorina et al. \cite{fiorina_modelling_2014}. The two sets of results arose
from a collaborative benchmarking exercise by researchers at Politecnico di
Milano and Technical University of Delft with two separate \gls{MSR}
simulation tools. Section \ref{sec:litrev} discusses these tools
in greater detail. The
secondary objective is to identify areas of improvement in Moltres for future
development.

\section{Thesis Outline}

The outline of this thesis is as follows. Chapter 2 discusses the history and
features of \glspl{MSR}, and a literature review of existing \gls{MSR}
simulation tools. The chapter also covers the \gls{MSFR} concept in greater
detail. Chapter 3 details the software and the general modeling
approach for generating the results in this thesis. Chapter 4 provides a
neutronics assessment by comparing key neutronics parameters from Moltres'
eigenvalue calculations to Serpent's Monte Carlo calculations. Chapter 5
presents steady-state results of coupled neutronics/thermal-hydraulics
\gls{MSFR} simulations in Moltres. Chapter 6 presents accident transient
simulation results for unprotected reactivity insertions, unprotected loss of
heat sink, unprotected loss of flow, and unprotected pump overspeed. Lastly,
Chapter 7 summarizes the key findings in this thesis
and posits some potential avenues for future work.
