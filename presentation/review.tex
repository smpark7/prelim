\begin{frame}
  \frametitle{Hybrid $S_N$-Diffusion Method: Literature Review}
  \textbf{Control Rods in MSR}
  \begin{itemize}
    \item MSRs contain comparatively fewer control rods than most other reactor types due to:
      \begin{itemize}
        \item Uniform liquid fuel burnup
        \item Strong passive safety of the liquid fuel form
        \item Low excess reactivity of fuel inventory
        \item Availability of other control mechanisms:
        \begin{itemize}
          \item Varying pump speeds
          \item Varying heat removal rates at the heat exchangers
          \item Online refueling and salt processing
        \end{itemize}
      \end{itemize}
    \item Many MSR designs include control rods to facilitate reactor start-up or shut-down rather
      than reactivity control during power operation.
  \end{itemize}
  \pause
  \textbf{Nevertheless, it is important to characterize control rod effects in all relevant
  transient scenarios.}
\end{frame}

\begin{frame}
  \frametitle{Hybrid $S_N$-Diffusion Method: Literature Review}
  \textbf{The Control Rod Modeling Dilemma}
  \begin{enumerate}
    \item Neutron diffusion, $P_1$, and $SP_N$ methods perform poorly near control rod regions
      due to the highly anisotropic neutron fluxes and steep flux gradients.
    \item High-fidelity neutron transport methods remain too computationally expensive for routine
      time-dependent multiphysics simulations.
  \end{enumerate}
\end{frame}

\begin{frame}
  \frametitle{Hybrid $S_N$-Diffusion Method: Literature Review}
  \textbf{Control Rod Modeling in MSR Multiphysics Studies}

  Common simplifications applied to control rod modeling include:
  \begin{itemize}
    \item Homogenized geometries containing static control rods with albedo neutron flux boundary
      conditions \cite{kophazi_development_2009} or transport-corrected cross sections
      \cite{cui_development_2021, jaradat_development_2021, yang_development_2022}
    \item Scaling the neutron source term to simulate moving control rods
      \cite{delpech_benchmark_2003, krepel_dyn3d-msr_2007, jaradat_development_2021,
      yang_development_2022}
  \end{itemize}

  \begin{block}{\textbf{Technical Gap}}
    \textbf{There are no existing MSR simulation tools or studies involving moving control rods
    which are explicitly modeled.}
  \end{block}
\end{frame}

\begin{frame}
  \frametitle{Hybrid $S_N$-Diffusion Method: Literature Review}
  \textbf{Transport-Correction Techniques For Neutron Diffusion-Based Solvers}
  \vspace{.3cm}

  Techniques for augmenting the neutron diffusion method with corrections derived from neutron
  transport
  \begin{itemize}
    \item Absorber Blackness
    \item Method of Equivalent Cross Sections (MECS)
    \item Response-Based Methods
    \item Ronen Method
    \item Transport-Corrected Diffusion Theory
    \item General Equivalence Theory (GET)
    \item Superhomogenization Method (SPH)
  \end{itemize}
\end{frame}

\begin{frame}
  \frametitle{Hybrid $S_N$-Diffusion Method: Literature Review}
  \begin{block}{\textbf{The Ronen Method \cite{ronen_accurate_2004}}}
    Starting with an initial neutron diffusion flux solution, a transport
    expression is used to iteratively improve the flux solution by updating the
    diffusion coefficients:
    %
    \begin{align}
      D(\vec{r},E) =& -\frac{J_{tr}(\vec{r},E)}{\nabla \phi(\vec{r},E)}
      \label{eq:ronen}
      \shortintertext{where}
      J_{tr} =& \mbox{ transport-derived neutron current.} \nonumber
    \end{align}
    %
    In 1-D slab geometry, the integral expression for calculating the current is:
    %
    \begin{align}
      J(x,E) = \frac{1}{2}\int^a_0 dx'\ &E_2[\tau(x',x,E)sgn(x-x')q_0(x',E) \nonumber \\
      &+\frac{3}{2}\int^a_0dx' \ E_3[\tau(x',x,E)]q_1(x',E)
    \end{align}
    \textbf{Limitation: Demonstrated for 1-D geometries only.}
  \end{block}
\end{frame}

\begin{frame}
  \frametitle{Hybrid $S_N$-Diffusion Method: Literature Review}
  \begin{block}{\textbf{Transport-Corrected Diffusion Theory \cite{pounders_diffusion_2009}}}
    Pounders \& Rahnema developed two separate methods which also
    generate space-dependent diffusion coefficients for transport corrections.
    \vspace{.1cm}
    \begin{columns}
      \column[t]{.5\textwidth}
      \textbf{1) Averaged Eddington Factor Diffusion Theory}
      %
      \begin{align}
        \small
        E_g(z) =& \frac{\int^1_{-1} \mu^2\psi(z,\mu)d\mu}{\int^1_{-1} \psi(z,\mu)d\mu}
      \end{align}
      %
      \begin{align}
        \small
        D^{AEF}_g(z) =& E^i_g\left[\hat{\Sigma}_{t,g}-\sum^G_{g'=1}\hat{\Sigma}^{g'\rightarrow g}_{s1}
        \frac{\hat{J}_{g'}}{\hat{J}_g}\right]^{-1}
      \end{align}
      \hfill
      \column[t]{.5\textwidth}
      \textbf{2) High-Order Empirical Diffusion Coefficients}
      \begin{align}
        D^i_g =& -\frac{\left(z_{i+1}-z_i\right) \bar{J}_g}{\left[\phi_g(z_{i+1})-\phi_g(z_i)\right]}
        \label{eq:emp}
      \end{align}
    \end{columns}
    \textbf{Limitation: Both methods require a priori knowledge of the neutron flux and current
    solution.}
  \end{block}
\end{frame}

\begin{frame}
  \frametitle{Hybrid $S_N$-Diffusion Method: Literature Review}
  \begin{block}{\textbf{Summary}}
    \begin{itemize}
      \item Demonstrations of the Ronen method are limited to 1-D geometries due to the difficulty of
        deriving transport operators for complex 2-D and 3-D geometries
      \item The transport-corrected diffusion theories by Pounders \& Rahnema require a priori
        knowledge of the neutron flux and current solution
      \item However, their work showed that transport-derived space-dependent diffusion
        coefficients are effective tools applying transport corrections to the neutron diffusion
        equation
    \end{itemize}
  \end{block}
\end{frame}
