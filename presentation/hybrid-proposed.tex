\begin{frame}
  \frametitle{Hybrid $S_N$-Diffusion Method: Proposed Work}
  \begin{block}{\textbf{Development of a Hybrid $S_N$-Diffusion Method for Control Rod Modeling}}
    \textbf{Sub-objectives}
    \begin{enumerate}
      \item Development \& Implementation of the Hybrid Method in Moltres
      \item Verification of the Hybrid Method Against Reference OpenMC Calculations
      \item Computational Performance Characterization of the Hybrid Method
    \end{enumerate}
  \end{block}
\end{frame}

\begin{frame}
  \frametitle{Hybrid $S_N$-Diffusion Method: Proposed Work}
  \textbf{Sub-objective 1: Development \& Implementation of the Hybrid Method in Moltres}
  \vspace{.2cm}

  \textbf{Implementation Details:}
  \begin{itemize}
    \item 3-D $S_N$ solver with a diffusion synthetic acceleration scheme
    \item Automate multigroup $S_N$-diffusion coupling through the MOOSE MultiApp and Action
      systems
    \item Supporting features (e.g., SVDC calculations, buffer zone calculations)
  \end{itemize}
\end{frame}

\begin{frame}
  \frametitle{Hybrid $S_N$-Diffusion Method: Proposed Work}
  \textbf{Sub-objective 1: Development \& Implementation of the Hybrid Method in Moltres}
  \vspace{.2cm}

  \textbf{\gls{SVDC} Calculation}

  The existing formulation for SVDCs are undefined at flux peaks and troughs:
  \begin{align}
    D^s_g(x) &= -J^{tr}_g(x)\bigg/\frac{d\phi^{tr}_g(x)}{dx} \nonumber
  \end{align}
  I will explore alternative formulations to avoid division by the flux gradient. For instance,
  Tomatis \& Dall'Osso \cite{tomatis_application_2011} developed the following formulation for
  additive corrections for the Ronen method:
  \begin{align}
    \delta D(x_{i+1/2},E) =& -\delta J(x_{i+1/2},E) \frac{(\Delta x_{i+1}+\Delta x_i)/2}{
    \phi(x_{i+1},E)+\phi(x_i,E)}
    \shortintertext{where}
    x_i =& \mbox{ $i$-th spatial interval,} \nonumber \\
    \delta J(x,E) =& J_{tr}(x,E) - J_D(x,E), \nonumber \\
    \Delta x_i =& \mbox{ size of $i$-th spatial interval.} \nonumber
  \end{align}
\end{frame}

\begin{frame}
  \frametitle{Hybrid $S_N$-Diffusion Method: Proposed Work}
  \textbf{Sub-objective 1: Development \& Implementation of the Hybrid Method in Moltres}
  \vspace{.2cm}
  
  \textbf{Correction Region Size}
  \vspace{.1cm}

  The correction region represents the problem domain of the $S_N$ sub-solver.
  \vspace{.2cm}
  
  I will investigate 2-D and 3-D problems derived from the MSRE design to create a set of criteria
  for the minimum required correction region size. My preliminary work showed that the:
  \begin{itemize}
    \item geometrical heterogeneity and
    \item optical thickness
  \end{itemize}
  strongly influence the \gls{SVDC} and the minimum correction region size.
\end{frame}

\begin{frame}
  \frametitle{Hybrid $S_N$-Diffusion Method: Proposed Work}
  \textbf{Sub-objective 1: Development \& Implementation the Hybrid Method in Moltres}
  \vspace{.2cm}

  \textbf{Autonomous Buffer Zone Determination}
  
\end{frame}

\begin{frame}
  \frametitle{Hybrid $S_N$-Diffusion Method: Proposed Work}
  \textbf{Sub-objective 2: Implement and extend the hybrid method for 2-D and 3-D reactor modeling
  in Moltres}
  \vspace{.2cm}

  This sub-objective involves four main tasks:
  \begin{itemize}
    \item Implement a $S_N$ solver in Moltres/MOOSE
    \item Implement a coupling framework between the $S_N$ solver and the existing neutron
      diffusion solver
    \begin{itemize}
      \item Meshing of correction region
      \item Data transfers (e.g., flux, boundary conditions)
    \end{itemize}
    \item Determination of the correction region
      \begin{itemize}
        \item Develop a set of criteria for setting up the correction region
      \end{itemize}
    \item Autonomous determination of the buffer zone
  \end{itemize}
\end{frame}

\begin{frame}
  \frametitle{Hybrid $S_N$-Diffusion Method: Proposed Work}
  \textbf{Sub-objective 4: Verify the hybrid method against reference OpenMC calculations}
  \vspace{.3cm}

  I will verify the hybrid method against reference OpenMC calculations of several toy problems,
  leading up to a 3-D model of the Molten Salt Reactor Experiment (MSRE). The verification study
  will include permutations of the following factors:
  \begin{itemize}
    \item 2-D and 3-D models
    \item Asymmetric control rod positions
    \item Static control rods at various levels of insertion
  \end{itemize}
  Stretch goal: Validate the hybrid method against the MSRE pump start-up and coast-down
  transients.
\end{frame}

\begin{frame}
  \frametitle{Hybrid $S_N$-Diffusion Method: Proposed Work}
  \textbf{Sub-objective 5: Characterize the computational performance of the hybrid method}
  \vspace{.3cm}

  The computational performance of the hybrid $S_N$-Diffusion method will be compared against the
  reference $S_N$ method. The hybrid method is expected to be faster due to:
  \begin{itemize}
    \item the small correction region size relative to the full reactor geometry
    \item faster convergence of SVDCs compared to neutron flux in $S_N$ calculations
  \end{itemize}
  Stretch goal: Explore the implementation of the hybrid method as a multischeme method,
  i.e., the $S_N$ and diffusion solvers run simultaneously.
\end{frame}
