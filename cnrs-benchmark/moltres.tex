\section{Modeling Approach with Moltres} \label{sec:model}

This section describes the specific modeling approach for
simulating the CNRS Benchmark cases in Moltres.

For this work\footnote{The input files for
all benchmark
cases are available on the Moltres GitHub repository at 
\url{https://github.com/arfc/moltres/tree/devel/problems/2021-cnrs-benchmark}.
}, I ran the benchmark cases on a uniformly-spaced mesh
of 200$\times$200 elements. Thus, the dimensions of each mesh element are
0.01m$\times$0.01m. I adopted the group constant data
provided by Tiberga et al. \cite{tiberga_results_2020}. Next, I
discretized most variables, i.e., neutron fluxes, velocity
components, pressure, and temperature, using continuous, first-order, Lagrange
shape functions. The only exception is the precursor concentration variables,
which I discretized using zeroth-order monomial shape functions and solved
using a \gls{DFEM}. I interpolated the resulting discontinuous,
cell-centered precursor values to obtain the nodal values for results
analysis.

The
\texttt{Navier-Stokes} and \texttt{Heat} \texttt{Conduction} modules from
\gls{MOOSE} provide some of the capabilities for
modeling incompressible flow and heat transfer. In particular, I stabilized
the incompressible flow and temperature governing equations using the
\gls{SUPG} stabilization method implemented in \gls{MOOSE}
\cite{peterson_overview_2018}. Without \gls{SUPG} stabilization, I
observed spurious numerical oscillations in the velocity and temperature near
the top boundary due to the singularity on the top left corner where different
velocity boundary conditions meet. I also applied the \gls{PSPG} stabilization
scheme \cite{hughes_new_1986} from the Navier-Stokes module
\cite{peterson_overview_2018},
which enables equal-order discretizations in the velocity and pressure
variables. Equal-order discretizations with \gls{PSPG} are computationally
cheaper and more convenient than implementing higher-order
velocity discretizations for stability without \gls{PSPG}
\cite{chapelle_inf-sup_1993}.

Using the inverse power method solver in \gls{MOOSE}, I ran all eigenvalue calculations in
Steps 0.2, 1.1, 1.2, 1.3, and 1.4. I ran all other steps
using the Preconditioned Newton-Krylov solver
\cite{gaston_physics-based_2015}. The coupled steady-state problems in
Steps 1.2, 1.3, and 1.4 required segregated solvers for the neutronics
and the thermal-hydraulics due to the unique problem setups involving a
criticality search problem for the neutron multiplication factor
and a steady-state problem in thermal-hydraulics simultaneously.

\begin{table}[tb]
    \caption{Timestep sizes used for the time-dependent cases in
    Step 2.1, corresponding to 1/200th of the perturbation period.}
	\centering
	\setlength\tabcolsep{2.5pt}
	\begin{tabular}{l l l l l l l l}
	    \toprule
	    Frequency [Hz] & 0.0125 & 0.025 & 0.05 & 0.1 & 0.2 & 0.4 & 0.8 \\
	    \midrule
	    Timestep size [s] & 0.2 & 0.2 & 0.1 & 0.05 & 0.025 & 0.0125 & 0.00625
	    \\
	    \bottomrule
	\end{tabular}
	\label{table:timestep}
\end{table}

For the time-dependent cases in Step 2.1, I employed full coupling with
a second-order implicit Backward Differential Formula (BDF2) time-stepping
scheme. I set the timestep sizes for each driving frequency in the heat transfer
coefficient to 1/200th of the perturbation period. Table
\ref{table:timestep} shows the timestep sizes. I assumed the
systems reached asymptotic behavior when the magnitudes of neighboring power
peaks differed by less than 0.001\% for at least ten wavelengths. Under this
assumption, the phase shift measurements between adjacent waves always
converged before the magnitude measurements of the power peaks.

Table \ref{table:software} compares the numerical methods, meshing schemes, and
neutronics models of Moltres and the four participating software packages in
the CNRS benchmark paper \cite{tiberga_results_2020}. The $SP_N$ and
$S_N$ neutronics models refer to the simplified $P_N$ spherical harmonics and
$S_N$ discrete ordinates neutron transport models, respectively. Based on the
solvers and methods of solution, Moltres is most similar to the
PHANTOM-$S_N$ + DGFlows \cite{tiberga_discontinuous_2019} multiphysics package
from \gls{TUD} with the $S_2$ neutron transport model. Participants from
\gls{CNRS} and \gls{PSI}
employed non-uniform meshes which were refined near the boundaries. In contrast,
we and the \gls{PoliMi} and \gls{TUD} participants employed uniform meshes.

\FloatBarrier

\begin{landscape}
\begin{table*}[p]
    \caption{List of software packages and their corresponding model
    specifications for the CNRS Benchmark simulations
    \cite{tiberga_results_2020}.}
    \centering
    \begin{tabular}{p{4.2cm} p{7cm} p{3.3cm} p{2cm} p{2.7cm}}
        \toprule
        Software & Institute & Numerical method & Mesh & Neutronics model \\
        \midrule
        OpenFOAM & Centre national de la recherche scientifique (CNRS) & Finite volume & 200$\times$200 \newline Non-uniform & $SP_1$ \& $SP_3$ \\
        OpenFOAM & Politecnico di Milano (PoliMi) & Finite volume & 400$\times$400 \newline Uniform & Neutron diffusion \\
        GeN-Foam & Paul Scherrer Institute (PSI) & Finite volume & 200$\times$200 \newline Non-uniform & Neutron diffusion \\
        PHANTOM-$S_N$+DGFlows & Delft University of Technology (TUD) & Discontinuous finite \newline element & 50$\times$50 \newline Uniform & $S_2$ \& $S_6$ \\
        Moltres (This work) & University of Illinois at Urbana-Champaign (UIUC) & Continuous \& discontinuous finite element & 200$\times$200 \newline Uniform & Neutron diffusion \\
        \bottomrule
    \end{tabular}
    \label{table:software}
\end{table*}
\end{landscape}

\FloatBarrier
